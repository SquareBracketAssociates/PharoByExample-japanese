% $Author$
% $Date$
% $Revision$

% HISTORY:
% 2006-11-19 - Stef added French version of Hilaire's morphic article
% 2006-12-10 - Pollet translating
% 2007-08-16 - Oscar edit
% 2007-11-05 - Andrew edit
% 2009-07-07 - Oscar migrate to Pharo

%=================================================================
\ifx\wholebook\relax\else
% --------------------------------------------
% Lulu:
	\documentclass[a4paper,10pt,twoside]{book}
	\usepackage[
		papersize={6.13in,9.21in},
		hmargin={.75in,.75in},
		vmargin={.75in,1in},
		ignoreheadfoot
	]{geometry}
	\input{../common.tex}
	\pagestyle{headings}
	\setboolean{lulu}{true}
% --------------------------------------------
% A4:
%	\documentclass[a4paper,11pt,twoside]{book}
%	\input{../common.tex}
%	\usepackage{a4wide}
% --------------------------------------------
    \graphicspath{{figures/} {../figures/}}
	\begin{document}
	% \renewcommand{\nnbb}[2]{} % Disable editorial comments
	\sloppy
\fi
%=================================================================
\chapter{Morphic}

%\sd{We should first give a conceptual overview.
%Then we need a cookbook of how to do simple things in Morphic.
%The observer pattern and its implementation with changed:  and update: messages could go here.  Or in ``Idiomatic design patterns'' later.}

\indmain{Morphic}は、\pharo のGUIに付けられた名前です。
Morphicは\st で書かれており、様々なOSの間で完全な互換性があります。その結果、\pharo はUnix, MacOS, Windowsのいずれにおいてもまったく同じ見た目となっています。
Morphicと他の多くのユーザ・インターフェース・ツールキットとの違いは、インターフェースを「構成すること」と「実行すること」とのモードを分けていないということです。画面上のすべてのグラフィック要素(graphical elements)は、いつでもユーザが組み立てたり分解したりできます。\footnote{Hilaire Fernandesに謝意を表します。彼のフランス語で書かれたオリジナル版をこの章のベースとすることを許諾してくれました。}

\ab{After the first printing, I took an editing pass, correcting some errors and grammatical infelicities.}

\on{I have commented out the LabelstickerMorph and PyramidMorph examples, as they do not really add much over the other examples we have already. The source code is now available in the example subdirectory, in case someone would like to try and use them after all.}

%=================================================================
\section{Morphicの歴史}

Morphicは、プログラミング言語\ind{Self}向けとして、John MaloneyとRandy Smithによって1993年頃から開発が始められました。
Maloneyは\squeak のために新しくMorphicを書き直しましたが、Self版の基本的な考え方は今でも\pharo のMorphicに受け継がれています。その基本的な考え方とは、\emph{直接性}と\emph{生命感}です。
直接性とは、画面上の図形がマウスでクリックして直接調べたり変更したりできるオブジェクトであるということです。
生命感とは、ユーザインターフェースがいつでもユーザの操作に反応できることです。つまり、画面上の情報は、それが表している世界が変わるたびに自動的に更新されます。
簡単な例を挙げれば、メニューから項目を取り外してボタンとして使うことができます。

\dothis{Worldメニューを出します。その上で\Metaclick を一回すると、Morphic ハローが現れます。\footnote{プリファレンス・ブラウザで、\button{halosEnabled}を設定することを忘れないでください。または、ワークスペース上で\ct{Preferences enable: \#halosEnabled}を実行します。}その後、取り外したいメニュー項目の上でさらに\metaclick をすると、メニュー項目の上にハローが現れます。\figref{detachingMenu}で示すように、黒いハンドル\grabHandle を掴んでその項目を画面の適当な場所にドラッグします。}
\index{Morphic!halo}
\index{blue button}

\begin{figure}[ht]
	\centerline{\includegraphics[width=0.3\textwidth]{detachingMenu}}
	\caption{\menu{Workspace}メニューのモーフを取り外して独立したボタンにしているところ。
		\figlabel{detachingMenu}}
\end{figure}

\pharo の起動後、画面上にあるすべてのオブジェクトは、\emph{モーフ}と呼ばれます。つまり、これらのオブジェクトは\ct{Morph}クラスのインスタンスです。
\mbox{\ct{Morph}}クラスは多くのメソッドを持つ大きなクラスであるため、少ないコードで面白い動作をするサブクラスを作れます。
あらゆるオブジェクトを表すモーフを作れますが、どのくらい良い表現が得られるかはオブジェクトによります。

\dothis{文字列オブジェクトを表すモーフを作るには、ワークスペースで以下のコードを実行します。} % , one line at a time.}
\begin{code}{}
'Morph' asMorph openInWorld
\end{code}
\cmindex{Morph}{openInWorld}

%\begin{code}{}
%s := 'Morph' asMorph openInWorld.
%s openViewerForArgument
%\end{code}
%\cmindex{Morph}{openInWorld}
% ON: openViewerForArgument is gone in pharo!

これは、\ct{'Morph'}という文字列を表すモーフを作り、「ワールド」上で開きます。(つまり、表示します)「ワールド」とは、\pharo の画面のことです。
これにより\metaclick で操作できるグラフィック要素 --- モーフ --- が得られます。
%The second line opens a ``viewer'' that shows you attributes of this Morph, such as its \ct{x} and \ct{y}  coordinates on the screen.  Clicking on one of the yellow exclamation marks sends a message to the Morph, which responds appropriately.

もちろん、今見ているものより面白い表現を持つモーフを定義することも可能です。
class \ct{Object}クラスにある\mthind{Object}{asMorph}メソッドのデフォルトの実装では、単にStringMorphを作るだけです。
例えば、\ct{Color tan asMorph}は、単に\clsind{Color} \ct{tan printString}の実行結果をラベルに持つStringMorphを返します。
それでは、これを変更して色の付いた長方形を代わりに返すようにしましょう。

\dothis{ブラウザを開き\ct{Color}クラスに以下のメソッドを追加します。}
\needlines{3}
\begin{method}{\ct{Color}のインスタンスを表すモーフを返す}
Color>>>asMorph
	^ Morph new color: self
\end{method}
\noindent
ワークスペースで\ct{Color orange asMorph} \mthind{Morph}{openInWorld}を実行すれば、文字列のモーフの代わりにオレンジ色の長方形が現れます。


%=================================================================
\section{モーフを操作する}

モーフはオブジェクトです。つまり、\st の他のオブジェクトと同様に操作できます。
メッセージを送ることで属性を変えたり、新しいMorphのサブクラスを作ったりできます。
どんなモーフも(たとえ画面に表示されていなくとも)位置と大きさを持っています。
便宜上、すべてのモーフは画面上の四角形の領域を占めていると考えます。不規則な形状の場合であっても、その位置と大きさは、そのモーフを囲む最小の四角形の「箱」となります。この箱を、モーフのバウンディングボックス、あるいは単に「bounds」と言います。
\mthind{Morph}{position}メソッドは、モーフの左上の座標(またはモーフのバウンディングボックスの左上の座標)を表す\ct{Point}オブジェクトを返します。
この座標系の原点は画面の左上の端にあります。$y$座標は画面の\emph{下方向に}増えていき、$x$座標は右方向に増えていきます。
\ct{extent}メソッドも\ct{Point}オブジェクトを返しますが、位置ではなくモーフの幅と高さからなる座標値を返します。

\dothis{ワークスペースに以下のコードを入力し\menu{do it}します。}
\begin{code}{}
joe := Morph new color: Color blue.
joe openInWorld.
bill := Morph new color: Color red .
bill openInWorld.
\end{code}
\noindent
さらに、\ct{joe position}と入力して\menu{print it}します。joeを動かすために、\ct{joe position: (joe position + (10@3))}を繰り返して実行します。

大きさについても同様のことが可能です。
\ct{joe} \mthind{Morph}{extent}はjoeの大きさを返します。
joeを大きくするには、\ct{joe extent: (joe extent * 1.1)}を実行します。
モーフの色を変えるには、好きな\ct{Color}オブジェクトを引数にして\mthind{Morph}{color:}メッセージを送ります。例えば、\ct{joe color: Color orange}のようにします。
透明度を設定するには、\ct{joe color: (Color orange alpha: 0.5)}を試してください。

\dothis{billにjoeの後を追いかけさせるには、以下のコードを繰り返して実行します。}
\begin{code}{}
bill position: (joe position + (100@0))
\end{code}
\noindent
マウスでjoeを移動した後にこのコードを実行すれば、billはjoeの100ピクセル右側に移動するでしょう。
\ab{It would seem that this would be a good place to introduce the \ct{step} method}

%=================================================================
\section{モーフの組み込み}

新しいグラフィカルな表現を作る一つの方法は、モーフの内側に別のモーフを置くことです。
これをモーフの\emph{組み込み}と言います。モーフはどんな深さにも組み込めます。
%
%To create new morphs, there are two main techniques that you can combine:
%\begin{enumerate}
%	\item by composing morphs one into another,
%	\item by subclassing \ct{Morph} and overriding \mthind{Morph}{drawOn:} to draw original morph shapes.
%\end{enumerate}
%}
\index{Morph!composing}
モーフを別のモーフに組み込むには、\mthind{Morph}{addMorph:}というメッセージを入れ物となるモーフに送ります。

\dothis{あるモーフを別のモーフに加えましょう。}
\begin{code}{}
star := StarMorph new color: Color yellow.
joe addMorph: star.
star position: joe position.
\end{code}

\noindent
最後の行では、星をjoeと同じ座標に置いています。
組み込まれるモーフの座標は、入れ物となるモーフを基準にするのではなく画面を基準とすることに気をつけてください。
モーフの位置を設定するための多くのメソッドがあります。
\ct{Morph}クラスの\protind{geometry}をブラウズすれば自分自身で確認できます。
例えば、joeの中心に星を置くには、\ct{star} \mthind{Morph}{center:} \ct{joe center}を実行します。

\begin{figure}[ht]
	\centerline{\includegraphics{joeStar}}
	\caption{青色の半透明なモーフであるjoeが星を含んでいる。
		\figlabel{joeStar}}
\end{figure}

マウスで星を掴もうとすると実際にはjoeを掴み、二つのモーフが一緒に動きます。星はjoeの内側に\emph{埋め込まれて}います。
joeの内側にもっと多くのモーフを埋め込むことができます。
プログラムによる埋め込みに加え、直接的な操作によっても埋め込むことができます。

%=================================================================
\section{自由にモーフを作り、描く}

モーフの組み込みによりたくさんの便利で面白いグラフィカルな表現を作れますが、いずれ今までとまったく違う何かを作りたいと思うでしょう。
\index{Morph!subclassing}
そのためには、\ct{Morph}のサブクラスを作り\mthind{Morph}{drawOn:}メソッドをオーバーライドすれば、見た目を変更できます。

Morphicのフレームワークは、モーフの再描画が必要となった際に、モーフに対して\ct{drawOn:}メッセージを送ります。\ct{drawOn:}メッセージの引数は、\clsind{Canvas}クラス(またはそのサブクラス)のインスタンスで、期待される振る舞いはモーフが自分自身をキャンバスのbounds内に描くことです。
この知識を使って十字モーフを作ってみましょう。
\index{Morph!subclassing}

\dothis{ブラウザを開き、\ct{Morph}を継承した\clsind{CrossMorph}クラスを新しく定義します。}
\begin{classdef}{\ct{CrossMorph}を定義する}
Morph subclass: #CrossMorph
	instanceVariableNames: ''
	classVariableNames: ''
	poolDictionaries: ''
	category: 'PBE-Morphic'
\end{classdef}

以下のように\ct{drawOn:}メソッドを定義できます。
\begin{method}[firstDrawOn]{\ct{CrossMorph}を描く}
drawOn: aCanvas 
	| crossHeight crossWidth horizontalBar verticalBar |
	crossHeight := self height / 3.0 .
	crossWidth := self width / 3.0 .
	horizontalBar := self bounds insetBy: 0 @ crossHeight.
	verticalBar := self bounds insetBy: crossWidth @ 0.
	aCanvas fillRectangle: horizontalBar color: self color.
	aCanvas fillRectangle: verticalBar color: self color
\end{method}


\begin{figure}[hbt]
	\ifluluelse
		{\centerline{\includegraphics[width=0.3\textwidth]{NewCross}}}
		{\centerline{\includegraphics{NewCross}}}
	\caption{Haloを伴った\ct{CrossMorph}。自由に大きさを変更できる。
		\figlabel{cross}}
\end{figure}


モーフに\mthind{Morph}{bounds}メッセージを送ると、\clsind{Rectangle}クラスのインスタンスであるバウンディングボックスが返されます。\clsind{Rectangle}クラスのインスタンスは数多くのメッセージを理解し、四角形に関連する他の四角形を作れます。この例では、引数に点を指定した\ct{insetBy:}メッセージを使って高さを低くした四角形を作り、さらに幅を縮めた四角形を作っています。

\dothis{新しいモーフをテストするために、\ct{CrossMorph new} \mthind{Morph}{openInWorld}を実行します。}
実行すると\figref{cross}のようになります。
しかし、マウスに反応する場所(つまりモーフを掴むのにクリックするところ)がバウンディングボックス全体になっていることに気づくと思います。これを解決しましょう。

Morphicフレームワークは、モーフがカーソルの下にあるかどうか知る必要があるとき、マウスポインタの下にバウンディングボックスがあるすべてのモーフに対して、\ct{containsPoint:}メッセージを送ります。
つまり、マウスに反応する場所をモーフの十字部分に制限するには、\ct{containsPoint:}メソッドをオーバーライドします。


\dothis{\ct{CrossMorph}クラスに以下のメソッドを定義します。}

\needlines{4}
\begin{method}[firstContains]{\ct{CrossMorph}のマウスに反応する場所を設定する。}
containsPoint: aPoint
	| crossHeight crossWidth horizontalBar verticalBar |
	crossHeight := self height / 3.0.
	crossWidth := self width / 3.0.
	horizontalBar := self bounds insetBy: 0 @ crossHeight.
	verticalBar := self bounds insetBy: crossWidth @ 0.
	^ (horizontalBar containsPoint: aPoint)
		or: [verticalBar containsPoint: aPoint]
\end{method}

このメソッドは\ct{drawOn:}と同じロジックを使っています。つまり、\ct{containsPoint:}が\ct{true}と答える領域は、\ct{drawOn}によって色が付けられた領域と一致すると確実に言えます。
\ct{Rectangle}クラスの\mthind{Rectangle}{containsPoint:}を活用すれば、この面倒な仕事をやりとげることができます。

メソッド\mthsref{firstDrawOn}と\ref{mth:firstContains}のコードには二つの問題点があります。
明らかな点は、コードが重複していることです。
これはとても重大な過ちです。というのも、\ct{horizonatalBar}や\ct{verticalBar}の計算方法を変える必要があるときに、二箇所にあるコードの一方を修正するのを簡単に忘れてしまうからです。
解決方法は、これらの計算を新たな二つのメソッドに抽出することです。どちらも\ct{private}プロトコルに格納します。

\needlines{4}
\begin{method}{\ct{horizontalBar}.}
horizontalBar
	| crossHeight |
	crossHeight := self height / 3.0.
	^ self bounds insetBy: 0 @ crossHeight
\end{method}

\needlines{4}
\begin{method}{\ct{verticalBar}.}
verticalBar
	| crossWidth |
	crossWidth := self width / 3.0.
	^ self bounds insetBy: crossWidth @ 0
\end{method}

\noindent
これらのメソッドを使って\ct{drawOn:}と\ct{containsPoint:}の両方を書き直します。

\needlines{4}
\begin{method}{リファクタリング後の\ct{CrossMorph>>>drawOn:}.}
drawOn: aCanvas 
	aCanvas fillRectangle: self horizontalBar color: self color.
	aCanvas fillRectangle: self verticalBar color: self color
\end{method}

\needlines{4}
\begin{method}{リファクタリング後の\ct{CrossMorph>>>containsPoint:}.}
containsPoint: aPoint 
	^ (self horizontalBar containsPoint: aPoint)
		or: [self verticalBar containsPoint: aPoint]
\end{method}

プライベートメソッドに意味のある名前を付けたおかげで、このコードはより簡単に理解できるようになりました。
実際、すぐに2番目の問題が明らかになります。それは、十字の中央の部分がhorizontalBarとverticalBarの両者の下になっており、2度描かれてしまうという問題です。
不透明な色を塗る場合には問題になりませんが、\figref{overdrawBug}に示すような半透明な色で描くとすぐにこのバグが露見します。

\begin{figure}[t]
\begin{minipage}{0.48\textwidth}
	\ifluluelse
		{\centerline{\includegraphics[scale=0.6]{overdrawBug}}}
		{\centerline{\includegraphics{overdrawBug}}}
	\caption{十字の中央が2度描かれる。
		\figlabel{overdrawBug}}
\end{minipage}
\hfill
\begin{minipage}{0.48\textwidth}
	\ifluluelse
		{\centerline{\includegraphics[scale=0.6]{hairlineBug}}}
		{\centerline{\includegraphics{bug}}}
	\caption{塗られていない部分のある十字モーフ。
		\figlabel{bug}}
\end{minipage}
\end{figure}

\needlines{4}
\dothis{ワークスペースで、以下のコードを1行ずつ実行します。}

\begin{code}{}
m := CrossMorph new bounds: (0@0 corner: 300@300).
m openInWorld.
m color: (Color blue alpha: 0.3).
\end{code}

\noindent
この解決法は、verticalBarを三つの部分に分け、上側と下側の部分だけ塗りつぶすということです。
\ct{Rectangle}クラスでこの面倒な仕事をしてくれるメソッドがまた見つかります。\ct{r1 areasOutside: r2}は、\ct{r2}の外側に出ている\ct{r1}を構成する四角形の配列を返します。
以下が改良されたコードになります。

\begin{method}{\ct{drawOn:}メソッドの改良版。十字の中心を一度だけ塗りつぶす。}
drawOn: aCanvas 
	| topAndBottom |
	aCanvas fillRectangle: self horizontalBar color: self color.
	topAndBottom := self verticalBar areasOutside: self horizontalBar. 
	topAndBottom do: [ :each | aCanvas fillRectangle: each color: self color]
\end{method}

このコードは正しく動作するように見えますが、何回か十字の大きさを変更しようとすると、\figref{bug}のように、ある大きさのときに1ピクセル分の幅の線が十字の下側と残りの部分を分けてしまうことに気づきます。
これは丸め誤差のせいです。塗りつぶそうとする四角形の大きさが整数でないときに、\ct{fillRectangle: color:}
が不適切に数値を丸めてしまい、1ピクセル分だけ塗りつぶされなくなるからです。
このバグは、四角形の大きさを計算する際に明示的に四捨五入することで解決できます。

\needlines{5}
\begin{method}{明示的に四捨五入した\ct{CrossMorph>>>horizontalBar}}
horizontalBar
	| crossHeight |
	crossHeight := (self height / 3.0) rounded.
	^ self bounds insetBy: 0 @ crossHeight
\end{method}

\needlines{5}
\begin{method}{明示的に四捨五入した\ct{CrossMorph>>>verticalBar}}
verticalBar
	| crossWidth |
	crossWidth := (self width / 3.0) rounded.
	^ self bounds insetBy: crossWidth @ 0
\end{method}



%=================================================================
%\section{Composing Morphs}

%\on{The source code is in the examples directory.
%For the moment I prefer to leave out the examples, as they do not add much.}

%Below, we present a few morphs that were designed for a course project.

%\paragraph{An adhesive Label} The \ct{LabelStickerMorph} is a metaphor for an adhesive label with a colored border and three lines of text (\figref{labeler}, \egref{labeler}).

%\begin{figure}[ht]
%	\centerline{\includegraphics[width=0.25\textwidth]{labeler}}
%	\caption{The sticker label morph.
%		\figlabel{labeler}}
%\end{figure}

%\begin{example}[labeler]{Creating a sticker label}{}
%label := LabelstickerMorph new openInWorld.
%label text1: 'Confiture sans sucre';
%	text2: 'Fraises du jardin';
%	text3: '9 mai 2006'.
%label lineColor: Color blue
%\end{example}

%\paragraph{A Number Pyramid}
%The previous morph is designed by overriding the \ct{drawOn:} method.
%We built \ct{PyramidMorph} by composing morphs: we used \ct{TextMorph}s to make the blocks and added them to a base morph (\figref{pyramid}, \egref{pyramid}). \damien{figure does not match text... no numbers? Where is the code?}
%\begin{figure}[ht]
%	\centerline{\includegraphics{pyramid}}
%	\caption{The number pyramid morph.
%		\figlabel{pyramid}}
%\end{figure}

%\begin{example}[pyramid]{Manipulating the number pyramid}{}
%pyramid := (PyramidMorph base: 4) openInWorld.
%pyramid block: 8 value: 2
%\end{example}


%=================================================================
\section{対話とアニメーション}

モーフを使って、いきいきとしたユーザインターフェースを実現するには、マウスやキーボードでモーフと対話できなければなりません。
さらに、モーフは自分の見た目や位置を変えることで(つまり自分自身のアニメーションによって)、ユーザからの入力に反応する必要があります。


\subsection{マウスイベント}

マウスボタンが押されたとき、Morphicはマウスポインタの下にある各モーフに対して\ct{handlesMouseDown:}メッセージを送ります。もし、モーフが\ct{true}を返すと、Morphicは直ちにそのモーフに対して\mthind{Morph}{mouseDown:}メッセージを送ります。ユーザがマウスボタンを離した際にも\mthind{Morph}{mouseUp:}メッセージを送ります。
すべてのモーフが\ct{false}を返すと、Morphicはドラッグ&ドロップの操作を開始します。
後述するように、\ct{mouseDown:}と\ct{mouseUp:}メッセージは、\clsind{MouseEvent}オブジェクトの引数を伴って送られます。この引数は、マウスの動作の詳細を表しています。

\ct{CrossMorph}クラスを拡張して、マウスイベントを扱うようにしましょう。
まず、すべての十字モーフが\mthind{Morph}{handlesMouseDown:}メッセージに確実に\ct{true}を返すことから始めます。

\dothis{\ct{CrossMorph}に以下のメソッドを追加します。}
\begin{method}{\ct{CrossMorph}がマウスクリックに反応するように宣言する。}
CrossMorph>>>handlesMouseDown: anEvent
	^true
\end{method}

十字モーフのクリックで赤色に変わり、\actclick で黄色に変わるようにしましょう。
これは\mthref{mouseDown}によって実現できます。

\needlines{7}
\begin{method}[mouseDown]{マウスクリックに反応してモーフの色を変えるようにする。}
CrossMorph>>>mouseDown: anEvent
	anEvent redButtonPressed "click"
		ifTrue: [self color: Color red].
	anEvent yellowButtonPressed "action-click"
		ifTrue: [self color: Color yellow].
	self changed
\end{method}

\ab{I added this note:}
このメソッドでは、モーフの色を変えた後で\ct{self changed}を送っていることに注意してください。
これによって、Morphicは直ちに\ct{drawOn:}を送るようになります。
\ab{However, the \ct{self changed} message seems to be entirely unnecessary; the colour changes instantly without it.}
なお、いったんモーフが\ind{マウスイベント}を処理すると、もはやモーフをマウスで掴んで動かすことができなくなることに気をつけてください。
その代わりにHaloを使う必要があります。モーフ上の\metaclick でHaloを出し、モーフの上部にある茶色の\moveHandle{}か、赤の\grabHandle{}のいずれかでモーフを掴みます。

\ct{mouseDown:}の引数の\ct{anEvent}は、\mbox{\clsind{MouseEvent}クラス}のインスタンスです。これは\lct{Mor\-phic\-Event}クラスのサブクラスです。\ct{MouseEvent}では、\mthind{MouseEvent}{redButtonPressed}と\mthind{MouseEvent}{yellowButtonPressed}メソッドを定義しています。マウスイベントを扱うメソッドにどんなものがあるか調べるには、このクラスをブラウズします。

\subsection{キーボードイベント}

\ind{キーボードイベント}を得るには、次の3ステップを必要とします。
\begin{enumerate}
	\item 「キーボードフォーカス」を特定のモーフに与えます。例えば、マウスがモーフの上にあるときに、フォーカスを与えることができます。
	\item \mthind{Morph}{handleKeystroke:}メソッドを使って、モーフがキーボードイベントを処理するようにします。このメッセージは、ユーザがキーを押した際に、キーボードフォーカスを持っているモーフへ送られます。
	\item マウスがモーフの上にないときに、キーボードフォーカスを解放します。
\end{enumerate}

キー入力に反応するように\ct{CrossMorph}クラスを拡張します。
まず、マウスがモーフ上にあることを通知するように変更します。
これは、\mthind{Morph}{handlesMouseOver:}メッセージに\ct{true}を返すことで実現できます。

\dothis{モーフがマウスポインタの下にあるときに、\ct{CrossMorph}が対応できるよう宣言します。}
\begin{method}{「マウスオーバー」イベントを扱えるようにする。} 
CrossMorph>>>handlesMouseOver: anEvent
	^true
\end{method}

\noindent
マウスの位置の取り扱いは\mthind{Morph}{handlesMouseDown:}の場合と同様です。
マウスポインタがモーフに入る、または離れる際に、\mthind{Morph}{mouseEnter:}と\mthind{Morph}{mouseLeave:}メッセージがモーフに送られます。

\dothis{\ct{CrossMorph}クラスがキーボードフォーカスの獲得と解放を行えるように、二つのメソッドを定義します。三つ目のメソッドで実際にキー入力を処理します。}
\begin{method}{マウスがモーフに入ったときにキーボードフォーカスを獲得する。}
CrossMorph>>>mouseEnter: anEvent
	anEvent hand newKeyboardFocus: self
\end{method}

\begin{method}{マウスポインタがモーフから離れたときに、キーボードフォーカスを解放する。}
CrossMorph>>>mouseLeave: anEvent
	anEvent hand newKeyboardFocus: nil
\end{method}

\begin{method}[handleKeystroke]{キーボードイベントを受け取って処理する。}
CrossMorph>>>handleKeystroke: anEvent
	| keyValue |
	keyValue := anEvent keyValue.
	keyValue = 30	 "up arrow"
		ifTrue: [self position: self position - (0 @ 1)].
	keyValue = 31	 "down arrow"
		ifTrue: [self position: self position + (0 @ 1)].
	keyValue = 29	 "right arrow"
		ifTrue: [self position: self position + (1 @ 0)].
	keyValue = 28	 "left arrow"
		ifTrue: [self position: self position - (1 @ 0)]
\end{method}

矢印キーを使ってモーフを動かすことができるようなメソッドを書きました。
マウスがモーフの上にないときには、\mthind{Morph}{handleKeystroke:}メッセージが送られないため、モーフはキーボードに反応しなくなることに注意してください。
キーの値を知りたいときは、トランスクリプトウィンドウを出し、\glbind{Transcript} \ct{show: anEvent keyValue}を\mthref{handleKeystroke}に加えてください。
\ct{handleKeystroke:}の引数の\ct{anEvent}は、\clsind{KeyboardEvent}のインスタンスで、\clsind{MorphicEvent}クラスのサブクラスです。このクラスをブラウズすれば、キーボードイベントについてさらに学習できます。

\subsection{Morphicアニメーション}

Morphicは主に二つのメソッドからなる単純なアニメーション機能を提供します。\mthind{Morphic}{step}が一定の間隔でモーフに送られ、\mthind{Morphic}{stepTime}は\ct{step}の間隔をミリ秒単位で指定します。\footnote{実際には、\ct{stepTime}は\ct{step}間の\emph{最小}時間を指定します。仮に\ct{stepTime}を1にしても、\pharo の負荷が高くなりすぎて、それほど頻繁にstepをモーフに送れないので注意してください。}
加えて、
\mthind{Morphic}{startStepping}はstepを開始させ、\mthind{Morphic}{stopStepping}はそれを停止させます。stepが始まっているかどうかを知るには、\mthind{Morphic}{isStepping}を使います。
\index{Morphic!animation}

\dothis{以下のようなメソッドを定義して、\ct{CrossMorph}を点滅させます。}
\begin{method}{アニメーションの間隔を設定する。}
CrossMorph>>>stepTime
	^ 100
\end{method}
\begin{method}{アニメーションのstepを作る。}
CrossMorph>>>step
	(self color diff: Color black) < 0.1
		ifTrue: [self color: Color red]
		ifFalse: [self color: self color darker]
\end{method}
\noindent
アニメーションを開始させるには、\ct{CrossMorph}上でインスペクタを開いて(Haloのデバッグハンドル\debugHandle{}を使います)、下部の小さなワークスペースペインに\ct{self startStepping}と入力し、\menu{do it}を実行します。
代わりに\ct{handleKeystroke:}メソッドを変更して、$+$キーと$-$キーでstepを開始、停止させることもできます。

\dothis{以下のコードを\mthref{handleKeystroke}に加えます。}

\begin{code}{}
	keyValue = $+ asciiValue 
		ifTrue: [self startStepping].
	keyValue = $- asciiValue
		ifTrue: [self stopStepping].
\end{code}

% \on{You can also \menu{debug \go inspect morph} and evaluate: \ct{self currentWorld startStepping: self}.}

%=================================================================
\section{対話機能}

ユーザに入力を促すために、\clsind{UIManager}クラスはすぐに使えるダイアログボックスを多く提供しています。
例えば、\mthind{UIManager}{request:initialAnswer:}メソッドは、ユーザによって入力された文字列を返します。(\figref{dialogName})
\begin{code}{}
UIManager default request: 'What''s your name?' initialAnswer: 'no name'
\end{code}

\begin{figure}[htb]
\begin{minipage}{0.48\textwidth}
	\centerline{\includegraphics[width=0.8\textwidth]{dialog}}
	\caption{入力ダイアログ}\figlabel{dialogName}
\end{minipage}
\hfill
\begin{minipage}{0.48\textwidth}
	\vfill
	\centerline{\includegraphics [width=0.8\textwidth]{popup}}
	\vfill
	\vspace{4ex}
	\caption{ポップアップメニュー}\figlabel{popup}
\end{minipage}
\end{figure}

ポップアップメニューを出すには、様々な\ct{chooseFrom:}メソッドを使います。(\figref{popup})
\begin{code}{}
UIManager default
	chooseFrom: #('circle' 'oval' 'square' 'rectangle' 'triangle')
	lines: #(2 4) message: 'Choose a shape'
\end{code}

\dothis{\clsind{UIManager}クラスをブラウズして、提供されている対話メソッドを試してみましょう。}

%\begin{figure}[ht]
%	\centerline{\includegraphics[width=5cm]{dialog}}
%	\caption{Dialog displayed by \ct{FillInTheBlank request: 'What''s your name?' initialAnswer: 'no name'}.
%		\figlabel{dialogName}}
%\end{figure}

%To display a pop-up menu, use the \clsind{PopupMenu} class:
%\begin{code}{}
%UIManager default chooseFrom: #('circle' 'oval' 'square' 'rectangle' 'triangle') lines: #(2 4) message: 'Choose a shape'
%\end{code}

%\begin{figure}[ht]
%	\centerline{\includegraphics[width=3cm]{popup}}
%	\caption{PopUp displayed by \ct{PopUpMenu>>>startUpWithCaption:}.}
%\end{figure}

%=================================================================
\section{ドラッグ&ドロップ}

Morphicはドラッグ&ドロップもサポートします。受け入れ側のモーフとドロップされるモーフの二つのモーフからなる簡単な例について考えてみましょう。
受け入れ側のモーフは、ドロップされるモーフが条件を満たしたときにだけドロップを受け入れます。この例ではモーフが青色であることを条件とします。受け入れが拒否された場合、ドロップされるモーフが何をするか決めます。

\dothis{まず受け入れ側のモーフを定義しましょう。}
\begin{classdef}{他のモーフを受け入れるモーフを定義する。}
Morph subclass: #ReceiverMorph
	instanceVariableNames: ''
	classVariableNames: ''
	poolDictionaries: ''
	category: 'PBE-Morphic'
\end{classdef}

\dothis{いつものやり方で初期化メソッドを定義します。}
\begin{method}{\ct{ReceiverMorph}を初期化する。}
ReceiverMorph>>>initialize
	super initialize.
	color := Color red.
	bounds := 0 @ 0 extent: 200 @ 200
\end{method}

受け入れ側のモーフがドロップを受け入れるか否かは、どのように判断されるのでしょうか?
一般的に、モーフの両者がドロップに同意する必要があります。
受け入れ側は\mthind{Morph}{wantsDroppedMorph:event:}メッセージに反応することで決定します。最初の引数はドロップされるモーフ、2番目の引数はマウスイベントです。例えば、受け入れ側はドロップ時に何らかの修飾キーが押されたかどうかを調べられます。
ドロップされるモーフも、\ct{wantsToBeDroppedInto:}メッセージが送られたときに、ドロップされる先のモーフが気に入るか調べるチャンスがあります。このメソッドのデフォルトの実装(\ct{Morph}クラスで定義されている)では、\ct{true}を返します。

\begin{method}{色を基準にしてドロップを受け入れる。}
ReceiverMorph>>>wantsDroppedMorph: aMorph event: anEvent
	^ aMorph color = Color blue
\end{method}

受け入れ側のモーフがドロップを望まない場合には、ドロップされるモーフに何が起こるのでしょうか?デフォルトの動作は、何もしないことです。つまり、ドロップされずに受け入れ側のモーフの上に置かれるだけです。より直感的な動作としては、ドロップされるモーフが元の位置に戻ることでしょう。これは、ドロップされるモーフを拒否する際に、受け入れ側のモーフが\mthind{Morph}{repelsMorph:event:}メッセージに対して\ct{true}を返すことで実現できます。

\needlines{4}
\begin{method}{ドロップ拒否時のモーフの振る舞いを変える。}
ReceiverMorph>>>repelsMorph: aMorph event: ev
	^ (self wantsDroppedMorph: aMorph event: ev) not
\end{method}

以上が、受け入れ側のモーフで必要なことすべてです。

\dothis{ワークスペースで\clsind{ReceiverMorph}と\clsind{EllipseMorph}のインスタンスを作ります。}
\begin{code}{}
ReceiverMorph new openInWorld.
EllipseMorph new openInWorld.
\end{code}
\noindent
黄色い\ct{EllipseMorph}を受け入れ側のモーフにドラッグ&ドロップしましょう。モーフはドロップを拒否され、元の位置に戻るはずです。

\dothis{この動作を変えるには、インスペクタを使って楕円のモーフの色を\ct{Color blue}に変えます。青いモーフは\ct{ReceiverMorph}に受け入れられるはずです。}

では、\ct{DroppedMorph}と名付けた\ct{Morph}クラスの特別なサブクラスを作って、もう少し経験を積みましょう。

\begin{classdef}{\ct{ReceiverMorph}にドラッグ&ドロップ可能なモーフを定義する。}
Morph subclass: #DroppedMorph
	instanceVariableNames: ''
	classVariableNames: ''
	poolDictionaries: ''
	category: 'PBE-Morphic'
\end{classdef}

\needlines{5}
\begin{method}{\ct{DroppedMorph}を初期化する。}
DroppedMorph>>>initialize
	super initialize.
	color := Color blue.
	self position: 250@100
\end{method}

ドロップされるモーフが受け入れ側に拒否されたときにすべきことを設定できます。ここでは、モーフをマウスポインタに付いたままにします。
\begin{method}{ドロップを拒否されたモーフの反応を定義する。}
DroppedMorph>>>rejectDropMorphEvent: anEvent
	| h |
	h := anEvent hand.
	WorldState
		addDeferredUIMessage: [h grabMorph: self].
	anEvent wasHandled: true
\end{method}

イベントに対して\mthind{MorphicEvent}{hand}メッセージを送ると、\ct{HandMorph}のインスタンスである\emph{ハンド}が返されます。このオブジェクトはマウスポインタを表していて、マウスポインタに関係するものは何でも持っています。
ここでは、ドロップを拒否されたモーフである\ct{self}をハンドが掴むように\ct{World}に指示しています。

\dothis{\ct{DroppedMorph}の二つのインスタンスを作成し、受け入れ側のモーフにドラッグ&ドロップします。}
\begin{code}{}
ReceiverMorph new openInWorld.
(DroppedMorph new color: Color blue) openInWorld.
(DroppedMorph new color: Color green) openInWorld.
\end{code}
\noindent
緑色のモーフはドロップを拒否されるため、マウスポインタに付いたままになります。

%=================================================================
\section{例題}

サイコロを転がすモーフを作りましょう。\footnote{}モーフをクリックするとすべての面が高速に表示され、再度クリックするとアニメーションが止まります。

\begin{figure}[ht]
	\centerline{\includegraphics[scale=0.65]{die}}
	\caption{Morphicで作られたサイコロ
		\figlabel{dialogDie}}
\end{figure}

\dothis{外枠を描くため、\ct{Morph}クラスの代わりに\clsind{BorderedMorph}のサブクラスとしてサイコロを定義します。}

\needlines{6}
\begin{classdef}{サイコロモーフを定義する。}
BorderedMorph subclass: #DieMorph
	instanceVariableNames: 'faces dieValue isStopped'
	classVariableNames: ''
	poolDictionaries: ''
	category: 'PBE-Morphic'
\end{classdef}

インスタンス変数の\ct{faces}は、サイコロの面の個数を記録します。ここで実装するサイコロモーフは、9までの面を許すことにします!\ct{dieValue}は現在表示されている面の値を表し、\ct{isStopped}はサイコロのアニメーションが停止しているときに\ct{true}となります。
サイコロのインスタンスを作るために、\clsind{DieMorph}の\emph{クラス}側に\ct{faces: n}メソッドを定義し、\ct{n}個の面を持つ新しいサイコロを生成するようにします。
\begin{method}{好きな個数の面を持つ新しいサイコロを生成する。}
DieMorph class>>>faces: aNumber
	^ self new faces: aNumber
\end{method}

\ct{initialize}メソッドは今までと同じやり方でインスタンス側に定義します。なお、\ct{new}は新しく生成されたインスタンスに\ct{initialize}メッセージを送ることを覚えておいてください。
\begin{method}{\ct{DieMorph}のインスタンスを初期化する。}
DieMorph>>>initialize
	super initialize.
	self extent: 50 @ 50.
	self useGradientFill; borderWidth: 2; useRoundedCorners.
	self setBorderStyle: #complexRaised.
	self fillStyle direction: self extent.
	self color: Color green.
	dieValue := 1.
	faces := 6.
	isStopped := false
\end{method}

サイコロの見た目をよくするために、\ct{BorderedMorph}クラスのメソッドをいくつか使います。浮き出る効果を得るために枠線を太くしたり、角を丸めたり、表示面にグラデーションをかける等をします。
インスタンス側の\ct{faces:}メソッドでは、正しい引数が与えられているかをチェックするように定義します。
\begin{method}{サイコロの面の個数を設定する。}
DieMorph>>>faces: aNumber
	"Set the number of faces"
	(aNumber isInteger
			and: [aNumber > 0]
			and: [aNumber <= 9])
		ifTrue: [faces := aNumber]
\end{method}
\on{Why not make this a pre-condition, \ie an assertion?}

サイコロが生成されたときにメッセージが送られる順序を知るのは良いことです。
例えば、\ct{DieMorph faces: 9}を評価したと仮定します。
\begin{enumerate}
	\item クラスメソッドの\ct{DieMorph class>>>faces:}では、\ct{DieMorph class}クラスに\ct{new}メッセージを送ります。
	\item (\ct{Behavior}クラスから継承された\ct{DieMorph class}クラスの)\ct{new}メソッドで、新しいインスタンスが生成され、そのインスタンスに\ct{initialize}メッセージが送られます。
	\item \ct{DieMorph}の\ct{initialize}メソッドが、\ct{faces}インスタンス変数に初期値6を設定します。
	\item \ct{DieMorph class>>>new}メソッドの結果できたインスタンスが、クラスメソッドの\ct{DieMorph class>>>faces:}に返り、その新しいインスタンスに\ct{faces: 9}メッセージが送られます。 
	\item インスタンスメソッドの\ct{DieMorph>>>faces:}が実行され、\ct{faces}インスタンス変数を9に設定します。
\end{enumerate}

\ct{drawOn:}を定義する前に、表示面に描く目の位置を求めるメソッドをいくつか定義します。
\begin{methods}{サイコロの目の位置を求める九つのメソッドを定義する。}
DieMorph>>>face1
	^{0.5@0.5}
DieMorph>>>face2
	^{0.25@0.25 . 0.75@0.75}
DieMorph>>>face3
	^{0.25@0.25 . 0.75@0.75 . 0.5@0.5}
DieMorph>>>face4
	^{0.25@0.25 . 0.75@0.25 . 0.75@0.75 . 0.25@0.75}
DieMorph>>>face5
	^{0.25@0.25 . 0.75@0.25 . 0.75@0.75 . 0.25@0.75 . 0.5@0.5}
DieMorph>>>face6
	^{0.25@0.25 . 0.75@0.25 . 0.75@0.75 . 0.25@0.75 . 0.25@0.5 . 0.75@0.5}
DieMorph>>>face7
	^{0.25@0.25 . 0.75@0.25 . 0.75@0.75 . 0.25@0.75 . 0.25@0.5 . 0.75@0.5 . 0.5@0.5}
DieMorph >>>face8
	^{0.25@0.25 . 0.75@0.25 . 0.75@0.75 . 0.25@0.75 . 0.25@0.5 . 0.75@0.5 . 0.5@0.5 . 0.5@0.25}
DieMorph >>>face9
	^{0.25@0.25 . 0.75@0.25 . 0.75@0.75 . 0.25@0.75 . 0.25@0.5 . 0.75@0.5 . 0.5@0.5 . 0.5@0.25 . 0.5@0.75}
\end{methods}
\on{kind of ugly boilerplate code -- should be a nice way to map these more elegantly to coordinates.}

上記のメソッドは、サイコロの各面について目の座標をコレクションの形で定義しています。それぞれの座標は$1\times1$の範囲に収めているので、拡大して実際の目の場所を求めます。

\ct{drawOn:}メソッドは、サイコロの背景を描くための\ct{super}送信と、目を描くという二つの仕事を行います。
\begin{method}{サイコロモーフを描く。}
DieMorph>>>drawOn: aCanvas
	super drawOn: aCanvas.
	(self perform: ('face' , dieValue asString) asSymbol)
		do: [:aPoint | self drawDotOn: aCanvas at: aPoint]
\end{method}

このメソッドの後半は、\st のリフレクション機能を使っています。
サイコロの面に対応する\ct{faceX}メソッドから得た座標のコレクションに対して、順に\ct{drawDotOn:at:}メッセージを送ることでサイコロの目を描いています。適切な\ct{faceX}メソッドを呼び出すために、\mthind{Object}{perform:}メソッドを使います。このメソッドが、\lct{('face', dieValue asString) asSymbol}により作られた文字列をメッセージとして送ります。このような\ct{perform:}の使い方は至るところで見られます。
\index{reflection}
\begin{method}{サイコロの目を描く。}
DieMorph>>>drawDotOn: aCanvas at: aPoint
	aCanvas
		fillOval: (Rectangle
			center: self position + (self extent * aPoint)
			extent: self extent / 6)
		color: Color black
\end{method}
\ew{I would not use reflection to call faceX on p.234. A statically filled Dictionary would be more appropriate. Get the point set with "MyDict at: X".}

座標が$[0{:}1]$の範囲で正規化されているため、サイコロの大きさに合わせるために、\ct{self extent * aPoint}で拡大します。

\dothis{ワークスペースでサイコロのインスタンスを作ることができます。}
\begin{code}{}
(DieMorph faces: 6) openInWorld.
\end{code}

表示面を変えるために、\ct{myDie dieValue: 4}のように使えるアクセサを作ります。
\begin{method}{サイコロの表示面を設定する。}
DieMorph>>>dieValue: aNumber
	(aNumber isInteger
			and: [aNumber > 0]
			and: [aNumber <= faces])
		ifTrue:
			[dieValue := aNumber.
			self changed]
\end{method}

サイコロの面を高速に表示するためにアニメーション機能を使います。
\needlines{3}
\index{Morphic!animation}
\begin{methods}{サイコロのアニメーションを作る。}
DieMorph>>>stepTime
	^ 100

DieMorph>>>step
	isStopped ifFalse: [self dieValue: (1 to: faces) atRandom]
\end{methods}
サイコロが回ってますね!

サイコロの回転をクリックで始めたり止めたりするには、前述したマウスイベントのやり方を使います。
まず、マウスイベントを受け入れるようにします。

\begin{methods}{アニメーションの開始と停止のためにマウスクリックを処理する。}
DieMorph>>>handlesMouseDown: anEvent
	^ true

DieMorph>>>mouseDown: anEvent
	anEvent redButtonPressed
		ifTrue: [isStopped := isStopped not]
\end{methods}
以上により、クリックでサイコロを振ったり止めたりできます。

% That's all for the essentials of Morphic!

% Most of the work on \ct{DieMorph} was done with an instance of it living in the environment; this is quite nice when to tweak programs.

%=================================================================
\section{キャンバスに関する追加情報}

\ct{drawOn:}メソッドは、\clsindmain{Canvas}クラスのインスタンスを引数として一つだけ取ります。
このキャンバスとは、モーフ自身を描く領域のことです。
キャンバスの描画メソッドを使うことで、モーフに自由な外観を与えられます。
\ct{Canvas}クラスの階層構造をブラウズすれば、いくつもの変形を見ることができます。
通常用いられる\ct{Canvas}クラスの変形は\clsind{FormCanvas}クラスなので、\ct{Canvas}クラスと\ct{FormCanvas}クラスの中から主要な描画メソッドを見つけられます。
これらのメソッドには、点や線、多角形や四角形、楕円やテキスト、画像などの描画や、回転や拡大・縮小などの機能があります。

透明なモーフやその他の描画メソッド、アンチエイリアスなどを実現するために、他の種類のキャンバスも利用できます。
これらの特徴を使うためには、\clsind{AlphaBlendingCanvas}クラスや\clsind{BalloonCanvas}クラスが必要となるでしょう。
しかし、\ct{drawOn:}が引数として\ct{FormCanvas}のインスタンスを受け取るとしたら、\ct{drawOn:}の中でどうやって他のキャンバスを得ればよいのでしょうか?
幸いにも、あるキャンバスを別の種類に変換することが可能です。

\dothis{\ct{DieMorph}のアルファチャンネル透過率を0.5にするキャンバスを使うように、\ct{drawOn:}を再定義します。}
\needlines{7}
\begin{method}{透明なサイコロを描く。}
DieMorph>>>drawOn: aCanvas
	| theCanvas |
	theCanvas := aCanvas asAlphaBlendingCanvas: 0.5.
	super drawOn: theCanvas.
	(self perform: ('face' , dieValue asString) asSymbol)
		do: [:aPoint | self drawDotOn: theCanvas at: aPoint]
\end{method}
\noindent
以上で終わりです。

\begin{figure}[ht]
	\centerline{\includegraphics[scale=0.7]{multiMorphs}}
	\caption{アルファチャンネル透過率を設定したサイコロ。
		\figlabel{multiMorphs}}
\end{figure}

% ON: This appears to be broken:

%If you're curious, have a look at the \mthind{Canvas}{asAlphaBlendingCanvas:} method.
%You can also get antialiasing by using \clsind{BalloonCanvas} and transforming the die drawing methods as shown in \mthsref{aadie}.

%\needlines{6}
%\begin{methods}[aadie]{Drawing an antialiased die.}
%DieMorph>>>drawOn: aCanvas
%	| theCanvas |
%	theCanvas := aCanvas asBalloonCanvas aaLevel: 3.
%	super drawOn: aCanvas.
%	(self perform: ('face' , dieValue asString) asSymbol)
%		do: [:aPoint | self drawDotOn: theCanvas at: aPoint]

%DieMorph>>>drawDotOn: aCanvas at: aPoint
%	aCanvas
%		drawOval: (Rectangle
%			center: self position + (self extent * aPoint)
%			extent: self extent / 6)
%		color: Color black
%		borderWidth: 0
%		borderColor: Color transparent
%\end{methods}

%=================================================================
\section{まとめ}

Morphicは、グラフィカルなフレームワークで、動的にGUI要素を組み合わせることができます。

\begin{itemize}
  \item \ct{asMorph openInWorld}メッセージをオブジェクトに送ると、モーフに変換して画面に表示できます。
  \item モーフを\metaclick すると、ハンドルが表示されモーフを操ることができるようになります。(ハンドルの機能は個々のバルーンヘルプの説明を見ましょう)
  \item ドラッグ&ドロップによるか\ct{addMorph:}メッセージを送ることで、モーフを別のモーフに組み込めます。
  \item 既にあるクラスのサブクラスを作り、\ct{initialize}や\ct{drawOn:}といった主要なメソッドを再定義できます。
  \item \ct{handlesMouseDown:}メソッドや\ct{handlesMouseOver:}メソッド\etc を再定義すれば、マウスやキーボードに反応するようにモーフを制御できます。
  \item \ct{step}メソッド(何をするか)と\ct{stepTime}メソッド(ミリ秒単位のstep間隔)を定義すれば、モーフのアニメーションを作れます。
  \item \ct{PopUpMenu}や\ct{FillInTheBlank}といった多くの定義済みのモーフを使って、ユーザと対話できます。
\end{itemize}

%=================================================================
\ifx\wholebook\relax\else\end{document}\fi
%=================================================================

%-----------------------------------------------------------------
