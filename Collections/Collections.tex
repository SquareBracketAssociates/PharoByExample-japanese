% $Author$
% $Date$
% $Revision$

% HISTORY:
% 2006-12-07 - Andrew started
% 2007-03-30 - Cassou moved Streams to separate chapter
% 2007-05-28 - Stef adds material
% 2007-08-13 - Oscar edits
% 2009-07-07 - Oscar migrated to Pharo; fixed broken tests
% 2011-xx-xx - Naruhiko translated to Japanese

%=================================================================
\ifx\wholebook\relax\else
% --------------------------------------------
% Lulu:
	\documentclass[a4paper,10pt,twoside]{book}
	\usepackage[
		papersize={6.13in,9.21in},
		hmargin={.75in,.75in},
		vmargin={.75in,1in},
		ignoreheadfoot
	]{geometry}
	\input{../common.tex}
	\pagestyle{headings}
	\setboolean{lulu}{true}
% --------------------------------------------
% A4:
%	\documentclass[a4paper,11pt,twoside]{book}
%	\input{../common.tex}
%	\usepackage{a4wide}
% --------------------------------------------
    \graphicspath{{figures/} {../figures/}}
	\begin{document}
	% \renewcommand{\nnbb}[2]{} % Disable editorial comments
	\sloppy
\fi
%=================================================================
\chapter{コレクション}
\chalabel{collections}

% \ew{Stack is a popular construction. How does it fit in the collection hierarchy?}
% \ab{ The material here is based on a section that Andrew wrote for the \st Collections refactoring paper. It's not necessarily all appropriate for the book, but much of it is, I think, useful.}
%\on{recycled some material from \url{https://www.iam.unibe.ch/scg/svn_repos/Lectures/ST-H07/03StandardClasses.ppt}}
%\sd{Did a first pass: I should have a look at Xavier Briffault's book and at the material mentionned by andrew + lalonde book?}
%=============================================================
\section{導入}

コレクションクラス群は \clsindmain{Collection} クラスと \clsind{Stream} クラスのサブクラスで、汎用目的なもののグループであるとやや曖昧に定義しておきます。「\ind{ブルーブック}」~\cite{Gold83a} に登場するこのクラスのグループには \ct{Collection} の 17 個のサブクラスと  \ct{Stream} の 9 個のサブクラス、合計 28 個のクラスが含まれています。これらは \st-80 システムがリリースされる前に何度か再設計されたものでした。これらのクラス群はしばしばオブジェクト指向設計の系統的な例として考えられます。

\pharo では、抽象クラス \ct{Collection} は 101 個のサブクラスを、抽象クラス \ct{Stream} は 50 個のサブクラスを持ちますが、これらのうち多く (\mbox{\clsind{Bitmap}、}\mbox{\clsind{FileStream}、}\clsind{CompiledMethod} など) はシステムやアプリケーションの一部として使われるように作られた特別な目的のものなので、システム構成として「Collection」としてはカテゴリー化されません。この章の目的のために、「コレクション階層」という言葉は \ct{Collection} と、そのサブクラス\emph{の中で} \scat{Collections-*} としてカテゴライズされたものを指します。
同様に「ストリーム階層」は \ct{Stream} と、そのサブクラス\emph{の中で} \scat{Collections-Streams} としてカテゴライズされたものを指します。
これら 56 個のクラスは 982 のメッセージに応答でき、合計 1609 個のメソッドを定義しているのです!

% See test case in PBE-Collections for statistics

%c := (Collection allSubclasses select: [:each | each category beginsWith: 'Collections']).
%s := (Stream allSubclasses select: [:each | each category beginsWith: 'Collections-Streams']).
%m := (c,s) gather: #methods.
%Transcript
%	show: 'Collection subclasses: ', Collection allSubclasses size printString; cr;
%	show: 'Stream subclasses: ', Stream allSubclasses size printString; cr;
%	show: 'Collection package subclasses: ', c size printString; cr;
%	show: 'Collection package subclasses: ', c size printString; cr;
%	show: 'Stream package subclasses: ', s size printString; cr;
%	show: 'Total messages: ', m size printString; cr;
%	show: 'Total messages: ', (m collect: #selector) asSet size printString; cr

%\begin{figure}
%\begin{center}
%\ifluluelse
%	{\fbox{\includegraphics[height=0.9\textheight]{CollectionHierarchyList}}}
%	{\fbox{\includegraphics[width=3in]{CollectionHierarchyList}}}
%\caption{The collection classes in \pharo. Indentation indicates subclassing.
%\textit{\textsf{Italicized}} classes are abstract.
%{\textbf{Bold}} classes are described in the ``Blue Book".}
%\figlabel{CollClassesList}
%\end{center}
%\end{figure}

この章ではコレクションクラスのうち \figref{CollClassesTree} に示す一部を主に取り上げます。
ストリームについては \charef{streams} で分けて論じます。

% Note that the stream classes suffer from aging problems since they contain duplicated code and other code smells.
% ON: We should avoid statements like this!

\begin{figure}
\begin{center}
\ifluluelse
	{\includegraphics[width=\textwidth]{CollectionHierarchy}}
	{\includegraphics[width=0.8\textwidth]{CollectionHierarchy}}
\caption{\pharo において主要なコレクション}
\figlabel{CollClassesTree}
\end{center}
\end{figure}

%=========================================================
\section{コレクションの種類}
\seclabel{varieties}

コレクションクラス群を上手く使うには、幅広い種類のコレクションについて、その実装と、それぞれの共通点と違いについて、少なくとも表層的には理解して置かなければなりません。

コレクションを用いたプログラミングは、個別の要素を扱うより、プログラムの抽象度を高める上でとても役立ちます。
\ind{Lisp} の \ct{map} 関数は、あるリストの各要素に対して引数に渡した関数を適用した結果のリストを作って返すものです。これはコレクションのようなスタイルの初期の例ですが、\st-80 はコレクション・ベースのプログラミングを中心に据えたのです。\ind{ML} や \ind{Haskell} のような現代的な関数型プログラミング言語は Smalltalk の後に続いたと言えます。

このアイディアはなぜよいのでしょう?
例えば学生のレコードのコレクションといったデータ構造があったとして、なにかの基準に一致するすべての学生のデータについて同じ処理をしたいと考えたとします。命令型言語に慣れたプログラマーなら、ループを使うことをすぐ思いつくでしょう。でも Smalltalk プログラマーならこう書くでしょう:
\begin{code}{}
students select: [ :each | each gpa < threshold ]
\end{code}
\noindent
これは角カッコで囲まれた式が \ct{true} を返す students だけを要素とする新しいコレクションという意味です\footnote{角カッコ内の式は無名関数 $\lambda x. x~{\sf gpa} < {\sf threshold}$ を定義する $\lambda$式だと考えることができます。}。この \st コードにはドメイン固有問い合わせ言語的なシンプルさとエレガンスがあります。

\st の\emph{すべての}コレクションはメッセージ \ct{select:} を理解することができます。学生のデータ構造が配列なのか線形リストなのかを知る必要はありません。どちらも \ct{select:} メッセージを理解できます。ループを使うやり方とは大きく異なるということに注意しましょう。ループでは \ct{students} が配列なのか線形リストなのか、あらかじめ知っておかなければなりません。

\st では、特定のコレクションを指定せずに単に「コレクション」といった場合には、正しく定義されたプロトコル、すなわちある要素が含まれているかをテストするものと、要素の数え上げを行うもの、をサポートしたオブジェクトを指します。\emph{すべての}コレクションは \protind{testing} メッセージである \mbox{\mthind{Collection}{includes:}、}\mbox{\mthind{Collection}{isEmpty}、}\mthind{Collection}{occurrencesOf:}を理解できます。\emph{すべての}コレクションは \prot{enumeration} メッセージである \mthind{Collection}{do:}、\mthind{Collection}{select:}、\mthind{Collection}{reject:} (\ct{select:} の反対の意)、\mthind{Collection}{collect:} (Lisp の \ct{map})、\mthind{Collection}{detect:ifNone:}、\mthind{Collection}{inject:into:} (左畳込みを行う) などなどを理解できます。emumeration プロトコルは普遍性と、十分なバラエティを備えているのでとても強力です。

\figref{protocols} はコレクション階層におけるクラスが標準でサポートしているプロトコルをまとめたものです。それぞれのメソッドは \ct{Collection} のサブクラスで定義または再定義され、最適化され、まれには削除されることもあります。

\begin{figure*}
\begin{center}
\ifluluelse
	{\begin{tabular}{|l|p{8cm}|}}
	{\begin{tabular}{|l|p{12cm}|}}
\hline
{\bf プロトコル} & {\bf メソッド}\\
\hline
\protind{accessing}	&	\lct{size}, \lct{capacity}, \lct{at: \emph{anIndex}}, \lct{at: \emph{anIndex} put: \emph{anElement}}	\\
\hline
\protind{testing}	&	\lct{isEmpty}, \lct{includes: \emph{anElement}}, \lct{contains: \emph{aBlock}}, \\
& \lct{occurrencesOf: \emph{anElement}}	\\
\hline
\protind{adding}	&	\lct{add: \emph{anElement}}, \lct{addAll: \emph{aCollection}}	\\
\hline
\protind{removing}		&	\lct{remove: \emph{anElement}}, \lct{remove: \emph{anElement} ifAbsent: \emph{aBlock}}, \lct{removeAll: \emph{aCollection}}	\\
\hline
\protind{enumerating}	&	\lct{do: \emph{aBlock}}, \lct{collect: \emph{aBlock}}, \lct{select: \emph{aBlock}}, \lct{ reject: \emph{aBlock}}, \lct{detect: \emph{aBlock}}, \lct{detect: \emph{aBlock} ifNone: \emph{aNoneBlock}}, \\
& \lct{inject: \emph{aValue} into: \emph{aBinaryBlock}}	\\
\hline
\protind{converting}	&	\lct{asBag}, \lct{asSet}, \lct{asOrderedCollection}, \lct{asSortedCollection}, \\
& \lct{asArray}, \lct{asSortedCollection: \emph{aBlock}}	\\
\hline
\protind{creation}		&	\lct{with: \emph{anElement}}, \lct{with:with:}, \lct{with:with:with:}, \\
& \lct{with:with:with:with:}, \lct{withAll: \emph{aCollection}}	\\
\hline
\end{tabular}
\caption{標準的なコレクションプロトコル\figlabel{protocols}}
\end{center}
\end{figure*}

%\begin{figure*}
%\begin{center}
%\includegraphics[width=\textwidth]{CollectionsBySeq}
%\caption{Collections can be categorized according to whether or not they are sequenceable, \ie whether there are clearly defined first and last elements. All of the sequenceable collections except linked lists can also be indexed by an integer key. Of the non-sequenceable collections, dictionaries can be accessed by an arbitrary key, such as a string, while sets and bags cannot.\figlabel{CollsBySeq}}%
%\end{center}
%\end{figure*}
%\on{A useless diagram -- refer to the class hierarchy instead!}

基本的な一貫性の裏で、違うプロトコルをサポートしたり、同じリクエスト対して異なる振る舞いをするさまざまなコレクションが存在しています。そのような違いの中で重要なものを見ていきましょう。

\begin{itemize}
  \item {\bf 順序性 (Sequenceable):}
  \clsind{SequenceableCollection} のすべてのサブクラスのインスタンスは \mthind{SequenceableCollection}{first} 要素から始まり、\mthind{SequenceableCollection}{last} 要素まで明確に決まった順序で並んでいます (sequenceable)。
  一方で、\mbox{\clsind{Set}、}\mbox{\clsind{Bag}、}\clsind{Dictionary} のインスタンスは順序性を持ちません。

  \item {\bf 整列性 (Sortable):}
  \clsind{SortedCollection} はその要素を常に整列された状態に保ちます。

  \item {\bf インデックス参照性 (Indexable):}
        ほとんどの順序性のあるコレクションはインデックスで参照できます (indexable)。つまり \ct{at:} で要素を取り出せるということです。
        \clsind{Array} はインデックス参照できるデータ構造のなかでもとても身近なもので、固定されたサイズを持ちます。\ct{anArray at: n} は \ct{anArray} の $\mbox{\ct{n}}^{th}$ 要素を取り出し、\ct{anArray at: n put: v} は $\mbox{\ct{n}}^{th}$ 要素の値を \ct{v} に変えます。
        \ct{LinkedList} 類と \ct{SkipList} 類は順序性はありますがインデックスでは参照できません。つまり \ct{first} と \ct{last} は理解できますが、\ct{at:} は理解できません。
	\clsindex{LinkedList}
	\clsindex{SkipList}

  \item {\bf キー参照性 (Keyed):}
	\clsind{Dictionary} とそのサブクラスのインスタンスは、インデックスの代わりにキーで要素を参照できます。

  \item {\bf 可変性 (Mutable):}
        ほとんどのコレクションは内容が可変 (mutable) ですが、\ct{Interval} 群と \ct{Symbol} 群は例外です。
        \clsind{Interval} クラスは Integer の範囲を示す内容が不変 (immutable) のコレクションです。例えば、\ct{5 to: 16 by: 2} は要素 5, 7, 9, 11,13, 15 を含むインターバルです。インターバルは \ct{at:} でインデックス参照できますが、\ct{at:put:} で値を変更することはできません。
	\clsindex{Symbol}

  \item {\bf 自動拡大性 (Growable):}
         \ct{Interval} や \ct{Array} のインスタンスは常にサイズが固定されています。他の種類のコレクション (整列コレクション、順序コレクション、線形リスト) は生成後にサイズが増えることがありえます。

         クラス \clsind{OrderedCollection} は \ct{Array} より汎用的です。\ct{OrderedCollection} は必要に応じてサイズを拡大し、\mthind{OrderedCollection}{at:} や \mthind{OrderedCollection}{at:put:} に加え、\mthind{OrderedCollection}{addFirst:} や \mthind{OrderedCollection}{addLast:} といったメソッドもあります。
  
  \item {\bf 重複許容性 (Accepts duplicates):}
  	\clsind{Set} は要素が重複しないようにしますが、\clsind{Bag} はそういったことをしません。
	\clsind{Dictionary}、\ct{Set}、\ct{Bag} は各要素の \ct{=} メソッドを用います。\ct{Identity} バリアントは二つのオブジェクトが同一であることを判定する the \ct{==} メソッドを用い、\ct{Pluggable} バリアントはコレクション作成時に与えられた任意の同値関係を用います。
	\index{Pluggable collections}

  \item {\bf 異種性 (Heterogeneous):}
        ほとんどのコレクションはどんな種類の要素も持ち得ます。
        一方で \clsind{String}、\clsind{CharacterArray}、\clsind{Symbol} は \ct{Character} だけを持ちます。
        \clsind{Array} はさまざまなオブジェクトを混在して保持できますが、\lct{ByteArray} は \lct{Byte} だけを、\clsind{IntegerArray} は \ct{Integer} だけを、\clsind{FloatArray} は \ct{Float} だけを持ち得ます。
	\clsind{LinkedList} の要素になるには、必ず \prot{Link \go accessing} プロトコルを受け付けなければならないという制約があります。
\end{itemize}

%  How is the order established? Sorted collections use a supplied total ordering function, intervals are implicitly ordered, while arrays and ordered collections are ordered explicitly when elements are inserted.

%=========================================================
\section{コレクションの実装}
\seclabel{implementation}

\begin{figure*}
\begin{center}
\includegraphics[width=\textwidth]{CollectionsByImpl}
\caption{実装技法によりカテゴリ分けしたコレクション
    \figlabel{collsByImpl}}
% some of collections だから「一部の」とすべきなのかもしれないが、ここは敢えて訳を落とす。
\end{center}
\end{figure*}

機能によるカテゴリーだけ考えれば良いというわけではありません。どうやってコレクションクラスが実装されているかも考えなければなりません。\figref{collsByImpl} に示すように、主要な実装技法としては五つのものがあります。

\begin{enumerate}
  \item Array は要素をコレクションオブジェクト自身の(インデックス参照できる)インスタンス変数に格納します。結果として、配列はサイズ固定となり、その代わりに一回のメモリ割り当てで生成できます。
  \clsindex{Array}
  \item \clsind{OrderedCollections} と \clsind{SortedCollections} は要素を配列に格納し、インスタンス変数でその配列を参照します。こうすることで、この内部的な配列のサイズよりもコレクションが大きくなったら、より大きい物と入れ替えることができます。
  \item set や dictionary の類もまた、記憶先として配列へのリファレンスを利用しますが、配列をハッシュテーブルとして用います。Bag は記憶先として、要素そのものをキー、出現回数を値として持つ \ct{Dictionaly} を使います。
  \clsindex{Bag}
  \item LinkedList は一般的な片方向リンクの形式を取ります。
  \clsindex{LinkedList}
  \item Interval は開始点・終了点・ステップサイズの三つの整数を格納します。
  \clsindex{Interval}
\end{enumerate}
これらのクラスに加え、\ct{Array}、\ct{Set}、それからディクショナリーの多くの種類は、``\subind{Collection}{weak}'' という変種を持ちます。これらの変種では要素を弱く保持します。\ie 要素のガーベージコレクションを禁止しないということです。
% 自信ないので原文残す。
% In addition to these classes, there are also ``\subind{Collection}{weak}'' variants of \ct{Array}, \ct{Set} and of the various kinds of dictionary.  These collections hold onto their elements weakly, \ie in a way that does not prevent the elements from being garbage collected.
\pharo の仮想マシンはこれらのクラスを判別し、特別に扱います。
\index{weak collections}

\st のコレクションについてより詳しく知りたい場合は LaLonde と Pugh のすばらしい本\cite{LaLo90a} を参照してください。

%=========================================================
\section{重要なクラスについての例}
よく使ったり重要だったりするコレクションクラスについて、簡単なコード例でご説明しましょう。
紹介するコレクションの主なプロトコルは以下のとおりです:\mthind{Collection}{at:}、\mthind{Collection}{at:put:} --- 要素へのアクセス、\mthind{Collection}{add:}、\mthind{Collection}{remove:} --- 要素の追加または削除、\mthind{Collection}{size}、\mthind{Collection}{isEmpty}、\mthind{Collection}{include:} --- コレクションについての情報の取得、\mthind{Collection}{do:}、\mthind{Collection}{collect:}、\mthind{Collection}{select:} --- コレクション全体への繰り返し。
それぞれのコレクションによってこれらのプロトコルはサポートされていたりしなかったりしますし、サポートされているときは、そのコレクションの意味に沿うように解釈します。それぞれのクラスをブラウズしてみて、特殊だったりより上級向けだったりするプロトコルを探してみるとよいでしょう。
\seeindex{do:@\ct{do:}}{Collection, \ct{do:}}
\seeindex{at:@\ct{at:}}{Collection, \ct{at:}}
\seeindex{at:put:@\ct{at:put:}}{Collection, \ct{at:put:}}

以下では次の最も一般的なクラスに的を絞ります。\clsind{OrderedCollection}、\clsind{Set}、\clsind{SortedCollection}、\clsind{Dictionary}、\clsind{Interval}、\clsind{Array}。

\paragraph{一般的な creation プロトコル}
コレクションのインスタンスを作る方法はいくつかあります。\mthind{Collection Class}{new:} メソッドと \mthind{Collection class}{with:} メソッドを使うのが一番汎用的なやり方でしょう。\ct{new: anInteger} はサイズが \ct{anInteger} ですべての要素が \ct{nil} であるコレクションを作ります。\mthind{Collection class}{with:} \ct{anObject} はコレクションを作り、それに \ct{anObject} を追加します。実際の振る舞いはコレクションが異なれば、それぞれ異なって見えるでしょう。

初期要素を持つコレクションを作るには、\mthind{Collection class}{with:}、\mthind{Collection class}{with:with:} などなど最高6個の要素まで用意されているメソッドを使うことができます。

\begin{code}{@TEST}
Array with: 1 --> #(1)
Array with: 1 with: 2 --> #(1 2)
Array with: 1 with: 2 with: 3 --> #(1 2 3)
Array with: 1 with: 2 with: 3 with: 4 --> #(1 2 3 4)
Array with: 1 with: 2 with: 3 with: 4 with: 5 --> #(1 2 3 4 5)
Array with: 1 with: 2 with: 3 with: 4 with: 5 with: 6 --> #(1 2 3 4 5 6)
\end{code}

あるコレクションにあるすべての要素を別の種類のコレクションに加えるには \ct{addAll:} が使えます。

\begin{code}{@TEST}
(1 to: 5) asOrderedCollection addAll: '678'; yourself --> an OrderedCollection(1 2 3 4 5 $6 $7 $8)
\end{code}
\noindent
\mthind{Collection}{addAll:} は引数を返し、レシーバーを返すわけではないのに注意してください!

\mthind{Collection class}{withAll:} や \mthind{Collection class}{newFrom:} で色々なコレクションを作ることができます。

\begin{code}{@TEST}
Array withAll: #(7 3 1 3)                      --> #(7 3 1 3)
OrderedCollection withAll: #(7 3 1 3) --> an OrderedCollection(7 3 1 3)
SortedCollection withAll: #(7 3 1 3)    --> a SortedCollection(1 3 3 7)
Set withAll: #(7 3 1 3)                         --> a Set(7 1 3)
Bag withAll: #(7 3 1 3)                        --> a Bag(7 1 3 3)
Dictionary withAll: #(7 3 1 3)               --> a Dictionary(1->7 2->3 3->1 4->3 )
\end{code}

\needlines{3}
\begin{code}{@TEST}
Array newFrom: #(7 3 1 3)                                          --> #(7 3 1 3)
OrderedCollection newFrom: #(7 3 1 3)                     --> an OrderedCollection(7 3 1 3)
SortedCollection newFrom: #(7 3 1 3)                       --> a SortedCollection(1 3 3 7)
Set newFrom: #(7 3 1 3)                                            --> a Set(7 1 3)
Bag newFrom: #(7 3 1 3)                                           --> a Bag(7 1 3 3)
Dictionary newFrom: {1 -> 7. 2 -> 3. 3 -> 1. 4 -> 3} --> a Dictionary(1->7 2->3 3->1 4->3 )
\end{code}
\noindent
この二つのメソッドは同じではないことにご注意ください。
具体的には、\cmind{Dictionary class}{withAll:} では与えられたコレクションを値と解釈しますが、\cmind{Dictionary class}{newFrom:} はアソシエーションのコレクションが与えられることを期待しています。

%---------------------------------------------------------
\subsection{Array}
\clsindmain{Array} (配列) はサイズが固定で、整数のインデックスで要素にアクセスできるコレクションです。
C 流とは異なり、\st の配列の先頭要素は位置 1 であり、0 ではありません。
配列の要素にアクセスする主なプロトコルは \mthind{Array}{at:} と \mthind{Array}{at:put:} です。\ct{at: anInteger} はインデックス \ct{anInteger} にある要素を返します。\ct{at: anInteger put: anObject} は \ct{anObject} をインデックス \ct{anInteger} に入れます。配列はサイズ固定のコレクションなので、配列の末尾の要素の後に要素を追加したり、末尾の要素を削除したりはできません。以下のコードはサイズ 5 の配列をつくり、最初の三つの要素にそれぞれ値を入れ、先頭要素の値を返します。

\begin{code}{@TEST | anArray | }
anArray := Array new: 5.
anArray at: 1 put: 4.
anArray at: 2 put: 3/2.
anArray at: 3 put: 'ssss'.
anArray at: 1 --> 4
\end{code}

クラス \clsind{Array} のインスタンスを作るためにはいくつか方法があります。\ct{new:}、\ct{with:}、それから \ct{#( )} と \ct|{ }| を使うこともできます。
% and  the constructs \ct{#( )} and \ct|{ }| の the constructs の定訳を探す必要あり。

\paragraph{\mthind{Array class}{new:} による生成} \ct{new: anInteger} はサイズ \ct{anInteger} の配列を生成します。
\ct{Array new: 5} はサイズ 5 の配列を生成します。

\paragraph{\mthind{Array class}{with:} による生成} \ct{with:} メソッドは要素の値を引数として一つ受け取ります。
次のコードでは値 \ct{4}、分数 \ct{3/2}、文字列 \ct{'lulu'} の三つの要素を持つ配列を生成します。

\begin{code}{@TEST | anArray |}
Array with: 4 with: 3/2 with: 'lulu' -->  {4. (3/2). 'lulu'}
\end{code}

\paragraph{\ct|\#()| を用いたリテラル生成}
\ct{#()} は静的な (または ``\subind{Array}{リテラル}'') 要素を持つ\ind{静的配列}を生成します。静的というのは式がコンパイルされたときに値が分かっているもので、実行時に評価されるものではないということです。以下のコードは最初の要素が (リテラル) 数値 \ct{1} で、次の要素が (リテラル) 文字列である、サイズ2の配列を生成します。

\seeindex{\#@{\textsf{\#( )}}}{Array, literal}
\seeindex{\{@{\textsf{\{ \}}}}{Array, dynamic}

\begin{code}{@TEST | anArray |}
#(1 'here') size --> 2
\end{code}

\ct{#(1+2)} を評価した場合、要素 \ct{3} 一つだけを持つ配列ではなく、\ct{#(1 #+ 2)} \ie 三つの要素、\ct{1}、シンボル \ct{#+}、数 \ct{2} を持つ配列になります。

\begin{code}{@TEST}
#(1+2) -->  #(1 #+ 2)
\end{code}

\noindent
これは \ct{#()} によって、配列に格納される式を定数としてコンパイラが解釈するからです。\ct{#()} 式はスキャンされて結果として得られた要素がそれぞれ生成される配列に格納されます。静的配列は数値、\ct{nil}、\ct{true}、\ct{false}、シンボル、文字列を格納できます。

\paragraph{\ct|\{ \}| を用いた動的生成}
\ct|{}| を使うことで動的配列を作ることができます。\ct|{ a , b }| は \lct{Array with: a with: b} と等価です。つまるところ、\ct|{| と \ct|}| で囲まれた式は評価時に実行されるということです。

\begin{code}{@TEST | anArray |}
{ 1 + 2 } --> #(3)
{(1/2) asFloat} at: 1 --> 0.5
{10 atRandom . 1/3} at: 2 --> (1/3)
\end{code}

\paragraph{要素へのアクセス}
順序性を持つすべてのコレクションは \mthind{Array}{at:} および \mthind{Array}{at:put:} で要素にアクセスできます。

\begin{code}{@TEST | anArray |}
anArray := #(1 2 3 4 5 6) copy.
anArray at: 3 --> 3
anArray at: 3 put: 33.
anArray at: 3 --> 33
\end{code}
\cmindex{Array}{copy}

\noindent
静的配列を変更するときには十分注意してください!
コンパイラは静的配列については、格納する領域を最初の一回だけ確保します。
配列をコピーしていないので、上の例を二回目に評価したとき、``静的'' 配列は期待通りの結果にならないでしょう。
(コピーを取らなければ、二回目の実行時には、静的配列 \ct{#(1 2 3 4 5 6)} は実際のところ\ct{#(1 2 33 4 5 6)} になっているのです!)
動的配列にはこのような問題はありません。

%---------------------------------------------------------
\subsection{OrderedCollection}
\clsindmain{OrderedCollection} は自動でサイズを拡張し、順序を保持して要素を追加できるコレクションの一種です。\mthind{OrderedCollection}{add:}、\mthind{OrderedCollection}{addFirst:}、\mthind{OrderedCollection}{addLast:}、\mthind{OrderedCollection}{addAll:} などの多様なメソッドを提供しています。

\begin{code}{@TEST | ordCol |}
ordCol := OrderedCollection new.
ordCol add: 'Seaside'; add: 'SqueakSource'; addFirst: 'Monticello'.
ordCol --> an OrderedCollection('Monticello' 'Seaside' 'SqueakSource')
\end{code}

\paragraph{要素の削除} メソッド \mthind{OrderedCollection}{remove:} \ct{anObject} は \ct{anObject} と一致する最初のオブジェクトをコレクションから削除jするものです。もしそのオブジェクトが存在しない場合は、エラーを起こします。

\begin{code}{@TEST | ordCol | ordCol := OrderedCollection with: 'Monticello' with: 'Seaside' with: 'SqueakSource'.}
ordCol add: 'Monticello'.
ordCol remove: 'Monticello'.
ordCol --> an OrderedCollection('Seaside' 'SqueakSource' 'Monticello')
\end{code}

\ct{remove:} の変種として \mthind{OrderedCollection}{remove:ifAbsent:} という名のものがあります。これは削除しようとした要素がコレクションに存在しないときの動作を第二引数としてブロックで渡すことができるものです。

\begin{code}{@TEST | ordCol res | ordCol := OrderedCollection with: 'Seaside' with: 'SqueakSource' with: 'Monticello'.}
res := ordCol remove: 'zork' ifAbsent: [33].
res --> 33
\end{code}

\paragraph{変換}
\ct{Array} から (またはその他どんなコレクションからも)、メッセージ \mthind{Collection}{asOrderedCollection}: を送ることで \ct{OrderedCollection} へ変換することができます。

\begin{code}{@TEST}
#(1 2 3) asOrderedCollection --> an OrderedCollection(1 2 3)
'hello' asOrderedCollection --> an OrderedCollection($h $e $l $l $o)
\end{code}

%\paragraph{A word about automatic growth.}
%Note that while \ct{OrderedCollection} grows automatically when you add new elements it can be better to create an \ct{OrderedCollection} with a good approximation of the numbers of elements it will contain. The following code creates an ordered collection whose size is equal to the number of classes in \pharo and adds all the classes to it.

% \on{I don't like this example. No one should ever write code like this. Furthermore it is not good advice, as you almost never need to do this.}

%\begin{code}{@TEST | classes res | }
%classes := Smalltalk allClasses.
%res := OrderedCollection new: classes size.
%classes do: [:each | res add: each].
%classes size = res size --> true
%\end{code}

%\noindent
%This can be faster than creating the ordered collection with \ct{new} and letting it grow (see \charef{profiling}).

%---------------------------------------------------------
\subsection{Interval}
\clsindmain{Interval} クラスは値の範囲を表現します。例えば 1 から 100 までの数の範囲は以下のように表現できます:
\cmindex{Interval class}{from:to:}
\begin{code}{@TEST}
Interval from: 1 to: 100 --> (1 to: 100)
\end{code}

\noindent
このインターバルの \mthind{Interval class}{printString} した結果により、\mthind{Number}{to:} というインターバルを生成する便利なメソッドがあることが分かります。

\begin{code}{}
(Interval from: 1 to: 100) = (1 to: 100) --> true
\end{code}

\cmind{Interval class}{from:to:by:} または \cmind{Number}{to:by:} によって、以下のように二つの値の間にある刻みを持った範囲を表現できます:

\begin{code}{@TEST}
(Interval from: 1 to: 100 by: 0.5) size --> 199
(1 to: 100 by: 0.5) at: 198 --> 99.5
(1/2 to: 54/7 by: 1/3) last --> (15/2)
\end{code}

%---------------------------------------------------------
\subsection{Dictionary}
ディクショナリーは要素をキーによってアクセスできる重要なコレクションです。
ディクショナリーのメッセージの中でよく使うものとしては \mthind{Dictionary}{at:}、\mthind{Dictionary}{at:put:}、\mthind{Dictionary}{at:ifAbsent:}、\mthind{Dictionary}{keys}、\mthind{Dictionary}{values} があります。
\seeindex{keys}{Dictionary, keys}
\seeindex{values}{Dictionary, values}

\begin{code}{@TEST | colors |}
colors := Dictionary new.
colors at: #yellow put: Color yellow.
colors at: #blue put: Color blue.
colors at: #red put: Color red.
colors at: #yellow --> Color yellow
colors keys          --> a Set(#blue #yellow #red)
colors values       --> {Color blue. Color yellow. Color red}
\end{code}

ディクショナリーはキーを同値性によって比較します。二つのキーを \ct{=} で比較したときに真であるならば、それは等しいとされます。よくあるけれど見つけにくいバグとしては、キーとして使うオブジェクトで \ct{=} を再定義しているにも関わらず、\ct{hash} メソッドを再定義していないというものです。これら二つのメソッドはどちらもディクショナリーの中でオブジェクトの比較に用いられます。
\index{Dictionary!overriding \ct{=} and \ct{hash}}

クラス \clsindmain{Dictionary} のクラス階層はサブクラスの考え方に基づいており、サブタイプに基づいているわけではないということが明確に示されています。\ct{Dictionary} は \ct{Set} のサブクラスではありますが、普通は \ct{Dictionary} を \ct{Set} のように使おうとは思わないでしょう。しかし実装においては、\ct{Dictionary} はメッセージ \mthind{Object}{->} により生成される関連 (キー - 値) の集合として表現できるのです。ということで \ct{Dictionary} は関連のコレクションから作ることもできますし、ディクショナリーを関連の配列に変換することもできます。
\seeindex{association}{Object, \ct{->}}

\needlines{5}
\begin{code}{@TEST | colors |}
colors := Dictionary newFrom: { #blue->Color blue . #red->Color red . #yellow->Color yellow }.
colors removeKey: #blue.
colors associations --> {#yellow->Color yellow. #red->Color red}
\end{code}

\paragraph{IdentityDictionary}
ディクショナリーが二つのキーが同じであるかを判定するのにメッセージ \ct{=} および \ct{hash} を用いるのに対し、クラス \clsindmain{IdentityDictionary} はキーの値ではなく、その同一性 (メッセージ \mthind{ProtoObject}{==} で判定する) を用います。\ie このクラスは二つのキーが同じオブジェクトであるとき\emph{だけ}一致するとします。

しばしばキーとしては \ct{Symbol} が用いられますが、\ct{IdentityDictionary} を使うならごく自然なことです。なぜなら \clsind{Symbol} はグローバルに一意であることが保証されているからです。
% globally unique = グローバルに一意、と訳。
言い換えれば、もしキーとして文字列を使うなら \ct{Dictionary} クラスを使うべきです。でないと問題が生じるでしょう:

\begin{code}{@TEST | a b trouble |}
a := 'foobar'.
b := a copy.
trouble := IdentityDictionary new.
trouble at: a put: 'a'; at: b put: 'b'.
trouble at: a          --> 'a'
trouble at: b          --> 'b'
trouble at: 'foobar' --> 'a'
\end{code}

\noindent
\ct{a} と \ct{b} はそれぞれ別のオブジェクトなので、異なるオブジェクトとして扱われます。
興味深いことですが、リテラル \mbox{\ct{'foobar'}} は一度しか確保されないので、\ct{a} と全く同じオブジェクトになります。
この振る舞いに依存したコードを書こうなんて思わないでくださいね!
ただの \ct{Dictionary} は \ct{'foobar'} と値が等しいキーであれば同じ値を返します。

\ct{IdentityDictionary} ではキーとしてグローバルに一意であるオブジェクト (\ct{Symbol} や \ct{SmallInteger} など) を使ってください。\ct{String} (やその他のオブジェクト) をキーとして使うならただの \ct{Dictionary} を使いましょう。

特筆すべきはグローバル変数 \glbind{Smalltalk} は \clsind{SystemDictionary} のインスタンスであり、これは \ct{IdentityDictionary} のサブクラスですので、すべてのキーは \ct {Symbol} (実際は 8bit 文字しか格納できない \ct{ByteSymbol}) だということです。

\begin{code}{@TEST}
Smalltalk keys collect: [ :each | each class ] --> a Set(ByteSymbol)
\end{code}
\noindent
\ct{keys} または \ct{values} を \ct{Dictionary} に送ると結果は \ct{Set} となりますが、このクラスについては次に説明します。
% Since every key has the same class, the set of classes of keys contains only a single element, \ct{ByteSymbol}.

%---------------------------------------------------------
\subsection{Set}
クラス \clsindmain{Set} (集合) は数学的な「集合」の意味を持つコレクションです。\ie 重複した要素を持たず、要素間に順序もありません。\ct{Set} の要素を加えるにはメッセージ \mthind{Set}{add:} を用い、また \ct{at:} によってアクセスすることはできません。Set に格納されるオブジェクトはメソッド \ct{hash} と \ct{=} が実装されていなければなりません。

\begin{code}{@TEST | s | }
s := Set new.
s add: 4/2; add: 4; add:2.
s size --> 2
\end{code}

Set を作るには \cmind{Set class}{newFrom:} または変換メッセージ \cmind{Collection}{asSet} を用いることができます:

\begin{code}{@TEST}
(Set newFrom: #( 1 2 3 1 4 )) = #(1 2 3 4 3 2 1) asSet --> true
\end{code}

\mthind{Collection}{asSet} offers us a convenient way to eliminate duplicates from a collection:
\begin{code}{@TEST}
{ Color black. Color white. (Color red + Color blue + Color green) } asSet size --> 2
\end{code}
\noindent
red + blue + green = white になることに注意してください。

\clsindmain{Bag} は \ct{Set} ととても似ていますが、重複を許すという違いがあります。
\begin{code}{@TEST}
{ Color black. Color white. (Color red + Color blue + Color green) } asBag size --> 3
\end{code}

集合演算である \emph{union (集合和)}、\emph{intersection (集合積)}、\emph{要素テスト} は \ct{Collection} メッセージの \mthind{Collection}{union:}、\mthind{Collection}{intersection:}、\mthind{Collection}{includes:} として実装されています。
これらの操作ではレシーバーは最初に \ct{Set} に変換されるので、すべてのコレクションでちゃんと動くのです!
\seeindex{Set!union}{Collection, \ct{union:}}
\seeindex{Set!intersection}{Collection, \ct{intersection:}}
\seeindex{Set!membership}{Collection, \ct{includes:}}

\begin{code}{@TEST}
(1 to: 6) union: (4 to: 10)  --> a Set(1 2 3 4 5 6 7 8 9 10)
'hello' intersection: 'there' --> 'he'
#Smalltalk includes: $k     --> true
\end{code}

あとで説明するように、Set の各要素はイテレータでアクセスできます (\secref{iterators} 参照)。

%---------------------------------------------------------
\subsection{SortedCollection}
\ct{OrderedCollection} とは対照的に、\clsindmain{SortedCollection} は要素をソート順を保って格納します。デフォルトでは、SortedCollection はソート順を決めるのに \mthind{Magnitude}{<=} を用いるので、オブジェクトを比較するプロトコル (\mthind{Magnitude}{<}, \mthind{Magnitude}{=}, \mthind{Magnitude}{>}, \mthind{Magnitude}{>=}, \mthind{Magnitude}{between:and:}...) を提供する抽象クラス \clsind{Magnitude} のサブクラスのインスタンスをソートできます。
(See \charef{basic}.)

\ct{SortedCollection} はまず new で生成し、要素を次のように追加していくことができます:
\begin{code}{@TEST}
SortedCollection new add: 5; add: 2; add: 50; add: -10; yourself. --> a SortedCollection(-10 2 5 50)
\end{code}

\noindent
しかし、もっと一般的なやり方は、すでにあるコレクションに変換メッセージ \mthind{Collection}{asSortedCollection} を送ることでしょう:
\begin{code}{@TEST}
#(5 2 50 -10) asSortedCollection --> a SortedCollection(-10 2 5 50)
\end{code}

この例は次の FAQ に対する答えとなります:

\important{FAQ: コレクションをソートするにはどうすればいいでしょうか?\\
{\sc Answer}: \ct{asSortedCollection} メッセージを送りなさい。}

\begin{code}{@TEST}
'hello' asSortedCollection --> a SortedCollection($e $h $l $l $o)
\end{code}

では結果を \ct{String} にもどすには?
残念なことに \ct{asString} は \ct{printString} 表現を返すので、望む結果は得られません。
\begin{code}{@TEST}
'hello' asSortedCollection asString --> 'a SortedCollection($e $h $l $l $o)'
\end{code}
\noindent
正解は \ct{String class>>>newFrom:}、\ct{String class>>>withAll:}、\ct{Object>>>as:} のどれかを用いることです:
\begin{code}{@TEST}
'hello' asSortedCollection as: String              --> 'ehllo'
String newFrom: ('hello' asSortedCollection) --> 'ehllo'
String withAll: ('hello' asSortedCollection)     --> 'ehllo'
\end{code}
\seeindex{Collection!sorting}{Collection, \ct{asSortedCollection}}

すべての要素に互換性がありさえすれば、\ct{SortedCollection} は異なった種類の要素を持つこともできます。例えば整数、浮動小数点数、分数といった異なった種類の数を混ぜることができます:
\begin{code}{@TEST | col |}
{ 5. 2/-3. 5.21 } asSortedCollection --> a SortedCollection((-2/3) 5 5.21)
\end{code}

もしかしたらメソッド \ct{<=} が定義されていないオブジェクトをソートしたかったり、\ct{<=} 以外の基準でソートしたかったりするかもしれません。この場合、ソートブロックと呼ばれる、二つの引数を持つブロックを SortedCollection に与えればよいのです。例えば、クラス \ct{Color} は大小関係を持ちません (Magnitude ではない) し、メソッド \ct{<=} も実装していませんが、次のようにブロックを指定することで色を明度 (明るさの尺度) によってソートすることができるのです:

\begin{code}{@TEST | col |}
col := SortedCollection sortBlock: [:c1 :c2 | c1 luminance <= c2 luminance].
col addAll: { Color red. Color yellow. Color white. Color black }.
col --> a SortedCollection(Color black Color red Color yellow Color white)
\end{code}
\cmindex{SortedCollection class}{sortBlock:}

%---------------------------------------------------------
\subsection{String}
\st の \clsindmain{String}(文字列) は \ct{Character} のコレクションとして表現されます。
文字列は順序性があり、インデックス参照可能で、変更可能で、同じ種類の要素、すなわち \clsind{Character} のインスタンスしか持つことができません。
\ct{Array} のように、\ct{String} は専用の文法を持ち、通常は \ct{String} リテラルは一重引用符で直接表現されますが、通常のコレクション作成メソッドもちゃんと動きます。

\begin{code}{@TEST | s1 s2 |}
'Hello'                                             --> 'Hello'
String with: $A                               --> 'A'
String with: $h with: $i with: $BANG       --> 'hiBANG'
String newFrom: #($h $e $l $l $o) --> 'hello'
\end{code}

実際は、\ct{String} は抽象クラスです。
\ct{String} のインスタンスを作ったときに実際に得られるのは8ビット文字の \clsind{ByteString} か32ビット文字の \clsind{WideString} です。
話を簡単にするために、その際は無視して単に \ct{String} のインスタンスについて述べます。

\ct{String} の二つのインスタンスはコンマによって結合できます。
\index{Collection!comma operator}
\begin{code}{@TEST |s|}
s := 'no', ' ', 'worries'.
s -->  'no worries'
\end{code}

String は変更可能コレクションなので \mthind{String}{at:put:} を用いることもできます。

\begin{code}{@TEST |s| s := 'no', ' ', 'worries'.}
s at: 4 put: $h; at: 5 put: $u.
s --> 'no hurries'
\end{code}

コンマメソッドは \ct{Collection} で定義されているので、どんな種類のコレクションでも使えることに注意してください!
\begin{code}{@TEST}
(1 to: 3) , '45' --> #(1 2 3 $4 $5)
\end{code}
\seeindex{String!concatenation}{Collection, comma operator}
\seeindex{String!comma}{Collection, comma operator}
\index{Collection!comma operator}

すでにある文字列を以下に示すように \mthind{String}{replaceAll:with:} または \mthind{String}{relpaceFrom:to:with:} を使って変更することができます。後者の文字数とインターバルは同じサイズでなければならないことに注意してください。

\begin{code}{@TEST |s| s := 'no hurries' copy.}
s replaceAll: $n with: $N.
s --> 'No hurries'
s replaceFrom: 4 to: 5 with: 'wo'.
s --> 'No worries'
\end{code}

上のメソッドと異なり、\mthind{String}{copyReplaceAll:} は新たな文字列を生成します。
(奇妙に見えますが、引数は個別の文字ではなく、どちらかといえば部分文字列であり、またサイズが一致している必要はありません。)

\begin{code}{@TEST |s| s:= 'No worries' copy.}
s copyReplaceAll: 'rries' with: 'mbats' --> 'No wombats'
\end{code}

実装をちょっと見てみれば、これらのメソッドは \ct{String} 専用ではなく、どんな \ct{SequenceableCollection} にも使えることが分かるので、例えば次のようなものもちゃんと動きます。

\begin{code}{@TEST}
(1 to: 6) copyReplaceAll: (3 to: 5) with: { 'three'. 'etc.' } --> #(1 2 'three' 'etc.' 6)
\end{code}

\paragraph{文字列のマッチ}
\index{String!pattern matching}
\mthind{String}{match:} メッセージを文字列に投げることによりパターンにマッチするかをテストできます。
パターンでは \ct{*} を任意の長さの文字列にマッチする意味で、また \# を一文字にマッチする意味で使用できます。注意したいのは \ct{match:} はパターンに送るもので、マッチさせたい文字列に送るメッセージではないということです。
\begin{code}{@TEST}
'Linux *' match: 'Linux mag'                      --> true
'GNU/Linux #ag' match: 'GNU/Linux tag' --> true
\end{code}

もう一つ便利なメソッドは \ct{findString:} です。
\begin{code}{@TEST}
'GNU/Linux mag' findString: 'Linux'                                                      --> 5
'GNU/Linux mag' findString: 'linux' startingAt: 1 caseSensitive: false  --> 5
\end{code}

Perl が提供しているようなより優れたパターンマッチングの機能も、\pkgind{Regex} パッケージで提供されています。

\paragraph{文字列に対する試験} 以下の例では文字列だけでなく他の一般的なコレクションにも定義されているメッセージ \mbox{\mthind{String}{isEmpty}、}\mthind{String}{includes:}、\mthind{String}{anySatisfy:} を文字列に使ったときの動作を示しています。

\begin{code}{@TEST}
'Hello' isEmpty --> false
'Hello' includes: $a --> false
'JOE' anySatisfy: [:c | c isLowercase] --> false
'Joe' anySatisfy: [:c | c isLowercase] --> true
\end{code}

\paragraph{文字列のテンプレート化}
文字列のテンプレート化に便利なメッセージが三つあります。\mthind{String}{format:}、\mthind{String}{expandMacros}、\mthind{String}{expandMacrosWith:} です。\index{String!テンプレート化}

\begin{code}{@TEST}
'{1} is {2}' format: {'Pharo' . 'cool'}  --> 'Pharo is cool'
\end{code}

expandMacros 族のメッセージは、\ct{<n>} によるキャリッジリターン、\ct{<t>} によるタブ、\ct{<1s>}、\ct{<2s>}、\ct{<3s>} による引数の展開 (\ct{<1p>}、\ct{<2p>} も同様、一重引用符で囲われた文字列)、\ct{<1?value1:value2>} による真偽値による値の埋め込み、などをサポートしています。

% 訳ちょっと強引。Sumim さんのブログを参考にしました。 http://d.hatena.ne.jp/sumim/20061130/p1
% The messages of the expandMacros family offer variable substitution, using \ct{<n>} for carriage return, \ct{<t>} for tabulation, \ct{<1s>}, \ct{<2s>}, \ct{<3s>} for arguments (\ct{<1p>}, \ct{<2p>}, surrounds the string with single quotes), and \ct{<1?value1:value2>} for conditional.

\begin{code}{@TEST}
'look-<t>-here' expandMacros                                         --> 'look-	-here'
'<1s> is <2s>' expandMacrosWith: 'Pharo' with: 'cool'   --> 'Pharo is cool'
'<2s> is <1s>' expandMacrosWith: 'Pharo' with: 'cool'   --> 'cool is Pharo'
'<1p> or <1s>' expandMacrosWith: 'Pharo' with: 'cool'  --> '''Pharo'' or Pharo'
'<1?Quentin:Thibaut> plays' expandMacrosWith: true     --> 'Quentin plays'
'<1?Quentin:Thibaut> plays' expandMacrosWith: false    --> 'Thibaut plays'
\end{code}

\paragraph{その他のユーティリティメソッド}
クラス \ct{String} は \mthind{String}{asLowercase}、\mthind{String}{asUppercase}、\mthind{String}{capitalized} などのユーティリティメッセージを受け取ります。

\begin{code}{@TEST}
'XYZ' asLowercase --> 'xyz'
'xyz' asUppercase   --> 'XYZ'
'hilaire' capitalized   --> 'Hilaire'
'1.54' asNumber      --> 1.54
'this sentence is without a doubt far too long' contractTo: 20 --> 'this sent...too long'
\end{code}

\mthind{Object}{printString} によってオブジェクトにその文字列表現を問い合わせたときと、\mthind{Object}{asString} によってオブジェクトを文字列に変換したときの結果は一般的に異なるということにご注意ください。
違いの例は次のとおりです。

\begin{code}{@TEST}
#ASymbol printString --> '#ASymbol'
#ASymbol asString    --> 'ASymbol'
\end{code}

シンボルは文字列と似ていますが、グローバルに単一であることが保証されるという点で異なります。このためディクショナリーの、とりわけ \ct{IdentifyDictionary} のインスタンスのキーとして、シンボルは文字列より適しています。
\clsind{String} と \clsind{Symbol} については \charef{basic} もご覧ください。

%=========================================================
\section{コレクションのイテレータ}
\seclabel{iterators}

\st では、ループや条件分岐は、コレクションや、整数やブロックのようなその他のオブジェクトへの単なるメッセージ送信です (\charef{syntax} も参照のこと)。先頭要素から最後の要素に対して引数付きのブロックを評価する \ct{to:do:} のような低水準のメッセージに加え、\st のコレクション階層はさまざまなイテレータを提供しています。イテレータを使うことでコードがより堅牢に、コンパクトになります。
\index{Collection!iteration}

%---------------------------------------------------------
\subsection{イテレーション (\lct{do:})}
メソッド \mthind{Collection}{do:} は基本となるコレクション・イテレータです。引数 (一つの引数を取るブロックです) をレシーバーの各要素に順に適用していきます。
次の例ではレシーバーに含まれるすべての文字列をトランスクリプトにプリントします。

\begin{code}{}
#('bob' 'joe' 'toto') do: [:each | Transcript show: each; cr].
\end{code}

\paragraph{派生系} \ct{do:} には沢山の派生系が存在します。例えば \mbox{\mthind{Collection}{do:without:}、}\mbox{\mthind{SequenceableCollection}{doWithIndex:}、}\mthind{OrderedCollection}{reverseDo:} などです:
インデックス可能コレクション (\ct{Array}、\ct{OrderedCollection}、\ct{SortedCollection}) には要素と同時にそのインデックスも参照できる \mthind{SequenceableCollection}{doWithIndex:} があります。このクラスは \ct{Number} クラスで定義されている \ct{to:do:} に関係しています。

\begin{code}{@TEST}
#('bob' 'joe' 'toto') doWithIndex: [:each :i | (each = 'joe') ifTrue: [ ^ i ] ] --> 2
\end{code}

順序ありコレクションの \mthind{OrderedCollection}{reverseDo:} はすべての要素を逆順に処理します。

次のコードは面白いメッセージを示しています:\mthind{Collection}{do:separatedBy:} は二番目のブロックを二つの要素の間でだけ実行します。
\begin{code}{@TEST | res | }
res := ''.
#('bob' 'joe' 'toto') do: [:e | res := res, e ] separatedBy: [res := res, '.'].
res --> 'bob.joe.toto'
\end{code}
\noindent
このコードは一時文字列を作っているという点でとてもできが良いとは言いがたく、結果をためておくのには書き込みストリームを用いたほうがいいということに注意しましょう (see \charef{streams} 参照):
\begin{code}{@TEST}
String streamContents: [:stream | #('bob' 'joe' 'toto') asStringOn: stream delimiter: '.' ] --> 'bob.joe.toto'
\end{code}

% DAMIEN: I would write it:
%res := String streamContents: [:stream |
%  #('bob' 'joe' 'toto')
%       do: [:e | stream nextPutAll: e]
%       separatedBy: [stream nextPut: $.]].
%res --> 'bob.joe.toto'
%Or even simpler:
%res := String streamContents: [:stream | #('bob' 'joe' 'toto')
%asStringOn: stream delimiter: '.' ].
%res --> 'bob.joe.toto'


\paragraph{ディクショナリー}
\mthind{Dictionary}{do:} をディクショナリーに送ると、要素のうち値だけが取り出され、関連が取り出されるわけではありません。\mthind{Dictionary}{keysDo:}、\mthind{Dictionary}{valuesDo:}、and \mthind{Dictionary}{associationsDo:} を使う方がより適当です。それぞれキー、値、関連を取り出します。

\begin{code}{}
colors := Dictionary newFrom: { #yellow -> Color yellow. #blue -> Color blue. #red -> Color red }.
colors keysDo: [:key | Transcript show: key; cr].                    "displays the keys"
colors valuesDo: [:value | Transcript show: value;cr].            "displays the values"
colors associationsDo: [:value | Transcript show: value;cr].  "displays the associations"
\end{code}

%---------------------------------------------------------
\subsection{結果の集積 (\lct{collect:})}
コレクションの要素に何かの処理を行い、その結果を新たなコレクションとしたい場合、\ct{do:} を用いるより、\ct{collect:} またはその他のイテレータメソッドを用いたほうが良いでしょう。
これらのメソッドのほとんどは \ct{Collection} およびそのサブクラスの \protind{enumerating} プロトコルに分類されています。

例えば各要素を倍にしたコレクションを新たに作る場合を考えてみましょう。メソッド \ct{do:} を使うなら次のように書くことになります:

\begin{code}{@TEST | double |}
double := OrderedCollection new.
#(1 2 3 4 5 6) do: [:e | double add: 2 * e].
double --> an OrderedCollection(2 4 6 8 10 12)
\end{code}

\noindent
メソッド \mthind{Collection}{collect:} は引数で渡されたブロックを各要素に対し実行し、その結果を入れた新たなコレクションを返します。
\ct{collect:} を代わりに用いることで、コードはとてもシンプルになります。
\begin{code}{@TEST}
#(1 2 3 4 5 6) collect: [:e | 2 * e] --> #(2 4 6 8 10 12)
\end{code}

\mthind{Collection}{do:} に比べて \ct{collect:} が優れていることは、次の例のほうがわかりやすいでしょう。整数のコレクションを取り、その絶対値からなるコレクションを生成するものです:

\begin{code}{@TEST |aCol result|}
aCol :=  #( 2 -3 4 -35 4 -11).
result := aCol species new: aCol size.
1 to: aCol size do: [ :each | result at: each put: (aCol at: each) abs].
result --> #(2 3 4 35 4 11)
\end{code}
\noindent
上の例を、ずっとシンプルな以下の式と比べてみてください。
\begin{code}{@TEST}
#( 2 -3 4 -35 4 -11) collect: [:each | each abs ] --> #(2 3 4 35 4 11)
\end{code}
\noindent
二つ目の解のもう一つ良い点は set や bug でも同じように動くことです。。

一般的にコレクションの要素それぞれにメッセージを送りたいのでなければ、\ct{do:} の利用は避けたほうが良いです。

メッセージ \ct{collect:} を送ると、レシーバーと同じ種類のコレクションが返ることに注意しましょう。
以下のコードが失敗するのはこれが原因です。
(\ct{String} は整数を保持できないので)
\begin{code}{}
'abc' collect: [:ea | ea asciiValue ]      "エラー発生"
\end{code}
\noindent
最初に文字列を \ct{Array} または \ct{OrderedCollection} に変換しておく必要があります:
\begin{code}{@TEST}
'abc' asArray collect: [:ea | ea asciiValue ] --> #(97 98 99)
\end{code}

実は \ct{collect:} はレシーバーと完全に同じクラスのコレクションを返すという保証はなく、単に同じ \emph{``種族''} であるということが保証されているだけなのです。\ct{Interval} の例では、実際の種族は \ct{Array} になるのです!
\begin{code}{@TEST}
(1 to: 5) collect: [ :ea | ea * 2 ] --> #(2 4 6 8 10)
\end{code}

%---------------------------------------------------------
\subsection{コレクションの選別}
% (\ct{select:}, \ct{reject:}, \ct{detect:})}

\mthind{Collection}{select:} はレシーバーの要素の中で与えられた条件を満たすものを返します:

\begin{code}{@TEST}
(2 to: 20) select: [:each | each isPrime] --> #(2 3 5 7 11 13 17 19)
\end{code}

\mthind{Collection}{reject:} はその反対です:
\begin{code}{@TEST}
(2 to: 20) reject: [:each | each isPrime] --> #(4 6 8 9 10 12 14 15 16 18 20)
\end{code}

%---------------------------------------------------------
\subsection{\lct{detect:} による要素の特定}
The method \mthind{OrderedCollection}{detect:} はレシーバーの要素のなかで、与えられた引数付きブロックを満たす最初のものを返します。

\begin{code}{@TEST}
'through' detect: [:each | each isVowel] --> $o
\end{code}

メソッド \mthind{Collection}{detect:ifNone:} は \ct{detect:} から派生したものです。二つ目のブロックは一つ目のブロックを満たす要素が存在しないときに呼ばれます。

\begin{code}{@TEST}
Smalltalk allClasses detect: [:each | '*cobol*' match: each asString] ifNone: [ nil ] --> nil
\end{code}

%---------------------------------------------------------
\subsection{\lct{inject:into:} による結果の累算}
関数型言語はしばしば \emph{fold} や \emph{reduce} といった、コレクションのすべての要素に二項演算子を繰り返し適用するような高階関数を用意しています。
\pharo でこのようなことをするには \cmind{Collection}{inject:into:} が使えます。

最初の引数は初期値であり、二つ目の引数は前の結果と、要素を一つ受け取る、二つの引数付きブロックです。

\ct{inject:into:} のつまらない例として数のコレクションの総和を求めてみましょう。
ガウス少年のように、\pharo では1から100までの自然数を以下のように書けるのです:
\begin{code}{@TEST}
(1 to: 100) inject: 0 into: [:sum :each | sum + each ] --> 5050
\end{code}

別の例として階乗を計算する一引数のブロックはこちらです:
\begin{code}{@TEST |factorial|}
factorial := [:n | (1 to: n) inject: 1 into: [:product :each | product * each ] ].
factorial value: 10 --> 3628800
\end{code}

%---------------------------------------------------------
\subsection{その他のメッセージ}

\paragraph{\mthind{Collection}{count:}} メッセージ \ct{count:} は与えられた条件を満たす要素の数を返します。条件は真偽ブロックで表現されます。

\begin{code}{@TEST}
Smalltalk allClasses count: [:each | 'Collection*' match: each asString ] --> 3
\end{code}

\paragraph{\mthind{Collection}{includes:}} メッセージ \ct{includes:} は引数がコレクションの中に含まれているかをチェックします。

\begin{code}{@TEST | colors |}
colors := {Color white . Color yellow. Color red . Color blue . Color orange}.
colors includes: Color blue. --> true
\end{code}

\paragraph{\mthind{OrderedCollection}{anySatisfy:}} メッセージ \ct{anySatisfy:} は引数で渡された条件を満足する要素がコレクション中に一つでもあれば true を返します。

\begin{code}{@TEST | colors | colors := {Color white . Color yellow. Color red . Color blue . Color orange}.}
colors anySatisfy: [:c | c red > 0.5] --> true
\end{code}
%=========================================================
\section{コレクションを用いるときのいくつかのヒント}

\paragraph{\mthind{OrderedCollection}{add:} を使うときのよくある間違い} 次のエラーは \st で最も頻繁に見られる失敗です。
\index{Collection!common errors}

\begin{code}{@TEST | collection | }
collection := OrderedCollection new add: 1; add: 2.
collection --> 2
\end{code}
\noindent
変数 \ct{collection} は新たに作られたコレクションではなく、最後に追加された値を保持しています。
これはメソッド \ct{add:} は追加された引数を返し、レシーバーを返すわけではないからです。

次のコードで期待した結果を得ることができます。
\begin{code}{@TEST | collection |}
collection := OrderedCollection new.
collection add: 1; add: 2.
collection --> an OrderedCollection(1 2)
\end{code}

また\ind{カスケード}されたメッセージのレシーバーを返すメッセージ \mthind{Object}{yourself} を用いることもできます。

\begin{code}{@TEST | collection |}
collection := OrderedCollection new add: 1; add: 2; yourself --> an OrderedCollection(1 2)
\end{code}

\paragraph{イテレート処理中のコレクションからの要素削除} もう一つよくある間違いとしては、今イテレートしているそのコレクションから、要素を削除しようとすることがあります。
\mthind{Collection}{remove:}
\begin{code}{@TEST |range|}
range := (2 to: 20) asOrderedCollection.
range do: [:aNumber | aNumber isPrime ifFalse: [ range remove: aNumber ] ].
range --> an OrderedCollection(2 3 5 7 9 11 13 15 17 19)
\end{code}
\noindent
9 と 15 が除外されなければいけないので、この結果は明らかに間違っています!

解決策は処理の前にコレクションをコピーしておくことです。
\begin{code}{@TEST |range|}
range := (2 to: 20) asOrderedCollection.
range copy do: [:aNumber | aNumber isPrime ifFalse: [ range remove: aNumber ] ].
range --> an OrderedCollection(2 3 5 7 11 13 17 19)
\end{code}

\paragraph{\ct{=} と \ct{hash} 両方の再定義}
発見が難しいエラーとしては \ct{=} を再定義しているけれども、\ct{hash} を再定義していないときに起きるものがあります。集合に入れた要素がいつの間にかなくなっていたりとか、そういうおかしな挙動が兆候として現れます。Kent Beck によって提案された一つの解決策は \ct{hash} の再定義に \ct{xor:} を用いることです。
例えば著者もタイトルも同じ二つの本は「等しい」と考えたくなるでしょうから、\ct{=} をそのように再定義したくなりますが、\ct{hash} も次のように再定義すべきなのです:
\index{Dictionary!overriding \ct{=} and \ct{hash}}

\begin{method}{\lct{=} と \lct{hash} の再定義}
Book>>>= aBook
   self class = aBook class ifFalse: [^ false].
   ^ title = aBook title and: [ authors = aBook authors]

Book>>>hash 
   ^ title hash xor: authors hash
\end{method}

可変なオブジェクトを扱っているときにはもう一ついやらしい問題があります。\ie \ct{Set}、あるいは \ct{Dictionary} のキーとしてそういったオブジェクトを用いると、ハッシュ値がいつの間にか変わってしまうということです。
デバッグがお好きでなければ、そういうことは止めておきましょう!

%=========================================================
\section{章のまとめ}

\st のコレクション階層は統一された操作を異なった種類のコレクションに適用できる共通の語彙を持っています。

\begin{itemize}
  \item キーとなる違いは、要素を与えられた順序で保持する \ct{SequenceableCollection} 群、キー-値の関連付けを保持する \ct{Dictionary} とそのサブクラス、順序を持たない \ct{Set} と \ct{Bag} の間に存在します。
  \item ほとんどのコレクションは、メッセージ \ct{asArray}、\ct{asOrderedCollection} \etc を送ることで、他のコレクションに変換できます。
  \item コレクションをソートするには、メッセージ \ct{asSortedCollection} を送ります。
  \item リテラルの \ct{Array} は特殊文法 \ct{#( ... )} で作ることができます。動的な \ct{Array} は文法 \ct|{ ... }| で作ることができます。
  \item \ct{Dictionary} はキーを同値性で比較します。この性質はキーとして \ct{String} のインスタンスを用いるときに最も便利です。一方、\ct{IdentifyDictionary} はオブジェクトの同一性でキーを判別します。この性質はキーとして \ct{Symbol} を用いるときか、オブジェクトの参照を値にマップするときなどに適しています。
  \item \ct{String} は一般的なコレクションメッセージを受け取れます。加えて、\ct{String} は簡単な形式のパターンマッチングもサポートしています。より高度な応用には、RegEx パッケージを参照してください。
  \item 基本となるイテレーションメッセージは \ct{do:} です。すべての要素に同じ処理を行うようなコード、例えばコレクションのすべての要素を変更するとか、同じメッセージを送るといったことをするときに有用です。
  \item \ct{do:} を用いる代わりに、より一般的には \ct{collect:}、\ct{select:}、\ct{reject:}、\ct{includes:}、\ct{inject:into:} またその他の、コレクションを統一的な方法で処理することができる高水準なメッセージを用います。
  \item イテレート中にそのコレクションの要素を削除しないでください。コレクションをイテレーション中に変更したいなら、コピーに対してイテレートします。
  \item \ct{=} の定義を上書きしたなら、\ct{hash} の再定義もお忘れなく!
\end{itemize}

%=========================================================
\ifx\wholebook\relax\else
   \bibliographystyle{jurabib}
   \nobibliography{scg}
   \end{document}
\fi
%=========================================================

