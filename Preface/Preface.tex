% $Author$
% $Date$
% $Revision$

% HISTORY:
% 2006-10-05 - Oscar started
% 2007-05-28 - Stef edit
% 2007-06-06 - Oscar first draft
% 2007-08-14 - Stef corrections
% 2007-09-06 - Lukas review
% 2009-08-12 - Oscar rewrite for Pharo

%=================================================================
\ifx\wholebook\relax\else
% --------------------------------------------
% Lulu:
	\documentclass[a4paper,10pt,twoside]{book}
	\usepackage[
		papersize={6.13in,9.21in},
		hmargin={.75in,.75in},
		vmargin={.75in,1in},
		ignoreheadfoot
	]{geometry}
	\input{../common.tex}
	\pagestyle{headings}
	\setboolean{lulu}{true}
% --------------------------------------------
% A4:
%	\documentclass[a4paper,11pt,twoside]{book}
%	\input{../common.tex}
%	\usepackage{a4wide}
% --------------------------------------------
    \graphicspath{{figures/} {../figures/}}
	\begin{document}
	% \renewcommand{\nnbb}[2]{} % Disable editorial comments
	\sloppy
	\frontmatter
\fi
%=================================================================
\chapter{Preface}\chalabel{intro}

%=================================================================
\section*{\pharo とは?}

\pharo は \st プログラミング言語・環境をフル機能で実装したもので、現在的、かつオープンソースな処理系です。 \pharo は古典的な \st-80 システムを再実装した、 \squeak\cite{Inga97a}から派生しています。\squeak が主に実験的な教育向けソフトウェアを開発するためのプラットフォームとして作られたのに比べ、\pharo はプロフェッショナルなソフトウェアを開発するための洗練されたオープンソース・プラットフォームを目指しています。また、研究と開発のための頑丈で安定したプラットフォームとして、動的な言語・環境を提供するように努めています。\pharo はSeasideウェブ開発フレームワークのリファレンス実装用にも使われています。

\pharo は \squeak にあったライセンスの問題を解決しています。\squeak の以前のバージョンとは異なり、\pharo のコア部分はMITライセンスで寄贈されたものです。\pharo プロジェクトは2008年3月に \squeak 3.9からのフォークとして始まり、最初の1.0ベータバージョンは2009年7月31日にリリースされました。

\pharo は \squeak からパッケージをいくつも取り去っている一方で、\squeak ではオプションであった機能をたくさん取り入れてもいます。例えば、\pharo には True Type フォントが入っています。\pharo は本物のブロッククロージャもサポートしています(訳注:\squeak ではBlockContextという再帰できないクロージャ実装でした)。ユーザインターフェースもシンプルで洗練したものになっています。

\pharo は極めて高い移植性を持っています。仮想マシンでさえもすべて \st で書かれていて、デバッグ・解析・変更をし易くなっています。\pharo はマルチメディア・アプリケーションから、教育向けプラットフォーム、商用ウェブ開発環境まで、広範囲のイノベーティブなプロジェクトの媒体となります。

\pharo の裏にはある重要な視点があります。「\pharo は過去の複製であってはならず、Smalltalkを\emph{再発明}すべきだ」というものです。ビッグバンアプローチが成功するのは稀です。\pharo は進化的でインクリメンタルな変更を支持します。重要な新機能やライブラリを実験できるようにしたいのです。進化とは\pharo が誤りを受け入れることを意味します。つまり、\pharo では(私たちが大好きな)大きなワンステップによる完璧な解答は目指しません。\pharo はたくさんの小さな変更をひとまとめにせず、インクリメンタルに適用することを支持します。\pharo の成功はコミュニティからの貢献にかかっています。

%=================================================================
\section*{誰のための本か?}

この本はオープンソースで公開されている\squeak の入門書 \emph{Squeak by Example}\footnote{\sbe} を元にしています。
\pharo と\squeak の違いに柔軟に適応し、違いを反映して改訂しています。
\pharo の基本からはじまり、先進的な話題まで多様な側面を紹介しています。

この本ではどうやってプログラムするかは学べないと思います。この本はプログラミング言語をいくつか知っている人に向けて書かれています。オブジェクト指向プログラミングの経験も役に立つでしょう。

この本では\pharo のプログラミング環境(言語と関連ツール)を紹介します。一般的なイディオムや実例に触れてもらいますが、オブジェクト指向設計よりも技術の方に焦点を当てています。可能限り例をたくさん示そうと思います(Alec Sharpによる素晴らしいSmalltalkの本\cite{Shar97a}にインスパイアされています)。
\index{Sharp, Alex}

ウェブ上には\st についての無料の本が他にもたくさんあります。しかし \pharo に特化したものはありません。参照:
\url{http://stephane.ducasse.free.fr/FreeBooks.html}

\ifluluelse{}{\newpage} % layout hint
%=================================================================
\section*{読者に向けてのアドバイス}

% http://www.surfscranton.com/architecture/KnightsPrinciples.htm

個々の\st の記述がすぐにわからないからといっていらだってはいけません。すべてを知る必要はないのですから!
Alan Knightは、以下のようにこの原則を表現しています。\footnote{\url{http://www.surfscranton.com/architecture/KnightsPrinciples.htm}}:
\index{Knight, Alan}

\important{{\bf 気にしないようにする}
\st を学び始めたプログラマは、どうやってそれが動くのかを詳細まですべて理解しなければならないと考え、困ってしまうことがよくあります。つまり、\ct{Transcript show: 'Hello World'} をマスターするまでにかなり時間がかかるということです。オブジェクト指向の優れた点は、「どうやって動くのだろう?」という問いに対して「気にしない」と答えられることなのです。}
%=================================================================

\section*{オープンな本}

この本は、以下の意味でオープンな本です。

\begin{itemize}


\item	この本の内容はクリエイティブ・コモンズの表示—継承(by-sa)ライセンスで公開されています。
		つまり、以下のURLにあるライセンス条件を尊重する限り、この本を自由に配布したり改変したりできます。
		URL: 
		\url{http://creativecommons.org/licenses/by-sa/3.0/}
%翻訳メモ: リンクを日本語のページにするか?


\item	この本は\pharo のコア部分のみを解説しています。
		理想をいうと、私たちが書かなかった部分も他の人が寄稿できるようにしたいと思っています。
		この取り組みに参加したい人は連絡してください。私たちはこの本が育つのを見たいのです!
		
\end{itemize}


詳しくは、\pbe を見てください。


%=================================================================
\section*{ \pharo のコミュニティ}


\pharo のコミュニティは親切で活動的です。
役に立ちそうなリソースをいくつかここに挙げておきます。

\begin{itemize}
\item \url{http://www.pharo-project.org} \pharo のメインとなるウェブサイト。

%environment built on top of \pharo but whose audience is elementary
%school teachers.) % I remove this [Martial: french contributor]

\item \url{http://www.squeaksource.com} \pharo プロジェクトにとってのSourceForge的なもの。\pharo 用の追加パッケージがここでたくさん生まれています。

\end{itemize}

%=================================================================
\section*{例と練習問題}

この本では特殊な記法を二つ決めました。

できる限りたくさんの例を示すようにしています。特に、実行可能な短いコードで示した例がたくさんあります。式を選択して\menu{print it}した結果を示すために、\ct{-->}記号を使っています。

\begin{code}{@TEST}
3 + 4 --> 7    "3+4を選択して'print it'を選ぶと7が得られる"
%翻訳メモ: メニューは日本語版にあわせて翻訳しておくか?
\end{code}

\pharo で実際に試すときのために、この本のサイト(\pbe)からすべてのコード例がテキストファイルでダウンロードできるようになっています。

二つめの決まりとして、皆さんにしてほしいことを示すため、\dothisicon{}アイコンを使います。

\dothis{次の章へと読み進みましょう!}

%=================================================================
\section*{謝辞}

まず、\squeak というこの驚くべき\st 開発環境を、オープンソース・プロジェクトとして公開したオリジナルの開発者たちに感謝します。

\st についてのコラムを翻訳することを許可してくれたHilaire FernandesとSerge Stinckwich、ストリームの章を寄稿してくれたDamien Cassouに感謝します。

査読をしてくれたAlexandre Bergel、Orla Greevy、Fabrizio Perin、Lukas Renggli、Jorge Ressia、Erwann Wernliに特に感謝します。

このオープンソース・プロジェクトを快く援助し、この本のウェブサイトを設置してくれたベルン大学(スイス)に感謝します。

この本のプロジェクトを熱心に支援して、初版の間違いを教えてくれたSqueakのコミュニティに感謝します。

%=============================================================
\ifx\wholebook\relax\else
   \bibliographystyle{jurabib}
   \nobibliography{scg}
   \end{document}
\fi
%=============================================================
