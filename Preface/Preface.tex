% $Author$
% $Date$
% $Revision$

% HISTORY:
% 2006-10-05 - Oscar started
% 2007-05-28 - Stef edit
% 2007-06-06 - Oscar first draft
% 2007-08-14 - Stef corrections
% 2007-09-06 - Lukas review
% 2009-08-12 - Oscar rewrite for Pharo

%=================================================================
\ifx\wholebook\relax\else
% --------------------------------------------
% Lulu:
	\documentclass[a4paper,10pt,twoside]{book}
	\usepackage[
		papersize={6.13in,9.21in},
		hmargin={.75in,.75in},
		vmargin={.75in,1in},
		ignoreheadfoot
	]{geometry}
	\input{../common.tex}
	\pagestyle{headings}
	\setboolean{lulu}{true}
% --------------------------------------------
% A4:
%	\documentclass[a4paper,11pt,twoside]{book}
%	\input{../common.tex}
%	\usepackage{a4wide}
% --------------------------------------------
    \graphicspath{{figures/} {../figures/}}
	\begin{document}
	% \renewcommand{\nnbb}[2]{} % Disable editorial comments
	\sloppy
	\frontmatter
\fi
%=================================================================
\chapter{Preface}\chalabel{intro}

%=================================================================
\section*{\pharo とは?}

\pharo は \st プログラミング言語・環境をフルに実装した、オープンソースによる新しい処理系です。 \pharo は古典的な \st-80 システムの再実装である、 \squeak\cite{Inga97a}から派生しました。\squeak が主に実験的な教育向けソフトウェアを開発するためのプラットフォームとして作られたのに対して、\pharo はプロフェッショナルなソフトウェアを開発するための、凝縮されたオープンソース・プラットフォームを目指しています。また、動的な言語・環境の研究開発のための、頑丈で安定したプラットフォームの提供にも努めます。\pharo はウェブアプリ開発フレームワークSeasideのリファレンス実装用にも使われています。

\pharo は \squeak にあったライセンスの問題を解決しています。\squeak の以前のバージョンとは異なり、\pharo のコア部分はMITライセンスで寄贈されたコードのみからできています。\pharo プロジェクトは2008年3月に \squeak 3.9からのフォークとして始まり、最初の1.0ベータバージョンは2009年7月31日にリリースされました。

\pharo は \squeak からパッケージをいくつも取り去った一方で、\squeak ではオプションだった機能をたくさん取り入れています。例えば、\pharo には True Type フォントが入っています。\pharo は本物のブロッククロージャもサポートしています(訳注:\squeak ではBlockContextという再帰できないクロージャ実装でした)。ユーザインターフェースもシンプルで洗練されたものになっています。

\pharo は極めて高い移植性を持っています。仮想マシンでさえもすべて \st で書かれていて、デバッグ・解析・変更をしやすくなっています。\pharo はマルチメディア・アプリケーションから、教育向けプラットフォーム、商用ウェブ開発環境まで、広い範囲で革新的なプロジェクトの媒体となります。

\pharo は重要な視点を背景として持っています。「\pharo は過去の単なる複製であってはならない。Smalltalkを\emph{再発明}すべきだ」というものです。ビッグバンアプローチは滅多に成功しません。\pharo では進化的でインクリメンタルな変化をさせる方法を取ります。重要な新機能やライブラリ開発を実験的に進められるようにしたいのです。進化とは\pharo が誤りを受け入れていくことを意味します。\pharo では一度で完璧に解決することを目指しません。そうしたくなる気持ちをこらえてでも、です。小さな変化をいくつも積み上げていくやり方が、\pharo ではうまくいくでしょう。\pharo の成功はコミュニティからの貢献にかかっているのです。

% \pharo コミュニティは皆さんから送られてくるコードに注目します。システムを向上させるために。

%=================================================================
\section*{この本が想定している読者}

この本はオープンソースで公開されている\squeak の入門書 \emph{Squeak by Example}\footnote{\sbe} を元にしています。
その名の通り、\pharo と\squeak の違いを反映して修正しました。
\pharo の基本からはじまり、先進的な話題まで、内容は多岐にわたります。

この本はどうやってプログラムするのかを教える本ではありません。この本は既にプログラミング言語をいくつか知っている人に向けて書かれています。オブジェクト指向プログラミングの経験も役に立つでしょう。

この本では\pharo のプログラミング環境(言語と関連ツール)を紹介します。一般的なイディオムや実例に触れてもらいますが、オブジェクト指向設計よりも技術の方に焦点を当てています。可能限り例をたくさん示そうと思います(Alec Sharpによる素晴らしいSmalltalkの本\cite{Shar97a}にインスパイアされています)。
\index{Sharp, Alex}

ウェブ上には\st についての無料の本が他にもたくさんあります。しかし \pharo に特化したものはありません。参照:
\url{http://stephane.ducasse.free.fr/FreeBooks.html}

\ifluluelse{}{\newpage} % layout hint
%=================================================================
\section*{読者に向けてのアドバイス}

% http://www.surfscranton.com/architecture/KnightsPrinciples.htm

個々の\st の記述がすぐにわからないからといっていらだってはいけません。すべてを知る必要はないのです!
Alan Knightは、以下のようにこの原則を表現しています。\footnote{\url{http://www.surfscranton.com/architecture/KnightsPrinciples.htm}}:
\index{Knight, Alan}

\important{{\bf 気にしないようにする}
\st を学び始めたプログラマは、どうやってそれが動くのかを詳細まですべて理解しなければならないと考え、困ってしまうことがよくあります。つまり、\ct{Transcript show: 'Hello World'} をマスターするまでにかなり時間がかかるということです。オブジェクト指向の優れた点は、「どうやって動くのだろう?」という問いに対して「気にしない」と答えられることなのです。}
%=================================================================

\section*{オープンな本}

この本は、以下の意味でオープンな本です。

\begin{itemize}


\item	この本の内容はクリエイティブ・コモンズの表示—継承(by-sa)ライセンスで公開されています。
		つまり、以下のURLにあるライセンス条件を尊重する限り、この本を自由に配布したり改変したりできます。
		URL: 
		\url{http://creativecommons.org/licenses/by-sa/3.0/deed.ja}

\item	この本は\pharo のコア部分のみを解説しています。
		理想を言うと、私たちが書かなかった部分も他の人が寄稿できるようにしたいと思っています。
		この取り組みに参加したい人は連絡してください。私たちはこの本が育つのを見たいのです!
		
\end{itemize}


詳しくは、\pbe を見てください。


%=================================================================
\section*{ \pharo のコミュニティ}


\pharo のコミュニティは親切で活動的です。
役に立ちそうなリソースをいくつかここに挙げておきます。

\begin{itemize}
\item \url{http://www.pharo-project.org} \pharo のメインとなるウェブサイト。

%environment built on top of \pharo but whose audience is elementary
%school teachers.) % I remove this [Martial: french contributor]

\item \url{http://www.squeaksource.com} \pharo プロジェクトにとってのSourceForge的なもの。\pharo 用の追加パッケージがここでたくさん生まれています。

\end{itemize}

%=================================================================
\section*{例と練習問題}

この本では特殊な記法を二つ決めました。

できる限りたくさんの例を示すようにしています。特に、実行可能な短いコードで示した例がたくさんあります。式を選択して\menu{print it}した結果を示すために、\ct{-->}記号を使っています。

\begin{code}{@TEST}
3 + 4 --> 7    "3+4を選択して'print it'を選ぶと7が得られる"
\end{code}
%翻訳メモ: メニューは日本語版にあわせて翻訳しておくか?
\pharo で実際に試すときのために、この本のサイト(\pbe)からすべてのコード例がテキストファイルでダウンロードできるようになっています。

二つ目の決まりとして、みなさんにしてほしいことを示すため、\dothisicon{}アイコンを使います。

\dothis{次の章へと読み進めましょう!}

%=================================================================
\section*{謝辞}

まず、\squeak というこの驚くべき\st 開発環境を、オープンソース・プロジェクトとして公開したオリジナルの開発者たちに感謝します。

% この本に貢献してくれた多種多様な皆さんに感謝しています。
% 特に、
\st についてのコラムを翻訳することを許可してくれたHilaire FernandesとSerge Stinckwich、ストリームの章を寄稿してくれたDamien Cassouに感謝します。

査読をしてくれたAlexandre Bergel、Orla Greevy、Fabrizio Perin、Lukas Renggli、Jorge Ressia、Erwann Wernliに特に感謝します。

このオープンソース・プロジェクトを快く援助し、この本のウェブサイトを設置してくれたベルン大学(スイス)に感謝します。

そして、Squeakのコミュニティに感謝します。この本のプロジェクトを熱意を持って助け、また、この本の初版の間違いを教えてくれました。

%=============================================================
\ifx\wholebook\relax\else
   \bibliographystyle{jurabib}
   \nobibliography{scg}
   \end{document}
\fi
%=============================================================
