% $Author$
% $Date$
% $Revision$

% HISTORY:
% 2007-06-06 - Stef started
% 2007-08-21 - Oscar edit
% 2007-09-06 - Lukas corrections
% 2007-09-11 - Orla corrections

%=================================================================
\ifx\wholebook\relax\else
% --------------------------------------------
% Lulu:
	\documentclass[a4paper,10pt,twoside]{book}
	\usepackage[
		papersize={6.13in,9.21in},
		hmargin={.75in,.75in},
		vmargin={.75in,1in},
		ignoreheadfoot
	]{geometry}
	\input{../common.tex}
	\pagestyle{headings}
	\setboolean{lulu}{true}
% --------------------------------------------
% A4:
%	\documentclass[a4paper,11pt,twoside]{book}
%	\input{../common.tex}
%	\usepackage{a4wide}
% --------------------------------------------
    \graphicspath{{figures/} {../figures/}}
	\begin{document}
	\renewcommand{\nnbb}[2]{} % Disable editorial comments
	\sloppy
\fi
%=================================================================
\chapter{メッセージ構文を理解しよう}
\chalabel{understanding}

\st の構文はとても単純ですが、標準的なものではないので、慣れるのに少し時間がかかるでしょう。
この章では、この特殊なメッセージ構文に順応するためのガイダンスを提供します。
もしすでに構文には慣れていると思われるなら、この章は飛ばしても構いませんし、あるいはまた後で読んでも良いでしょう。

%=============================================================
\section{メッセージを読み取る}

\st では \charef{syntax} で紹介した構文要素 (\ct+:= ^ . ; # () {} [ : | ]+)を除けば、すべてがメッセージ送信です。
\ind{C++} のように、\ct{+} のような演算子を自分で作ったクラスに定義することもできますが、すべての演算子の優先順位は同じです。
さらに、あるメソッドが受け取る引数の個数も変えられません。例えば ``\ct{-}'' は、常に二項演算子で、単項の ``\ct{-}'' をオーバーロードして作ることはできません。

\st では、メッセージの優先順位、すなわち記述されたメッセージがどのような順番で送信されるかは、メッセージの種類によって決定されます。そのメッセージの種類は 3 種類だけです: \emphsubind{メッセージ}{単項}、\emphsubind{メッセージ}{二項}および\emphsubind{メッセージ}{キーワード}メッセージです。単項メッセージは最高の優先順位であり、その次が二項、そして最後がキーワードメッセージです。他の多くの言語と同様に、\ind{括弧}を使って評価順を変えることができます。これらの規則によって、\st のコードは可能な限り読みやすく、かつ規則のことに頭を悩ませなくてもよいようになっています。

基本的に、\st でのすべての計算はメッセージ送信によって行われるので、メッセージを正しく読み取ることが非常に重要です。以下にまとめた用語と概念を理解することが役に立つでしょう。

\begin{itemize}
  \item メッセージは、\emphsubind{メッセージ}{セレクタ}およびオプションのメッセージ引数からなりたっています。
  \item メッセージは、\emphsubind{メッセージ}{レシーバ}に送信されます。
  \item レシーバとメッセージの組み合わせが、\emph{メッセージ} \emphsubind{メッセージ}{送信}と呼ばれます(\figref{firstScriptMessage})。
%  \item We call expression a message send, a variable, an assignment, a literal or a block. 
\end{itemize}

\begin{figure}[htb]
\begin{minipage}{0.53\textwidth}
	\begin{center}
	\includegraphics[width=0.95\textwidth]{message}
	\caption{レシーバ、セレクタ、引数からなるメッセージ送信の例。二つ。\figlabel{firstScriptMessage}}\end{center}
\end{minipage}
\hfill
\begin{minipage}{0.43\textwidth}
	\begin{center}
	\ifluluelse
		{\includegraphics[width=0.9\textwidth]{uKeyUnOne}}
		{\includegraphics[width=6cm]{uKeyUnOne}}
	\caption{\ct{aMorph color: Color yellow} という式は、\ct{Color yellow} および \ct{aMorph color: Color yellow} という二つのメッセージ送信からなりたっている。\figlabel{ellipse}}
	\end{center}
\end{minipage}
\end{figure}

%\begin{figure}[ht]
%\begin{center}
%\includegraphics[width=0.5\textwidth]{message}
%\caption{Two messages composed of a receiver, a method selector, and a set of arguments.\figlabel{firstScriptMessage}}\end{center}
%\end{figure}

\important{メッセージは、常にレシーバに送信されます。レシーバは、リテラル・ブロック・変数・他のメッセージを評価した結果、のいずれかです。}
%sd-ま、代入を考えてないからかんぜんただしいというわけではないがな

以下の図ではメッセージのレシーバーを同定しやすくなるよう、レシーバーに下線を引くことにします。
また、それぞれのメッセージ送信を点線の楕円で囲み、送信される順番に従って番号をつけてあります。

%\begin{figure}[!ht]
%\begin{center}
%\includegraphics[width=6cm]{uKeyUnOne}
%\end{center}
%\caption{\ct{aMorph color: Color yellow} is composed of two expressions: \ct{Color yellow} and \ct{aMorph color: Color yellow}.\figlabel{ellipse}}
%\end{figure}

\figref{ellipse}は\ct{Color yellow}と\ct{aMorph color: Color yellow}という2つのメッセージ送信を表していますので、楕円も2つあります。メッセージ送信\ct{Color yellow}がまず実行されるので、そこについている楕円に番号\ct{1}がついています。全体の中にはレシーバーが2つあります。ひとつは\ct{aMorph}で、メッセージ\ct{color: ...}が送られており、もう一つは\ct{Color}で、メッセージ\ct{yellow}が送られています。それぞれに下線がついています。

レシーバーは、例えばメッセージ送信\ct{100 + 200}の\ct{100}や、\ct{Color yellow}の中の\ct{Color}のように、メッセージ送信記述の最初の要素であることが多いです。しかしながら、ほかのメッセージの結果がレシーバーとして使われることもよくあります。例えば、\ct{Pen new go: 100}というメッセージ式では、\ct{go: 100}というメッセージのレシーバーは\ct{Pen new}というメッセージ送信から結果として返されるオブジェクトとなります。いずれにせよ、メッセージは\emph{レシーバー}と呼ばれるオブジェクトに送られ、そのオブジェクトはほかのメッセージ送信の結果の場合もあるということです。

\begin{table}\centering
	\begin{tabularx}{\linewidth}{llX}
		\toprule
		メッセージ送信式 & メッセージの種類 & 結果 \\
		\midrule
		\lct{Color yellow}
			& 単項
			& 色を表すオブジェクト。
		\\
		\lct{aPen  go: 100}
			& キーワード
			& レシーバーであるペン・オブジェクト100ピクセル移動。
		\\
		\lct{100 + 20}
			& 2項
			& 数値オブジェクト100に数値20を引数とする+メッセージを送信。
		\\
		\lct{Browser open}
			& 単項
			& 新しいブラウザを開く。
		\\
		\lct{Pen new  go: 100}
			& 単項およびキーワード
			& ペン・オブジェクトが生成され、100ピクセル移動する。
		\\
		\lct{aPen go: 100 + 20}
			& キーワードおよび2項
			& レシーバーであるペン・オブジェクトが120ピクセル移動。
		\\
		\bottomrule
	\end{tabularx}
	\caption{メッセージ送信と種類の例}\tablabel{messageExamples}
\end{table}

\tabref{messageExamples}にいくつかのメッセージ式の例を挙げました。
必ずしもすべてのメッセージ式に引数があるわけではないということに注意してください。\ct{open}のような単項メッセージには引数はありません。\ct{go: 100}と\ct{+ 20}のような単独のキーワードと2項メッセージは引数を一つとります。
単純なメッセージと複合的なものとがあります。\ct{Color yellow}や\ct{100 + 20}は単純です。一つのメッセージが一つのオブジェクトに送られるだけです。一方、\ct{aPen go: 100 + 20}は2つのメッセージ送信が複合したものです。\ct{+ 20}が\ct{100}に送られ、その結果を引数として\ct{go:}が\ct{aPen}に送られます。
値を返す式(代入、メッセージ送信およびリテラル)がレシーバーとなりえます。\ct{Pen new go: 100}では、メッセージ\ct{go: 100}が\ct{Pen new}の結果として返されるオブジェクトに送られます。


%=============================================================
\section{3種類のメッセージ}

\st ではメッセージの送信順序を決定するいくつかの簡単な規則が定義されています。これらの規則は、以下のような3種類のメッセージ式の種類に基づいています。
\begin{itemize}
\item \emph{単項メッセージ}はあるオブジェクトに他の追加情報なしで送られるようなメッセージです。 例えば、\ct{3 factorial}という式の\ct{factorial}は単項メッセージです。
\item  \emph{2項メッセージ}は演算子(しばしば算術的な)からなるメッセージです。これらは常にレシーバーと引数という2つのオブジェクトが関与しているために2項(binary)と呼ばれています。\ct{10 + 20}の場合は、\ct{+}が演算子であり、 引数\ct{20}と共に\ct{10}に送られています。
\item  \emph{キーワードメッセージ}は1つあるいはそれ以上のキーワードからなっています。それぞれのキーワードはコロン記号(\ct{:})がついており、さらにひとつ引数をとります。
例えば、\ct{anArray at: 1 put: 10}という式の場合、キーワード\ct{at:}は\ct{1}という引数をとり、\ct{put:}というキーワードが引数\ct{10}をとっています。
\end{itemize}

%-------------------------------------------------------------
\subsection{単項メッセージ}
単項メッセージは引数を必要としないメッセージであり、その構文は\ct{receiver selector}という形式です。セレクターは\ct{:}を含まない単純な文字列です(例: \ct{factorial}, \ct{open}, \ct{class})。
\needlines{4}
\begin{code}{@TEST}
89 sin           --> 0.860069405812453
3 sqrt           --> 1.732050807568877
Float pi         --> 3.141592653589793
'blop' size     --> 4
true not        --> false
Object class --> Object class  "The class of Object is Object class (BANG)"
\end{code}
% ON: I changed the examples to things we can test

\important{単項メッセージは引数を必要とないメッセージです。\\
その構文は\lct{receiver \textbf{selector}}という形式です。}

%-------------------------------------------------------------
\subsection{2項メッセージ} 
2項メッセージは引数を1つだけとり\emph{かつ}セレクターが以下の文字集合からの1文字以上の繰り返しからなるメッセージです。\ct{+}、\ct{-}、\ct{*}、\ct{/}、\ct{\&}、\ct{=}、\ct{>}、\ct{|}、\ct{<}、\ct{\~}、および\ct{@}。構文解析時に問題が起こるので、\ct{--}は不正なセレクターであるということに注意してください。

\begin{code}{@TEST}
100@100      --> 100@100  "Pointオブジェクトを生成"
3 + 4              --> 7
10 - 1            --> 9
4 <= 3            --> false
(4/3) * 3 = 4   --> true  "同値性のテストも単なる2項メッセージ。分数オブジェクトは値を正確に表現できる"
(3/4) == (3/4) --> false  "別個に作られた2つの分数オブジェクトは同一ではない"
\end{code}

\important{2項メッセージは引数を1つだけとり\emph{かつ}セレクターが以下の文字集合からの1文字以上の繰り返しからなるメッセージです。\ct{+}、\ct{-}、\ct{*}、\ct{/}、\ct{\&}、\ct{=}、\ct{>}、\ct{|}、\ct{<}、\ct{\~}、および\ct{@}。\ct{--}は許されません。 \\
その構文は\lct{receiver \textbf{selector} argument}という形式です。}

%-------------------------------------------------------------
\subsection{キーワードメッセージ} 

キーワードメッセージは1つ以上の引数を必要とします。そのセレクターは\ct{:}で終わるキーワードを1つ以上並べたものからなります。キーワードメッセージの構文は
\lct{receiver \textbf{selectorWordOne:} argument\-One \textbf{wordTwo:} argumentTwo}という形式です。

それぞれのキーワードが引数をひとつずつとります。すなわち、\ct{r:g:b:}というセレクターは3引数であり、\ct{playFileNamed:}や\ct{at:}は1引数、\ct{at:put:}は2引数ということになります。\ct{Color}クラスのインスタンスを作るためには、例えば\ct{r:g:b:}を使い、\ct{Color r: 1 g: 0 b: 0}のようなメッセージにより、赤色を表すColorオブジェクトを作ることができます。ここでは、コロンもセレクターの一部であるということに注意してください。

\important{\ind{Java}や\ind{C++}であれば、\st における\ct{Color r: 1 g: 0 b: 0}という式は
\ct{Color.rgb(1,0,0)}と書かれることになるでしょう。}

\begin{code}{@TEST | nums |}
1 to: 10                        --> (1 to: 10)  "区間オブジェクトを生成"
Color r: 1 g: 0 b: 0       --> Color red  "色オブジェクトを返す"
12 between: 8 and: 15 --> true

nums := Array newFrom: (1 to: 5).
nums at: 1 put: 6.
nums --> #(6 2 3 4 5)
\end{code}
% ON: Changed to real examples that we can test

\important{キーワードメッセージは1つ以上の引数をとるメッセージです。そのセレクターはそれぞれがコロン(\ct{:})で終わる1つ以上のキーワードからなります。構文は\\
\lct{receiver \textbf{selectorWordOne:} argumentOne \textbf{wordTwo:} argumentTwo}という形式をとります。}

%=============================================================
\section{複合したメッセージ}
これまで述べた3種類のメッセージはそれぞれ異なる優先順位をもっており、それらをエレガントに組み合わせて使えるようになっています。

\begin{enumerate}
\item 単項メッセージがまず最初に送信され、次に2項、そして最後にキーワードメッセージが送られます。
\item \ind{かっこ}で囲まれたメッセージは他のものよりも優先して送信されます。
\item 同じ種類のメッセージは左から右に評価されます。
\end{enumerate}
\index{message!evaluation order}

このルールにより、プログラムを読む際にも自然に読めるようになっています。メッセージが意図した通りの順番で送られることを明示的に示したいのであれば、\figref{uKeyUn}にあるように、かっこを足していくこともできます。この図では、メッセージ\ct{yellow}が単項メッセージであり、\ct{color:}はキーワードメッセージであるため、\ct{Color yellow}がまず先に実行されます。しかし、かっこで囲まれたメッセージがまず実行されるため、(冗長な)かっこを\ct{Color yellow}の周りに書くことにより、ここがまず先に実行されるということを強調することができます。以下では、これらの点について説明します。

\begin{figure}[ht]
\ifluluelse
	{\centerline{\includegraphics[width=0.9\textwidth]{uKeyUn}} }
	{\centerline{\includegraphics[width=10cm]{uKeyUn}} }
\caption{単項メッセージが最初に送られるので、\ct{Color yellow}がまず送られ、返されたColorオブジェクトが引数として\ct{aPen color:}に渡されます。\figlabel{uKeyUn}}
\end{figure}

%---------------------------------------------------------
\subsection*{単項 > 2項 > キーワード}
単項メッセージがまず送られ、次に2項、そしてキーワードメッセージが最後に送られます。しばしば、単項が他の種類のメッセージよりも高い優先順位を持つということがあります。

\important{\textbf{規則1:} 単項メッセージがまず送られ、次に2項、そしてキーワードメッセージが最後に送られます。\\
\centerline{単項 \ct{>} 2項 \ct{>} キーワード}
}

以下の例が示すように、\st の構文では複合したメッセージ式を多くの場合自然な形で読むことができます。
\begin{code}{@TEST}
1000 factorial / 999 factorial --> 1000
2 raisedTo: 1 + 3 factorial     --> 128
\end{code}
\noindent

ただし、残念ながら、算術演算の場合にはこの規則はやや単純すぎます。そのため、かっこを使って2項演算子を複数組み合わせたときに評価順を強制する必要があります。
\begin{code}{@TEST}
1 + 2 * 3   --> 9
1 + (2 * 3) --> 7
\end{code}

以下のさらに複雑な(!)例は、さらに混みいった\st の式も自然に読めるということの良い例になっています。
\begin{code}{@TEST}
[:aClass | aClass methodDict keys select: [:aMethod | (aClass>>aMethod) isAbstract ]] value: Boolean --> an IdentitySet(#or: #| #and: #& #ifTrue: #ifTrue:ifFalse: #ifFalse: #not #ifFalse:ifTrue:)
\end{code}
\noindent
ここでは、\ct{Boolean}クラスのどのメソッドが抽象メソッドであるかを調べようとしています\footnote{実をいうと、同等の式はさらに簡単に\ct{Boolean methodDict select: #isAbstract thenCollect: #selector}と書くこともできます。}\damien{I've added this footnote, just remove it if you don't like it :-)}.
まず、引数として渡されたクラス\ct{aClass}に, そのメソッド辞書のキー集合を問い合わせ、その中から、抽象メソッドに対応しているものを選択(select)するブロックを記述します。
そのブロックの引数\ct{aClass}にあるクラス\ct{Boolean}を束縛しています。
この式の中では、かっこはメソッドをクラスから取り出すを2項演算子\ct{>>}を、単項メッセージ\mbox{\ct{isAbstract}}の前に送るところだけに必要でした。全体を評価した結果は、\ct{Boolean}のサブクラスである\ct{True}および\ct{False}で実装しなくてはならないメソッドの集合となっています。

%\begin{code}{}
%Pen new go: 30 + 50          "create a turtle and moves it forward 80 pixels"
%Display restoreAfter: [WarpBlt test4] 					
%	"Keyword message, try test1, test12, test3, test4 and test 5"
%#($t $e $s $t) at: 3 --> $s 
%#($a $b $c $d) at: 2 factorial put: $z 
%\end{code}

%As you can see the syntax and in particular the keyword messages as in
%the example \ct{array at: 1 put: 4} make it possible to write code
%with a structure approaching that of natural language.
% This was one of the initial objectives so that the children can program.

\paragraph{例:}
式\ct{aPen color: Color yellow}の中には、\ct{Color}に送られる\emph{単項}メッセージ\ct{yellow}と、\ct{aPen}に送られる\emph{keyword}メッセージ\ct{color:}があります。単項メッセージがまず送られるので、\egref{decColor}にあるように\ct{Color yellow}がまず送られ(1)、その結果として返されたColorオブジェクトがメッセージ\ct{aPen color: aColor}の引数となります(2)。
\figref{uKeyUn} shows graphically how messages are sent. TODO

\needlines{5}
\begin{example}[decColor]{\ct{aPen color: Color yellow}の評価順を分解}{}
        aPen color: Color yellow
(1)                       Color yellow        "単項メッセージがまず送られる"
                        --> aColor
(2)   aPen color: aColor                 "キーワードメッセージが次に送られる"
\end{example}

\paragraph{例:} 式\ct{aPen go: 100 + 20}の中には、\emph{2項}メッセージ\ct{+ 20}と\emph{キーワード}メッセージ\ct{go:}があります。2項メッセージはキーワードメッセージの先に送られるので\ct{100 + 20}がまず送信されます(1)。\ct{+ 20}が数値オブジェクト\ct{100}に送信され、数値\ct{120}が返ります。次に、メッセージ\ct{aPen go: 120}が\ct{120}を引数として送信されます(2)。
\Egref{decGo}にこのメッセージを図示しました。

\begin{example}[decGo]{\ct{aPen go: 100 + 20}の評価順を分解}{}
      aPen go: 100 + 20   
(1)                 100 + 20           "2項メッセージがまず送られる"
                   -->   120
(2)  aPen go: 120                   "キーワードメッセージが次に送られる"
\end{example}

\begin{figure}[htb]
\begin{minipage}{0.48\textwidth}
	\ifluluelse
		{\centerline{\includegraphics[width=0.9\textwidth]{uKeyBin}}}
		{\centerline{\includegraphics[width=6cm]{uKeyBin}}}
	\caption{2項メッセージはキーワードメッセージの前に送られる。\figlabel{uKeyBin}}
\end{minipage}
\hfill
\begin{minipage}{0.48\textwidth}
	\begin{center}
	\ifluluelse
		{\includegraphics[width=0.9\textwidth]{uunKeyBin}}
		{\includegraphics[width=6cm]{uunKeyBin}}
\caption{\ct{Pen new go: 100 + 20}の評価順}\figlabel{unKeyBin}
\end{center}
\end{minipage}
\end{figure}

%\begin{figure}[ht]
%\centerline{\includegraphics[width=6cm]{uKeyBin}} 
%\caption{Unary messages are sent first so \ct{Color yellow} is sent. This returns a color object which is passed as argument of the message \ct{aPen color:}.\figlabel{uKeyBin}}
%\end{figure}

%\paragraph{Example 3.}
%The message \ct{aPen penSize: aPen penSize + 2} contains one unary message \ct{penSize}, one binary message \ct{+},  and one keyword message \ct{penSize:}.
%The unary message \ct{aPen penSize} is sent first (1), this message returns a number representing the current size of the receiver pen. Then the binary message is sent (2), the returned number is sent the message \ct{+ 2} which in its turn returns another number. Finally the keyword message 
%\ct{penSize:} is sent with the last number as argument. The expression increases the receiver pen size by two pixels. \Egref{decpen} illustrates the decomposition of message evaluation.

% penSize does not exist on Pen
%\

%\begin{scriptfigwithsize}[0.65]{\includegraphics[width=6cm]{uKeyUnBin}}{Decomposing \ct{ aPen penSize: aPen penSize + 2}}\scrlabel{decpen}
%      aPen penSize: aPen penSize + 2
%(1)                        aPen penSize            "unary"
%                              $\arrow$  aNumber
%(2)                              aNumber + 2	             "binary"
%                                    $\arrow$   anotherNumber	
%(3)   aPen penSize: anotherNumber           "keyword"      
%\end{scriptfigwithsize}

\paragraph{例:} 練習として、式\ct{Pen new go: 100 + 20}の評価順を分解してみてください。この式には単項メッセージがひとつ、キーワードメッセージがひとつ、2項メッセージがひとつあります。(\figref{unKeyBin}を参照。).
% The unary message \ct{Pen new} is first sent. It returns a new bot, then the binary message \ct{100 + 20} is sent and returns \ct{120}. Finally the message \ct{go:} is sent to the newly created robot with \ct{120}.

%\begin{figure}[ht]
%\begin{center}
%\includegraphics[width=8cm]{uunKeyBin}
%\caption{Decomposing \ct{Pen new go: 100 + 20}}\figlabel{unKeyBin}
%\end{center}
%\end{figure}

%-------------------------------------------------------------
\subsection{かっこ優先}

\important{\textbf{規則2:} かっこで囲まれたメッセージが他のメッセージよりも優先して送信されます。\\
\centerline{\ct{(}任意のメッセージ\ct{) >} 単項 \ct{>} 2項 \ct{>} キーワード}}

\begin{code}{@TEST}
1.5 tan rounded asString = (((1.5 tan) rounded) asString) --> true    "かっこは不要"
3 + 4 factorial   --> 27    "(5040ではない)"
(3 + 4) factorial --> 5040
\end{code}

以下の例では、\ct{lowMajorScaleOn:}を\ct{play}よりも先に送るために\ind{かっこ}が使われています。
\begin{code}{}
(FMSound lowMajorScaleOn: FMSound clarinet) play 
"(1) メッセージclarinetがFMSoundクラスに送られ、クラリネットの音色オブジェクトが作られる。
 (2) このオブジェクトがFMSoundに送られるlowMajorScaleOn:キーワードメッセージの引数として送られる。
 (3) 結果の音オブジェクトにplayが送られ、演奏される。"
\end{code}

% ON: This has nothing to do with parentheses!
%RecordingControlsMorph new openInWorld
%"An instance of the digitizer is created then visualized. If your microphone is plugged in try a sampleBANG"

% ON: This link is broken, and the result does not understand display!
%(HTTPSocket httpShowGif:
%   'www.altavista.digital.com/av/pix/default/av-adv.gif') display

\paragraph{例:}
式\ct{(65@325 extent: 134@100) center}は左上が座標$(65, 325)$で大きさが$134{\times}100$であるような長方形の中央座標を返します。\Egref{decExtent}に評価順を分解したところが図示されています。まず、かっこで囲まれた部分が実行されます。その部分には2つの2項メッセージ\ct{65@325}と\ct{134@100}があるので、それらがまず実行されてそれぞれPointオブジェクトを返します。次に、キーワードメッセージ\ct{extent:}が送られ、Rectangleオブジェクトが得られます。最後に、単項メッセージ\ct{center}がその長方形オブジェクトに送られ、点が返ります。
もしかっこがなかったとすると、数値\ct{100}はメッセージ\ct{center}を理解できないため、評価時にエラーとなります。

\needlines{9}
\begin{example}[decExtent]{かっこの例。}{}
      (65@325 extent: 134@100) center
(1)   65@325                                                    "2項"
    --> aPoint
(2)                                134@100                     "2項"
                                 --> anotherPoint
(3)   aPoint extent: anotherPoint                       "キーワード"
      --> aRectangle
(4)   aRectangle center                                     "単項"
      --> 132@375
\end{example}

\subsection{左から右への評価}
ここまでで、異なる種類のメッセージがどのような優先順位で処理されるかを学びました。残る疑問は、同じ優先順位のメッセージがどのような順番で送信されるかということです。その答えは、左から右に送られるというものです。このことは、実は\egref{decExtent}の中で、2つのPointオブジェクト生成メッセージ(\ct{@})が送られているときにすでに登場していました。

\important{{\textbf{規則3:} 同じ種類のメッセージがある場合は、評価は左から右の順で行われる。}}

%\begin{figure}
%\centerline{\includegraphics[width=8cm]{ucompoUn}} 
%\caption{The message \ct{Pen new east} is composed of two unary messages. Therefore the leftmost one, \ct{new},  is sent and it returns a new robot to which the second message \ct{east} is sent. \figlabel{compoUn}}
%\end{figure}

\paragraph{例:} 式\ct{Pen new down}では単項メッセージのみが使われているので、左端の\ct{Pen new}がまず送信されます。その結果は新しいペン・オブジェクトで、それに次のメッセージ\ct{down}が送られます。\figref{unaryMessages}を参照してください。

\begin{figure}
	\centering
	\includegraphics[width=8cm]{ucompoUn}
	\caption{\ct{Pen new down}の評価順の分解\figlabel{unaryMessages}}
\end{figure}

%-------------------------------------------------------------
\subsection{算術演算での非一貫性}
複合メッセージに関する規則はごく単純なものですが、算術演算が2項メッセージとして記述されている場合には伝統的な記法との非一貫性が問題となります。以下はいくつかよけいなかっこが必要となる例です。

\needlines{6}
\begin{code}{@TEST}
3 + 4 * 5      --> 35    "(23ではない)  2項メッセージが左から右に実行されるため"
3 + (4 * 5)    --> 23
1 + 1/3         --> (2/3)    "4/3ではない"
1 + (1/3)       --> (4/3)
1/3 + 2/3       --> (7/9)    "1ではない"
(1/3) + (2/3)  --> 1
\end{code}

\paragraph{例:}
式\ct{20 + 2 * 5}では\ct{+}と\ct{*}という2項メッセージだけが使われています。しかしながら、\st では\ct{+}と\ct{*}の間には特に優先順位の違いがありません。どちらも2項メッセージであるだけなので、\ct{*}は\ct{+}よりも優先順位が高い訳ではありません。\egref{binaryMessages1}のように、ここでは左端の\ct{+}がまず送信され(1)、その結果に\ct{*}が送信されます。

\begin{example}[binaryMessages1]{Decomposing \ct{20 + 2 * 5}}{}
"2項メッセージ同士には優先順位の違いがないので、算数の約束としては*がまず先に実行されるべきですが、ここでは左端の+がまず評価されます。"

      20 + 2 * 5 
(1)  20 + 2 --> 22
(2)  22       * 5 --> 110
\end{example}

\begin{figure}
\begin{center}\includegraphics[width=8cm]{ucompoNoBracketPar}\end{center}
\end{figure}
\noindent
\egref{binaryMessages1}の結果は、\ct{30}ではなく\ct{110}となります。これはもしかすると書かれた意図とは反しているかもしれませんが、メッセージ送信の規則をそのまま適用した結果であり、\st の簡素なモデルに付随する問題ではあります。正しい結果を得るためには、かっこを使う必要があります。かっこで囲まれたメッセージが先に評価されるため、メッセージ式\ct{20 + (2 * 5)}は\egref{mathcorrect}に示されたように評価され、正しい結果を返します。

\needlines{4}
\begin{example}[mathcorrect]{\ct{20 + (2 * 5)}の評価順の分解}{}
"かっこで囲まれたメッセージが先に評価されるので、*が+の前に実行され、正しく動作します。"

    20 + (2 * 5) 
(1)        (2 * 5) --> 10
(2) 20 + 10      --> 30
\end{example}

\begin{figure}
\begin{center}
\includegraphics[width=8cm]{ucompoNumberBracket}
\end{center}
\end{figure}

\important{\st では、+ や * のような算術演算にも特別な優先順位はありません。\ct{+}や\ct{*}も単なる2項メッセージであり、\ct{*}は\ct{+}に優先するわけではありません。意図通りの結果を得るためにはかっこを使ってください。}

%  At the beginning put parenthesis when you have multiple binary messages.}  HUH?  At the beginning of what?!

\begin{figure}
\begin{center}
\ifluluelse
	{\includegraphics[width=\textwidth]{uKeyUnBinPar}}
	{\includegraphics[width=0.8\textwidth]{uKeyUnBinPar}}
\ifluluelse
	{\includegraphics[width=\textwidth]{uunKeyBinPar}}
	{\includegraphics[width=10cm]{uunKeyBinPar}}
\end{center}
\caption{かっこの有無に関わらない等価なメッセージ式の\figlabel{uKeyUnBinPar}}
\end{figure}

最初の規則は単項メッセージは2項メッセージやキーワードメッセージに優先するということであったので、単項メッセージのまわりにはかっこを書く必要はないということに注目してください。\tabref{expressions}に普通に書いた式と、仮に優先順位規則が存在しなかったとしたらどのようにかっこをつけて等価な式とするかという対応を挙げました。この表では右も左も同じ実行順序となり、同じ結果を返します。

\begin{figure}\centering
	\begin{tabular}{l@{\qquad}l}
	\toprule
	規則に基づく評価順 & 明示的にかっこを書いた等価な式 \\
	\midrule
	\lct{aPen color: Color yellow}
		& \lct{aPen color: (Color yellow)}
		\\
	\lct{aPen go: 100 + 20}
		& \lct{aPen go: (100 + 20)}
		\\
	\lct{aPen penSize: aPen penSize + 2}
		& \lct{aPen penSize: ((aPen penSize) + 2)}
		\\
	\lct{2 factorial + 4}
		& \lct{(2 factorial) + 4}
		\\
	\bottomrule
	\end{tabular}
	\caption{メッセージ式とかっこを完全につけた等価な式の対応例\tablabel{expressions}}
\end{figure}

%=============================================================
\section{キーワードメッセージの切れ目を見つけるためのヒント}
初心者には、いつかっこを書かなくてはならないのかを理解することがしばしば問題となります。以下で、コンパイラがどのようにキーワードメッセージを認識するのかを紹介しましょう。

%-------------------------------------------------------------
\subsection{かっこかかっこなしか?}
構文的には、文字\ct{[}、\ct{]}、\ct{(}および\ct{)}が領域を区切るために使われます。
それぞれの領域の中では\ct{.}や\ct{;}で切られておらず、\ct{:}で終端されている語の列がキーワードメッセージを構成します。
もし、コロンで終端されている語が文字\ct{[}、\ct{]}、\ct{(}、\ct{)}で区切られた領域内にあるのであれば、それらの語はその領域の中で独立した別のメッセージ式を構成することになります。

以下の例では\ct{rotatedBy:magnify:smoothing:}と\ct{at:put:}という2つの独立したキーワードメッセージが使われています。

\begin{code}{}
aDict
   at: (rotatingForm 
          rotateBy: angle	
          magnify: 2 
          smoothing: 1)
   put: 3
\end{code}

\important{
以下のような区切り文字、\lct{[}、 \lct{]}、 \lct{(}および\lct{)}がコードの領域を区切るために使われます。
それぞれの領域の中では、\lct{:}で終端され、\lct{,}や\lct{;}で分断されていない語の最長
の列がキーワードメッセージとなっています。区切り文字\lct{[}、\lct{]}、\lct{(}および\lct{)}がコロンのついた語を囲んでいる場合は、それらの語の列がその領域内のキーワードメッセージになります。}

\on{Sounds terribly complicated.  yo-ちげーねーな}

\paragraph{ヒント:} もしまだ優先順位規則がよくわからないのであれば、まずはかっこをなるべくたくさんつけるようにして、同じ優先順位を持つメッセージ同士であっても確実に区別できるようにしてもよいかもしれません。

以下の例では\ct{x isNil}は単項でありキーワードメッセージ\ct{ifTrue:}よりも先に送られるため、かっこは本来必要ありません。
\begin{code}{}
(x isNil)
   ifTrue:[...]
\end{code}

以下の例では、\ct{includes:}と\ct{ifTrue:}はどちらもキーワードメッセージなので、かっこを使う必要があります。
\begin{code}{}
ord := OrderedCollection new.
(ord includes: $a)
   ifTrue:[...]
\end{code}
\noindent
かっこを付けなかったとしたら、\ct{includes:ifTrue:}という意味不明なメッセージがordコレクションに送られてしまいます!

%-------------------------------------------------------------
\subsection{\lct{[ ]}や\lct{( )}をいつ使うべきか}
もしかすると、大かっこ(鍵かっこ)をいつかっこの代わりに使うべきかをりかいするのにとまどっているかもしれませんね。
基本原則としては、もしある式が何回(もしかすると0回かも)評価されるかあらかじめ決定できない時には、その式を\ct{[ ]}で囲めばよいと言えます。
\lct{[\emph{expression}]}という式は、実行時に\ind{ブロック}クロージャ (\ie 、あるオブジェクト)を\lct{\emph{expression}}から生成します。このクロージャは必要に応じて何度でも(0回の可能性も含めて)評価することできます。\ct{[ ]}の中には、メッセージ式、変数、リテラル、代入、または別のブロックを書くことができます。

\ct{ifTrue:}や\ct{ifTrue:ifFalse:}のような条件分岐がブロックを必要とするのはこのためです。同じ原理に従い、\ct{whileTrue:}メッセージのレシーバーと引数も何度評価されるかはあらかじめ決定できないので大かっこが必要です。

一方、通常のかっこはメッセージ送信の順序のみに影響を与えます。
すなわち、\lct{(\emph{expression})}という式では\lct{\emph{expression}}の部分は\emph{常に}一度だけ評価されます。

\begin{code}{}
[ x isReady ] whileTrue: [ y doSomething ]   "レシーバーと引数の両方ともブロックでなくてはならない"
4 timesRepeat: [ Beeper beep ]                   "引数は2回以上実行されるので、ブロックでなくてはならない"
(x isReady) ifTrue: [ y doSomething ]           "引数は一度だけ評価されるのでブロックではない"
\end{code}

%=============================================================
\section{式の列(文)}
ピリオドで区切られた複数の式(\ie メッセージ送信や代入など)は順番に評価されます。
ただし、変数宣言とその直後の式の間にはピリオドは書かないということに注意してください。
最後の式の値が式の列全体の結果となります。
最後の式が以外の式が返す値は無視されます。ここでは、ピリオドは\subind{文}{の区切り}であり、終端子ではないということに注意してください。式の列全体の最後にはピリオドを付けても付けなくてもかまいません。

\begin{code}{@TEST}
| box |
box := 20@30 corner: 60@90.
box containsPoint: 40@50 --> true
\end{code}

%=============================================================
\section{カスケードメッセージ式}
\st には、複数のメッセージを同じ同じレシーバーに送信するためのセミコロン(\ct{;})を使った記法があります。これは\st 用語では\emphind{カスケード}と呼ばれています。

\important{Expression Msg1 ; Msg2}

\begin{minipage}{0.35\textwidth}
\begin{code}{}
Transcript show: 'Pharo is '.
Transcript show: 'fun '.
Transcript cr.
\end{code}
\end{minipage}
\emph{is equivalent to:}
\begin{minipage}{0.35\textwidth}
\begin{code}{}
Transcript        
   show: 'Pharo is';
   show: 'fun ';
   cr
\end{code}
\end{minipage}

カスケードメッセージを受け取っているオブジェクト自身が、メッセージ送信の結果として返される可能性があるということに注意してください。
さらには、すべてのカスケードメッセージのレシーバーは、一連のカスケードメッセージの中で最初のものを受け取ったオブジェクトが使われます。以下の例では、最初のカスケードメッセージは\ct{setX:setY:}です(後にカスケードが続いているため)。カスケードメッセージ\ct{setX:setY:}のレシーバーは\ct{Point new}によって新たに作られたPointオブジェクトであり、\ct{Point}ではありません。引き続き、同じレシーバーにメッセージ\ct{isZero}が送信されます。

\begin{code}{}
Point new setX: 25 setY: 35; isZero --> false
\end{code}

%=============================================================
\section{章のまとめ}

\begin{itemize}
\item メッセージは常に\emph{レシーバー}と呼ばれるオブジェクトに送信されます。レシーバーは他のメッセージ送信の結果であることもあります。

\item 単項メッセージは引数をとらないメッセージです。\\
その構文は\lct{receiver \textbf{selector}}です。

\item 2項メッセージはレシーバーと引数という2つのオブジェクトが関与し、\emph{かつ}そのセレクターが文字集合\ct{+}、\ct{-}、\ct{*}、\ct{/}、\ct{|}、\texttt{\&}、\ct{=}、\ct{>}、\ct{<}、\texttt{\~}および\ct{@}の要素の列からなるメッセージです。その構文は\lct{receiver \textbf{selector} argument}です。

\item キーワードメッセージは2つ以上のオブジェクトが関与し、そのセレクターにコロン(\ct{:})が使われているようなメッセージです。\\
その構文は
\lct{receiver \textbf{selectorWordOne:} argumentOne \textbf{wordTwo:} argumentTwo}です。

\item \textbf{規則1:} 単項メッセージがまず送信され、次に2項メッセージ、そして最後にキーワードメッセージが送信されます。
\item \textbf{規則2:} かっこで囲まれたメッセージが他のメッセージよりも先に送信されます。
\item \textbf{規則3:} 同じ種類のメッセージが使われている場合は、評価左から右という順序で行われます。
\item \st では、\ct{+}や\ct{*}のような一般の算術演算もすべて同じ優先順位となっています。\ct{+}や\ct{*}は普通の2項メッセージなので、\ct{*}が\ct{+}よりも高い優先順位を持つということはありません。異なる結果を求めたい場合はかっこを使用してください。
\end{itemize}

%\end{document}
% ON: Don't ever put an \end{document} in a chapter
% It will make the book stop there!
%=================================================================
\ifx\wholebook\relax\else\end{document}\fi
%=================================================================

%=================================================================
%%% Local Variables:
%%% coding: utf-8
%%% mode: latex
%%% TeX-master: t
%%% TeX-PDF-mode: t
%%% ispell-local-dictionary: "english"
%%% End:

%---------------------------------------------------------
