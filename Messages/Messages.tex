% $Author$
% $Date$
% $Revision$

% HISTORY:
% 2007-06-06 - Stef started
% 2007-08-21 - Oscar edit
% 2007-09-06 - Lukas corrections
% 2007-09-11 - Orla corrections

%=================================================================
\ifx\wholebook\relax\else
% --------------------------------------------
% Lulu:
	\documentclass[a4paper,10pt,twoside]{book}
	\usepackage[
		papersize={6.13in,9.21in},
		hmargin={.75in,.75in},
		vmargin={.75in,1in},
		ignoreheadfoot
	]{geometry}
	\input{../common.tex}
	\pagestyle{headings}
	\setboolean{lulu}{true}
% --------------------------------------------
% A4:
%	\documentclass[a4paper,11pt,twoside]{book}
%	\input{../common.tex}
%	\usepackage{a4wide}
% --------------------------------------------
    \graphicspath{{figures/} {../figures/}}
	\begin{document}
	\renewcommand{\nnbb}[2]{} % Disable editorial comments
	\sloppy
\fi
%=================================================================
\chapter{メッセージ構文を理解しよう}
\chalabel{understanding}

\st のメッセージ構文はとても単純ですが、標準的なものではないので慣れるのに少し時間がかかるでしょう。
この章では、この特殊なメッセージ構文に順応するためのガイダンスを提供します。
もし既にこの構文に慣れているのなら、この章は飛ばしても構いませんし後で読んでも良いでしょう。

%=============================================================
\section{メッセージを読み取る}

\st では、\charef{syntax} で紹介した構文要素 (\ct + := ^ . ; # () {} [ : | ]+)を除けばすべてがメッセージ送信です。
\ind{C++} のように、\ct{+} などの演算子を自作のクラス用に定義することもできますが、すべての演算子の優先順位は同じです。
さらに、あるメソッドが受け取る引数の個数は変えられません。例えば『\ct{-}』は、常に二項演算子です; 『\ct{-}』をオーバーロードして単項演算子にすることはできません。

\st では、メッセージがどのような順番で送信されるかは、メッセージの種類によって決定されます。そしてメッセージには 3 種類しかありません: \emphsubind{メッセージ}{単項}、\emphsubind{メッセージ}{二項}および\emphsubind{メッセージ}{キーワード}メッセージです。単項メッセージの優先順位が最高で、その次が二項メッセージ、そして最後がキーワードメッセージです。他の多くの言語と同様に \ind{括弧}を使って評価順を変えることもできます。これらの規則によって \st のコードは、可能な限り読みやすく、かつ規則のことで頭を悩ませなくても良いようになっています。

\st のほとんどの計算はメッセージパッシングによって行われるので、メッセージを正しく読み取ることが非常に重要です。次の用語と概念を理解することが役に立つでしょう:

\begin{itemize}
  \item メッセージは、\emphsubind{メッセージ}{セレクタ}およびオプションのメッセージ引数からなります。
  \item メッセージは、\emphsubind{メッセージ}{レシーバ}に送信されます。
  \item レシーバとメッセージの組み合わせが、\emph{メッセージ} \emphsubind{メッセージ}{送信}と呼ばれます(\figref{firstScriptMessage})。
%  \item We call expression a message send, a variable, an assignment, a literal or a block. 
\end{itemize}

\begin{figure}[htb]
\begin{minipage}{0.53\textwidth}
	\begin{center}
	\includegraphics[width=0.95\textwidth]{message}
	\caption{レシーバ、セレクタ、引数からなるメッセージ送信の例(二つ)。\figlabel{firstScriptMessage}}\end{center}
\end{minipage}
\hfill
\begin{minipage}{0.43\textwidth}
	\begin{center}
	\ifluluelse
		{\includegraphics[width=0.9\textwidth]{uKeyUnOne}}
		{\includegraphics[width=6cm]{uKeyUnOne}}
	\caption{\ct{aMorph color: Color yellow} は、二つのメッセージ送信の組み合わせ: \ct{Color yellow} および \ct{aMorph color: Color yellow}。\figlabel{ellipse}}
	\end{center}
\end{minipage}
\end{figure}

%\begin{figure}[ht]
%\begin{center}
%\includegraphics[width=0.5\textwidth]{message}
%\caption{Two messages composed of a receiver, a method selector, and a set of arguments.\figlabel{firstScriptMessage}}\end{center}
%\end{figure}

\important{メッセージは、常にレシーバに送信されます。レシーバは、リテラル・ブロック・変数・他のメッセージを評価した結果、のいずれかです。}
%sd-ま、代入を考えてないからかんぜんただしいというわけではないがな

\figref{ellipse} では、メッセージのレシーバが同定しやすくなるようにレシーバに下線を引きました。
また、それぞれのメッセージ送信を楕円で囲み、送信される順番に従って番号を付けました。

%\begin{figure}[!ht]
%\begin{center}
%\includegraphics[width=6cm]{uKeyUnOne}
%\end{center}
%\caption{\ct{aMorph color: Color yellow} is composed of two expressions: \ct{Color yellow} and \ct{aMorph color: Color yellow}.\figlabel{ellipse}}
%\end{figure}

\figref{ellipse} は、\ct{Color yellow}と \ct{aMorph color: Color yellow} という二つのメッセージ送信を表しています。したがって楕円も二つあります。メッセージ送信 \ct{Color yellow} がまず実行されるので、それを囲む楕円に番号 \ct{1} が付いています。全体ではレシーバが二つあります: 一つは \ct{aMorph} でメッセージ \ct{color: ...} が送られており、もう一つは \ct{Color} でメッセージ \ct{yellow} が送られています。レシーバにはそれぞれ、既に述べたように下線が引いてあります。

レシーバは、メッセージ送信の最初の要素であることが多いです。例えば、\ct{100 + 200} の \ct{100} や \ct{Color yellow} の \ct{Color} がそうです。しかし、他のメッセージ送信の結果がレシーバとして使われることも良くあります。例えば \ct{Pen new go: 100} というメッセージ送信では、メッセージ \ct{go: 100} のレシーバは \ct{Pen new} オブジェクト(\ct{Pen new} の結果として返されるオブジェクト)です。いずれにせよメッセージは、\emph{レシーバ} と呼ばれるオブジェクトに送信され、レシーバは他のメッセージ送信の結果であっても良い、ということです。

\begin{table}\centering
	\begin{tabularx}{\linewidth}{llX}
		\toprule
		メッセージ送信 & メッセージの種類 & 結果 \\
		\midrule
		\lct{Color yellow}
			& 単項
			& 色を表すオブジェクト。
		\\
		\lct{aPen  go: 100}
			& キーワード
			& レシーバであるペン・オブジェクトが 100 ピクセル移動。
		\\
		\lct{100 + 20}
			& 二項
			& 数値オブジェクト 100 に、メッセージ + を、数値 20 とともに送信。
		\\
		\lct{Browser open}
			& 単項
			& 新しいブラウザを開く。
		\\
		\lct{Pen new  go: 100}
			& 単項およびキーワード
			& ペン・オブジェクトが生成され、100 ピクセル移動する。
		\\
		\lct{aPen go: 100 + 20}
			& キーワードおよび二項
			& レシーバであるペン・オブジェクトが 120 ピクセル移動。
		\\
		\bottomrule
	\end{tabularx}
	\caption{メッセージ送信と種類の例}\tablabel{messageExamples}
\end{table}

\tabref{messageExamples}にいくつかのメッセージ送信の例を挙げました。
必ずしもすべてのメッセージ送信に引数があるわけではない、ということに注意してください。\ct{open} のような単項メッセージには引数はありません。\ct{go: 100} のような単独のキーワードメッセージや \ct{+ 20} のような二項メッセージは、引数を一つとります。
単純なメッセージ・組み合わされたメッセージ、という分類もできます。\ct{Color yellow} や \ct{100 + 20} は単純です: 一つのメッセージが一つのオブジェクトに送信されるだけです。一方 \ct{aPen go: 100 + 20}は二つのメッセージ送信が組み合わさったものです: \ct{+ 20} が \ct{100} に送信され、その結果を引数として \ct{go:} が \ct{aPen} に送信されます。
オブジェクトを返す式(代入、メッセージ送信およびリテラル)であればレシーバになることができます。\ct{Pen new go: 100}では、メッセージ \ct{go: 100}が \ct{Pen new} オブジェクトに送信されます。


%=============================================================
\section{3種類のメッセージ}

\st では、メッセージの送信順序を決定するいくつかの単純な規則が定義されています。これらの規則は、以下の 3 種類のメッセージの別に基づきます:
\begin{itemize}
\item \emph{単項メッセージ}は、あるオブジェクトに他の追加情報無しで送信されるメッセージです。 例えば \ct{3 factorial} の \ct{factorial} は、単項メッセージです。
\item  \emph{二項メッセージ}は、演算子(しばしば算術的な)からなるメッセージです。これらは、常にレシーバと引数という二つのオブジェクトが関与するため、二項メッセージと呼ばれます。\ct{10 + 20} の場合は、\ct{+} が二項メッセージで引数 \ct{20} と共にレシーバ \ct{10} に送信されています。
\item  \emph{キーワードメッセージ}は、一つあるいはそれ以上のキーワードからなっています。それぞれのキーワードの最後にはコロン (\ct{:}) がついており、キーワードごとに一つ引数を取ります。
例えば \ct{anArray at: 1 put: 10} の場合、キーワード \ct{at:} は \ct{1} という引数を取り、キーワード \ct{put:} が引数 \ct{10} を取っています。
\end{itemize}

%-------------------------------------------------------------
\subsection{単項メッセージ}
単項メッセージは引数を必要としないメッセージです。その構文は、\ct{receiver selector} という形式です。セレクタ (\ct{selector}) は、\ct{:} を含まない単純な文字列です (\eg \ct{factorial}、\ct{open}、\ct{class})。
\needlines{4}
\begin{code}{@TEST}
89 sin           --> 0.860069405812453
3 sqrt           --> 1.732050807568877
Float pi         --> 3.141592653589793
'blop' size     --> 4
true not        --> false
Object class --> Object class  "Object のクラスは Object class (BANG)"
\end{code}
% ON: I changed the examples to things we can test

\important{単項メッセージは引数を必要とないメッセージです。\\
その構文は、\lct{receiver \textbf{selector}}という形式です。}

%-------------------------------------------------------------
\subsection{二項メッセージ} 
二項メッセージは引数を一つだけ取ります。\emph{また}二項メッセージのセレクタは、次の文字セットからの 1 文字以上の繰り返しでなければなりません: \ct{+}、\ct{-}、\ct{*}、\ct{/}、\ct{\&}、\ct{=}、\ct{>}、\ct{|}、\ct{<}、\ct{\~}、および\ct{@}。構文解析時に問題が起こるので、\ct{--} は不正なセレクタであるということに注意してください。

\begin{code}{@TEST}
100@100      --> 100@100  "Point オブジェクトを生成"
3 + 4              --> 7
10 - 1            --> 9
4 <= 3            --> false
(4/3) * 3 = 4   --> true  "同値性のテストも、\st では単なる二項メッセージ。分数オブジェクトは値を正確に表現できる"
(3/4) == (3/4) --> false  "別個に作られた二つの分数は、同じオブジェクトではない"
\end{code}

\important{二項メッセージは引数を一つだけ取ります。\emph{また}二項メッセージのセレクタは、次の文字セットからの 1 文字以上の繰り返しでなければなりません: \ct{+}、\ct{-}、\ct{*}、\ct{/}、\ct{\&}、\ct{=}、\ct{>}、\ct{|}、\ct{<}、\ct{\~}、および\ct{@}。\ct{--}は許されません。 \\
二項メッセージの構文は、\lct{receiver \textbf{selector} argument} という形式です。}

%-------------------------------------------------------------
\subsection{キーワードメッセージ}

キーワードメッセージは一つ以上の引数を取るメッセージです。そのセレクタは、それぞれが \ct{:} で終わる一つ以上のキーワードを連結したものです。キーワードメッセージの構文は次の形式です:
\lct{receiver \textbf{selectorWordOne:} argument\-One \textbf{wordTwo:} argumentTwo}

それぞれのキーワードが引数を一つずつとります。すなわち \ct{r:g:b:}というセレクタは 3 引数であり、\ct{playFileNamed:} や \ct{at:} は 1 引数、\ct{at:put:} は 2 引数ということになります。\ct{Color} クラスのインスタンスを作るには、クラスに \ct{r:g:b:} メッセージを送ります(例えば \ct{Color r: 1 g: 0 b: 0} によって赤色を表すオブジェクトを作ることができます)。ここでは、コロンもセレクタの一部であるということに注意してください。

\important{\ind{Java} や \ind{C++} であれば、\st における \ct{Color r: 1 g: 0 b: 0} というメッセージ送信は、
\ct{Color.rgb(1,0,0)}と書かれることになるでしょう。}

\begin{code}{@TEST | nums |}
1 to: 10                        --> (1 to: 10)  "区間オブジェクトを生成"
Color r: 1 g: 0 b: 0       --> Color red  "色オブジェクトを生成"
12 between: 8 and: 15 --> true

nums := Array newFrom: (1 to: 5).
nums at: 1 put: 6.
nums --> #(6 2 3 4 5)
\end{code}
% ON: Changed to real examples that we can test

\important{キーワードメッセージは一つ以上の引数を取るメッセージです。そのセレクタは、それぞれがコロン (\lct{:}) で終わる一つ以上のキーワードを連結したものです。構文は次の形式です:\\
\lct{receiver \textbf{selectorWordOne:} argumentOne \textbf{wordTwo:} argumentTwo}}

%=============================================================
\section{メッセージ式の組み合わせ}
これまでに述べた 3 種類のメッセージにはそれぞれ異なる優先順位があるので、これらをエレガントに組み合わせて使うことができます。

\begin{enumerate}
\item 単項メッセージがまず最初に送信され、次に二項、そして最後にキーワードメッセージが送信されます。
\item \ind{括弧}で囲まれたメッセージは、他のメッセージよりも優先して送信されます。
\item 同じ種類のメッセージは、左から右に評価されます。
\end{enumerate}
\index{message!evaluation order}

この規則により、\st のプログラムは、極めて自然に読み進められるようになっています。メッセージが意図した通りの順番で送信されることを確実にしたいのであれば、\figref{uKeyUn}にあるように、括弧を足していくこともできます。この図では、メッセージ \ct{yellow} が単項メッセージであり、\ct{color:} はキーワードメッセージであるため、\ct{Color yellow} がまず先に送信されます。しかし、括弧で囲まれたメッセージがまず送信されるため、(冗長な)括弧を \ct{Color yellow} の周りに書くことにより、これがまず先に送信されるということを強調することができます。節の残りでは、これらの点について説明します。

\begin{figure}[ht]
\ifluluelse
	{\centerline{\includegraphics[width=0.9\textwidth]{uKeyUn}} }
	{\centerline{\includegraphics[width=10cm]{uKeyUn}} }
\caption{単項メッセージが最初に送信されるので、\ct{Color yellow} がまず送信され、返された色オブジェクトがメッセージ \ct{aPen color:} の引数として渡される。\figlabel{uKeyUn}}
\end{figure}

%---------------------------------------------------------
\subsection*{単項 > 二項 > キーワード}
単項メッセージがまず送信され、次に二項、そしてキーワードメッセージが最後に送信されます。単項メッセージの優先順位が他の種類のメッセージよりも高い、とも言います。

\important{\textbf{規則 1:} 単項メッセージがまず送信され、次に二項、そしてキーワードメッセージが最後に送信されます。\\
\centerline{単項 \ct{>} 二項 \ct{>} キーワード}
}

以下の例が示すように、\st の構文では、多くの場合、メッセージ式の組み合わせも自然に読むことができます。
\begin{code}{@TEST}
1000 factorial / 999 factorial --> 1000
2 raisedTo: 1 + 3 factorial     --> 128
\end{code}
\noindent

ただし残念ながら、算術演算の場合にはこの規則はやや単純過ぎます。そのため、二項演算子を複数組み合わせるときは、場合によって括弧を使って評価順を強制する必要があります。
\begin{code}{@TEST}
1 + 2 * 3   --> 9
1 + (2 * 3) --> 7
\end{code}

以下のさらに複雑な(!) 例は、込み入った \st の式も自然に読めるということの良い例になっています。
\begin{code}{@TEST}
[:aClass | aClass methodDict keys select: [:aMethod | (aClass>>aMethod) isAbstract ]] value: Boolean --> an IdentitySet(#or: #| #and: #& #ifTrue: #ifTrue:ifFalse: #ifFalse: #not #ifFalse:ifTrue:)
\end{code}
\noindent
ここでは、\ct{Boolean} クラスのどのメソッドが抽象メソッドであるかを調べようとしています\footnote{実を言うと同等の式は、さらに簡単に \ct{Boolean methodDict select: #isAbstract thenCollect: #selector} と書くこともできます。}\damien{I've added this footnote, just remove it if you don't like it :-)}。
このブロックは、引数として渡されたクラス \ct{aClass} に、そのメソッド辞書のキー集合を問い合わせ、その中から抽象メソッドを選択(\ct{select:})するようになっています。
ここでは、\ct{aClass} に、\ct{Boolean} がバインドされます。
この式の中で括弧は、二項メッセージ \ct{>>} を単項メッセージ \mbox{\ct{isAbstract}} より先に送信するためだけに必要でした(\ct{>>} は、メソッドをクラスから取り出す二項演算子です)。全体を評価した結果は、メソッド名の集合です。これらのメソッドは、\ct{Boolean} の具象サブクラス (\ct{True} と \ct{False}) で実装しなくてはなりません。

%\begin{code}{}
%Pen new go: 30 + 50          "create a turtle and moves it forward 80 pixels"
%Display restoreAfter: [WarpBlt test4] 					
%	"Keyword message, try test1, test12, test3, test4 and test 5"
%#($t $e $s $t) at: 3 --> $s 
%#($a $b $c $d) at: 2 factorial put: $z 
%\end{code}

%As you can see the syntax and in particular the keyword messages as in
%the example \ct{array at: 1 put: 4} make it possible to write code
%with a structure approaching that of natural language.
% This was one of the initial objectives so that the children can program.

\paragraph{例。}
メッセージ \ct{aPen color: Color yellow} の中には、\ct{Color} に送信される\emph{単項}メッセージ \ct{yellow}と、\ct{aPen} に送信される \emph{キーワード}メッセージ \ct{color:} があります。単項メッセージがまず送信されるので、\egref{decColor} にあるように、\ct{Color yellow} がまず送信(評価)され (1)、その結果として返された色オブジェクトがメッセージ \ct{aPen color: aColor} の引数 (\ct{aColor}) となります (2)。
\figref{uKeyUn} は、これらのメッセージがどのように送信されるかをグラフィカルに表しています。

\needlines{5}
\begin{example}[decColor]{\ct{aPen color: Color yellow} の評価順を分解する}{}
        aPen color: Color yellow
(1)                       Color yellow        "単項メッセージがまず送信される"
                        --> aColor
(2)   aPen color: aColor                 "キーワードメッセージが次に送信される"
\end{example}

\paragraph{例。} メッセージ \ct{aPen go: 100 + 20} の中には、\emph{二項}メッセージ \ct{+ 20} と\emph{キーワード}メッセージ \ct{go:} があります。二項メッセージはキーワードメッセージより先に送信されるので、\ct{100 + 20} がまず評価されます (1): \ct{+ 20} がオブジェクト \ct{100} に送信され、数値 \ct{120} が返ります。次に、\ct{aPen} にメッセージ \ct{go: 120} が送信されます。\ct{120} は引数です (2)。
\Egref{decGo}に、このメッセージ送信がどのように実行されるかを示しました。

\begin{example}[decGo]{\ct{aPen go: 100 + 20} を分解する}{}
      aPen go: 100 + 20   
(1)                 100 + 20           "二項メッセージがまず送信される"
                   -->   120
(2)  aPen go: 120                   "キーワードメッセージが次に送信される"
\end{example}

\begin{figure}[htb]
\begin{minipage}{0.48\textwidth}
	\ifluluelse
		{\centerline{\includegraphics[width=0.9\textwidth]{uKeyBin}}}
		{\centerline{\includegraphics[width=6cm]{uKeyBin}}}
	\caption{二項メッセージはキーワードメッセージより先に送信される。\figlabel{uKeyBin}}
\end{minipage}
\hfill
\begin{minipage}{0.48\textwidth}
	\begin{center}
	\ifluluelse
		{\includegraphics[width=0.9\textwidth]{uunKeyBin}}
		{\includegraphics[width=6cm]{uunKeyBin}}
\caption{\ct{Pen new go: 100 + 20} を分解する}\figlabel{unKeyBin}
\end{center}
\end{minipage}
\end{figure}

%\begin{figure}[ht]
%\centerline{\includegraphics[width=6cm]{uKeyBin}} 
%\caption{Unary messages are sent first so \ct{Color yellow} is sent. This returns a color object which is passed as argument of the message \ct{aPen color:}.\figlabel{uKeyBin}}
%\end{figure}

%\paragraph{Example 3.}
%The message \ct{aPen penSize: aPen penSize + 2} contains one unary message \ct{penSize}, one binary message \ct{+},  and one keyword message \ct{penSize:}.
%The unary message \ct{aPen penSize} is sent first (1), this message returns a number representing the current size of the receiver pen. Then the binary message is sent (2), the returned number is sent the message \ct{+ 2} which in its turn returns another number. Finally the keyword message 
%\ct{penSize:} is sent with the last number as argument. The expression increases the receiver pen size by two pixels. \Egref{decpen} illustrates the decomposition of message evaluation.

% penSize does not exist on Pen
%\

%\begin{scriptfigwithsize}[0.65]{\includegraphics[width=6cm]{uKeyUnBin}}{Decomposing \ct{ aPen penSize: aPen penSize + 2}}\scrlabel{decpen}
%      aPen penSize: aPen penSize + 2
%(1)                        aPen penSize            "unary"
%                              $\arrow$  aNumber
%(2)                              aNumber + 2	             "binary"
%                                    $\arrow$   anotherNumber	
%(3)   aPen penSize: anotherNumber           "keyword"      
%\end{scriptfigwithsize}

\paragraph{例。} 練習として、メッセージ \ct{Pen new go: 100 + 20} の評価順を分解してみてください。このメッセージには、単項メッセージが一つ、キーワードメッセージが一つ、二項メッセージが一つあります (\figref{unKeyBin})。
% The unary message \ct{Pen new} is first sent. It returns a new bot, then the binary message \ct{100 + 20} is sent and returns \ct{120}. Finally the message \ct{go:} is sent to the newly created robot with \ct{120}.

%\begin{figure}[ht]
%\begin{center}
%\includegraphics[width=8cm]{uunKeyBin}
%\caption{Decomposing \ct{Pen new go: 100 + 20}}\figlabel{unKeyBin}
%\end{center}
%\end{figure}

%-------------------------------------------------------------
\subsection{括弧優先}

\important{\textbf{規則 2。} 括弧で囲まれたメッセージは、他のメッセージよりも優先して送信されます。\\
\centerline{\lct{(任意のメッセージ) > 単項 > 二項 > キーワード}}}

\begin{code}{@TEST}
1.5 tan rounded asString = (((1.5 tan) rounded) asString) --> true    "括弧は不要"
3 + 4 factorial   --> 27    "(5040 ではない)"
(3 + 4) factorial --> 5040
\end{code}

以下の例では、\ct{lowMajorScaleOn:} を \ct{play} よりも先に送信するために、\ind{括弧}が使われています。
\begin{code}{}
(FMSound lowMajorScaleOn: FMSound clarinet) play 
"(1) メッセージ clarinet が FMSound クラスに送信され、クラリネットの音色オブジェクトが作られる。
 (2) この音色オブジェクトが lowMajorScaleOn: キーワードメッセージの引数として FMSound に送信される。
 (3) 結果の音オブジェクトに play が送信され、演奏される。"
\end{code}

% ON: This has nothing to do with parentheses!
%RecordingControlsMorph new openInWorld
%"An instance of the digitizer is created then visualized. If your microphone is plugged in try a sampleBANG"

% ON: This link is broken, and the result does not understand display!
%(HTTPSocket httpShowGif:
%   'www.altavista.digital.com/av/pix/default/av-adv.gif') display

\paragraph{例。}
メッセージ \ct{(65@325 extent: 134@100) center} は、左上が座標 $(65, 325)$ で大きさが $134{\times}100$ であるような長方形の、中央の座標を返します。\Egref{decExtent} は、このメッセージがどのように分解されて送信されるかを示しています。まず、括弧で囲まれたメッセージが送信されます: 括弧の内側には、二つの二項メッセージ \ct{65@325} と \ct{134@100} があるので、これらがまず送信(評価)されて、二つの点オブジェクトが返ります。次に、最初の点オブジェクトにキーワードメッセージ \ct{extent:} が送信され、長方形オブジェクトが返ります。最後に、単項メッセージ \ct{center} がその長方形オブジェクトに送信され、点オブジェクトが返ります。
もし括弧が無かったとすると、数値 \ct{100} はメッセージ \ct{center} を理解できないため、評価時にエラーとなります。

\needlines{9}
\begin{example}[decExtent]{括弧の例。}{}
      (65@325 extent: 134@100) center
(1)   65@325                                                    "二項"
    --> aPoint
(2)                                134@100                     "二項"
                                 --> anotherPoint
(3)   aPoint extent: anotherPoint                       "キーワード"
      --> aRectangle
(4)   aRectangle center                                     "単項"
      --> 132@375
\end{example}

\subsection{左から右への評価}
ここまでに、異なる種類のメッセージがどのような優先順位で処理されるかを学びました。残る疑問は、同じ優先順位のメッセージがどのような順番で送信されるかということです。これらは、左から右に送信されます。この振る舞いは実は、\egref{decExtent} の中で確認しています。\egref{decExtent} では、二つの点オブジェクトの生成メッセージ (\ct{@}) が先に送信されていました。

\important{{\textbf{規則 3。} 同じ種類のメッセージがある場合は、評価は左から右の順で行われる。}}

%\begin{figure}
%\centerline{\includegraphics[width=8cm]{ucompoUn}} 
%\caption{The message \ct{Pen new east} is composed of two unary messages. Therefore the leftmost one, \ct{new},  is sent and it returns a new robot to which the second message \ct{east} is sent. \figlabel{compoUn}}
%\end{figure}

\paragraph{例。} メッセージ送信 \ct{Pen new down} では、単項メッセージのみが使われているので、左端の \ct{Pen new} がまず送信されます。その結果は新しいペン・オブジェクトで、このペン・オブジェクトに 2 番目のメッセージ \ct{down} が送信されます (\figref{unaryMessages})。

\begin{figure}
	\centering
	\includegraphics[width=8cm]{ucompoUn}
	\caption{\ct{Pen new down} を分解する\figlabel{unaryMessages}}
\end{figure}

%-------------------------------------------------------------
\subsection{伝統的な算術記法との食い違いに関する問題}
メッセージ式の組み合わせに関する規則はごく単純なものですが、算術演算が二項メッセージとして記述されている場合には、伝統的な記法との食い違いが問題になります。以下は、いくつか余計な括弧が必要となる例です。

\needlines{6}
\begin{code}{@TEST}
3 + 4 * 5      --> 35    "(23 ではない)  二項メッセージが左から右に実行されるため"
3 + (4 * 5)    --> 23
1 + 1/3         --> (2/3)    "4/3 ではない"
1 + (1/3)       --> (4/3)
1/3 + 2/3       --> (7/9)    "1 ではない"
(1/3) + (2/3)  --> 1
\end{code}

\paragraph{例。}
メッセージ送信 \ct{20 + 2 * 5} では、\ct{+} と \ct{*} という二項メッセージだけが使われています。しかし、\st では \ct{+} と \ct{*} の間に特に優先順位の違いはありません。どちらも単なる二項メッセージであって、\ct{*} は \ct{+} よりも優先順位が高いわけではありません。\egref{binaryMessages1} のように、ここでは左端の \ct{+} がまず送信され (1)、その結果に \ct{*} が送信されます (2)。

\begin{example}[binaryMessages1]{\ct{20 + 2 * 5} を分解する}{}
"二項メッセージ間に優先順位の違いが無いので、算数の約束としては * がまず先に送信されるべきだが、ここでは左端の + がまず評価される。"

      20 + 2 * 5 
(1)  20 + 2 --> 22
(2)  22       * 5 --> 110
\end{example}

\begin{figure}
\begin{center}\includegraphics[width=8cm]{ucompoNoBracketPar}\end{center}
\end{figure}
\noindent
\egref{binaryMessages1}の結果は、\ct{30} ではなく \ct{110} となります。これはもしかすると、意図したものと違うかも知れませんが、メッセージ送信の規則をそのまま適用した結果であり、\st の簡素なモデルの代償と言えるかも知れません。正しい(意図した)結果を得るためには、括弧を使う必要があります。括弧で囲まれたメッセージが先に評価されるため、メッセージ送信 \ct{20 + (2 * 5)} は、\egref{mathcorrect} に示されたような結果を返します。

\needlines{4}
\begin{example}[mathcorrect]{\ct{20 + (2 * 5)} を分解する}{}
"括弧で囲まれたメッセージが先に評価されるので、* が + の前に送信される。これは正しい振る舞いである。"

    20 + (2 * 5)
(1)        (2 * 5) --> 10
(2) 20 + 10      --> 30
\end{example}

\begin{figure}
\begin{center}
\includegraphics[width=8cm]{ucompoNumberBracket}
\end{center}
\end{figure}

\important{\st では、\ct{+} や \ct{*} のような算術演算にも特別な優先順位はありません。\ct{+} や \ct{* }も単なる二項メッセージであり、\ct{*} は \ct{+} に優先するわけではありません。意図通りの結果を得るためには括弧を使ってください。}

%  At the beginning put parenthesis when you have multiple binary messages.}  HUH?  At the beginning of what?!

\begin{figure}
\begin{center}
\ifluluelse
	{\includegraphics[width=\textwidth]{uKeyUnBinPar}}
	{\includegraphics[width=0.8\textwidth]{uKeyUnBinPar}}
\ifluluelse
	{\includegraphics[width=\textwidth]{uunKeyBinPar}}
	{\includegraphics[width=10cm]{uunKeyBinPar}}
\end{center}
\caption{括弧の有無に関わらない等価なメッセージ。\figlabel{uKeyUnBinPar}}
\end{figure}

最初の規則は、単項メッセージは二項メッセージやキーワードメッセージに優先する、というものでした。だから、単項メッセージの周りに括弧を書く必要はありません。\tabref{expressions} に、普通に書いたメッセージ送信と、優先順位規則を無視して明示的に括弧を付けて書いた等価なメッセージ送信を並べました。この表では、左右どちらのメッセージ送信も同じ実行順序となり、同じ結果を返します。

\begin{figure}\centering
	\begin{tabular}{l@{\qquad}l}
	\toprule
	規則に基づく評価順 & 明示的に括弧を付けて書いた等価なメッセージ送信 \\
	\midrule
	\lct{aPen color: Color yellow}
		& \lct{aPen color: (Color yellow)}
		\\
	\lct{aPen go: 100 + 20}
		& \lct{aPen go: (100 + 20)}
		\\
	\lct{aPen penSize: aPen penSize + 2}
		& \lct{aPen penSize: ((aPen penSize) + 2)}
		\\
	\lct{2 factorial + 4}
		& \lct{(2 factorial) + 4}
		\\
	\bottomrule
	\end{tabular}
	\caption{メッセージ送信と、括弧を完全につけた等価な式の対応例\tablabel{expressions}}
\end{figure}

%=============================================================
\section{キーワードメッセージの切れ目を見つけるためのヒント}
初心者にとってしばしば問題なのは、いつ括弧が必要なのかがわかりにくいことです。そこで、コンパイラがどのようにキーワードメッセージを認識するのかを見てみましょう。

%-------------------------------------------------------------
\subsection{括弧か括弧無しか?}
文字 \ct{[}、\ct{]}、\ct{(} および \ct{)} がコードの領域を区切るために使われます。
それぞれの領域の中では、\ct{:} で終端され \ct{.} や \ct{;} で分断されていない語の並びのうち最長のものが、キーワードメッセージになります。
\ct{[}、\ct{]}、\ct{(} および \ct{)} がコロンのついた語を囲んでいる場合は、それらの語の並びが、(囲まれた)領域内のキーワードメッセージになります。

以下の例では \ct{rotatedBy:magnify:smoothing:} と \ct{at:put:} という二つの独立したキーワードメッセージが使われています。

\begin{code}{}
aDict
   at: (rotatingForm 
          rotateBy: angle	
          magnify: 2 
          smoothing: 1)
   put: 3
\end{code}

\important{
文字 \lct{[}、\lct{]}、\lct{(} および \lct{)} がコードの領域を区切るために使われます。
それぞれの領域の中では、\lct{:} で終端され \lct{.} や \lct{;} で分断されていない語の並びのうち最長のものが、キーワードメッセージになります。
\lct{[}、\lct{]}、\lct{(} および \lct{)} がコロンのついた語を囲んでいる場合は、それらの語の並びが、領域内のキーワードメッセージになります。}

\on{Sounds terribly complicated.  yo-ちげーねーな}

\paragraph{ヒント。} もしまだ優先順位規則が良く分からないのであれば、まずは括弧をなるべく沢山付けるようにして、同じ優先順位を持つメッセージ同士であっても確実に区別できるようにしても構いません。

以下の例では、\ct{x isNil} は単項メッセージでありキーワードメッセージ \ct{ifTrue:} よりも先に送信されるため、括弧は本来必要ありません。
\begin{code}{}
(x isNil)
   ifTrue:[...]
\end{code}

以下の例では、\ct{includes:} と \ct{ifTrue:} はどちらもキーワードメッセージなので、括弧を使う必要があります。
\begin{code}{}
ord := OrderedCollection new.
(ord includes: $a)
   ifTrue:[...]
\end{code}
\noindent
括弧を付けなかったとしたら、\ct{includes:ifTrue:} という意味不明なメッセージがコレクション (ord) に送信されてしまいます!

%-------------------------------------------------------------
\subsection{\lct{[ ]} や \lct{( )} をいつ使うべきか}
もしかすると、いつ鍵括弧を括弧の代わりに使うべきか、良く分からないかもしれませんね。
基本原則としては、もしある式が何回(もしかすると 0 回かも)評価されるかあらかじめ分からない時には、その式を \ct{[ ]} で囲みます。
\lct{[\emph{expression}]} は、実行時に \ind{ブロック} クロージャ (\ie オブジェクト)を \lct{\emph{expression}} から生成します。このクロージャは、必要に応じて何回でも (0 回の可能性も含めて)評価することできます。\ct{[ ]} の中には、メッセージ送信、変数、リテラル、代入、または別のブロックを書くことができます。

\ct{ifTrue:} や \ct{ifTrue:ifFalse:} のような条件分岐がブロックを必要とするのはこのためです。同じ原則に従い、\ct{whileTrue:} メッセージのレシーバと引数も、何回評価されるかはあらかじめ決定できないので、これらを鍵括弧で囲む必要があります。

一方通常の括弧は、メッセージ送信の順序のみに影響を与えます。
すなわち、\lct{(\emph{expression})} という式では \lct{\emph{expression}} の部分は、\emph{常に}一回だけ評価されます。

\begin{code}{}
[ x isReady ] whileTrue: [ y doSomething ]   "レシーバと引数の両方ともブロックでなくてはならない"
4 timesRepeat: [ Beeper beep ]                   "引数は 2 回以上実行されるので、ブロックでなくてはならない"
(x isReady) ifTrue: [ y doSomething ]           "レシーバは 1 回だけ評価されるので、ブロックではない"
\end{code}

%=============================================================
\section{式の並び}
複数の式をピリオドで区切ったものを、式の並びと呼びます。ここで式(\ie メッセージ送信や代入\dots) は順番に評価されます。
ただし、変数宣言とその直後の式の間にはピリオドは書かないので注意してください。
式の並びにおいて、最後の式の値が全体の値となり、
これ以外の式が返した値は、無視されます。ピリオドは\subind{文}{区切り}であり、終端子ではないということに注意してください。したがって式の並びの最後には、ピリオドを付けても付けなくても構いません。

\begin{code}{@TEST}
| box |
box := 20@30 corner: 60@90.
box containsPoint: 40@50 --> true
\end{code}

%=============================================================
\section{カスケードメッセージ}
\st には、複数のメッセージを同じレシーバに送信するためのセミコロン (\ct{;}) を使った記法があります。これは \st 用語では \emphind{カスケード} と呼ばれています。

\important{式 メッセージ1 ; メッセージ2}

\begin{minipage}{0.35\textwidth}
\begin{code}{}
Transcript show: 'Pharo is '.
Transcript show: 'fun '.
Transcript cr.
\end{code}
\end{minipage}
\emph{is equivalent to:}
\begin{minipage}{0.35\textwidth}
\begin{code}{}
Transcript        
   show: 'Pharo is';
   show: 'fun ';
   cr
\end{code}
\end{minipage}

カスケードメッセージを受け取っているオブジェクト自身が、メッセージ送信の結果であっても良い、ということに注意してください。
実際、すべてのカスケードメッセージのレシーバは、一連のカスケードの中で最初のメッセージを受け取ったオブジェクトになります。以下の例では、最初のカスケードメッセージは \ct{setX:setY:} です(後にカスケードが続いているため)。カスケードメッセージ \ct{setX:setY:} のレシーバは \ct{Point new} によって新たに作られた点オブジェクトであり、\ct{Point} \emph{ではありません}。引き続き、同じレシーバ(点オブジェクト)にメッセージ \ct{isZero} が送信されます。

\begin{code}{}
Point new setX: 25 setY: 35; isZero --> false
\end{code}

%=============================================================
\section{まとめ}

\begin{itemize}
\item メッセージは常に、\emph{レシーバ} と呼ばれるオブジェクトに送信されます。レシーバは他のメッセージ送信の結果であることもあります。

\item 単項メッセージは、引数を取らないメッセージです。\\
単項メッセージの形式は、\lct{receiver \textbf{selector}}です。

\item 二項メッセージは、レシーバと引数という二つのオブジェクトが関与するメッセージです。二項メッセージのセレクタは\emph{また}、次の文字: \ct{+}、\ct{-}、\ct{*}、\ct{/}、\ct{|}、\texttt{\&}、\ct{=}、\ct{>}、\ct{<}、\texttt{\~} および \ct{@} からなります。
二項メッセージの形式は、\lct{receiver \textbf{selector} argument}です。
\item キーワードメッセージは、二つ以上のオブジェクトが関与し、そのセレクタにコロン (\ct{:}) が使われているようなメッセージです。\\
キーワードメッセージの形式は、
\lct{receiver \textbf{selectorWordOne:} argumentOne \textbf{wordTwo:} argumentTwo}です。

\item \textbf{規則 1:} 単項メッセージがまず送信され、次に二項メッセージ、そして最後にキーワードメッセージが送信されます。
\item \textbf{規則 2:} 括弧で囲まれたメッセージが他のメッセージよりも先に送信されます。
\item \textbf{規則 3:} 同じ種類のメッセージが使われている場合は、左から右の順に評価されます。
\item \st では、\ct{+} や \ct{*} のような伝統的な算術演算の優先順位も、すべて同じです。\ct{+} や \ct{*} も単なる二項メッセージで、\ct{*} の優先順位の方が \ct{+} よりも高い、ということはありません。異なる結果を求めたい場合は、括弧を使用しなければなりません。
\end{itemize}

%\end{document}
% ON: Don't ever put an \end{document} in a chapter
% It will make the book stop there!
%=================================================================
\ifx\wholebook\relax\else\end{document}\fi
%=================================================================

%=================================================================
%%% Local Variables:
%%% coding: utf-8
%%% mode: latex
%%% TeX-master: t
%%% TeX-PDF-mode: t
%%% ispell-local-dictionary: "english"
%%% End:

%---------------------------------------------------------
