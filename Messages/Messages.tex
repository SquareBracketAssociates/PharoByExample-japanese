% $Author$
% $Date$
% $Revision$

% HISTORY:
% 2007-06-06 - Stef started
% 2007-08-21 - Oscar edit
% 2007-09-06 - Lukas corrections
% 2007-09-11 - Orla corrections

%=================================================================
\ifx\wholebook\relax\else
% --------------------------------------------
% Lulu:
	\documentclass[a4paper,10pt,twoside]{book}
	\usepackage[
		papersize={6.13in,9.21in},
		hmargin={.75in,.75in},
		vmargin={.75in,1in},
		ignoreheadfoot
	]{geometry}
	\input{../common.tex}
	\pagestyle{headings}
	\setboolean{lulu}{true}
% --------------------------------------------
% A4:
%	\documentclass[a4paper,11pt,twoside]{book}
%	\input{../common.tex}
%	\usepackage{a4wide}
% --------------------------------------------
    \graphicspath{{figures/} {../figures/}}
	\begin{document}
	\renewcommand{\nnbb}[2]{} % Disable editorial comments
	\sloppy
\fi
%=================================================================
\chapter{メッセージ構文を理解しよう}
\chalabel{understanding}

\st の構文はとても簡素ではありますが、標準的なものではないためになれるのに少し時間が必要でしょう。
この章ではメッセージ構文の特徴に順応するためのガイダンスを提供します。
もし構文になれていると思われるなら、この章は飛ばしてもよいですし、あるいはまたあとで読むことにしてもよいでしょう。

%=============================================================
\section{Identifying messages}

\st では\charef{syntax}で紹介した構文要素 (\ct+:= ^ . ; # () {} [ : | ]+)を除けば、, すべてがメッセージ送信です。
\ind{C++}と同様に、\ct{+}のような演算子を自分のクラスに定義することができますが、すべての演算子は同じ優先順位となることに気をつけてください。
また、あるメソッドが受け取る引数の個数も変えられません。例えば、``\ct{-}''は常に2項演算子で、単項の``\ct{-}''をオーバーロードして作ることなどはできません。

\st では、メッセージ構文の優先順位、すなわち記述されたメッセージがどのような順番で送信されるかはメッセージ構文の種類によって決定されます。そのメッセージ構文の種類は3種類だけです。それらは\emphsubind{message}{unary(単項)}、\emphsubind{message}{binary(2項)}および\emphsubind{message}{keyword(キーワード)}メッセージです。単項メッセージは最高の優先順位であり、その次が2項、そして最後がキーワードメッセージです。ほかの多くの言語と同様に、\ind{parentheses}を使って評価順を変えることができます。これらの規則によって、\st のコードは可能な限り読みやすく、かつ規則のことに頭を悩ませなくてもよいようになっています。

基本的に\st でのすべての計算はメッセージ送信によって行われるので、メッセージを正しく読み取ることが非常に重要です。以下にまとめた用語と概念を理解することが役に立つでしょう。

\begin{itemize}
  \item メッセージは\emphsubind{message}{セレクタ}および必要な場合に指定されるメッセージ引数からなりたっています。
  \item メッセージは\emphsubind{message}{レシーバ}に送信されます。
  \item レシーバとメッセージの組み合わせが、図\figref{firstScriptMessage}にあるように\emph{メッセージ} \emphsubind{message}{送信}と呼ばれています。
%  \item We call expression a message send, a variable, an assignment, a literal or a block. 
\end{itemize}

\begin{figure}[htb]
\begin{minipage}{0.53\textwidth}
	\begin{center}
	\includegraphics[width=0.95\textwidth]{message}
	\caption{レシーバ、セレクタ、引数からなるメッセージ送信の例2つ。\figlabel{firstScriptMessage}}\end{center}
\end{minipage}
\hfill
\begin{minipage}{0.43\textwidth}
	\begin{center}
	\ifluluelse
		{\includegraphics[width=0.9\textwidth]{uKeyUnOne}}
		{\includegraphics[width=6cm]{uKeyUnOne}}
	\caption{\ct{aMorph color: Color yellow}という式は\ct{Color yellow}および\ct{aMorph color: Color yellow}という二つのメッセージ送信からなりたっている。\figlabel{ellipse}}
	\end{center}
\end{minipage}
\end{figure}

%\begin{figure}[ht]
%\begin{center}
%\includegraphics[width=0.5\textwidth]{message}
%\caption{Two messages composed of a receiver, a method selector, and a set of arguments.\figlabel{firstScriptMessage}}\end{center}
%\end{figure}

\important{メッセージは常にレシーバに送信されます。レシーバはリテラル、ブロック、変数または他のメッセージを評価した結果のいずれかです。}
%sd-ま、代入を考えてないからかんぜんただしいというわけではないがな

メッセージのレシーバを同定しやすくなるよう、レシーバに下線を引くことにします。
また、それぞれのメッセージ送信を点線の楕円で囲み、送信される順番に従って番号をつけることにもします。

%\begin{figure}[!ht]
%\begin{center}
%\includegraphics[width=6cm]{uKeyUnOne}
%\end{center}
%\caption{\ct{aMorph color: Color yellow} is composed of two expressions: \ct{Color yellow} and \ct{aMorph color: Color yellow}.\figlabel{ellipse}}
%\end{figure}

\figref{ellipse}は\ct{Color yellow}と\ct{aMorph color: Color yellow}という2つのメッセージ送信を表していますので、楕円も2つあります。

メッセージ送信\ct{Color yellow}がまず実行されるので、そこについている楕円に番号\ct{1}がついています。全体の中にはレシーバが2つあります。ひとつは\ct{aMorph}で、メッセージ\ct{color: ...}が送られており、もう一つは\ct{Color}で、メッセージ\ct{yellow}が送られています。それぞれに下線がついています。

レシーバは、例えばメッセージ送信\ct{100 + 200}の\ct{100}や、\ct{Color yellow}の中の\ct{Color}のように、メッセージ送信記述の最初の要素であることが多いです。しかしながら、ほかのメッセージの結果がレシーバとして使われることもよくあります。例えば、\ct{Pen new go: 100}というメッセージ式では、\ct{go: 100}というメッセージのレシーバはは\ct{Pen new}というメッセージ送信から結果として返されるオブジェクトとなります。いずれにせよ、メッセージは\emph{レシーバ}と呼ばれるオブジェクトに送られ、そのオブジェクトはほかのメッセージ送信の結果の場合もあるということです。

\begin{table}\centering
	\begin{tabularx}{\linewidth}{llX}
		\toprule
		メッセージ送信式 & メッセージの種類 & 結果 \\
		\midrule
		\lct{Color yellow}
			& 単項
			& 色を表すオブジェクト。
		\\
		\lct{aPen  go: 100.}
			& キーワード
			& レシーバであるペンオブジェクト100ピクセル移動。
		\\
		\lct{100 + 20}
			& 2項
			& 数値オブジェクト100に数値20を引数とする+メッセージを送信。
		\\
		\lct{Browser open}
			& 単項
			& 新しいブラウザを開く。
		\\
		\lct{Pen new  go: 100}
			& 単項およびキーワード
			& ペンオブジェクトが生成され、100ピクセル移動する。
		\\
		\lct{aPen go: 100 + 20}
			& キーワードおよび2項
			& レシーバであるペンオブジェクトが120ピクセル移動。
		\\
		\bottomrule
	\end{tabularx}
	\caption{メッセージ送信と種類の例}\tablabel{messageExamples}
\end{table}

表\tabref{messageExamples}にいくつかのメッセージ式の例を挙げました。
必ずしもすべてのメッセージ式に引数がある訳ではないということに注意してください。\ct{open}のような単項メッセージには引数はありません。\ct{go: 100}と\ct{+ 20}のような単独のキーワードと2項メッセージは引数を一つとります。
単純なメッセージと複合的なものとがあります。\ct{Color yellow}や\ct{100 + 20}は単純です。一つのメッセージが一つのオブジェクトに送られるだけです。一方、\ct{aPen go: 100 + 20}は2つのメッセージ送信が複合したものです。\ct{+ 20}が\ct{100}に送られ、その結果を引数として\ct{go:}が\ct{aPen}に送られます。
値を返す式(代入、メッセージ送信およびリテラル)がレシーバとなりえます。\ct{Pen new go: 100}では、メッセージ\ct{go: 100}が\ct{Pen new}の結果として返されるオブジェクトに送られます。

%=============================================================
\section{3種類のメッセージ}

\st ではメッセージの送信順序を決定するいくつかの簡単な規則が定義されています。これらの規則は、以下のような3種類のメッセージ式の種類に基づいています。
\begin{itemize}
\item \emph{単項メッセージ}はあるオブジェクトに他の追加情報なしで送られるようなメッセージです。 例えば、\ct{3 factorial}という式の\ct{factorial}は単項メッセージです。
\item  \emph{2項メッセージ}は演算子(しばしば算術的な)からなるメッセージです。これらは常にレシーバと引数という2つのオブジェクトが関与しているために2項(binary)と呼ばれています。\ct{10 + 20}の場合は、\ct{+}が演算子であり、 引数\ct{20}と共に\ct{10}に送られています。
\item  \emph{キーワードメッセージ}は1つあるいはそれ以上のキーワードからなっています。それぞれのキーワードはコロン記号(\ct{:})がついており、さらにひとつ引数をとります。例えば、\ct{anArray at: 1 put: 10}という式の場合、キーワード\ct{at:}は\ct{1}という引数をとり、\ct{put:}というキーワードが引数\ct{10}をとっています。
\end{itemize}

%-------------------------------------------------------------
\subsection{単項メッセージ}
単項メッセージは引数を必要としないメッセージであり、その構文は\ct{receiver selector}という形式です。セレクタは\ct{:}を含まない単純な文字列です(例: \ct{factorial}, \ct{open}, \ct{class})。
\needlines{4}
\begin{code}{@TEST}
89 sin           --> 0.860069405812453
3 sqrt           --> 1.732050807568877
Float pi         --> 3.141592653589793
'blop' size     --> 4
true not        --> false
Object class --> Object class  "The class of Object is Object class (BANG)"
\end{code}
% ON: I changed the examples to things we can test

\important{単項メッセージは引数を必要とないメッセージです。\\
その構文は\lct{receiver \textbf{selector}}という形式です。}

%-------------------------------------------------------------
\subsection{2項メッセージ} 
2項メッセージは引数を1つだけとり\emph{かつ}セレクタが以下の文字集合からの1文字以上の繰り返しからなるメッセージです。\ct{+}、\ct{-}、\ct{*}、\ct{/}、\ct{&}、\ct{=}、\ct{>}、\ct{|}、\ct{<}、\ct{~}、および\ct{@}。構文解析時に問題が起こるので、\ct{--}は不正なセレクタであるということに注意してください。

\begin{code}{@TEST}
100@100      --> 100@100  "Pointオブジェクトを生成"
3 + 4              --> 7
10 - 1            --> 9
4 <= 3            --> false
(4/3) * 3 = 4   --> true  "同値性のテストも単なる2項メッセージ。分数オブジェクトは値を正確に表現できる"
(3/4) == (3/4) --> false  "別個に作られた2つの分数オブジェクトは同一ではない"
\end{code}

\important{2項メッセージは引数を1つだけとり\emph{かつ}セレクタが以下の文字集合からの1文字以上の繰り返しからなるメッセージです。\ct{+}、\ct{-}、\ct{*}、\ct{/}、\ct{&}、\ct{=}、\ct{>}、\ct{|}、\ct{<}、\ct{~}、および\ct{@}。\ct{--}は許されません\\
その構文は\lct{receiver \textbf{selector} argument}という形式です。}

%-------------------------------------------------------------
\subsection{キーワードメッセージ} 

キーワードメッセージは1つ以上の引数を必要とします。そのセレクタは\ct{:}で終わるキーワードを1つ以上並べたものからなります。キーワードメッセージの構文は
\lct{receiver \textbf{selectorWordOne:} argument\-One \textbf{wordTwo:} argumentTwo}という形式です。

それぞれのキーワードが引数をひとつずつとります。すなわち、\ct{r:g:b:}というセレクタは3引数であり、\ct{playFileNamed:}や\ct{at:}は1引数、\ct{at:put:}は2引数ということになります。\ct{Color}クラスのインスタンスを作るためには、例えば\ct{r:g:b:}を使い、\ct{Color r: 1 g: 0 b: 0}のようなメッセージにより、赤色を表すColorオブジェクトを作ることができます。ここでは、コロンもセレクタの一部であるということに注意してください。

\important{\ind{Java}や\ind{C++}であれば、\st における\ct{Color r: 1 g: 0 b: 0}という式は
\ct{Color.rgb(1,0,0)}と書かれることになるでしょう。}

\begin{code}{@TEST | nums |}
1 to: 10                        --> (1 to: 10)  "区間オブジェクトを生成"
Color r: 1 g: 0 b: 0       --> Color red  "色オブジェクトを返す"
12 between: 8 and: 15 --> true

nums := Array newFrom: (1 to: 5).
nums at: 1 put: 6.
nums --> #(6 2 3 4 5)
\end{code}
% ON: Changed to real examples that we can test

\important{キーワードメッセージは1つ以上の引数をとるメッセージです。そのセレクタはそれぞれがコロン(\ct{:})で終わる1つ以上のキーワードからなります。構文は\\
\lct{receiver \textbf{selectorWordOne:} argumentOne \textbf{wordTwo:} argumentTwo}という形式をとります。}

%=============================================================
\section{複合メッセージ}
これまで述べた3種類のメッセージはことなる優先順位をもっており、それらを組み合わせて使うときに、エレガントに行うことができます。

\begin{enumerate}
\item 単項メッセージがまず最初に送信され、次に2項、そして最後にキーワードメッセージが送られます。
\item \ind{括弧}で囲まれたメッセージは他のものよりも優先して送信されます。
\item 同じ種類のメッセージは左から右に評価されます。
\end{enumerate}
\index{message!evaluation order}

このルールにより、プログラムを読む際にも自然に読めるようになっています。メッセージが意図した通りの順番で送られることを明示的に示したいのであれば、\figref{uKeyUn}にあるように、括弧を足していくこともできます。この図では、メッセージ\ct{yellow}が単項メッセージであり、\ct{color:}はキーワードメッセージであるため、\ct{Color yellow}がまず先に実行されます。しかし、括弧で囲まれたメッセージがまず実行されるため、(冗長な)括弧を\ct{Color yellow}の周りに書くことにより、ここがまず先に実行されるということを強調することができます。以下では、これらの点について説明します。

\begin{figure}[ht]
\ifluluelse
	{\centerline{\includegraphics[width=0.9\textwidth]{uKeyUn}} }
	{\centerline{\includegraphics[width=10cm]{uKeyUn}} }
\caption{単項メッセージが最初に送られるので、\ct{Color yellow}がまず送られ、返されたColorオブジェクトが引数として\ct{aPen color:}に渡されます。\figlabel{uKeyUn}}
\end{figure}

%---------------------------------------------------------
\subsection*{単項 > 2項 > キーワード}
単項メッセージがまず送られ、次に2項、そしてキーワードメッセージが最後に送られます。しばしば、単項が他の種類のメッセージよりも高い優先順位を持つということがあります。

\important{\textbf{規則1.} 単項メッセージがまず送られ、次に2項、そしてキーワードメッセージが最後に送られます。\\
\centerline{\ct{単項 > 2項 > キーワード}}
}

以下の例が示すように、\stの構文では複合したメッセージ式を多くの場合自然な形で読むことができます。
\begin{code}{@TEST}
1000 factorial / 999 factorial --> 1000
2 raisedTo: 1 + 3 factorial     --> 128
\end{code}
\noindent

ただし、残念ながら、算術演算の場合にはこの規則はやや単純すぎます。そのため、かっこを使って2項演算子を複数組み合わせたときに評価順を強制する必要があります。
\begin{code}{@TEST}
1 + 2 * 3   --> 9
1 + (2 * 3) --> 7
\end{code}

以下のさらに複雑な(!)例は、さらに混みいった\stの式も自然に読めるということ良い例になっています。
\begin{code}{@TEST}
[:aClass | aClass methodDict keys select: [:aMethod | (aClass>>aMethod) isAbstract ]] value: Boolean --> an IdentitySet(#or: #| #and: #& #ifTrue: #ifTrue:ifFalse: #ifFalse: #not #ifFalse:ifTrue:)
\end{code}
\noindent
ここでは、\ct{Boolean}クラスのどのメソッドが抽象メソッドであるかを調べようとしています\footnote{実をいうと、同等の式はさらに簡単に\ct{Boolean methodDict select: #isAbstract thenCollect: #selector}と書くこともできます。}\damien{I've added this footnote, just remove it if you don't like it :-)}.
まず、引数として渡されたクラス\ct{aClass}に, そのメソッド辞書のキー集合を問い合わせ、その中から、抽象メソッドに対応しているものを選択(select)するブロックを記述します。
そのブロックの引数\ct{aClass}にあるクラス\ct{Boolean}を束縛しています。
この式の中では、括弧はメソッドをクラスから取り出すを2項演算子\ct{>>}を、単項メッセージ\mbox{\ct{isAbstract}}の前に送るところだけに必要でした。全体を評価した結果は、\ct{Boolean}のサブクラスである\ct{True}および\ct{False}で実装しなくてはならないメソッドの集合となっています。

%\begin{code}{}
%Pen new go: 30 + 50          "create a turtle and moves it forward 80 pixels"
%Display restoreAfter: [WarpBlt test4] 					
%	"Keyword message, try test1, test12, test3, test4 and test 5"
%#($t $e $s $t) at: 3 --> $s 
%#($a $b $c $d) at: 2 factorial put: $z 
%\end{code}

%As you can see the syntax and in particular the keyword messages as in
%the example \ct{array at: 1 put: 4} make it possible to write code
%with a structure approaching that of natural language.
% This was one of the initial objectives so that the children can program.

\paragraph{例.}
式\ct{aPen color: Color yellow}の中には、\ct{Color}に送られる\emph{単項}メッセージ\ct{yellow}と、\ct{aPen}に送られる\emph{keyword}メッセージ\ct{color:}があります。単項メッセージがまず送られるので、図\egref{decColor}にあるように\ct{Color yellow}がまず送られ(1)、その結果として返されたColorオブジェクトがメッセージ\ct{aPen color: aColor}の引数となります(2)。

\needlines{5}
\begin{example}[decColor]{\ct{aPen color: Color yellow}の評価順を分解}{}
        aPen color: Color yellow
(1)                       Color yellow        "単項メッセージがまず送られる"
                        --> aColor
(2)   aPen color: aColor                 "キーワードメッセージが次に送られる"
\end{example}

\paragraph{例.} In the message \ct{aPen go: 100 + 20}, there is a \emph{binary} message \ct{+ 20} and a \emph{keyword} message \ct{go:}. Binary messages are sent prior to keyword messages so \ct{100 + 20} is sent first (1): the message \ct{+ 20} is sent to the object \ct{100} and returns the number \ct{120}. Then the message \ct{aPen go: 120} is sent with \ct{120} as argument (2).
\Egref{decGo} shows how the message send is executed. 

\begin{example}[decGo]{Decomposing \ct{aPen go: 100 + 20}}{}
      aPen go: 100 + 20   
(1)                 100 + 20           "binary message first"
                   -->   120
(2)  aPen go: 120                   "then keyword message"
\end{example}

\begin{figure}[htb]
\begin{minipage}{0.48\textwidth}
	\ifluluelse
		{\centerline{\includegraphics[width=0.9\textwidth]{uKeyBin}}}
		{\centerline{\includegraphics[width=6cm]{uKeyBin}}}
	\caption{Binary messages are sent before keyword messages.\figlabel{uKeyBin}}
\end{minipage}
\hfill
\begin{minipage}{0.48\textwidth}
	\begin{center}
	\ifluluelse
		{\includegraphics[width=0.9\textwidth]{uunKeyBin}}
		{\includegraphics[width=6cm]{uunKeyBin}}
\caption{Decomposing \ct{Pen new go: 100 + 20}}\figlabel{unKeyBin}
\end{center}
\end{minipage}
\end{figure}

%\begin{figure}[ht]
%\centerline{\includegraphics[width=6cm]{uKeyBin}} 
%\caption{Unary messages are sent first so \ct{Color yellow} is sent. This returns a color object which is passed as argument of the message \ct{aPen color:}.\figlabel{uKeyBin}}
%\end{figure}

%\paragraph{Example 3.}
%The message \ct{aPen penSize: aPen penSize + 2} contains one unary message \ct{penSize}, one binary message \ct{+},  and one keyword message \ct{penSize:}.
%The unary message \ct{aPen penSize} is sent first (1), this message returns a number representing the current size of the receiver pen. Then the binary message is sent (2), the returned number is sent the message \ct{+ 2} which in its turn returns another number. Finally the keyword message 
%\ct{penSize:} is sent with the last number as argument. The expression increases the receiver pen size by two pixels. \Egref{decpen} illustrates the decomposition of message evaluation.

% penSize does not exist on Pen
%\

%\begin{scriptfigwithsize}[0.65]{\includegraphics[width=6cm]{uKeyUnBin}}{Decomposing \ct{ aPen penSize: aPen penSize + 2}}\scrlabel{decpen}
%      aPen penSize: aPen penSize + 2
%(1)                        aPen penSize            "unary"
%                              $\arrow$  aNumber
%(2)                              aNumber + 2	             "binary"
%                                    $\arrow$   anotherNumber	
%(3)   aPen penSize: anotherNumber           "keyword"      
%\end{scriptfigwithsize}

\paragraph{Example.} As an exercise we let you decompose the evaluation of the message \ct{Pen new go: 100 + 20} which is composed of one unary, one keyword and one binary message (see \figref{unKeyBin}).
% The unary message \ct{Pen new} is first sent. It returns a new bot, then the binary message \ct{100 + 20} is sent and returns \ct{120}. Finally the message \ct{go:} is sent to the newly created robot with \ct{120}.

%\begin{figure}[ht]
%\begin{center}
%\includegraphics[width=8cm]{uunKeyBin}
%\caption{Decomposing \ct{Pen new go: 100 + 20}}\figlabel{unKeyBin}
%\end{center}
%\end{figure}

%-------------------------------------------------------------
\subsection{Parentheses first}

\important{\textbf{Rule Two.} Parenthesised messages are sent prior to other messages.\\
\centerline{\ct{(Msg) > Unary > Binary > Keyword}}}

\begin{code}{@TEST}
1.5 tan rounded asString = (((1.5 tan) rounded) asString) --> true    "parentheses not needed here"
3 + 4 factorial   --> 27    "(not 5040)"
(3 + 4) factorial --> 5040
\end{code}

Here we need the \ind{parentheses} to force sending \ct{lowMajorScaleOn:} before \ct{play}.
\begin{code}{}
(FMSound lowMajorScaleOn: FMSound clarinet) play 
"(1) send the message clarinet to the FMSound class to create a clarinet sound.
 (2) send this sound to FMSound as argument to the lowMajorScaleOn: keyword message.
 (3) play the resulting sound."
\end{code}

% ON: This has nothing to do with parentheses!
%RecordingControlsMorph new openInWorld
%"An instance of the digitizer is created then visualized. If your microphone is plugged in try a sampleBANG"

% ON: This link is broken, and the result does not understand display!
%(HTTPSocket httpShowGif:
%   'www.altavista.digital.com/av/pix/default/av-adv.gif') display

\paragraph{Example.}
The message \ct{(65@325 extent: 134@100) center} returns the center of a rectangle whose top left point is $(65, 325)$ and whose size is $134{\times}100$. \Egref{decExtent} shows how the message is decomposed and sent. First the message between parentheses is sent: it contains two binary messages \ct{65@325} and \ct{134@100} that are sent first and return points, and a keyword message \ct{extent:} which is then sent and returns a rectangle. Finally the unary message \ct{center} is sent to the rectangle and a point is returned. 
Evaluating the message without parentheses would lead to an error because the object \ct{100} does not understand the message \ct{center}.

\needlines{9}
\begin{example}[decExtent]{Example of Parentheses.}{}
      (65@325 extent: 134@100) center
(1)   65@325                                                    "binary"
    --> aPoint
(2)                                134@100                     "binary"
                                 --> anotherPoint
(3)   aPoint extent: anotherPoint                       "keyword"
      --> aRectangle
(4)   aRectangle center                                     "unary"
      --> 132@375
\end{example}

\subsection{From left to right}
Now we know how messages of different kinds or priorities are handled. The final question to be addressed is how messages with the same priority are sent. They are sent from the left to the right. Note that you already saw this behaviour in \egref{decExtent} where the two point creation messages (\ct{@}) were sent first.

\important{{\textbf{Rule Three.} When the messages are of the same kind, the order of evaluation is from left to right.}}

%\begin{figure}
%\centerline{\includegraphics[width=8cm]{ucompoUn}} 
%\caption{The message \ct{Pen new east} is composed of two unary messages. Therefore the leftmost one, \ct{new},  is sent and it returns a new robot to which the second message \ct{east} is sent. \figlabel{compoUn}}
%\end{figure}

\paragraph{Example.} In the message sends \ct{Pen new down} all messages are unary messages, so the leftmost one, \ct{Pen new}, is sent first. This returns a newly created pen to which the second message \ct{down} is sent, as shown in \figref{unaryMessages}.

\begin{figure}
	\centering
	\includegraphics[width=8cm]{ucompoUn}
	\caption{Decomposing \ct{Pen new down}\figlabel{unaryMessages}}
\end{figure}

%-------------------------------------------------------------
\subsection{Arithmetic inconsistencies}
The message composition rules are simple but they result in inconsistency for the execution of arithmetic message sends expressed in terms of binary messages. Here we see the common situations where extra parentheses are needed. 

\needlines{6}
\begin{code}{@TEST}
3 + 4 * 5      --> 35    "(not 23)  Binary messages sent from left to right"
3 + (4 * 5)    --> 23
1 + 1/3         --> (2/3)    "and not 4/3"
1 + (1/3)       --> (4/3)
1/3 + 2/3       --> (7/9)    "and not 1"
(1/3) + (2/3)  --> 1
\end{code}

\paragraph{Example.} 
In the message sends \ct{20 + 2 * 5}, there are only binary messages \ct{+} and \ct{*}. However in \st there is no specific priority for the operations \ct{+} and \ct{*}. They are just binary messages, hence \ct{*} does not have priority over \ct{+}. Here the leftmost message \ct{+} is sent first (1) and then the \ct{*} is sent to the result as shown in \egref{binaryMessages1}.  

\begin{example}[binaryMessages1]{Decomposing \ct{20 + 2 * 5}}{}
"As there is no priority among binary messages, the leftmost message + is evaluated first even if by the rules of arithmetic the * should be sent first."

      20 + 2 * 5 
(1)  20 + 2 --> 22
(2)  22       * 5 --> 110
\end{example}

\begin{figure}
\begin{center}\includegraphics[width=8cm]{ucompoNoBracketPar}\end{center}
\end{figure}
\noindent
As shown in \egref{binaryMessages1} the result of this message send is not \ct{30} but \ct{110}. This result is perhaps unexpected but follows directly from the rules used to send messages. This is somehow the price to pay for the simplicity of the \st model. To get the correct result, we should use parentheses. When messages are enclosed in parentheses, they are evaluated first. Hence the message send \ct{20 + (2 * 5)} returns the result as shown in \egref{mathcorrect}.

\needlines{4}
\begin{example}[mathcorrect]{Decomposing \ct{20 + (2 * 5)}}{}
"The messages surrounded by parentheses are evaluated first therefore * is sent prior to + which produces the correct behaviour."

    20 + (2 * 5) 
(1)        (2 * 5) --> 10
(2) 20 + 10      --> 30
\end{example}

\begin{figure}
\begin{center}
\includegraphics[width=8cm]{ucompoNumberBracket}
\end{center}
\end{figure}

\important{In \st, arithmetic operators such as + and * do not have different priority. \ct{+} and \ct{*} are just binary messages, therefore \ct{*} does not have priority over \ct{+}. Use parentheses to obtain the desired result.}

%  At the beginning put parenthesis when you have multiple binary messages.}  HUH?  At the beginning of what?!

\begin{figure}
\begin{center}
\ifluluelse
	{\includegraphics[width=\textwidth]{uKeyUnBinPar}}
	{\includegraphics[width=0.8\textwidth]{uKeyUnBinPar}}
\ifluluelse
	{\includegraphics[width=\textwidth]{uunKeyBinPar}}
	{\includegraphics[width=10cm]{uunKeyBinPar}}
\end{center}
\caption{Equivalent messages using parentheses. \figlabel{uKeyUnBinPar}}
\end{figure}

Note that the first rule stating that unary messages are sent prior to binary and keyword messages avoids the need to put explicit parentheses around them. \tabref{expressions} shows message sends written following the rules and equivalent message sends if the rules would not exist. Both message sends result in the same effect or return the same value. 

\begin{figure}\centering
	\begin{tabular}{l@{\qquad}l}
	\toprule
	Implicit precedence & Explicitly parenthesized equivalent \\
	\midrule
	\lct{aPen color: Color yellow}
		& \lct{aPen color: (Color yellow)}
		\\
	\lct{aPen go: 100 + 20}
		& \lct{aPen go: (100 + 20)}
		\\
	\lct{aPen penSize: aPen penSize + 2}
		& \lct{aPen penSize: ((aPen penSize) + 2)}
		\\
	\lct{2 factorial + 4}
		& \lct{(2 factorial) + 4}
		\\
	\bottomrule
	\end{tabular}
	\caption{Message sends and their fully parenthesized equivalents\tablabel{expressions}}
\end{figure}

%=============================================================
\section{Hints for identifying keyword messages}
Often beginners have problems understanding when they need to add parentheses. Let's see how keywords messages are recognized by the compiler. 

%-------------------------------------------------------------
\subsection{Parentheses or not?}
The characters \ct{[}, \ct{]}, \ct{(} and \ct{)} 
delimit distinct areas. Within such an area, a keyword message is the longest sequence of words terminated  by \ct{:} that is not cut by the characters \ct{.},  or \ct{;}. 
When the characters \ct{[}, \ct{]}, \ct{(} and \ct{)} surround some words with colons, these words participate in the keyword message \emph{local} to the area defined.

In this example, there are two distinct keyword messages:  \ct{rotatedBy:magnify:smoothing:} and \ct{at:put:}.

\begin{code}{}
aDict
   at: (rotatingForm 
          rotateBy: angle	
          magnify: 2 
          smoothing: 1)
   put: 3
\end{code}

\important{
The characters \lct{[}, \lct{]}, \lct{(} and \lct{)} delimit distinct areas. Within such an area, 
a keyword message is the longest sequence  of words terminated by \lct{:} that is not cut by the characters \lct{.},  or \lct{;}. 
When the characters \lct{[}, \lct{]}, \lct{(} and \lct{)} surround some words with colons, these words participate in the keyword message local to the area defined.}

\on{Sounds terribly complicated.}

\paragraph{Hints.} If you have problems with these precedence rules, you may start simply by putting parentheses whenever you want to distinguish two messages having the same precedence.

The following piece of code does not require parentheses because the message send  \ct{x isNil} is unary hence is sent prior to the keyword message \ct{ifTrue:}.
\begin{code}{}
(x isNil)
   ifTrue:[...]
\end{code}

The following piece of code requires parentheses because the messages \ct{includes:} and \ct{ifTrue:} are both keyword messages. 
\begin{code}{}
ord := OrderedCollection new.
(ord includes: $a)
   ifTrue:[...]
\end{code}
\noindent
Without parentheses the unknown message \ct{includes:ifTrue:} would be sent to the collection!

%-------------------------------------------------------------
\subsection{When to use \lct{[ ]} or \lct{( )}}
You may also have problems understanding when to use square brackets rather than parentheses.
The basic principle is that you should use \ct{[ ]} when you do not know how many times, potentially zero, an expression should be evaluated.
\lct{[\emph{expression}]} will create a \ind{block} closure (\ie an object) from \lct{\emph{expression}}, which may be evaluated any number of times (possibly zero), depending on the context. Here note that an expression can either be a message send, a variable, a literal, an assignment or a block.

Hence the conditional branches of \ct{ifTrue:} or \ct{ifTrue:ifFalse:} require blocks. Following the same principle both the receiver and the argument of a \ct{whileTrue:} message require the use of square brackets since we do not know how many times either the receiver or the argument should be evaluated.

Parentheses, on the other hand, only affect the order of sending messages.
So in \lct{(\emph{expression})}, the \lct{\emph{expression}} will \emph{always} be evaluated exactly once.

\begin{code}{}
[ x isReady ] whileTrue: [ y doSomething ]   "both the receiver and the argument must be blocks"
4 timesRepeat: [ Beeper beep ]                   "the argument is evaluated more than once, so must be a block"
(x isReady) ifTrue: [ y doSomething ]           "receiver is evaluated once, so is not a block"
\end{code}

%=============================================================
\section{Expression sequences}
Expressions (\ie message sends, assignments\dots) separated by periods are evaluated in sequence.
Note that there is no period between a variable definition and the following expression.
The value of a sequence is the value of the last expression. The values returned by all the 
expressions except the last one are ignored. Note that the period is a \subind{statement}{separator} and not a terminator. Therefore a final period is optional.

\begin{code}{@TEST}
| box |
box := 20@30 corner: 60@90.
box containsPoint: 40@50 --> true
\end{code}

%=============================================================
\section{Cascaded messages}
\st offers a way to send multiple messages to the same receiver using a semicolon (\ct{;}). This is called the \emphind{cascade} in \st jargon.

\important{Expression Msg1 ; Msg2}

\begin{minipage}{0.35\textwidth}
\begin{code}{}
Transcript show: 'Pharo is '.
Transcript show: 'fun '.
Transcript cr.
\end{code}
\end{minipage}
\emph{is equivalent to:}
\begin{minipage}{0.35\textwidth}
\begin{code}{}
Transcript        
   show: 'Pharo is';
   show: 'fun ';
   cr
\end{code}
\end{minipage}

Note that the object receiving the cascaded messages can itself be the result of a message send. 
In fact the receiver of all the cascaded messages is the receiver of the first message involved in a cascade. In the following example, the first cascaded message is \ct{setX:setY} since it is followed by a cascade. The receiver of the cascaded message \ct{setX:setY:} is the newly created point resulting from the evaluation of \ct{Point new}, and \emph{not} \ct{Point}. The subsequent message \ct{isZero} is sent to that same receiver. 

\begin{code}{}
Point new setX: 25 setY: 35; isZero --> false
\end{code}

%=============================================================
\section{Chapter summary}

\begin{itemize}
\item A message is always sent to an object named the \emph{receiver} which may be the result of other message sends.

\item Unary messages are messages that do not require any argument.\\
They are of the form of \lct{receiver \textbf{selector}}.

\item Binary messages are messages that involve two objects, the receiver and another object \emph{and} whose selector is composed of  one or more characters from the following list: \ct{+}, \ct{-}, \ct{*}, \ct{/}, \ct{|}, \texttt{\&}, \ct{=}, \ct{>}, \ct{<}, \texttt{\~}, and \ct{@}.
They are of the form: \lct{receiver \textbf{selector} argument}
\item Keyword messages are messages that involve more than one object and that contain at least one colon character (\ct{:}). \\
They are of the form:
\lct{receiver \textbf{selectorWordOne:} argumentOne \textbf{wordTwo:} argumentTwo}

\item \textbf{Rule One.} Unary messages are sent first, then binary messages, and finally keyword messages.
\item \textbf{Rule Two.} Messages in parentheses are sent before any others.
\item \textbf{Rule Three.} When the messages are of the same kind, the order of evaluation is from left to right.
\item In \st, traditional arithmetic operators such as + and * have the same priority. \ct{+} and \ct{*} are just binary messages, therefore \ct{*} does not have priority over \ct{+}. You must use parentheses to obtain a different result.
\end{itemize}

%\end{document}
% ON: Don't ever put an \end{document} in a chapter
% It will make the book stop there!
%=================================================================
\ifx\wholebook\relax\else\end{document}\fi
%=================================================================

%=================================================================
%%% Local Variables:
%%% coding: utf-8
%%% mode: latex
%%% TeX-master: t
%%% TeX-PDF-mode: t
%%% ispell-local-dictionary: "english"
%%% End:

%---------------------------------------------------------
