% $Author$
% $Date$
% $Revision$

% HISTORY:
% 2007-08-13 - Oscar starts
% 2007-09-01 - Marcus review

%=================================================================
\ifx\wholebook\relax\else
% --------------------------------------------
% Lulu:
	\documentclass[a4paper,10pt,twoside]{book}
	\usepackage[
		papersize={6.13in,9.21in},
		hmargin={.75in,.75in},
		vmargin={.75in,1in},
		ignoreheadfoot
	]{geometry}
	\input{../common.tex}
	\pagestyle{headings}
	\setboolean{lulu}{true}
% --------------------------------------------
% A4:
%	\documentclass[a4paper,11pt,twoside]{book}
%	\input{../common.tex}
%	\usepackage{a4wide}
% --------------------------------------------
    \graphicspath{{figures/} {../figures/}}
	\begin{document}
	\renewcommand{\nnbb}[2]{} % Disable editorial comments
	\sloppy
\fi
%=================================================================
\chapter{よくある質問}
\label{app:faq}

\on{These should be *real* (not invented) FAQs.
Here are a few that I have collected.
Check the ST lecture notes for more FAQs.
We should also try to mine more from newbies.}

%=================================================================
\section{はじめに}
\begin{faq}
最新のPharoはどこで入手出来ますか?
\end{faq}
\answer
\url{http://www.pharo-project.org/}
\index{download}

\begin{faq}
この本ではどの \pharo のイメージを使うのがいいですか?
\end{faq}
\answer
どのPharoイメージを使っても大丈夫ですが、Pharo by Example のWEBサイト(\url{http://PharoByExample.org})に置いてあるイメージを使うことをお勧めします。
また、他のほとんどのイメージも使用可能ですが演習ではあっと驚くくらい動作が異なることがあります。

%=================================================================
\section{コレクション}

\begin{faq}
\clsind{OrderedCollection} を並べ替えるにはどうすればいいですか?
\end{faq}
\answer
インスタンスに \mthind{Collection}{asSortedCollection} メッセージを送ります。

\begin{code}{@TEST}
#(7 2 6 1) asSortedCollection --> a SortedCollection(1 2 6 7)
\end{code}

\begin{faq}
文字の配列(collectionクラス)を文字列(\clsind{String})に変換するにはどうすればいいですか?
\end{faq}
\answer
\begin{code}{@TEST}
String streamContents: [:str | str nextPutAll: 'hello' asSet] --> 'hleo'
\end{code}

%=================================================================
\section{システムブラウザー}

% See \faqref{packagebrowser}

\begin{faq}\faqlabel{packagebrowser}
ブラウザーがこの本で使用されているものとは違います。どうしたらいいですか?
\end{faq}
\answer
おそらく異なるバージョンの \ind{OmniBrowser} がデフォルトのブラウザーとしてインストールされているイメージを使用しているからです。このチュートリアルでは、デフォルトブラウザーとして OmniBrowser (\emph{Package Browser})がインストールされていることを前提としています。デフォルトブラウザーはブラウザーのメニューバーをクリックして変更することができます。ウィンドウの右上隅にある灰色の角丸長方形のバーをクリックし、“Choose new default Browser”を選択し、O2PackageBrowserAdaptorを選びます。次回以降開かれるブラウザーはパッケージブラウザーになります。

\begin{figure}[tbh]
	\centering
	\subfigure[Choose new default browser を選択]{\figlabel{chooseNewBrowser}
		\includegraphics[width=0.45\linewidth]{chooseNewBrowser}}\hfill
	\subfigure[O2PackageBrowserAdaptor を選択]{\figlabel{O2PackageBrowserAdaptor}
		\includegraphics[width=0.45\linewidth]{O2PackageBrowserAdaptor}}\hfill
	\caption{デフォルトブラウザーの変更}
\end{figure}
\seeindex{morphic halo}{Morphic}

\begin{faq}
クラスを検索するにはどうしたらいいですか?
\end{faq}
\answer
クラス名の文字列を選択して \short{b} (browse)、または、ブラウザーのカテゴリーペーン上にマウスカーソルがある状態で \short{f}
\index{keyboard shortcut!browse it}
\index{keyboard shortcut!find ...}

\begin{faq}
スーパークラスにメッセージを送っている全てのメソッド検索したりブラウズするにはどうしたらいいですか?
\end{faq}
\answer
ふたつ目の手法の方が高速です:
\begin{code}{}
SystemNavigation default browseMethodsWithSourceString: 'super'.
SystemNavigation default browseAllSelect: [:method | method sendsToSuper ].
\end{code}
\index{browsing programmatically}
\clsindex{SystemNavigation}

\begin{faq}
特定の階層内でスーパークラスにメッセージを送っている全てのメソッドをブラウズするにはどうしたらいいですか?
\end{faq}
\answer
\begin{code}{}
browseSuperSends := [:aClass | SystemNavigation default
	browseMessageList: (aClass withAllSubclasses gather: [ :each |
		(each methodDict associations
			select: [ :assoc | assoc value sendsToSuper ])
				collect: [ :assoc | MethodReference class: each selector: assoc key ] ])
	name: 'Supersends of ' , aClass name , ' and its subclasses'].
browseSuperSends value: OrderedCollection.
\end{code}

\begin{faq}
サブクラスによって新たに追加されたメソッドを見つけるにはどうしたらいいですか?(すなわち、親クラスのメソッドをオーバーライドしていないメソッドのこと)
\end{faq}
\answer
\ct{True}クラスによって追加された新しいメソッドを調べるのは次の通りです:
\begin{code}{@TEST | newMethods |}
newMethods := [:aClass| aClass methodDict keys select:
	[:aMethod | (aClass superclass canUnderstand: aMethod) not ]].
newMethods value: True --> an IdentitySet(#asBit #xor:)
\end{code}

\begin{faq}
特定のクラスにおいてどのメソッドが抽象的メソッドであるかわかりますか?
\end{faq}
\answer
\begin{code}{@TEST | abstractMethods |}
abstractMethods :=
	[:aClass | aClass methodDict keys select:
		[:aMethod | (aClass>>aMethod) isAbstract ]].
abstractMethods value: Collection --> an IdentitySet(#remove:ifAbsent: #add: #do:)
\end{code}

\begin{faq}
\ind{AST} (抽象構文木)のビューを生成するにはどうしたらいいですか?
\end{faq}
\answer
squeaksource.com からASTをロードし、
\begin{code}{}
(RBParser parseExpression: '3+4') explore
\end{code}
を実行します。(またはそれを \emph{explore it} します)
\clsindex{RBParser}

\begin{faq}
システムにある全てのトレイトを見つけるにはどうしたらいいですか?
\end{faq}
\answer
\begin{code}{}
Smalltalk allTraits
\end{code}

\begin{faq}
トレイトを使用しているクラスを探すにはどうしたらいいですか?
\end{faq}
\answer
\begin{code}{}
Smalltalk allClasses select: [:each | each hasTraitComposition and: [each traitComposition notEmpty ]]
\end{code}
% Should be:
% Smalltalk allClasses select: [:each | each hasTraitComposition ]
% But this is broken for some reason!

%=================================================================
\section{Monticello で SqueakSource を使う}

\begin{faq}
\ind{SqueakSource} からプロジェクトをロードするにはどうしたらいいですか?
\index{Monticello}
\end{faq}
\answer
\begin{enumerate}
  \item ロードしたいプロジェクトを \url{http://squeaksource.com} から探します
  \item プロジェクトのリポジトリ情報をクリップボードにコピーします
  \item \menu{open \go Monticello browser} を選択して Monticello browser を開きます
  \item \menu{+Repository \go HTTP} を選択します
  \item リポジトリ情報をペーストし、必要ならパスワードを入力してOKします
  \item いま作ったばかりの新しいリポジトリを選択して \menu{Open} を選択します
  \item 最新のバージョンのファイル名を選択してLoadします
\end{enumerate}

\begin{faq}
SqueakSourceプロジェクトを新規作成するにはどうしたらいいですか?
\end{faq}
\answer
\begin{enumerate}
  \item ウェブブラウザーで \url{http://squeaksource.com} を開きます
  \item メンバー登録を行います
  \item プロジェクトを登録します(名前=カテゴリーとなります)
  \item リポジトリー情報をクリップボードにコピーします
  \item \menu{open \go Monticello browser} を選択して Monticello browser を開きます
  \item \menu{+Package} を選択してカテゴリーを登録します
  \item パッケージを選択します
  \item \menu{+Repository \go HTTP} を選択します
  \item リポジトリ情報をペーストし、必要ならパスワードを入力してOKします
  \item 最初のバージョンを \menu{Save} します
\end{enumerate}

\begin{faq}
既存の \ct{Number} クラスに追加した \ct{Number>>>chf} メソッドを自分の \ct{Money} プロジェクトのパッケージに入れるにはどうしたらいいですか?
\end{faq}
\answer
そのメソッドのメソッドカテゴリを \ct{*Money} にします。Monticello は‘*package’のような名前で始まるカテゴリのメソッドをまとめてそのパッケージに入れます。

%=================================================================
\section{ツール}

\begin{faq}
プログラムソース中から \ind{SUnit} \ind{TestRunner} を開くにはどうしたらいいですか?
\end{faq}
\answer
 \ct{TestRunner open} を 実行します

\begin{faq}
\ind{Refactoring Browser} はどこにありますか? 
\end{faq}
\answer
ASTをロードして、その後、squeaksource.com のリファクタリングエンジンをロードして下さい:
\url{http://www.squeaksource.com/AST}
\url{http://www.squeaksource.com/RefactoringEngine}

\begin{faq}
デフォルトのブラウザーを設定するにはどうしたらいいですか?
\end{faq}
\answer
ブラウザーウィンドウの右上隅にある灰色の菱形のバーをクリックします。

%=================================================================
\section{正規表現と構文解析}

%\begin{faq}
%How can I work with regular expressions?
%\index{regular expression package}
%\end{faq}
%\answer
%\ifluluelse
%	{Load Vassili Bykov's RegEx package from: \\
%	\url{http://www.squeaksource.com/Regex.html}}
%	{Load Vassili Bykov's RegEx package from:
%	\url{http://www.squeaksource.com/Regex.html}}
%\index{Bykov, Vassili}

\begin{faq}
RegEx(正規表現)パッケージに関するドキュメントはどこにありますか?
\end{faq}
\answer
\menu{VB-Regex} カテゴリにある \ct{RxParser class} クラスの \menu{DOCUMENTATION} プロトコルを参照して下さい。

\begin{faq}
パーサーを書くための何かいいツールはありますか?
\end{faq}
\answer
\ind{SmaCC}(the Smalltalk Compiler Compiler)を使って下さい。
その場合SmaCC-lr.13以降をインストールして下さい。
\url{http://www.squeaksource.com/SmaccDevelopment.html} からロード出来ます。
ここにはとても良く出来たオンラインチュートリアルかあります:
\url{http://www.refactory.com/Software/SmaCC/Tutorial.html}

\begin{faq}
パーサーを書くためには SqueakSource の SmaccDevelopment にあるどのパッケージをロードしたらいいですか?
\end{faq}
\answer
\ind{SmaCCDev} の最新バージョンをロードして下さい。ランタイムはすでに存在します
(SmaCC-Development は \squeak 3.8 用です)

%=================================================================
\ifx\wholebook\relax\else\end{document}\fi
%=================================================================

%\begin{faq}
%How do I run a headless image with a file argument?
%\end{faq}
%\answer
%Right now you can't do it with the MacOSX VM.

%\begin{faq}
%How do I find out which methods access the Smalltalk dictionary?
%\end{faq}
%\answer
%???

%\begin{faq}
%How do I get the tree view of an AST?
%\end{faq}
%\answer
%???

%\begin{faq}
%How do I browse all references to a given class?
%\end{faq}
%\answer
%???
