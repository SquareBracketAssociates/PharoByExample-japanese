% $Author$
% $Date$
% $Revision$

% HISTORY:
% 2007-08-13 - Oscar starts
% 2007-09-01 - Marcus review

%=================================================================
\ifx\wholebook\relax\else
% --------------------------------------------
% Lulu:
	\documentclass[a4paper,10pt,twoside]{book}
	\usepackage[
		papersize={6.13in,9.21in},
		hmargin={.75in,.75in},
		vmargin={.75in,1in},
		ignoreheadfoot
	]{geometry}
	\input{../common.tex}
	\pagestyle{headings}
	\setboolean{lulu}{true}
% --------------------------------------------
% A4:
%	\documentclass[a4paper,11pt,twoside]{book}
%	\input{../common.tex}
%	\usepackage{a4wide}
% --------------------------------------------
    \graphicspath{{figures/} {../figures/}}
	\begin{document}
	\renewcommand{\nnbb}[2]{} % Disable editorial comments
	\sloppy
\fi
%=================================================================
\chapter{よくある質問}
\label{app:faq}

\on{These should be *real* (not invented) FAQs.
Here are a few that I have collected.
Check the ST lecture notes for more FAQs.
We should also try to mine more from newbies.}

%=================================================================
\section{はじめに}
\begin{faq}
最新のPharoはどこで入手できますか?
\end{faq}
\answer
\url{http://www.pharo-project.org/}
\index{download}

\begin{faq}
この本ではどの \pharo のイメージを使うべきですか?
\end{faq}
\answer
どのPharoイメージを使っても大丈夫ですが、Pharo by Example のwebサイト(\url{http://PharoByExample.org})に置いてあるイメージを使うことをお勧めします。ほとんどの場合、他のイメージも使用できますが、演習によっては驚くほど動作が異なることがあります。

%=================================================================
\section{コレクション}

\begin{faq}
\clsind{OrderedCollection} を並べ替えるにはどうすればいいですか?
\end{faq}
\answer
インスタンスに \mthind{Collection}{asSortedCollection} メッセージを送ります。

\begin{code}{@TEST}
#(7 2 6 1) asSortedCollection --> a SortedCollection(1 2 6 7)
\end{code}

\begin{faq}
文字の配列を文字列(\clsind{String})に変換するにはどうすればいいですか?
\end{faq}
\answer
\begin{code}{@TEST}
String streamContents: [:str | str nextPutAll: 'hello' asSet] --> 'hleo'
\end{code}

%=================================================================
\section{システムブラウザ}

% See \faqref{packagebrowser}

\begin{faq}\faqlabel{packagebrowser}
ブラウザがこの本で使用されているものとは違います。どうすればいいですか?
\end{faq}
\answer
おそらくデフォルトのブラウザとして、異なるバージョンの \ind{OmniBrowser} がインストールされたイメージを使用しているからです。この本では、デフォルトとして OmniBrowserの (\emph{Package Browser})がインストールされていることを前提としています。デフォルトブラウザはブラウザのメニューバーをクリックすると変更できます。ウィンドウの右上隅にある灰色の角丸長方形のボタンをクリックし、“Choose new default Browser”を選択し、O2PackageBrowserAdaptorを選んでください。次回以降開かれるブラウザはパッケージブラウザになります。

\begin{figure}[tbh]
	\centering
	\subfigure[Choose new default browser を選択]{\figlabel{chooseNewBrowser}
		\includegraphics[width=0.45\linewidth]{chooseNewBrowser}}\hfill
	\subfigure[O2PackageBrowserAdaptor を選択]{\figlabel{O2PackageBrowserAdaptor}
		\includegraphics[width=0.45\linewidth]{O2PackageBrowserAdaptor}}\hfill
	\caption{デフォルトブラウザの変更}
\end{figure}
\seeindex{morphic halo}{Morphic}

\begin{faq}
クラスを検索するにはどうすればいいですか?
\end{faq}
\answer
クラス名の文字列を選択して \short{b} (browse)か、ブラウザのカテゴリペイン上にマウスカーソルがある状態で \short{f} です。
\index{keyboard shortcut!browse it}
\index{keyboard shortcut!find ...}

\begin{faq}
superにメッセージを送っているすべてのメソッドを検索したりブラウズするにはどうすればいいですか?
\end{faq}
\answer
二つ目の手法の方が高速です:
\begin{code}{}
SystemNavigation default browseMethodsWithSourceString: 'super'.
SystemNavigation default browseAllSelect: [:method | method sendsToSuper ].
\end{code}
\index{browsing programmatically}
\clsindex{SystemNavigation}

\begin{faq}
あるクラス階層内で、superにメッセージを送っているメソッドをすべてブラウズするにはどうすればいいですか?
\end{faq}
\answer
\begin{code}{}
browseSuperSends := [:aClass | SystemNavigation default
	browseMessageList: (aClass withAllSubclasses gather: [ :each |
		(each methodDict associations
			select: [ :assoc | assoc value sendsToSuper ])
				collect: [ :assoc | MethodReference class: each selector: assoc key ] ])
	name: 'Supersends of ' , aClass name , ' and its subclasses'].
browseSuperSends value: OrderedCollection.
\end{code}

\begin{faq}
サブクラスで新たに追加されたメソッド(親クラスのメソッドをオーバーライドしていないメソッド)を見つけるには、どうすればいいですか?
\end{faq}
\answer
\ct{True}クラスで追加されたメソッドを調べる例です:
\begin{code}{@TEST | newMethods |}
newMethods := [:aClass| aClass methodDict keys select:
	[:aMethod | (aClass superclass canUnderstand: aMethod) not ]].
newMethods value: True --> an IdentitySet(#asBit #xor:)
\end{code}

\begin{faq}
クラスの抽象メソッドはどうすればわかりますか?
\end{faq}
\answer
\begin{code}{@TEST | abstractMethods |}
abstractMethods :=
	[:aClass | aClass methodDict keys select:
		[:aMethod | (aClass>>aMethod) isAbstract ]].
abstractMethods value: Collection --> an IdentitySet(#remove:ifAbsent: #add: #do:)
\end{code}

\begin{faq}
\ind{AST}(抽象構文木)のビューを生成するにはどうすればいいですか?
\end{faq}
\answer
squeaksource.com からASTパッケージをロードし、
\begin{code}{}
(RBParser parseExpression: '3+4') explore
\end{code}
を実行します。(または \emph{explore it} します)
\clsindex{RBParser}

\begin{faq}
システムにあるすべてのトレイトを見つけるにはどうすればいいですか?
\end{faq}
\answer
\begin{code}{}
Smalltalk allTraits
\end{code}

\begin{faq}
トレイトを使用しているクラスを探すにはどうすればいいですか?
\end{faq}
\answer
\begin{code}{}
Smalltalk allClasses select: [:each | each hasTraitComposition and: [each traitComposition notEmpty ]]
\end{code}
% Should be:
% Smalltalk allClasses select: [:each | each hasTraitComposition ]
% But this is broken for some reason!

%=================================================================
\section{Monticello と SqueakSource を使う}

\begin{faq}
\ind{SqueakSource} からプロジェクトをロードするにはどうすればいいですか?
\index{Monticello}
\end{faq}
\answer
\begin{enumerate}
  \item ロードしたいプロジェクトを \url{http://squeaksource.com} から探します
  \item プロジェクトのリポジトリ情報をクリップボードにコピーします
  \item \menu{open \go Monticello browser} を選択して Monticelloブラウザを開きます
  \item \menu{+Repository \go HTTP} を選択します
  \item リポジトリ情報をペーストし、必要ならパスワードを入力してOKします
  \item できたリポジトリを選択して \menu{Open} を選択します
  \item 最新のバージョンのファイル名を選択してロードします
\end{enumerate}

\begin{faq}
SqueakSourceにプロジェクトを新規作成するにはどうすればいいですか?
\end{faq}
\answer
\begin{enumerate}
  \item ウェブブラウザで \url{http://squeaksource.com} を開きます
  \item メンバー登録を行います
  \item プロジェクトを登録します(名前=カテゴリーとなります)
  \item リポジトリー情報をクリップボードにコピーします
  \item \menu{open \go Monticello browser} を選択して Monticelloブラウザを開きます
  \item \menu{+Package} を選択してカテゴリーを登録します
  \item パッケージを選択します
  \item \menu{+Repository \go HTTP} を選択します
  \item リポジトリ情報をペーストし、必要ならパスワードを入力してOKします
  \item \menu{Save}ボタンで最初のバージョンを保存します
\end{enumerate}

\begin{faq}
既存の \ct{Number} クラスに追加した \ct{Number>>>chf} メソッドを、自分の \ct{Money} プロジェクトのパッケージに入れるにはどうすればいいですか?
\end{faq}
\answer
そのメソッドのメソッドカテゴリを \ct{*Money} にします。Monticello は‘*package’のような名前で始まるカテゴリのメソッドをまとめて、そのパッケージに入れます。

%=================================================================
\section{ツール}

\begin{faq}
プログラムソース中から \ind{SUnit} \ind{TestRunner} を開くにはどうすればいいですか?
\end{faq}
\answer
 \ct{TestRunner open} を実行します

\begin{faq}
\ind{Refactoring Browser} はどこにありますか? 
\end{faq}
\answer
squeaksource.com からASTパッケージをロードし、その後リファクタリングエンジンをロードして下さい:
\url{http://www.squeaksource.com/AST}
\url{http://www.squeaksource.com/RefactoringEngine}

\begin{faq}
デフォルトのブラウザを設定するにはどうすればいいですか?
\end{faq}
\answer
ブラウザウィンドウの右上隅にある灰色のボタンをクリックします。

%=================================================================
\section{正規表現と構文解析}

%\begin{faq}
%How can I work with regular expressions?
%\index{regular expression package}
%\end{faq}
%\answer
%\ifluluelse
%	{Load Vassili Bykov's RegEx package from: \\
%	\url{http://www.squeaksource.com/Regex.html}}
%	{Load Vassili Bykov's RegEx package from:
%	\url{http://www.squeaksource.com/Regex.html}}
%\index{Bykov, Vassili}

\begin{faq}
RegEx(正規表現)パッケージに関するドキュメントはどこにありますか?
\end{faq}
\answer
\menu{VB-Regex} カテゴリにある \ct{RxParser class} クラスの \menu{DOCUMENTATION} プロトコルを参照して下さい。

\begin{faq}
パーサを書くための何かいいツールはありますか?
\end{faq}
\answer
\ind{SmaCC}(the Smalltalk Compiler Compiler)を使って下さい。その場合少なくともSmaCC-lr.13以降をインストールする必要があります。
\url{http://www.squeaksource.com/SmaccDevelopment.html} からロードできます。
ここにはとても良くできたオンラインチュートリアルかあります:
\url{http://www.refactory.com/Software/SmaCC/Tutorial.html}

\begin{faq}
パーサを書くためには SqueakSource の SmaccDevelopment にあるどのパッケージをロードすればいいですか?
\end{faq}
\answer
\ind{SmaCCDev} の最新バージョンをロードして下さい。ランタイムも含まれています
(SmaCC-Development は \squeak 3.8 用です)。

%=================================================================
\ifx\wholebook\relax\else\end{document}\fi
%=================================================================

%\begin{faq}
%How do I run a headless image with a file argument?
%\end{faq}
%\answer
%Right now you can't do it with the MacOSX VM.

%\begin{faq}
%How do I find out which methods access the Smalltalk dictionary?
%\end{faq}
%\answer
%???

%\begin{faq}
%How do I get the tree view of an AST?
%\end{faq}
%\answer
%???

%\begin{faq}
%How do I browse all references to a given class?
%\end{faq}
%\answer
%???
