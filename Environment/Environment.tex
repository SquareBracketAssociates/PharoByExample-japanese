% $Author$
% $Date$
% $Revision$

% HISTORY:
% 2006-10-24 - Pollet started
% 2006-12-09 - Andrew adds material
% 2007-08-30 - Andrew completes first draft
% 2007-08-31 - Oscar edits
% 2007-09-07 - Stef corrections
% 2007-10-08 - Cassou corrections
% 2010-03-05 - Alexandre minor correction (thanks Ralph Boland)

%=================================================================
\ifx\wholebook\relax\else
% --------------------------------------------
% Lulu:
	\documentclass[a4paper,10pt,twoside]{book}
	\usepackage[
		papersize={6.13in,9.21in},
		hmargin={.75in,.75in},
		vmargin={.75in,1in},
		ignoreheadfoot
	]{geometry}
	\input{../common.tex}
	\pagestyle{headings}
	\setboolean{lulu}{true}
% --------------------------------------------
% A4:
%	\documentclass[a4paper,11pt,twoside]{book}
%	\input{../common.tex}
%	\usepackage{a4wide}
% --------------------------------------------
    \graphicspath{{figures/} {../figures/}}
	\begin{document}
	% \renewcommand{\nnbb}[2]{} % Disable editorial comments
	\sloppy
\fi
%=================================================================
\chapter{\pharo のプログラミング環境}
\chalabel{env}

% Email 2009-10-21
\hw{The fist paragraph talks about error handling. Maybe it is a good idea to reference point 1.4 for the problems related with error handling. I know that forward references are not good, but changing the order to talk about error codes first is a big change. \\
Just as a comment, in my class of exceptions, I start talking about how to write safe code and talk about error code as a common technique. Then we discuss about the problems of error code handling (as you do later in the chapter) and then I make them think about how to "remove" all that code duplication... the solution then is obviuos, the problem needs to be reified and a object model has to be created to solve the problem... then we talk about whose responsible for what so, processes, contexts and blocks came into play and we deduce the final model together... Anyway, the idea of seen exceptions as a model that replaces error codes makes a lot of sense to them and help them understand exceptions from the conceptual point of view \\
- Section 1.4: it says: "If Smalltalk did provide exception-handling".
I think it should say: "If Smalltalk did NOT provide exception-handling" \\
- Something that is missing but I don't know if it is one goal of the chapter is some words about whose responsible to signal exceptions (the objects at the end of the tree call) and whose responsible to handle exceptions (the objects at the top of the tree call) \\
Section 1.7 sounded to me a little out of order... because you need to talk about defaultAction but you have not talked about how handlers are found yet... I would move this secction to the end, as an example of exception usage \\
- Page 12, point 4: it says: "send it value: E". I think it should say "send TO it value: E". I also would write "value: E" in italics because it makes more clear that you are talking about a message. \\
- Section Nested Exceptions: "Exception handlers are outside of their own scope" For me is confusing... maybe "Exception handlers do not handle exceptions signaled during its evaluation" or something like that?
The first example of Nested Exceptions does:
\ct{result := [[ Error signal ....]}\\
But nowhere before is said what \#on:do: returns...  \\
- Section 1.9: Does it make sense to talk about abandon? It is the same as return... 
Section 1.10: I would put section 1.10 as part of 1.9. I mean, it is the same from the organization point of view as Retry Execution, etc. \\
- Page 16, second example: ResumableLoader>>readOptionsFrom:aStrem. A the end of the method does "aStream close". I think it should not close the stream, that is responsibility of the one that opened it and passed it as parameter.
Page 16, third example, the "stream close" should be inside an \#ensure: :-) \\
- I would move the Deprication example to a different section. I think it is better to finish with how exception can be handle and then give examples. \\
- About deprication, why is it an exception? why not having a global object to wich one can send the message \#deprecated:on:in: ? my feeling is that there is no need to use exceptions for this... even more, I'm starting to think that they are a bad example of exception usage... \\
- Section 1.11: I would move it inside section 1.9, at the same level as Retry excecution, Returning execution, etc.
Section 1.12: Same as section 1.11. I would put Resending exceptions as part of 1.9
Section 1.13: Same as section 1.11 (does it make sense to have outer? I think it only adds confusion... just a coment) \\
- Page 23: It seems to me that the second example is not doing what it is expected. The handler variable references the doit closure, not the handler. I think it should be:
\ct{| exception handler |
[ exception := thisContext sender at: 1. 
1/0 ]
on: Error
do: [ :ex | handler := thisContext ].
{ exception . handler } explore}
I would also suggest to send the message \#at: 1 to the handler \\
- Section 1.16: I don't like the title... Specialized Exceptions or Specific Exceptions ?
I think a good discussion about when to subclass an exception and when to reuse an exception is valuable. Something to help avoid a class explosion of exceptions... That is an interesting problem... \\
}

この章の目的は、\pharo のプログラミング環境を使い、どのようにプログラムを開発するかを示すことです。既にBrowserを使いメソッドやクラスをどのように定義するかについて見てきたはずです。この章ではBrowserのその他の機能について示し、他のBrowserのいくつかについて紹介しましょう。

もちろん作ったプログラムがが期待した動作をしないことを頻繁に見つけるかも知れません。\pharo には優れたdebuggerがあります。もっとも役立つツールではありますが、最初は使うときに悩むかも知れません。デバッグのセッションで練習し、debuggerの幾つかの機能について実演します。

\st 固有の特徴の1つにプログラミングをしているとき、静止したプログラムテキストの世界ではなく、生きている物としての世界に住んでいる気持ちになります。このことにより、プログラミング中に非常に素早いフィードバックを得ることが可能で、このフィードバックにより更に生産性が上がります。生きている物を見、実際に変更するために2つのツールがあります。それは\emph{inspector}と\emph{explorer}です。
ファイルやテキストエディタではなく、生きている物の世界上でのプログラミングの結果、\st イメージファイルからプログラムを取り出すため、何か明示的なことをしなければいけません。それを行うための\emph{fileout}や\emph{change set}といった古い方法も全ての\st の方言でサポートされています。それらの方法は基本的に他のシステムに読み込むことができるコード化されたテキストです。\pharo 上でプログラムを取り出すための新しい方法は、コードをサーバー上でバージョン管理されたリポジトリーにアップロードすることです。Monticelloと呼ばれるツールを使うことでアップロードすることができます。この方法は特にチームで作業する時に、よりパワフルで効果的です。
\seeindex{change set}{file, filing out}
\index{file!filing out}

%Finally, you may find a bug in \pharo as you work; we explain how to report bugs, and how to submit bug fixes.
%\ab{Or I would, if I knew how.   We should do this, or remove the paragraph.}

%=========================================================
\section{概要}
\seclabel{overview}

\st や現在のグラフィカルインターフェースは一緒に発展してきました。\st の最初の公開リリースである1983年以前でさえ、\st は独自のグラフィカルな開発環境を提供していました。そして全ての\st の発展はその環境上で発展してきました。では今から\pharo の主なツールを見ていきましょう。

\begin{itemize}
	\item{\menu{Browser}}は中心的な開発ルールです。これを使いクラスやメソッドを作成、定義、構築することができます。これを使うことで全てのクラスライブラリーを閲覧することもできます。分割保存されたソースコードを使う他の環境とは違い、\st では全てのクラスやメソッドはイメージファイルに含まれます。
	\index{Browser}

	\item{\menu{Message Names}}ツールを使うことにより、特定のセレクターもしくは部分文字列を含んだセレクターで全てのメソッドを探すことができます。
	\index{message name finder}
	
	\item{\menu{Method Finder}}ツールはメソッドを探すのに役立ちます。ただし何と呼ばれるのかと同様に何を\emph{する}のかによりメソッドを検索します。
	\index{method finder}
	
	\item{\menu{Monticello Browser}}はMonticelloパッケージからコードを読み込んだり、保存したりする出発点です。
	
	\item{\menu{Process Browser}}は\st 上で実行されている全てのプロセス(スレッド)を見るのに役立ちます。
	\index{process Browser}
	
	\item{\menu{Test Runner}}はSUnitテストを実行、デバッグすることができます。詳細については\charef{SUnit}を参照して下さい。
	\index{Test Runner}
	\index{SUnit}
	
	\item{\menu{Transcript}}は\glbind{Transcript}ストリームをそのウィンドウに出力します。これは\secref{transcript}で既に述べた通り、ログメッセージを書くのに役立ちます。
	
	\item{\menu{Workspace}}は入力することができるウィンドウです。 これは色々な目的に使えます。最もよく使われるのは\st 式を打ち込み、\menu{do it}で実行することです。\ind{workspace}の使い方は\secref{transcript}にも図解されています。
\end{itemize}

\menu{Debugger}は名前から判断できる明白な役目を持っています。しかし他のプログラミング言語のdebuggerと比較し、より重要な役目を担っていることに気づくでしょう。\st では\ind{debugger}上で\emph{プログラム}ができるのです。debuggerはメニュー画面から起動しません。通常、失敗するテストプログラムを実行したり、\short{\textbf{.}}と打ち込むことにより起動しているプロセスを中断したり、コード中に\lct{自己停止}のための式を入れることで、debuggerを起動します。
\index{process!interrupting}

%=========================================================
\section{Browser}
\seclabel{browser}

多くの異なったclass Browserが\st の歴史と共に長年にわたり開発されました。\pharo では様々なビューを1つのBrowserに統合し提供することでこの長年にわたる物語を簡単なものにしました。
\figref{SystemBrowser0}は最初にBrowserを開いたときに表示されるものです。\footnote{もしBrowserが\figref{classBrowser}のように表示されない場合はデフォルトのBrowserを変更する必要があるかも知れないことを思い出して下さい。\faqref{packagebrowser}参照}

\begin{figure}[htbp]
   \centering
   \ifluluelse
	 {\includegraphics[width=\textwidth]{SystemBrowser0} }
	 {\includegraphics[width=0.7\textwidth]{SystemBrowser0} }
   \caption{Browser}
   \figlabel{SystemBrowser0}
\end{figure}

Browser上部の4つの小さなペインはシステム内メソッドの階層ビューを表しています。これはディスク上のファイルを見るためのNeXTstepの\textit{File Viewer}やMac OS Xの\textit{Finder}のカラムモードと同様のビューを提供しています。左端のペインはクラスの\emph{パッケージ}リストです。その内の1つ(例えば\scat{Kernel})を選択するとすぐ右隣にあるペインにパッケージ中のすべてのクラスが表示されます。
% \on{I adopted the spelling of NeXTstep recommended by wikipedia}

\begin{figure}[htbp]
   \centering
   \ifluluelse
	   {\includegraphics[width=\textwidth]{SystemBrowser1} }
	   {\includegraphics[width=.7\textwidth]{SystemBrowser1} }
   \caption{Browserで\ct{Model}クラスを選択
   \figlabel{SystemBrowserModel}}
\end{figure}

同様に左から2番目のペインでクラスの中の1つ、例えば\menu{Model}(\figref{SystemBrowserModel}参照)を選択すると左から3番目のペインに、そのクラスに対して定義された\emph{プロトコル}が表示されます。すると仮想プロトコル\prot{-{}-all-{}-}がデフォルトで選択されます。プロトコルはメソッドをカテゴリー化するためのもので、概念的に首尾一貫して細分化することにより、クラスを見つけたりその振る舞いを考えたりするのを助けてくれます。
左から4つめのペイン選択したプロトコルで定義された全てのメソッド名が表示されます。メソッド名を選択すると、そのメソッドのソースコードがBrowser下部にある大きなペインに表示されます。そのペインではソースコードを閲覧したり、編集したり、編集後のソースコードを保存したりすることができます。\menu{Model}クラスから,\protind{dependents}プロトコル、そして\menu{myDependents}へと選択すると、\figref{SystemBrowserMyDependents}のように表示されるはずです。
\protindex{all}
\cmindex{Model}{myDependents}

\begin{figure}[htbp]
   \centering
   \ifluluelse
	   {\includegraphics[width=\textwidth]{SystemBrowserMyDependents}}
	   {\includegraphics[width=0.7\textwidth]{SystemBrowserMyDependents}}
   \caption{\ct{Model}クラスの\ct{myDependents}メソッドをBrowserで表示
   \figlabel{SystemBrowserMyDependents}}
\end{figure}

Mac OS X の\emph{Finder}とは違い、Browser上部4つのペインはディレクトリではありません。
クラスやメソッドは\st 言語の一部であるのに対し、パッケージとプロトコルはそうではありません。パッケージとプロトコルはそれぞれのペインに表示する必要がある情報量を制限するのに役立てるために導入されたものです。たとえば、もしプロトコルがなければ、選択したクラスの全てのメソッドのリストをBrowserに表示するでしょう。多くのクラスでは、このリストは大きすぎて効率良く操作できません。 
\index{Mac OS X Finder}

このことにより、新パッケージや新プロトコルを作る方法は、新クラスや新メソッドを作る方法と異なっています。新パッケージを作るためにはパッケージペインで\actclick し\menu{new package}を選択します。新プロトコルを作るためにはプロトコルペインで\actclick し\menu{new protocol}を選択します。
ダイアログの中に新パッケージや新プロトコルの名前を入力すれば完了です。その作業後はパッケージやプロトコルの名前とその内容以外何も表示されないはずです。
\index{package!creating}

\begin{figure}[htbp]
   \centering
   \ifluluelse
	   {\includegraphics[width=\textwidth]{SystemBrowserClassCreation}}
	   {\includegraphics[width=0.7\textwidth]{SystemBrowserClassCreation}}
   \caption{クラス作成テンプレートをBrowserに表示
   \figlabel{SystemBrowserClassCreation}}
\end{figure}

新クラスや新メソッドを作るのとは対照的に、新パッケージや新プロトコルを作成するときには何らかの\st のコードを書かなくてはいけません。(左端のぺいんにある)現在選択されているパッケージをクリックするとBrowserペインのボタンが、クラス作成テンプレートを表示します(\figref{SystemBrowserClassCreation})。テンプレートを編集し新クラスを作ります。\ct{Object}の箇所を、今すでにありサブクラスを作りたいと思うクラスに置き換えて下さい。\ct{NameOfSubclass}の箇所を、新サブクラスに置き換えて下さい。もし既に決めているのであればインスタンス変数名を入れてください。新クラスのためのカテゴリーは、デフォルトでは現在選択しているパッケージのカテゴリーとなります\footnote{パッケージとカテゴリーが正確には同じものではないことを思い出して下さい。\secref{packages}にてそれらの関係について詳しく見ていきます}。しかしこれはもし望むのであれば変更することはできます。もし作りたいサブクラスの元となる既存のクラスを既にBrowserで見ているのならば、クラスペインを\actclick し\menu{class templates \ldots \go subclass template}を選択することにより、少しだけ初期化処理が違いますが、同じテンプレートを得ることができます。また既存のクラス定義を編集し、クラス名を何か新しいものに変更することでも、テンプレートの替りにすることができます。全てのケースにおいて、新しい定義をacceptしたときに、(\ct{#}から始まる名前の)新クラスが(対応するメタクラスとして)作られます。クラスを作ることは、そのクラスを参照するグローバル変数を作ることでもあります。それは全ての既存のクラスをそれらの名前を参照することができる理由です。
\index{class!creation}
\index{browser!defining a class}

新クラス名が(\ie \ct{#}で始まる)\lclsind{シンボル}としてクラスの作ったテンプレートの中に現れる理由や、クラスが作られた後、(\ie \ct{#}のない)識別子としてクラス名を使うことでそのクラスをコードから参照することができます理由が理解できたでしょうか?

新メソッドを作るプロセスも同様です。最初にメソッドを作りたいクラスを選択します、その後プロトコルを選択します。\figref{SystemBrowserMethodTemplate}にあるように、Browserは編集することのできるメソッド作成テンプレートを表示します。
\index{method!creation}
\index{browser!defining a method}

\begin{figure}[htbp]
   \centering
   \ifluluelse
	   {\includegraphics [width=\textwidth]{SystemBrowserMethodTemplate}}
	   {\includegraphics[width=0.7\textwidth]{SystemBrowserMethodTemplate}}
   \caption{メソッド作成テンプレートをBrowserに表示
   \figlabel{SystemBrowserMethodTemplate}}
\end{figure}

%---------------------------------------------------------
\subsection{コード空間を渡り歩く}

Browserはコードを探したり分析するためのいくつかのツールを提供します。様々なコンテキストメニューを\actclick することにより、もしくは最も頻繁に使われるツールの場合はキーボードショートカットにより、これらのツールを使うことができます。

\subsubsection{新しいBrowserウインドウを開く}
\seclabel{browsing}

いくつかのBrowserウインドウを開きたいと思うときがあると思います。コードを書いているときは少なくとも2つ: 1つは打ち込んでいるメソッドのため、もう1つはシステム周りがどのように動いているのかを見るためです。クラス名上で何かテキストを選択し\ind{キーボードショートカット}\short{b}を使うことでBrowserを開くことができます。
\index{browser!browse button}
\index{keyboard shortcut!browse it}

\dothis{次のことを試してみてください: workspaceウインドウ内でクラス名(たとえば\ct{Morph})を打ち込み、それを選択し\short{b}を押して下さい。この方法はよく役に立ち、どのテキストウィンドウでも使えます。}

\subsubsection{センダーとメッセージ作成者}
\seclabel{sendersImplementors}

\index{browser!senders}
メソッドペインで\menu{browse \ldots \go senders (n)}を\actclick すると、選択したメソッドを使用している全てのメソッドのリストが表示されます。Browserを開き\ct{Morph}を選択し、メソッドペインの中の\mthind{Morph}{drawOn:}メソッドをクリックすると、\ct{drawOn:}本体がBrowser下部に表示されます。 \menu{senders (n)} (\figref{SendersOfDrawOn})を選択すると、\ct{drawOn:}を一番上の項目として下に向かって\ct{drawOn:}が送る全てのメッセージ(\figref{SendersOfDrawOn2})のメニューが表示されます。 このメニューの中の項目を選択すると、選択したメッセージ(\figref{CanvasDraw})を送る、イメージファイル内の全メソッドのリストをbrwoserで開きます。

%\begin{figure}[htb]
%\begin{minipage}[b]{0.74\textwidth}
%\centerline {\includegraphics[width=\textwidth]{SendersOfDrawOn}}
%\caption{The \menu{senders (n)} menu item.\figlabel{SendersOfDrawOn}}
%\end{minipage}
%\hfill
%\begin{minipage}[b]{0.24\textwidth}
%\centerline {\includegraphics[width=\textwidth]{SendersOfDrawOn2}}
%\caption{Choose senders of which message.\figlabel{SendersOfDrawOn2}}
%\end{minipage}
%\end{figure}

\begin{figure}[htb]
\centerline {\includegraphics[width=\textwidth]{SendersOfDrawOn}}
\caption{\menu{senders (n)}メニュー項目。\figlabel{SendersOfDrawOn}}
\end{figure}

\begin{figure}[htb]
\centerline {\includegraphics[width=0.4\textwidth]{SendersOfDrawOn2}}
\caption{メッセージのセンダーを選択。\figlabel{SendersOfDrawOn2}}
\end{figure}

\menu{senders (n)}の中の``n''はキーボードショートカットがあることを伝えています。メッセージのセンダーを探すためのキーボードショートカットは\short{n}となります。この方法は、\emph{どの}テキストウインドウでも使えます。

\dothis{コードペインにある``drawOn:''テキストを選択し、\short{n}を押すと、すぐに\ct{drawOn:}のセンダーが表示されます。}

\begin{figure}[htbp]
	\begin{center}
   \ifluluelse
		{\includegraphics[width=\textwidth]{CanvasDraw}}
		{\includegraphics[width=0.7\textwidth]{CanvasDraw}}
	\end{center}
	\caption{Senders Browserは \ct{Canvas>>>draw}メソッドが\ct{drawOn:}メッセージをその引数としていることを表しています。\figlabel{CanvasDraw}}
\end{figure}

\ct{AtomMorph>>>drawOn:}にある\ct{drawOn:}センダーを探すとき、これはスーパークラスが\subind{super}{送信}していることがわかるでしょう。そのためこのメソッドは\ct{AtomMorph}のスーパークラスが実行していることがわかります。これは何のクラスなんでしょう? ~ \menu{browse \go hierarchy implementors}を\actclick するとそれが\ct{EllipseMorph}クラスであることがわかります。 
\index{browser!hierarchy button}

\figref{CanvasDraw}に示されている、リストにある6番目のセンダー、\ct{Canvas>>>draw}、を見てみましょう。このメソッドは\ct{drawOn:}を、それが引数として受け渡される全てのオブジェクトや、任意のクラスのインスタンスとなる可能性があるオブジェクトに対し、送信していることがわかります。
データフロー解析はいくつかのメッセージのレシーバークラスを理解する助けになります。しかし一般的に、Browserを使い、どのメッセージ送信がどのメソッドの実行につながるのかを知る簡単な方法はありません。このような理由から``senders'' Browserは、選択したセレクターと一緒にメッセージの全センダー名を連想させるものを正確に示します。
senders Browserはそれだけでなく、どのようにメソッドを\emph{使う}のかを理解する必要があるときに、即座にそのメソッドの使用例へ導いたりするなど、非常に役立ちます。
同じセレクターによる全てのメソッドは同じように使われるべきであるため、定められたメッセージの使い方は似ています。
\index{browser!senders}

\index{browser!implementors}
implementors Browserも似たような動作をします。しかしメッセージのセンダーリストの替りに、同じセレクターを使ったメソッドによって実装した全クラスのリストを表示します。これを見るためにはメソッドペインにある\lct{drawOn:}を選択し、そして\menu{browse \go implementors (m)}を選択して下さい(もしくはコードペインにある``drawOn:''テキストを選択し、\short{m}を押して下さい)。そうすればメソッドのリストウインドウが表示され、スクロールすると90余りのクラスが\ct{drawOn:}メソッドを実装していることがわかります。非常に多くのクラスがこのメソッドを実装しているということだけが驚くことではありません。\ct{drawOn:}は、それ自身をスクリーン上に描写することができる全てのオブジェクトによって理解されるメッセージなのです。
\subsubsection{メソッドのバージョン}
\seclabel{versions}

メソッドの新しい\subind{method}{version}を保存すると、古いものは失われます。\pharo は全ての古いバージョンを保存し、異なるバージョンを比較したり、古いバージョンに戻したり(``復帰'')することができます。
\begin{figure}[btp]
   \centering
   \includegraphics[width=\textwidth]{Versions}
   \caption{\ind{Versions Browser}が2つの\ct{TheWorldMenu>>>buildWorldMenu:}メソッドを表示}
   \figlabel{buildWorldMenuVersions}
\end{figure}
\menu{browse \go versions (v)}メニューアイテムにより、選択したメソッドの連続して修正することができます。\figref{buildWorldMenuVersions}では\ct{buildWorldMenu:}メソッドの2つのバージョンを示しています。

\index{browser!versions button}
表示されている上部のペインには、そのメソッドのそれぞれのバージョンの1行が表示されています。そこではこのメソッドを書いたプログラマーのイニシャル、保存した日付、クラス名とメソッド名、そしてそれらが定義されているプロトコルを尋ねてきます。
現在の(アクティブな)バージョンはリストの一番上にあります。どちらのバージョンが選択されてもそれは下のペインに表示されます。
% If the \menu{diffs} checkbox is selected, as it is in \figref{mouseUpVersions}, the display also shows the differences between the selected version and the one immediately older. 
選択されたメソッドと現在のバージョンの違いを表示したり、選択したバージョンを復帰するためのボタンも用意されています。
% The \menu{prettyDiffs} checkbox is useful if there have been changes to layout: it pretty-prints both versions before differencing, so that the differences that are displayed exclude formatting changes. 

\ind{Versions Browser}があるのはコードを提供するときに、もはや必要ないので削除したほうが良いかどうかを悩まなくても済むようにするためにあります。
もしそのコードが\emph{本当に}必要だと分かったら、いつでも古いバージョンに復帰することができます。また古いバージョンから必要なコードの断片を得、別のメソッドに貼り付けることもできます。
今使っているバージョンで、もはや必要のないコードをコメントアウトするのは悪い習慣です。なぜならそれによって現在のコードが読みにくくなるからです。Smalltalkerは読みやすいコードを非常に高く評価します。

\hint{もしメソッド全体を削除した後、それを戻そうと決めた場合はどうすればいいでしょう? change setの中に削除したものを見つけることができます。そこで\actclick することにより以前のバージョンを見るかどうかを確認することができます。
Change Set Browserは\secref{env:changeSet}に表示されています。}

\subsubsection{メソッドのオーバーライド}
\seclabel{overriding}

inheritance Browserは、表示されたメソッドによって全てのオーバーライドされたメソッドを表示します。それらどのように動作するかを見るには、Browserにある\cmind{ImageMorph}{drawOn:}メソッドを選択してください。
(\figref{OBinheritanceBrowser})メソッドの隣にある三角形のアイコンに注目してください。
上向きの三角形は\ct{ImageMorph>>>drawOn:}が継承されたメソッドであること(\ie \ct{Morph>>>drawOn:})を教えてくれます。そして下向きの三角形はそれがサブクラスによってオーバーライドされたことを教えてくれます。(アイコンをクリックすることで、これらのメソッドを操作することもできます。)
では\menu{browse \go inheritance}を選択してください。
inheritance Browserによりオーバーライドされたメソッドの階層関係を見ることができます。(\figref{OBinheritanceBrowser}参照)

\begin{figure}[btp]
	\begin{center}
   \ifluluelse
		{\includegraphics[width=\textwidth]{OBInheritanceOverriding}}
		{\includegraphics[width=0.7\textwidth]{OBInheritanceOverriding}}
	\end{center}
	\caption{\ct{ImageMorph>>>drawOn:} とそれがオーバーライドしたメソッド。選択したメソッドの兄弟関係がリストに表示されている。}
	\figlabel{OBinheritanceBrowser}
\end{figure}

\subsubsection{階層ビュー}
\seclabel{hierarchy}

標準では、Browserは左端のペインにパッケージのリストを表示します。
しかしクラスの階層ビューを入れ替えることは可能です。
関心のある特定のクラスを選択してください。例えば\ct{ImageMorph}を選択し、\button{hier.}ボタンをクリックしてください。
選択したクラスの全てのスーパークラスとサブクラスの階層関係が、左端のペインで見ることができます。
2つ目のペインは選択したクラスで実装されているメソッドのパッケージがリスト表示されています。
%The \button{hierarchy} button opens a \ind{hierarchy browser} on the current class; this
%browser can also be opened by using the \menu{browse hierarchy} menu item in the class pane.
%The hierarchy browser is similar to the browser, but instead of displaying the categories and the classes in each category, it shows a single list of classes, indented to represent inheritance.
%The category of the selected class is displayed in the small annotation pane at the top of the browser.
\figref{hierarchyBrowser}では、hierarchy Browserが\clsind{ImageMorph}のスーパクラスが \clsind{Morph}であることを示しています。
\index{browser!hierarchy button}

\begin{figure}[btp]
	\begin{center}
	\ifluluelse
		{\includegraphics[width=\textwidth]{hierarchyBrowser}}
		{\includegraphics[width=0.7\textwidth]{hierarchyBrowser}}
	\end{center}
	\caption{\ct{ImageMorph}の階層ビュー。}
	\figlabel{hierarchyBrowser}
\end{figure}

\subsubsection{参照している変数を見つける}
\seclabel{variables}

\index{browser!variables}
クラスペインであるクラスを\actclick し、\menu{browse \go chase variables}を選択することで、インスタンス変数やクラス変数が使われている箇所を見つけることができます。

\emph{chasing Browser}を使うことで、全てのインスタンス変数とクラス変数、次にこれらのアクセサーを送るメソッド、そしてその他と順番にアクセサーを見ることができます(\figref{chasingBrowser})。

\begin{figure}[btp]
	\begin{center}
	\ifluluelse
		{\includegraphics[width=\textwidth]{chasingBrowser}}
		{\includegraphics[width=0.7\textwidth]{chasingBrowser}}
	\end{center}
	\caption{\ct{Morph}のためのchasing Browser。}
	\figlabel{chasingBrowser}
\end{figure}

% The menu also includes \menu{inst var defs}, which shows the subset of the instance variable references that assign to the variable. 
%Once you click on the button or select the menu item, you will be presented with a dialog that invites you to choose a variable from all of the variables defined in the current class, and all of the variables that it inherits.  The list is in inheritance order; it can often be useful to bring up this list just to remind yourself of the name of an instance variable.  If you click outside the list, it will go away and no variable browser will be created.

%Also available by \actclick{ing} the class pane is \menu{class vars}, which opens an inspector showing the class variables of the current class \emph{and their values}, and \menu{class refs (N)} which displays a list of all of the methods that directly reference the current class.

\subsubsection{ソース}
\seclabel{sources}

\index{browser!view}
メソッドペインを\actclick することで利用可能な\menu{various \go view \ldots}メニューアイテムにより、sourceペインで選択したメソッドをどのように見せるかを可能にする``how to show''メニューを表示します。\menu{source}のソースコード、\menu{prettyPrint}表示されたソースコード、そして\menu{byteCode}とバイトコードから\menu{decompile}されたソースコードといったオプションも選択できます。
% The label on the button changes if you select one of the other modes.  There are other options too; if you let the mouse linger over the names, a help balloon will appear.  Try some of them.
\index{method!pretty-print}
\index{method!decompile}
\index{method!byte code}

``how to show''メニューで選択した\menu{prettyPrint}はセーブする前に整形出力したメソッドと同じもの\footnote{\menu{pretty print (r)}はメソッドペインの最初のメニューアイテムもしくはcodeペインの真ん中下側にあります。}ではないことに注意してください。このメニューは、Browserがどのように表示するかを単に操作するだけで、システムに保存されたコードに対しては何の影響も与えません。2つのBrowserを開き、一方のBrowserで\menu{prettyPrint}を選択し、もう一方のBrowserで\menu{source}を選択することにより、このことを確認することができます。実際、2つのBrowserで同じメソッドを一方で\menu{byteCode}を選択し、もう一方で\menu{decompile}選択することで\pharo の仮想マシンのバイトコード化命令セットについて学ぶことができます。

\subsubsection{リファクタリング}

contextualメニューから多くの標準的なリファクタリングができます。4つのうちのどのペインを\actclick しても、現在リファクタリングすることが可能なものを見ることができます。\figref{refactoring}参照。

以前は、リファクタリングをするためにはrefactoring Browserと呼ばれる特殊なBrowserを使う必要がありましたが、現在はどのBrowserからもアクセスすることができます。

\begin{figure}[btp]
	\begin{center}
	\ifluluelse
		{\includegraphics[width=\textwidth]{refactoring}}
		{\includegraphics[width=0.7\textwidth]{refactoring}}
	\end{center}
	\caption{リファクタリング操作。}
	\figlabel{refactoring}
\end{figure}

%---------------------------------------------------------
\subsection{Browserのメニュー}

Browserペインを\actclick することで多くの追加機能を利用することができます。たとえメニューアイテム上のラベルが同じであっても、その\emph{意味}は文脈に依存します。例えばパッケージペイン、クラスペイン、プロトコルペイン、そしてメソッドペイン全てが\menu{File out}メニューアイテムを持っています。しかしそれらは違う動作をします。パッケージペインの\menu{File out}メニューはパッケージ全てをファイル出力します。クラスペインの\menu{File out}メニューは全てのクラスをファイル出力します。プロトコルペインの\menu{File out}メニューは全てのプロトコルをファイル出力します。そしてメソッドペインの\menu{File out}メニューは表示されたメソッドだけをファイル出力します。このことはすぐに分かることのように見えるかもしてませんが、初心者にとっては混乱の元になりえます。
\index{file!filing in}
\index{file!filing out}

おそらくパッケージペインの中で最も便利なメニューアイテムは\menu{find class\ldots (f)}です。カテゴリは活発に開発しているコードのために役立ちますが、システム全体のカテゴライズ化について、ほとんどの人は知りません。また、探しているクラスがどのパッケージにあるのか推測するために、クラス名の最初の数文字を打ち込んだ後に続けてshort{f}と打ち込んだ方がはるかに速いということもほとんどの人は知りません。\menu{recent classes\ldots}はたとえクラス名を忘れてしまったとしても、最近閲覧したクラスへ素早く戻るのに役立ちます。
\index{class!finding}
\index{class!recent}

特定のクラスやメソッドも、Browserの左上にある検索ボックスにその名前を打ち込むことで探すことができます。リターンキーを打ち込むと検索が始まりシステムを一時的に停止し、その後検索結果を表示します。検索語の前に\ct{#}を付けることで、クラスを参照しているものや、メッセージを送信したクラスをさがすことができることに注目してください。
%To browse to a particular method, you can 
%In the class pane, there are two menu items \menu{find method} and \menu{find method wildcard\ldots} that can be useful if you want to browse a particular method.  
もしあるクラスの特定のメソッドを探しているのなら、\prot{-{}-all-{}-}プロトコルで閲覧する方が速い時があります(これはデフォルトです)。メソッドペインにマウスを移動し、探しているメソッド名の最初の文字を打ち込んでください。この方法により探しているメソッド名が表示されるよう、ペインがスクロールします。
\index{method!finding}
\protindex{all}

\dothis{\cmind{OrderedCollection}{removeAt:}で2つの操作方法を試してください。}

その他多くのオプションがメニューにはあります。これを使うことでBrowser上の作業や何がどこにあるのか探すのに費やす時間を減らすことができます。
\dothis{クラスペインメニューの\menu{Browse Protocol}, \menu{Browse Hierarchy}, そして\menu{Show Hierarchy}を実行した結果を比較してください。}

%---------------------------------------------------------
\subsection{プログラムでBrowserを操作}

\glbind{SystemNavigation}クラスはシステム周りを操作するために役に立つユーティリティメソッドのいくつかを提供しています。その機能の大半は古典的なBrowserにより提供され、\ct{SystemNavigation}によって実装されます。\index{browsing programmatically}

\dothis{
workspaceを開き、\ct{drawOn:}:センダーを見るために以下のコードを評価してください。}
\begin{code}{}
SystemNavigation default browseAllCallsOn: #drawOn:
\end{code}
特定のクラスのメソッドへのセンダーを検索を制限するためには次のようにコードを評価してください。
\begin{code}{}
SystemNavigation default browseAllCallsOn: #drawOn: from: ImageMorph
\end{code}
開発ツールもオブジェクトなので、プログラムからそれらに完全にアクセスすることができます。そのため独自のツールを開発したり、今あるツールを必要に応じて取り込むこともできます。

\menu{implementors}メニューアイテムと同じ動作をプログラム上行うには次のように評価してください。
\begin{code}{}
SystemNavigation default browseAllImplementorsOf: #drawOn:
\end{code}

他に何が出来るのかを知るにはBrowserで\ct{SystemNavigation}クラスを探してみてください。もっと多くの操作例についてはFAQ(\appref{faq})で見つけることができます。

%=========================================================
\section{Monticello}

\secref{Monticello}において、\pharo のパッケージ作成ツール、\ind{Monticello}の概要についてお見せしました。しかしMonticelloは、その章に書かれていた以上の機能を持っています。Monticelloは\emph{パッケージ}を扱うため、Monticelloについてお伝えする前に、\ind{パッケージ}とは何なのかについて正確に説明する必要があります。

%---------------------------------------------------------
\subsection{\pharo コードを宣言型のカテゴリー化}\seclabel{packages}

\secref{categoriesPackages}において、パッケージとは多かれ少なかれカテゴリーと同等のものであると指摘しました。これからそれらの関係について詳しく見ていきましょう。
パッケージシステムは、カテゴリーとプロトコルのために開発された単純な命名規則を活用した\st ソースコードを簡単で軽量にまとめる方法の1つです。

例を使って説明しましょう。\pharo からリレーショナル型データベースを簡単に使うために命名されたフレームワークを開発していると思ってください。このフレームワークを\ct{PharoLink}と呼ぶことにします。そして作った全てのクラスを含む一連のカテゴリーを作ったとします。
例えば、\ct{'PharoLink-Connections'}カテゴリーは\ct{OracleConnection MySQLConnection PostgresConnection}を含み、 \ct{'PharoLink-Model'}カテゴリーは\ct{DBTable DBRow DBQuery}等を含みます。ただし、これらのクラスの中に作ったコード全てが含まれていないこととします。例えばSQLが扱いやすい形式にオブジェクトを変換する一連のメソッドがあるかも知れません。

\begin{code}{}
Object>>>asSQL
String>>>asSQL
Date>>>asSQL
\end{code}

\noindent
% \damien{don't we say 'belong to'?}
% \on{no -- belong in = they should be there; belong to = they are there}
これらのメソッドは、\ct{PharoLink-Connections}や\ct{PharoLink-Model}カテゴリとして同じパッケージに入っています。
しかし明らかに\ct{Object}クラス全体はパッケージの中に入っていません。そのため、クラスの残りの部分は別のパッケージに入れるとしても、特定の\emph{メソッド}を任意のパッケージに入れる方法が必要になります。
\index{package!extension}
\seeindex{extension package}{package, extension}

これらのメソッドを、prot{*PharoLink}(最初のアスタリスクに注意してください)と呼ばれる(\ct{Object}, \ct{String} \ct{Date}等)のプロトコルに置くことで、それは実現できます。\scat{PharoLink-\ldots}カテゴリーと\prot{*PharoLink}プロトコルの組み合わせにより\ct{PharoLink}と呼ばれるパッケージは形作られます。
正確に言うと以下のようにパッケージに納められるルールの事です。

\ct{Foo}と名付けられたパッケージは次のものを含みます。

\begin{enumerate}		\seclabel{packageRules}
	\item{} \scat{Foo}カテゴリ内もしくは\scat{Foo-}で始まる名前のカテゴリ内にある、クラスの全ての\emph{クラス定義}
	\item{} \label{env:extensions} \prot{*Foo}や\prot{*foo}\footnote{この比較を行うときには、名前の中にある大文字小文字は無視されます。}という名のプロトコル内もしくは\prot{*Foo-}や\prot{*foo-}で始まる名前のプロトコル内にある、\emph{任意のクラス}内の全ての\emph{メソッド}
	\item{} \scat{Foo}カテゴリー内もしくは\scat{Foo-}で始まる名前のカテゴリー内にある、クラス内の全ての\emph{メソッド}。ただし\prot{*}で始まる名前のプロトコル内にあるメソッドは\emph{除く}。
	
\end{enumerate}
\noindent
これらのルールの結果、各クラスの定義と各メソッドは、厳密に1つのパッケージに属するということになります。
最後のルールにある\emph{例外}があることを除き、これらのメソッドは他のパッケージに属している必要があります。ルール\ref{env:extensions}において無視するケースの理由は、慣習的にプロトコル名が通常は(必ずというわけではないですが)小文字(空白ふくまない)であることです。その一方でカテゴリー名はCamelCase(空白を含まない)を使います。
\index{camelCase}

\ct{PackageInfo}クラスはこれらのルールに従い実装されています。その結果の感触はこのクラスを試すことで得られます。

\dothis{workspace上で以下の式を評価して下さい。}

\begin{code}{}
mc := PackageInfo named: 'Monticello'
\end{code}

このパッケージについて深く考える事ができるようになりました。例えばworkspaceペイン上で\ct{mc classes}を表示することにより、Monticelloパッケージを構成するクラスの長いリストが返ります。

\ct{mc coreMethods}により、これらのクラスにある全てのメソッドのための\ct{MethodReference}のリストが返されます。
\ct{mc extensionMethods}はおそらく最も興味深い検索語でしょう。つまり\ct{mc extensionMethods}は\lct{Monticello}クラスを除き、\ct{Monticello}パッケージに含まれる全てのメソッドを返します。

パッケージは比較的新しく\pharo に追加されたものです。しかしパッケージの命名規則は、すでに使われているものがベースとなっているため、\ct{PackageInfo}を使用し明らかに命名規則に則っていない古いコードをも分析することができます。

\dothis{\ct{(PackageInfo named: 'Collections') externalSubclasses}を使いを\lct{表示}してください。この式により\ct{Collection}パッケージに含まれ\emph{ない}、\ct{Collections}の全てのサブクラスのリストが返ります。}

%---------------------------------------------------------
\subsection{Monticelloの基礎}

\ind{Monticello}は、合衆国3番目の大統領であり「the Statute of Virginia for Religious Freedom」の著者であるThomas Jeffersonの山頂の家にちなんで名付けられました。この名前の意味はイタリア語で``小さな山''であり、イタリア語の``c''、chairの``ch''の音を使いと一般的にMonticelloと発音されます。


\begin{figure}[btp]
	\begin{center}
	\ifluluelse
		{\includegraphics[width=\textwidth]{freshMonticello}}
		{\includegraphics[width=0.7\textwidth]{freshMonticello}}
	\end{center}
	\caption{Monticello Browser.}
	\figlabel{freshMonticello}
\end{figure}

Monticello Browserを開くと、\figref{freshMonticello}に示されるように2つのペインと1列に並ぶボタンが表示されます。左側のペインには、今動かしているイメージファイル内にロードされた全てのパッケージがリスト表示されています。また、名前の後にあるカッコの中にパッケージを特定するバージョンが表示されます。

右側のペインにはMonticelloが知っているソースコードリポジトリ全体のリストが表示されている。通常はそのリストからコードを読み込みます。左側のペインにあるパッケージを選択すると、右側のペインには選択したパッケージのバージョンを含んだリポジトリだけを表示するようフィルタリングされます。
リポジトリの1つは\emph{package-cache}と呼ばれるディレクトリーで、実行されているイメージファイルのあるディレクトリーの下にあるディレクトリーです。もし遠隔地にあるリポジトリーからコードを読み込んだり、書き込んだりする場合、そのコピーはpackage-cacheにも保存されます。ネットワークが使えないときでパッケージにアクセスする必要がある場合、このpackage-cacheは役立ちます。また、Monticelloファイル(.mcz)をemailの添付などで直接渡された場合、そのファイルをpackage-cacheディレクトリに入れるのが一番手軽な方法です。
\index{package!cache}

リストに新たなリポジトリを追加するには、\button{+Repository}ボタンをクリックし、ポップアップメニューからリポジトリーの種類を選択してください。ではHTTPリポジトリーを追加してみましょう。

\dothis{Monticelloを開き、\button{+Repository}ボタンをクリック。読むためのダイアログを編集します。}
%\ab{How does one continue the $\backslash$dothis to include the code?}
%\on{Don't.  Just close the \dothis{} and follow with the code.}
\needlines{4}
\begin{code}{}
MCHttpRepository
	location: 'http://squeaksource.com/PharoByExample'
	user: ''
	password: ''
\end{code}

\begin{figure}[btp]
	\begin{center}
	\ifluluelse
		{\includegraphics[width=0.7\textwidth]{SqueakSource-PBE}}
		{\includegraphics[width=0.7\textwidth]{SqueakSource-PBE}}
	\end{center}
	\caption{Repository Browser.}
	\figlabel{SqueakSource:PBE}
\end{figure}
\noindent
その後このリポジトリをRepository Browserを上に表示するため\button{Open}ボタンをクリックします。すると\figref{SqueakSource:PBE}のように見えるはずです。左側のペインはリポジトリー内の全てのパッケージのリストです。その内の1つを選択すると右側のペインにこのリポジトリー内にある選択されたパッケージの全てのバージョンが表示されます。

その内の1つのバージョンを選択すると、\button{Browse}ボタンを押すことで(イメージファイルに読み込むことなしに)閲覧でき、\button{Load}ボタンで読み込むことができ、\button{Changes}ボタンで選択したバージョンを読み込んだことによるイメージファイルの変更点を見ることができます。 また、\button{Copy}ボタンを押すことにより任意のバージョンのパッケージをコピーすることができ、そしてそれを別のリポジトリーに書きこむこともできます。

ご覧のように、バージョンの名前には、パッケージの名前、バージョンの作成者のイニシャル、およびバージョン番号が含まれています。バージョン名は、リポジトリー内のファイル名でもあります。これらの名前を変更してはいけません、なぜならMonticelloの動作とこれらの名前は依存関係にあるからです! Monticelloのバージョンファイルは単なるzipアーカイブであり、興味があればzipツール使用してファイルを解凍できます。しかしその内容を見る一番の方法は、Monticelloを使用することです。Monticelloでパッケージを作成するには、2つのことを行う必要があります。1つはいくつかのコードを書くこと、もう1つはそれについての情報をMonticelloに伝えることです。

\dothis{\scat{PBE-Monticello}と呼ばれるパッケージを作成し、\figref{MCnewcategory}に示すように2つのクラスをその中に置いて下さい。もちろん、\ct{Object}などの実在するクラスのメソッドを作成し、\pageref{sec:packageRules}ページにあるルールを用いて独自のクラスとして同じパッケージ内に置いて下さい\,---\,\figref{MCnewmethod}参照。}

\begin{figure}[btp]
	\begin{center}
	\ifluluelse
		{\includegraphics[width=\textwidth]{MCnewcategory}}
		{\includegraphics[width=0.7\textwidth]{MCnewcategory}}
	\end{center}
	\caption{``PBE''パッケージ中の2つのクラス。}
	\figlabel{MCnewcategory}
\end{figure}

\begin{figure}[btp]
	\begin{center}
	\ifluluelse
		{\includegraphics[width=\textwidth]{MCnewmethod}}
		{\includegraphics[width=0.7\textwidth]{MCnewmethod}}
	\end{center}
	\caption{``PBE''パッケージにこれから作成される拡張メソッド。}
	\figlabel{MCnewmethod}
\end{figure}

作ったパッケージをMonticelloに伝えるために、\button{+Package}ボタンをクリックして下さい。そしてパッケージ名、この場合は``PBE''を打ち込んで下さい。Monticelloは\ct{PBE}をパッケージリストに加えます。パケージの見出しがアスタリスクでマークされ、イメージ内のバージョンがまだどのリポジトリーにも書き込まれていないことを示しています。Monticelloに2つのパッケージがあることに注目して下さい。1つは\ct{PBE}、もう1つは\ct{PBE-Monticello}と呼ばれています。\ct{PBE}は\ct{PBE-Monticello}や\ct{PBE-}で始まる他の全てのパッケージを含みます。

最初はこのパッケージに関連している唯一のリポジトリーは\figref{MC+PBE}に示されているようにpackage-cacheだけです。それで問題はないので、コードをセーブして、package-cacheに書きこんで下さい。\button{Save}ボタンをクリックすると、\figref{PBE-on}のように、セーブするパッケージのバージョンのためのログメッセージを入力するよう求められます。このメッセージをAcceptするとMonticelloはパッケージをセーブします。セーブしたことを示すため、Monticelloのパッケージペインにマークされていたアスタリスクがなくなり、バージョン番号が付与されます。

もしパッケージに変更を加える\,---\,例えばクラスの1つにメソッドを追加する\,---\,場合、変更がセーブされていないことを示すため、アスタリスクが再度表示されます。package-cache上のRepository Browserを開くと、セーブしたバージョンを選択することや、\button{Changes}ボタンや他のボタンを押すことができます。リポジトリーに新たなバージョンを保存することもできます。\button{Refresh}ボタンを押し、リポジトリービューを一度更新して下さい。すると\figref{package-cache-browser}のようになるはずです。
\index{package!cache}

\begin{figure}[tbp]
	\begin{center}
		\includegraphics[width=\textwidth]{MC+PBE}
	\end{center}
	\caption{Monticello上ではPBEパッケージはまだ保存されていない。}
	\figlabel{MC+PBE}
\end{figure}

\begin{figure}[tbp]
	\begin{center}
	\includegraphics[width=0.8\textwidth]{PBE-on}
	\end{center}
	\caption{パッケージの新バージョンについてのログメッセージを入力する。}
	\figlabel{PBE-on}
\end{figure}

\begin{figure}[tbp]
	\begin{center}
	\includegraphics[width=\textwidth]{package-cache-browser}
	\end{center}
	\caption{package-cacheに今2つのバージョンのパッケージがあります。}
	\figlabel{package-cache-browser}
\end{figure}

新たなパッケージをpackage-cache以外のリポジトリーにセーブするためには、まず最初に、必要に応じてリポジトリーを追加するなどして、Monticelloにそのリポジトリーについて知らせる必要があります。その後package-cache Repository Browser内の\button{Copy}ボタンを使いコピーして下さい。\figref{associateRepository}に示されているように、リポジトリー上で\actclick し、メニューで\menu{add to package \ldots}を選択することで、特定のパッケージとお望みのリポジトリーを関連付けることができます。1度パッケージとリポジトリーを関連付けると、Monticello Browser内のリポジトリーやパッケージを選択し\button{Save}ボタンをクリックすることにより、新しいバージョンをセーブすることができます。もちろん書き込むリポジトリーへのパーミッションは必要です。
\emphind{\sqsrc}上の\ct{PharoByExample}リポジトリーは世界中いたる所で読むことができますが、書き込むことはできません。そのためそのリポジトリーにセーブしようとしてもエラーメッセージが表示されます。
ただし、\url{http://www.squeaksource.com}にあるWebのインターフェースから\sqsrc 上に独自のリポジトリーを作り、プログラムをセーブする場所として使うことはできます。
これは、作ったコードを友人と共有するためのメカニズムとして、特に役立ちます。

\begin{figure}[tbp]
	\begin{center}
		\includegraphics[width=\textwidth]{MCaddToPackage}
	\end{center}
	\caption{パッケージに関連付けられているリポジトリーのセットをリポジトリーに追加。}
	\figlabel{associateRepository}
\end{figure}

もしパーミッションのないリポジトリーに対しセーブをしようとすると、そのバージョンは意に反しpackage-cacheに書き込まれます。

それは、リポジトリの情報を編集したり(Monticello Browserで\actclick )、別のリポジトリを選択し、package-cache browserから\button{Copy}ボタンを押すことで復帰させることができます。

%=========================================================
\section{The Inspector and the Explorer}
\seclabel{inspector} % (fold)

One of the things that makes \st so different from many other programming environments is that it is provides you with a window onto a world of live objects, not a world of static code.
Any of those objects can be examined by the programmer, and even changed\,---\,although some care is necessary when changing the basic objects that support the system.  
By all means experiment, but save your image first!

%---------------------------------------------------------
\subsection{The Inspector}

\dothis{As an illustration of what you can do with an \ind{inspector}, type  \ct{TimeStamp now} in a workspace, and then \actclick and choose \menu{inspect it}.} 
(It's not necessary to select the text before using the menu; if no text is selected, the menu operations work on the whole of the current line.
You can also type \short{i} for \menu{\textbf{i}nspect it}.)
\clsindex{TimeStamp}
\index{keyboard shortcut!inspect it}

\begin{figure}[btp]
	\begin{center}
		\includegraphics[width=\textwidth]{inspectTimeNow1}
	\end{center}
	\caption{Inspecting \ct{TimeStamp now}}
	\figlabel{inspectTimeNow1}
\end{figure}

A window like that shown in \figref{inspectTimeNow1} will appear.   
This is an inspector, and can be thought of as a window onto the internals of a particular object\,---\,in this case, the particular instance of \mbox{\ct{TimeStamp}} 
% the \mbox is here because without it, the listings macros puts a space between TimeStamp 
% and the following word, and that space happens to come out at the start of a line.
that was created when you evaluated the expression \ct{TimeStamp now}.
The title bar of the window shows the printable representation of the object that is being inspected.
If you select \menu{self} at the top of the left pane, the right pane will show the printstring of the object.
% If you select \menu{all inst vars} in the left pane, the right pane will show a list of the instance variables in the object, and the printstring for each one.  
% The remaining items in the left pane represent the instance variables; this makes it easy to examine them one at a time, and also to change them.
The left pane shows a tree view of the object, with \self at the root.
Instance variables can be explored by expanding the triangles next to their names.

The horizontal pane at the bottom of the inspector is a small workspace window.  It is useful because in this window, the pseudo-variable \ct{self} is bound to the object that you have selected in the left pane.
So, if you \menu{inspect it} on
\begin{code}{}
self - TimeStamp today
\end{code}
in the workspace pane, the result will be a \clsind{Duration} object that represents the time interval between midnight today and the instant at which you evaluated  \ct{TimeStamp now} and created the \ct{TimeStamp} object that you are inspecting.
You can also try evaluating \ct{TimeStamp now - self}; this will tell you how long you have spent reading this section of this book!

In addition to \ct{self}, all the instance variables of the object are in scope in the workspace pane, so you can use them in expressions or even assign to them.  For example, if you select the root object in the left pane and evaluate \ct{jdn  := jdn - 1} in the workspace pane, you will see that the value of the \ct{jdn} instance variable will indeed change, and the value of \ct{TimeStamp now - self} will increase by one day.

% ON: Does not work anymore
%You can change instance variables directly by selecting them, replacing the old value in the right-hand  pane by a \pharo expression, and accepting.  
%\pharo will evaluate the expression and assign the result to the instance variable.

There are special variants of the inspector for Dictionaries, OrderedCollections, CompiledMethods and a few other classes that make it easier to examine the contents of these special objects.

%---------------------------------------------------------
\subsection{The Object Explorer}

The \emph{object explorer} is conceptually similar to the inspector, but presents its information in a different way.
To see the difference, we'll \emph{explore} the same object that we were just inspecting.

\begin{figure}[tbp]
\begin{minipage}{0.48\textwidth}
	\begin{center}
	\ifluluelse
		{\includegraphics[width=\textwidth]{exploreTimeStampNow}}
		{\includegraphics[width=0.7\textwidth]{exploreTimeStampNow}}
	\end{center}
	\caption{Exploring \ct{TimeStamp now}}
	\figlabel{exploreTimeStampNow}
\end{minipage}
\hfill
\begin{minipage}{0.48\textwidth}
	\begin{center}
	\ifluluelse
		{\includegraphics[width=\textwidth]{exploreTimeStampNow2}}
		{\includegraphics[width=0.7\textwidth]{exploreTimeStampNow2}}
	\end{center}
	\caption{Exploring the instance variables}
	\figlabel{exploreTimeStampNow2}
\end{minipage}
\end{figure}

\dothis{Select \menu{self} in the inspector's left-hand pane, then \actclick and choose \menu{explore (I)}.}
The \ind{explorer} window looks like \figref{exploreTimeStampNow}.
If you click on the small triangle next to \ct{root}, the view will change to \figref{exploreTimeStampNow2}, which shows the instance variables of object that you are exploring.
Click on the triangle next to \ct{offset}, and you will see \emph{its} instance variables.  
The explorer is really useful when you need to explore a complex hierarchic structure\,---\,hence the name.
\index{keyboard shortcut!explore it}

The workspace pane of the object explorer works slightly differently from that of the inspector.
\ct{self} is not bound to the root object, but rather to the object that is currently selected; the instance variables of the selected object are also in scope.

To see the value of the explorer, let's use it to explore a deeply-nested structure of objects.

\dothis{Evaluate \ct{Object explore} in a workspace.}
This is the object that represents the class \ct{Object} in \pharo.
Note that you can navigate directly to the objects representing the method dictionary and even the compiles methods of this class (see \figref{ExploreObject}).

\begin{figure}[tbp]
	\begin{center}
		\includegraphics[width=0.5\textwidth]{ExploreObject}
	\end{center}
	\caption{Exploring a \ct{Object}}
	\figlabel{ExploreObject}
\end{figure}

%\dothis{Open a browser, and \metaclick five times on the method pane to bring-up the Morphic halo on the \ct{OBPluggableListMorph} that is used to represent the list of messages. 
%Click on the \emph{debug} handle \debugHandle{} and select \menu{explore morph}
%from the menu that appears.  This will open an Explorer on the \clsind{OBPluggableListMorph} object that represents the method list on the screen.  
%Open the root object (by clicking in its triangle), open its \ct{submorphs}, and continue exploring the structure of the objects that underlie this Morph, as shown in \figref{explorePluggableListMorph}.}
%
%\begin{figure}[tbp]
%	\begin{center}
%		\includegraphics[width=0.7\textwidth]{explorePluggableListMorph}
%	\end{center}
%	\caption{Exploring a \ct{PluggableListMorph}}
%	\figlabel{explorePluggableListMorph}
%\end{figure}

%=========================================================
\section{The Debugger}
\seclabel{debugger} % (fold)

The \ind{debugger} is arguably the most powerful tool in the \pharo tool suite.  It is used not just for debugging, but also for writing new code.
To demonstrate the debugger, let's start by creating a bug!

\dothis{Using the browser, add the following method to the class \ct{String}:}

\needlines{7}
\begin{method}[buggy]{A buggy method}
suffix
	"assumes that I'm a file name, and answers my suffix, the part after the last dot"
	| dot dotPosition |
	dot := FileDirectory dot.
	dotPosition := (self size to: 1 by: -1) detect: [ :i | (self at: i) = dot ].
	^ self copyFrom: dotPosition to: self size 
\end{method}

Of course, we are sure that such a trivial method will work, so instead of writing an SUnit test, we just type
\ct{'readme.txt' suffix} in a workspace and \menu{print it (p)}.
What a surprise!  Instead of getting the expected answer \ct{'txt'}, a \clsind{PreDebugWindow} pops up, as shown in \figref{PreDebugWindow}.

\begin{figure}[btp]
	\begin{center}
	\includegraphics[width=0.8\textwidth]{PreDebugWindow}
	\end{center}
	\caption{A \ct{PreDebugWindow} notifies us of a bug.}
	\figlabel{PreDebugWindow}
\end{figure}

The \ct{PreDebugWindow} has a title-bar that tells us what error occurred, and shows us a \emphind{stack trace} of the messages that led up to the error. 
Starting from the bottom of the trace, \ct{UndefinedObject>>>DoIt} represents the code that was compiled and run when we selected \ct{'readme.txt' suffix} in the workspace and asked \pharo to \menu{print it}.
This code, of course, sent the message \ct{suffix} to a \clsind{ByteString} object (\ct{'readme.txt'}). 
This caused the inherited \ct{suffix} method in class \ct{String} to execute; all this information is encoded in the next line of the stack trace, \ct{ByteString(String)>>>suffix}.  
Working up the stack, we can see that \ct{suffix} sent \ct{detect:}\ldots and eventually \ct{detect:ifNone} sent \ct{errorNotFound}.
\clsindex{UndefinedObject}

\begin{figure}[btp]
	\begin{center}
	\ifluluelse
		{\includegraphics[width=\textwidth]{debuggerDetectIfNone}}
		{\includegraphics[width=0.7\textwidth]{debuggerDetectIfNone}}
	\end{center}
	\caption{The debugger.}
	\figlabel{debuggerDetectIfNone}
\end{figure}

To find out \emph{why} the dot was not found, we need the debugger itself, so click on \button{Debug}.

%\dothis{You can also open the debugger by clicking on any of the lines on the stack trace.  If you do this, the debugger will open already focussed on the corresponding method.}

The debugger is shown in \figref{debuggerDetectIfNone}; it looks intimidating at first, but it is quite easy to use.
The title-bar and the top pane are very similar to those that we saw in the \lct{PreDebugWindow}.  
However, the debugger combines the stack trace with a method browser, so when you select a line in the stack trace, the corresponding method is shown in the pane below.
It's important to realize that the execution that caused the error is still in your image, but in a suspended state.  
Each line of the stack trace represents a frame on the execution stack that contains all of the information necessary to continue the execution.  This includes all of the objects involved in the computation, with their instance variables, and all of the temporary variables of the executing methods.

In \figref{debuggerDetectIfNone} we have selected the \ct{detect:ifNone:} method in the top pane.
The method body is displayed in the center pane; the blue highlight around the message \ct{value} shows that the current method has sent the message \ct{value} and is waiting for an answer.

The four panes at the bottom of the debugger are really two mini-inspectors (without workspace panes).
The inspector on the left shows the current object, that is, the object named \self in the center pane.
As you select different stack frames, the identity of \self may change, and so will the contents of the 
\self{}-inspector.
If you click on \self in the bottom-left pane, you will see that \self is the interval \ct{(10 to: 1 by -1)}, which is what we expect.
The workspace panes are not needed in the debugger's mini-inspectors because all of the variables are also in scope in the method pane; you should feel free to type or select expressions in this pane and evaluate them.  
You can always \menu{cancel (l)} your changes using the menu or \short{\textit{l}}. 
% apb: that lower-case-L is in italics so that it doesn't look like a 1 or a |
\index{keyboard shortcut!cancel}

The inspector on the right shows the temporary variables of the current context.
In \figref{debuggerDetectIfNone},
\ct{value} was sent to the parameter \ct{exceptionBlock}.

%\dothis{To see the current value of this parameter, click on \ct{exceptionBlock} in the context inspector.
%This will tell you that \ct{exceptionBlock} is \ct{[self errorNotFound: ...]}.
%\on{no longer true!}

As we can see one method lower in the stack trace, the \ct{exceptionBlock} is \ct{[self errorNotFound: ...]}, so, it is not surprising that we see the corresponding error message.

Incidentally, if you want to open a full inspector or explorer on one of the variables shown in the mini-inspectors, just double-click on the name of the variable, or select the name of the variable and \actclick to choose \menu{inspect (i)} or \menu{explore (I)}.
This can be useful if you want to watch how a variable changes while you execute other code. 
\index{keyboard shortcut!inspect it}
\index{keyboard shortcut!explore it}

Looking back at the method window, we see that we expected the penultimate line of the method to find \ct{dot} in the string \ct{'readme.txt'}, and that execution should never have reached the final line.
\pharo does not let us run an execution backwards, but it does let us start a method again, which works very well in code such as this that does not mutate objects, but instead creates new ones.  

\dothis{Click \button{Restart}, and you will see that the locus of execution returns to the first statement of the current method.  
The blue highlight shows that the next message to be sent will be {\ct{do:}} (see \figref{RestartDetectIfNone}).}

\begin{figure}[btp]
	\begin{center}
	\ifluluelse
		{\includegraphics[width=\textwidth]{RestartDetectIfNone}}
		{\includegraphics[width=0.7\textwidth]{RestartDetectIfNone}}
	\end{center}
	\caption{The debugger after restarting the \ct{detect: ifNone:} method}
	\figlabel{RestartDetectIfNone}
\end{figure}

The \button{Into} and \button{Over} buttons give us two different ways to step through the execution.  If you click \button{Over}, \pharo executes the current message-send (in this case the \ct{do:}) in one step, unless there is an error.  
So \button{Over} will take us to the next message-send in the current method, which is \ct{value}\,---\,this is exactly where we started, and not much help. 
What we need to do is to find out why the \ct{do:} is not finding the character that we are looking for.

\dothis{After clicking \button{Over}, click \button{Restart} to get back to the situation shown in \figref{RestartDetectIfNone}.}

\dothis{Click \button{Into}; \pharo will go into the method corresponding to the highlighted message-send, in this case, \ct{Collection>>>do:}.}

However, it turns out that this is not much help either: we can be fairly confident that \ct{Collection>>>do:} is not broken.  The bug is much more likely to be in \emph{what} we asked \pharo to do.
\button{Through} is the appropriate button to use in this case: we want to ignore the details of the \ct{do:} itself and focus on the execution of the argument block. 

\dothis{Select the \ct{detect:ifNone:} method again and \button{Restart} to get back to the state shown in \figref{RestartDetectIfNone}.
Now click on \button{Through} a few times.  Select \ct{each} in the context window as you do so.
You should see \ct{each} count down from \ct{10} as the \ct{do:} method executes.}

When \ct{each} is \ct{7} we expect the \ct{ifTrue:} block to be executed, but it isn't.
To see what is going wrong, go \button{Into} the execution of \ct{value:} as illustrated in \figref{steppingIntoValue}.

\begin{figure}[btp]
	\begin{center}
	\ifluluelse
		{\includegraphics[width=\textwidth]{steppingIntoValue}}
		{\includegraphics[width=0.7\textwidth]{steppingIntoValue}}
	\end{center}
	\caption{The debugger after stepping \lct{Through} the \ct{do:} method several times.}
	\figlabel{steppingIntoValue}
\end{figure}

After clicking \button{Into}, we find ourselves in the position shown in \figref{dotIsAString}.
It looks at first that we have gone \emph{back} to the \ct{suffix} method, but this is because we are now executing the block that \ct{suffix} provided as argument to \ct{detect:}.
%\on{does not work any more! the debugger does not know about block variables!}
%If you select \ct{i} in the context inspector, you can see its current value, which should be \ct{7} if you have been following along.  
%You can then select the corresponding element of \self from the \self{}-inspector.
%In  \figref{dotIsAString} you can see that element \ct{7} of the string is character 46, which is indeed a dot.
If you select \ct{dot} in the context inspector, you will see that its value is \ct{'.'}.
And now you see why they are not equal: the seventh character of \ct{'readme.txt'} is of course a \ct{Character}, while \ct{dot} is a \ct{String}.

\begin{figure}[btp]
	\begin{center}
	\ifluluelse
		{\includegraphics[width=\textwidth]{dotIsAString}}
		{\includegraphics[width=0.7\textwidth]{dotIsAString}}
	\end{center}
	\caption{The debugger showing why \ct{'readme.txt' at: 7} is not equal to \ct{dot}}
	\figlabel{dotIsAString}
\end{figure}

Now that we see the bug, the fix is obvious: we have to convert \ct{dot} to a character before starting to search for it.  

\begin{figure}[btp]
	\begin{center}
	\ifluluelse
		{\includegraphics[width=\textwidth]{revertDialog}}
		{\includegraphics[width=0.7\textwidth]{revertDialog}}
	\end{center}
	\caption{Changing the \ct{suffix} method in the debugger: asking for confirmation of the exit from an inner block}
	\figlabel{revertDialog}
\end{figure}

\dothis{Change the code right in the debugger so that the assignment reads \ct{dot := FileDirectory dot first} and \menu{accept} the change.}

Because we are executing code inside a block that is inside a \lct{detect:}, several stack frames will have to be abandoned in order to make this change.  \pharo asks us if this is what we want (see \figref{revertDialog}), and, assuming that we click \menu{yes}, will save (and compile) the new method.

%\dothis{Click \button{Restart} and then \button{Proceed}; the debugger window will vanish, and the evaluation of the expression \ct{'readme.txt' suffix} will complete, and print the answer \ct{'.txt'}}

The evaluation of the expression \ct{'readme.txt' suffix} will complete, and print the answer \ct{'.txt'}.

Is the answer correct?  Unfortunately, we can't say for sure.  Should the suffix be \ct{.txt} or \ct{txt}?
The method comment in \ct{suffix} is not very precise.  
The way to avoid this sort of problem is to write an \ind{SUnit} test that defines the answer.

\begin{method}[testSuffix]{A simple test for the \ct{suffix} method}
testSuffixFound
	self assert: 'readme.txt' suffix = 'txt'
\end{method}

The effort required to do that was little more than to run the same test in the workspace, but using \sunit saves the test as executable documentation, and makes it easy for others to run.
Moreover, if you add \mthref{testSuffix} to the class \ct{StringTest} and run that test suite with \sunit, you can very quickly get back to debugging the error.
\sunit opens the debugger on the failing assertion, but you need only go back down the stack one frame, \button{Restart} the test and go \button{Into} the \ct{suffix} method, and you can correct the error, as we are doing in \figref{fixOffByOne}.
It is then only  a second of work to click on the \button{Run Failures} button in the \sunit Test Runner, and confirm that the test now passes.

\begin{figure}[btp]
	\begin{center}
		\includegraphics[width=\textwidth]{fixOffByOne}
	\end{center}
	\caption{Changing the \ct{suffix} method in the debugger: fixing the off-by-one error after an \sunit assertion failure}
	\figlabel{fixOffByOne}
\end{figure}

Here is a better test:

\begin{method}[testSuffix2]{A better test for the \ct{suffix} method}
testSuffixFound
	self assert: 'readme.txt' suffix = 'txt'.
	self assert: 'read.me.txt' suffix = 'txt'
\end{method}
\noindent
Why is this test better?  Because it tells the reader what the method should do if there is more than one dot in the target String.

There are a few other ways to get into the debugger in addition to catching errors and assertion failures.
If you execute code that goes into an infinite loop, you can interrupt it and open a debugger on the computation by typing \short{.} (that's a full stop or a period, depending  on where you learned English).\footnote{It is also useful to know that you can bring up an emergency debugger at any time by typing \short{{\sc shift--}.}}
You can also just edit the suspect code to insert \ct{self halt}.
So, for example, we might edit the \ct{suffix} method to read as follows:
\index{process!interrupting}

\needspace{11ex}
\begin{method}[suffix]{Inserting a \ct{halt} into the \ct{suffix} method.}
suffix
	"assumes that I'm a file name, and answers my suffix, the part after the last dot"
	| dot dotPosition |
	dot := FileDirectory dot first.
	dotPosition := (self size to: 1 by: -1) detect: [ :i | (self at: i) = dot ].
	self halt.
	^ self copyFrom: dotPosition to: self size 
\end{method}

When we run this method, the execution of the \ct{self halt} will bring up the \ind{pre-debugger}, from where we can proceed, or go into the debugger and look at variables, step the computation, and edit the code.

That's all there is to the debugger, but it's not all there is to the \ct{suffix} method.  
The initial bug should have made you realize that if there is no dot in the target string, the \ct{suffix} method will raise an error.  
This isn't the behaviour that we want, so let's add a second test to specify what should happen in this case.

\needlines{3}
\begin{method}[testNoSuffix]{A second test for the \ct{suffix} method: the target has no suffix}
testSuffixNotFound
	self assert: 'readme' suffix = ''
\end{method}

\needlines{2}
\dothis{Add \mthref{testNoSuffix} to the test suite in class \clsind{StringTest}, and watch the test raise an error.
Enter the debugger by selecting the erroneous test in \sunit, and edit the code so that the test passes.
The easiest and clearest way to do this is to replace the \ct{detect:} message by \ct{detect: ifNone:}, where  the second argument is a block that simply returns the string size.}

We will learn more about SUnit in \charef{SUnit}.

% section debugger (end)

%=========================================================
\section{The Process Browser}

\st is a multi-threaded system: there are many lightweight processes (also known as threads) running concurrently in your image. 
In the future the \pharo virtual machine may take advantage of multiprocessors when they are available, but at present concurrency is implemented by time-slicing.

\begin{figure}[btp]
	\begin{center}
	\ifluluelse
		{\includegraphics[width=\textwidth]{processBrowser}}
		{\includegraphics[width=0.7\textwidth]{processBrowser}}
	\end{center}
	\caption{The Process Browser}
	\figlabel{processBrowser}
\end{figure}

The process \subind{process}{browser} is a cousin of the debugger that lets you look at the various processes running inside \pharo.
\figref{processBrowser} shows a screenshot.
The top-left pane lists all of the processes in \pharo, in priority order, from the timer interrupt watcher at priority 80 to the idle process at priority 10.
Of course, on a uniprocessor, the only process that can be running when you look is the UI process; all others will be waiting for some kind of event.
%:===> Process browser context menu is broken!
\on{broken -- to be fixed!}
By default, the display of processes is static; it can be updated by \actclick{ing} and selecting \menu{turn on auto-update (a)}

If you select a process in the top-left pane, its stack trace is displayed in the top-right pane, just as with the debugger.
If you select a stack frame, the corresponding method is displayed in the bottom pane.
The process browser is not equipped with mini-inspectors for \self and \lct{thisContext}, but \actclick{ing} on the stack frames provide equivalent functionality.

%=========================================================
\section{Finding methods}
\seclabel{methodFinder} 

There are two tools in \pharo to help you find messages.
They differ in both interface and functionality.

The \emph{method finder} was described at some length in \secref{quick:methodFinder}; you can use it to find methods by name or by functionality. 
However, to look at the body of a method, the method finder opens a new browser.
This can quickly become overwhelming.

\begin{figure}[btp]
	\begin{center}
	\ifluluelse
		{\includegraphics[width=\textwidth]{methodNamesRandom}}
		{\includegraphics[width=0.7\textwidth]{methodNamesRandom}}
	\end{center}
	\caption{The message names browser showing all methods containing the substring \ct{random} in their selectors.}
	\figlabel{methodNamesRandom} % should be *message* names!
\end{figure}

\index{message names browser}
The \emph{message names} browser has more limited search functionality: you type a fragment of a message selector in the search box, and the browser lists all methods that contain that fragment in their names, as shown in \figref{methodNamesRandom}.
However, it is a full-fledged browser:
if you select one of the names in the left pane, all of the methods with that name are listed in the right pane, and can be browsed in the bottom pane.
As with the browser, the message names browser has a button bar that can be used to open other  browsers on the selected method or its class.


% section methodFinder (end)

%=========================================================
\section{Change sets and the Change Sorter}
\seclabel{env:changeSet} % (fold)

Whenever you are working in \pharo, any changes that you make to methods and classes are recorded in a \ct{change set}.
This includes creating new classes, re-naming classes, changing categories, adding methods to existing classes\,---\,just about everything of significance.  
However, arbitrary \emph{doits} are not included, so if, for example, you create a new global variable by assigning to it in a workspace, the variable creation will not make it into a \subind{file}{change set}.
\index{change sorter}

At any time, many change sets exist, but only one of them\,---\,\ct{ChangeSet current}\,---\,is collecting the changes that are being made to the image.  
You can see which change set is current and can examine all of the change sets using the  change sorter, available by selecting \menu{World \go Tools \ldots \go Change Sorter}.

\begin{figure}[btp]
	\begin{center}
		\includegraphics[width=\linewidth]{changeSorter}
	\end{center}
	\caption{The Change Sorter}
	\figlabel{changeSorter}
\end{figure}

\figref{changeSorter} shows this browser.  The title bar shows which change set is current, and this change set is selected when the change sorter opens. 

Other change sets can be selected in the top-left pane; the \actclick menu allows you to make a different change set current, or to create a new change set.
The next pane lists all of the classes affected by the selected change set (with their categories).
Selecting one of the classes displays the names of those of its methods that are also in the change set (\emph{not} all of the methods in the class) in the left central pane, and selecting a method name displays the method definition in the bottom pane.
Note that the change sorter does \emph{not} show you whether the creation of the class itself is part of the change set, although this information is stored in the object structure that is used to represent the change set.

The change sorter also lets you delete classes and methods from the change set using the \actclick menu on the corresponding items.
%  However, for more elaborate editing of change sets, you should use a second tool, the \textit{change sorter}, available by selecting \menu{World\go{}open \ldots \go{}dual change sorter}, which is shown in \figref{changeSorter}.
% The change sorter is essentially two change set browsers side by side; each side can focus on a different change set, class, or method.

The change sorter allows you to simultaneously view two change sets, one on the left hand side and the other on the right.
This layout supports the change sorter's main feature, which is the ability to move or copy changes from one change set to another, as shown by the \actclick menu in \figref{changeSorter}.
It is also possible to copy individual methods from one side to the other.

You may be wondering why you should care about the composition of a change set.
the answer is that change sets provide a simple mechanism for exporting code from \pharo to the file system, from where it can be imported into another \pharo image, or into another non-\pharo \st.
Change set export is known as ``filing-out'', and can be accomplished using the \actclick menu on any change set, class or method in either browser.
Repeated file outs create new versions of the file, but change sets are not a versioning tool like Monticello:
they do not keep track of dependencies.
\index{file!filing out}

Before the advent of Monticello, change sets were the main means for exchanging code between smalltalkers.
They have the advantage of simplicity (the file out is just a text file, although we \emph{don't} recommend that you try to edit them with a text editor), and a degree of portability.  
%It's also quite easy to create a change set that makes changes to many different, unrelated parts of the system\,---\,something that Monticello is not yet equipped to do.
%\ab{Or is it?}
%\on{you mean something different than extensions to foreign packages using the *package protocol notation?}

The main drawback of change sets, compared to \ind{Monticello} packages, is that they do not support the notion of dependencies.
A filed-out change set is a set of \emph{actions} that change any image into which it is loaded. To successfully load a change set requires that the image be in an appropriate state.
For example, the change set might contain an action to add a method to a class; this can only be accomplished if the class is already defined in the image.
Similarly, the change set might rename or re-categorize a class, which obviously will only work if the class is present in the image; methods may use instance variables that were declared when they were filed out, but which do not exist in the image into which they are imported.
The problem is that change sets do not explicitly describe the conditions under which they can be filed in:
the file in process just hopes for the best, usually resulting in a cryptic error message and a stack trace when things go wrong.
Even if the file in works, one change set might silently undo a change made by another change set.

In contrast, Monticello packages represent code in a declarative fashion: they describe the state of the image should be in after they have been loaded.
This permits Monticello to warn you about conflicts (when two packages require contradictory final states)
and to offer to load a series of packages in dependency order.

In spite of these shortcomings, change sets still have their uses; in particular, you may find change sets on the Internet that you want to look at and perhaps use.
So, having filed out a change set using the change sorter, we will now tell you how to file one in.
This requires the use of another tool, the file list browser.

% section changeSet (end)

%=========================================================
\section{The File List Browser}

\begin{figure}[btp]
	\begin{center}
	\ifluluelse
		{\includegraphics[width=\textwidth]{fileList}}
		{\includegraphics[width=0.7\textwidth]{fileList}}
	\end{center}
	\caption{A file list browser}
	\figlabel{fileList}
\end{figure}

The \ind{file list browser} is in fact a general-purpose tool for browsing the file system (and also FTP servers) from \pharo. 
You can open it from the \menu{World\go{}Tools \ldots \go{}File Browser} menu.
What you see of course depends on the contents of your local file system, but a typical view is shown in \figref{fileList}.
\seeindex{file!browsing}{file list browser}

When you first open a file list browser it will be focussed on the current directory, that is, the one from which you started \pharo. The title bar shows the path to this directory.
The larger pane on the left-hand side can be used to navigate the file system in the conventional way.
When a directory is selected, the files that it contains (but not the directories) are displayed on the right.
This list of files can be filtered by entering a Unix-style pattern in the small box at the top-left of the window.  
Initially, this pattern is \ct{*}, which matches all file names, but you can type a different string there and accept it, changing the pattern.  (Note that a \ct{*} is implicitly prepended and appended to the pattern that you type.)
The sort order of the files can be changed using the \button{name}, \button{date} and \button{size} buttons.
The rest of the buttons depend on the name of the file selected in the browser.
In \figref{fileList}, the file name has the suffix \ct{.cs}, so the browser assumes that it is a change set, and provides buttons to \button{install} it (which \textit{files it in} to a new change set whose name is derived from the name of the file),  to browse the \button{changes} in the file, to examine the \button{code} in the file, 
and to \button{filein} the code into the \emph{current} change set.
You might think that the \button{conflicts} button would tell you about changes in the change set that conflicted with existing code in the image, but it doesn't.
\ab{Does anyone know what it does do?  I've never found it useful.}
\on{I tried it and found that it complained about linefeeds.}
Instead it just checks for potential problems in the file that might indicate that the file cannot properly be loaded (such as the presence of linefeeds).

\begin{figure}[btp]
	\begin{center}
	\ifluluelse
		{\includegraphics[width=\textwidth]{fileContentsBrowser}}
		{\includegraphics[width=0.7\textwidth]{fileContentsBrowser}}
	\end{center}
	\caption{A File Contents Browser}
	\figlabel{fileContentsBrowser}
\end{figure}

Because the choice of buttons to display depends on the file's \emph{name}, and not on its contents, sometimes the button that you want won't be on the screen.  
However, the full set of options is always available from the \actclick \menu{more \ldots} menu, so
you can easily work around this problem.

The \button{code} button is perhaps the most useful for working with change sets; it opens a browser on the contents of the change set file; an example is shown in \figref{fileContentsBrowser}.
The file contents browser is similar to the browser except that it does not show categories, just classes, protocols and methods.
For each class, the browser will tell you whether the class already exists in the system and whether it is defined in the file (but \emph{not} whether the definitions are identical).  
It will show the methods in each class, and (as shown in \figref{fileContentsBrowser}) will show you the differences between the current version and the version in the file.
Contextual menu items in each of the top four panes will also let you file in the whole of the change set, or the corresponding class, protocol or method. 

%=========================================================
\section{In Smalltalk, you can't lose code}
\seclabel{cantLoseCode} % (fold)

It is quite possible to crash \pharo: as an experimental system, \pharo lets you change anything, including things that are vital to make \pharo work! 

\dothis{To maliciously crash \pharo, try \ct{Object become: nil}.}

The good news is that you need never lose any work, even if you crash and go back to the last saved version of your image, which might be hours old.
This is because all of the code that you executed is saved in the \emph{.changes} file.
All of it!
This includes one liners that you evaluate in a workspace, as well as code that you add to a class while programming.
\index{changes}

So here are the instructions on how to get your code back.
There is no need to read this until you need it. 
However, when you do need it, you'll find it here waiting for you.

In the worst case, you can use a text editor on the \emph{.changes} file, but since it is many megabytes in size, this can be slow and is not recommended. 
\pharo offers you better ways.

%---------------------------------------------------------
\subsection{How to get your code back}
Restart \pharo from the most recent snapshot, and select \menu{World\go{}Tools \ldots \go{}Recover lost changes}. 
%This will open a workspace full of useful expressions. The first three,

%\begin{code}{}
%Smalltalk recover: 10000.
%ChangeList browseRecentLog.
%ChangeList browseRecent: 2000.
%\end{code}

%\noindent
%are most useful for recovery.

% If you execute \ct{ChangeList browseRecentLog}, you will be given 

This will give you the opportunity to decide how far back in history you wish to browse. 
Normally, it's sufficient to browse changes as far back as the last snapshot. (You can get much the same effect by editing \ct{ChangeList browseRecent: 2000} so that the number \ct{2000} becomes something else, using trial and error.)

One you have a \emph{recent changes} browser, showing, say, changes back as far as your last snapshot, you will have a list of everything that you have done to \pharo during that time. 
You can delete items from this list using the \actclick menu.
When you are satisfied, you can file-in what is left, thus incorporating the changes into your new image.
It's a good idea to start a new change set, using the ordinary change set browser, before you do the file in, so that all of your recovered code will be in a new change set. 
You can then file out this change set.

One useful thing to do in the \emph{recent changes} browser is to \menu{remove doIts}. 
Usually, you won't want to file in (and thus re-execute) doIts. 
However, there is an exception. 
Creating a class shows up as a \menu{doIt}.
\emph{Before you can file in the methods for a class, the class must exist.}
So, if you have created any new classes, \emph{first} file-in the class creation doIts, then \menu{remove doIts} and file in the methods.
\lr{Maybe mention that class renames are not logged and completely screw up the change-set mechanism. (p. 174)}

When I am finished with the recovery, I like to file out my new change set, quit \pharo without saving the image, restart, and make sure that the new change set files back in cleanly.
% section cantLoseCode (end)

%=========================================================
\section{Chapter summary}

In order to develop effectively with \pharo, it is important to invest some effort into learning the tools available in the environment.

\begin{itemize}
  \item The standard \emph{browser} is your main interface for browsing existing categories, classes, method protocols and methods, and for defining new ones.
  The browser offers several useful buttons to directly jump to senders or implementors of a message, versions of a method, and so on.
  \item There exist several different browsers (such as the OmniBrowser and the Refactoring Browser), and several specialized browsers (such as the hierarchy browser) which provide different views of classes and methods.
  \item From any of the tools, you can highlight the name of a class or a method and immediately jump to a browser by using the keyboard shortcut \short{b}.
  \item You can also browse the \st system programmatically by sending messages to \ct{SystemNavigation default}.
  \item \emph{Monticello} is a tool for exporting, importing, versioning and sharing packages of classes and methods.
  A Monticello package consists of a category, subcategories, and related methods protocols in other categories.
  \item The \emph{inspector} and the \emph{explorer} are two tools that are useful for exploring and interacting with live objects in your image.
  You can even inspect tools by \metaclick{ing} to bring up their morphic halo and selecting the debug handle \debugHandle.
  \item The \emph{debugger} is a tool that not only lets you inspect the run-time stack of your program when an error is raised, but it also enables you to interact with all of the objects of your application, including the source code. In many cases you can modify your source code from the debugger and continue executing. The debugger is especially effective as a tool to support test-first development in tandem with SUnit (\charef{SUnit}).
  \item The \emph{process browser} lets you monitor, query and interact with the processes current running in your image.
  \item The \emph{method finder} and the \emph{message names browser} are two tools for locating methods. The first is more useful when you are not sure of the name, but you know the expected behaviour. The second offers a more advanced browsing interface when you know at least a fragment of the name.
  \item \emph{Change sets} are automatically generated logs of all changes to the source code of your image. They have largely been superseded by Monticello as a means to store and exchange versions of your source code, but are still useful, especially for recovering from catastrophic failures, however rare these may be.
  \item The \emph{file list browser} is a tool for browsing the file system. It also allows you to \menu{filein} source code from the file system.
  \item In case your image crashes before you could save it or backup your source code with Monticello, you can always recover your most recent changes using a \emph{change list browser}. You can then select the changes you want to replay and file them into the most recent copy of your image.
\end{itemize}

%=================================================================
\ifx\wholebook\relax\else\end{document}\fi
%=================================================================

%=========================================================
%---------------------------------------------------------
