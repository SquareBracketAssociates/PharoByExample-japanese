% $Author$
% $Date$
% $Revision$

% HISTORY:
% 2006-12-05 - Stef started
% 2006-12-30 - Andrew new material
% 2007-01-10 - Stef edit
% 2007-01-12 - Andrew edit
% 2007-06-07 - Oscar edit
% 2007-07-03 - Stef edit
% 2007-09-06 - Lukas review
% 2007-09-06 - Cassou corrections
% 2007-09-24 - Cassou review
% 2009-07-01 - Oscar migrated to Pharo

%=================================================================
\ifx\wholebook\relax\else
% --------------------------------------------
% Lulu:
	\documentclass[a4paper,10pt,twoside]{book}
	\usepackage[
		papersize={6.13in,9.21in},
		hmargin={.75in,.75in},
		vmargin={.75in,1in},
		ignoreheadfoot
	]{geometry}
	\input{../common.tex}
	\pagestyle{headings}
	\setboolean{lulu}{true}
% --------------------------------------------
% A4:
%	\documentclass[a4paper,11pt,twoside]{book}
%	\input{../common.tex}
%	\usepackage{a4wide}
% --------------------------------------------
    \graphicspath{{figures/} {../figures/}}
	\begin{document}
	% \renewcommand{\nnbb}[2]{} % Disable editorial comments
	\sloppy
\fi
%=================================================================
\chapter{最初のアプリケーション}
\chalabel{firstApp}

この章において、簡単なゲーム:\ind{Lights Out}\footnote{\url{http://en.wikipedia.org/wiki/Lights_Out_(game)}}を作成することにします。\pharoプログラマーがプログラムを作成したりデバッグをするために使用する多くのツールを通じ、どのように他の開発者とプログラムを受け渡しするのかを実際に体験します。
ここでは、ブラウザー、オブジェクトインスペクター、デバッガー、\ind{Monticello} \ind{package}ブラウザーのツールの使い方を見ることになります。Smalltalkを使いプログラムをするのはとても効率的です。なぜそう言えるのかは、コードを書くためにいかに多くの時間を費やすことができ、開発プロセスに割く時間をどれほど省くことができるかということ、をこの章を読むことでわかるでしょう。Smallltalkを使ったプログラミングが効率的なのは、Smallltalkがプログラミング言語として簡潔であること、そしてプログラムを開発するための環境がプログラミング言語ととてもうまく統合されていること、によるためです。

%=================================================================
\section{The Lights Out game}

% DON'T USE WRAPFIGURE CLOSE TOO A PAGE BREAK!!! (ON)
%\begin{wrapfigure}[13]{r}{0.35\linewidth}%
%	\vskip -\baselineskip
%	\centerline{\includegraphics[width=.8\linewidth]{GameBoard}}
%	\caption{The Lights Out game board. The user has just clicked the mouse as shown by the cursor.
%	\figlabel{gameBoard}}
%\end{wrapfigure}

\begin{figure}[ht]
	\vskip -\baselineskip
	\centerline{\includegraphics[width=.3\linewidth]{GameBoard}}
	\caption{Lights Outゲームボード。ユーザーは単にボード上のセルをマウスでクリックするだけです。
	\figlabel{gameBoard}}
\end{figure}

\pharoのプログラミングツールの使い方を見せるため\emph{Lights Out}と呼ばれる簡単なゲームを作ります。ゲームボードは図2.1の通り、淡黄色の\emph{cells}の長方形配列で構成されます。いずれかのcellをマウスでクリックすると、周囲の4つのcellが青色に変わります。もう一度\Clickすると、それらの色の変わったcellは、淡黄色に戻ります。このゲームの目的は、できるだけ多くのcellを青色に変えることです。
\figref{gameBoard}に示されているゲームはゲームボード本体と100個からなるcellの2種類のオブジェクトで構成されています。このゲームを実装するための\pharoのコードは2つのクラスから成ります。1つはゲームクラス、もう1つはcellクラスです。\pharoのプログラミングツールを使い、どのようにこれらのクラスを定義していくか見ていきましょう。

%=================================================================
\section{Creating a new Package}

\charef{quick}の\ind{browser}の使い方において、クラスやメソッドを見る方法や新たなメソッドを定義する方法を学びました。ここではパッケージ、カテゴリ、クラスの作り方を学びます。
\index{category!creating}
\index{package!creating}

\dothis{ブラウザーを開きパッケージペインで\actclick。その後\menu{create package}を選択。\footnote{パッケージブラウザーが標準ブラウザーとしてインストールされているものとします。もし\figref{addPackage}で示されているようなブラウザーを使っていない場合はデフォルトのブラウザー変更する必要があります。もし\figref{addPackage}で示されているようなブラウザーを使っていない場合はデフォルトのブラウザー変更する必要があります。\faqref{packagebrowser}参照。}}

\begin{figure}[htb]
\begin{minipage}[b]{0.48\textwidth}
\ifluluelse
	{\centerline {\includegraphics[width=0.9\textwidth]{AddPackage}}}
	{\centerline {\includegraphics[scale=0.7]{AddPackage}}}
	\caption{Adding a package.
	\figlabel{addPackage}}
\end{minipage}
\hfill
\begin{minipage}[b]{0.48\textwidth}
\ifluluelse
	{\centerline {\includegraphics[width=0.8\textwidth]{ClassTemplate}}}
	{\centerline {\includegraphics[scale=0.6]{ClassTemplate}}}
	\caption{The class template.
	\figlabel{classTemplate}}
\end{minipage}
\end{figure}

ダイアログボックスに新たなパッケージ名を入力してください(ここではパッケージ名を\scat{PBE-LightsOut}とします)。その後\menu{accept}をクリックしてください(または単にリターンキーを押してください)。すると新たなパッケージが作成され、パッケージ一覧にアルファベット順に表示されます。

%=================================================================
\section{Defining the class LOCell}

あたり前のことですが、新しいパッケージにはまだ何のクラスも定義されていません。しかし、メインの編集ペインには新たなクラスを作りやすいように、自動的にクラスのテンプレートが用意されます(\figref{classTemplate}参照)。

このテンプレートでは、\ct{Object}と呼ばれるクラスに\st式のメッセージ送る\ct{NameOfSubClass}と呼ばれるサブクラスを作成する作成するように求められます。新しいクラスには変数はありません。またこのクラスは\scat{PBE-LightsOut}カテゴリに属していなければなりません。

\subsection{On Categories and Packages}
\seclabel{categoriesPackages}

歴史的には、\stは\emph{categories}は知っていますが、パッケージは知りません(意味不明、後に再チェック)。この2つ用語の違いとは何か、疑問に思うことと思います。カテゴリとは\stのイメージファイル内に、関係するクラスを単に集めたものです。\emph{package}とは関係するクラスと\emph{拡張メソッド}を集めたもので、バージョン管理ツールであるモンティチェロを用いて管理されるものを集めたものです。慣習的にパッケージ名とカテゴリ名は同じものを使います。通常、この用語の違いを気にする必要はありません。しかしこの本においては、この用語の違いが重要な意味を持つため、各所にて正しい用語を注意深く使用することとします。実際にモンティチェロを使うときに、これらの違いについてより深く知ることにしましょう。

\index{package}
\index{category}

\subsection{Creating a new class}

クラスのテンプレートを修正し、目的に合ったクラスを作ってみましょう。

\dothis{自動的に作成されたテンプレートのクラスを修正する手順は以下の通りです:}
\begin{itemize}
  \item \clsind{Object}を\clsind{SimpleSwitchMorph}と書き換えます。
  \item \ct{NameOfSubClass}を\clsind{LOCell}と書き換えます。
  \item instance variablesのリストに\ct{mouseAction}を追加します。
\end{itemize}
結果は\clsref{firstClassDef}の通りです。

\needlines{5}
\begin{classdef}[firstClassDef]{Defining the class \ct| LOCell|}
SimpleSwitchMorph subclass: #LOCell
   instanceVariableNames: 'mouseAction'
   classVariableNames: ''
   poolDictionaries: ''
   category: 'PBE-LightsOut'
\end{classdef}
\index{browser!defining a class}
\index{class!creation}
\index{Morphic}

この新しく定義したSmalltalk式は、SimpleSwitchMorphという名の既存のクラスに対し、LOCellという名のサブクラスを作るかどうかをメッセージを送り尋ねます。(実際にはLOCellサブクラスはまだ作られていないため、これから作作るサブクラスの名前、つまり#LOCellシンボルを引数として渡します。)
また、マウスでクリックしたときにcellがどのような動作をするか、ということもこの後定義するmouseActionインスタンス変数に対し伝えなければなりません。
この時点ではまだ何も作られてはいません。クラステンプレートペインの境界線が赤に変わっていることに注目してください(図2.4)。これはまだプログラムを変更したことがセーブされていないことを意味します。実際にセーブするためにacceptする必要があります。

\begin{figure}[h!t]
\ifluluelse
	{\centerline {\includegraphics[width=\textwidth]{AcceptClassDef}}}
	{\centerline {\includegraphics[scale=0.7]{AcceptClassDef}}}
\caption{The class-creation Template.
\figlabel{acceptClassDef}}
\end{figure}

\dothis{Accept the new class definition.}
action-clickをしてacceptを選択、もしくはショートカットのCMD-s(セーブのため)を実行して下さい。SimpleSwitchMorphにメッセージが送られ、新しいクラスがコンパイルされます。
1度クラス定義がacceptされると、クラスが作られ、ブラウザのクラスペインにそのクラスが表示されます(図2.5)。今、編集ペイン(1章と整合性をとる)にはクラス定義が表示されているはずです。またその下にある小さなペインではそのクラスの目的を説明を書くことをうながしてきます。これはclass commentと呼ばれ、他のプログラマーに対し、このクラスの目的を伝えるために非常に役に立つ概要を与えてくれます。Smalltalkプログラマーは自分が書くコードの読みやすさを重視します。そのためメソッド内に書かれる詳細なコメントの質は並外れていて、そのコードについての哲学が語られています。(もしそうなっていない場合は、そうなるまでリファクタリングするべきです。)
\index{keyboard shortcut!accept}

1度クラス定義がacceptされると、クラスが作られ、ブラウザのクラスペインにそのクラスが表示されます(図2.5)。今、編集ペイン(1章と整合性をとる)にはクラス定義が表示されているはずです。またその下にある小さなペインではそのクラスの目的を説明を書くことをうながしてきます。これはclass commentと呼ばれ、他のプログラマーに対し、このクラスの目的を伝えるために非常に役に立つ概要を与えてくれます。Smalltalkプログラマーは自分が書くコードの読みやすさを重視します。そのためメソッド内に書かれる詳細なコメントの質は並外れていて、そのコードについての哲学が語られています。(もしそうなっていない場合は、そうなるまでリファクタリングするべきです。)
\index{refactoring}
クラスのコメントに、クラスの詳細な説明が含まれている必要はありません。しかし、クラス全体の目的を数語で表現できれば、後進のプログラマーがこのクラスを時間を費やしてでも読む気にさせてくれるでしょう。

\dothis{LOCellクラスのコメントを入力し、acceptして下さい。これで後にそれを改良することができます。}

\begin{figure}[h!t]
\ifluluelse
	{\centerline {\includegraphics[width=\textwidth]{LOCell}}}
	{\centerline {\includegraphics[scale=0.7]{LOCell}}}
\caption{The newly-created class \ct{LOCell}\figlabel{LOCell}}
\end{figure}

%=================================================================
\section{Adding methods to a class}

クラスにメソッドを追加します

今度はこのクラスにいくつかのメソッドを追加してみましょう。
プロトコルペインにある--all--プロトコルを選択します。それを選択し、メソッド2.2に示されている内容と置き換えてください。
\protindex{all}
\index{method!creation}
\index{browser!defining a method}

\needlines{10}
\begin{numMethod}[scbecellinitialize]{Initializing instances of \ct{LOCell}}
initialize
   super initialize.
   self label: ''.
   self borderWidth: 2.
   bounds := 0@0 corner: 16@16.
   offColor := Color paleYellow.
   onColor := Color paleBlue darker.
   self useSquareCorners.
   self turnOff
\end{numMethod}
\index{initialization}

\noindent
3行目''の文字は、間に何の文字も挟んでいない2つの単一引用符です。二重引用符ではないことに注意してください。''空の文字列を表しています。

\dothis{このメソッドの定義をAccept}

上記のコードは何をするのでしょう?ここではその詳細全てには触れません(本の残りの部分はそのためにあるのです!)。しかし簡単に内容を見てみましょう。1行ずつ進んでいきましょう。

initializeと呼ばれるメソッドに注目してください。この名前は非常に重要です。慣習的にinitializeと名付けられたメッソドは、そのメソッドが属するクラスのオブジェクト作られた直後に呼び出されれます。つまり、LOCell new が評価されたとき、initialize メッセージが自動的に新しく作られたオブジェクトに送られます。initializeメソッドはオブジェクトの状態、主にそのインスタンス変数、を設定するために使われます。
\seeindex{Object!initialization}{initialization}
\index{initialization}

このメソッドが最初にすること(2行目)は、そのスーパークラスであるSimpleSwitchMorphのinitializeメソッドです。スーパークラスのinitializeメソッドにより、全ての継承された状態が適切に初期化された、とここでは考えてください。何かをする前に、スーパークラスのinitializeメソッドにより、継承された状態が初期化されるという考えることは常に正しい事です。つまり、SimpleSwitchMorphのinitializeメソッドが何をするかは知らなくとも気にしなくとも構いません。しかし、妥当なデフォルト値を維持するため、いくつかのインスタンス変数を設定するのはまっとうな方法であり、その値を呼び出すことはよいことです。さもなくば、何かをするときに初期化されていない状態から始める危険性をともないます(訳良くない、修正の必要あり)。

このメソッドの残りの部分はこのオブジェクトの状態を設定します。例えば、self label:'' をオブジェクトに送ることで、このオブジェクトのラベルに空の文字列を設定します。

0@0 corner: 16@16 で表される式については、恐らくいくつかの説明が必要でしょう。0@0は、両方共0に設定されたx,y座標のPointオブジェクトを意味します。実際に 0@0 で表される式は 数値 0 に、引数が 0 の @ メッセージを送ります。その結果、数値 0 はPointクラスに、座標(0,0)で新たなインスタンスを作るよう指示します。新しく作られた(座標の)点に、corner:16@16 メッセージを送ってみましょう。その結果、角が 0@0 と 16@16の長方形が作られます。この新しく作られた長方形には、スーパークラスから継承された bounds 変数が割り当てられます。Pharo の画面の原点は左上であり、下に向かってy座標の値が増えることに注意してください。
(以下の行 全部 意味不明、後でやり直し)
メソッドの残りの部分は自己説明的であるべきです。良いSmalltalkのコードを書く方法の1つは、Smalltalkのコードが英語の混成語のように読めるような良いメソッド名をつけることです。オブジェクトがそれ自体と話したり、”自分で広場の角を使う”、”自分で電源をオフにする”というイメージを持つことができるでしょう。
%=================================================================
\section{オブジェクトを調査}

新たなLOCellを作り、それを調査することによって書いたコードの効果をテストできます。

\dothis{workspaceを開き、そこでLOCell newと打ち込み inspect it して下さい。}

\begin{figure}[htbp]
   \centering
   \includegraphics[width=\textwidth]{LOCellInspector} 
   \caption{LOCellオブジェクトを調べるために使用したインスペクタ。}
\end{figure}

インスペクタの左側のペインには、調査したインスタンス変数のリストが表示されます。その中の1つ例えば bounds)を選んで見てください、右側のペインにそのインスタンス変数の値が表示されます。インスペクターの下側にあるペインはmini-workspaceです。擬似変数selfが選択されたオブジェクトに束縛されているため、このworkspaceは便利です(意味不明 後でやり直し)。

\dothis{
インスペクタウィンドウのルートにあるLOCellを選択してください。下部にあるペインに、self bounds:(200@200 coner: 250@250)と打ち込み do it して下さい。インスペクター上で束縛された値が変化します。次にmini-workspaceに self openinWorldと打ち込み do it して下さい。}
画面の左上の端にcellが表示されます。実際、boundsメッセージで表示するよう指示した場所に正確に表示されます。cell上でmeta-clickをするとmorphic haroが現れます。cellを右上2番目の茶色いハンドルを使って移動したり、右下にある黄色のハンドルを使ってリサイズしてみましょう。bounds の値がどのように変化するか、インスペクターを使って確認して下さい。(新たな bounds の値を見るために refresh action-click が必要かもしれません。)

\begin{figure}[htbp]
\centering
\ifluluelse
	{\includegraphics[width=\textwidth]{LOCellResize} }
	{\includegraphics[scale=0.7]{LOCellResize} }
\caption{cellのリサイズ。\figlabel{cellresize}}
\end{figure}

\dothis{ピンク色のハンドルをクリックすることでcellは消せます。}

%=================================================================
\section{クラスLOGameの定義}

これから、LOGameと呼ばれるゲームを作るのに必要な他のクラスを作りましょう。

\dothis{ブラウザーのメインウインドウで、クラス定義テンプレートを作ってください。}
これはパッケージ名をクリックすればできます。以下の通りにコードを書いてください。

\needlines{6}
\begin{classdef}[sbegame]{LOGameクラスを定義します。}
BorderedMorph subclass: #LOGame
   instanceVariableNames: ''
   classVariableNames: ''
   poolDictionaries: ''
   category: 'PBE-LightsOut'
\end{classdef}

ここではBorderedMorphをサブクラス化しています。MorphはPharoのグラフィック図形すべてのスーパークラスです。そして(驚くことに)、BorderedMorphは境界線を持ったMorphです(意味不明 再度翻訳する)。2行目にある引用符の間にインスタンス変数の名前を入れることができますが、今はここは空のままにしておきましょう。

Now let's define an \mthind{LOGame}{initialize} method for \ct{LOGame}.

\dothis{Type the following into the browser as a method for \ct{LOGame} and try to \menu{accept} it:}

\begin{numMethod}[sbegameinitialize]{Initializing the game}
initialize
   | sampleCell width height n |
   super initialize.
   n := self cellsPerSide.
   sampleCell := LOCell new.
   width := sampleCell width.
   height := sampleCell height.
   self bounds: (5@5 extent: ((width*n) @(height*n)) + (2 * self borderWidth)).
   cells := Matrix new: n tabulate: [ :i :j | self newCellAt: i at: j ].
\end{numMethod}

%\sd{it would be nicer if we would not have to create an instance of LOCell for nothing}
%\on{yes}

\pharo will complain that it doesn't know the meaning of some of the terms.
\pharo tells you that it doesn't know of a message \ct{cellsPerSide}, and suggests a number of corrections, in case it was a spelling mistake.


\begin{figure}[htb]
\begin{minipage}{0.48\textwidth}
	\centering
	\ifluluelse
		{\includegraphics[width=\textwidth]{UnknownSelector}}
		{\includegraphics[scale=0.7]{UnknownSelector}}
	\caption{\pharo detecting an unknown selector.\figlabel{unknownSelector}}
\end{minipage}
\hfill
\begin{minipage}{0.48\textwidth}
	\centering
	\ifluluelse
		{\includegraphics[width=\textwidth]{DeclareInstanceVar}}
		{\includegraphics[scale=0.7]{DeclareInstanceVar}}
	\caption{Declaring a new instance variable.\figlabel{declareInstance}}
\end{minipage}
\end{figure}

But \ct{cellsPerSide} is not a mistake\,---\,it is just a method that we haven't yet defined\,---\,we will do so in a minute or two.

\dothis{So just select the first item from the menu, which confirms that we really meant \ct{cellsPerSide}.}

Next, \pharo will complain that it doesn't know the meaning of \ct{cells}.  It offers you a number of ways of fixing this.

\dothis{Choose \menu{declare instance} because we want \ct{cells} to be an instance variable.}
Finally, \pharo will complain about the message \ct{newCellAt:at:} sent on the last line; this is also not a mistake, so confirm that message too.
\index{on the fly variable definition}
\index{instance variable definition} 

If you now look at the class definition once again (which you can do by clicking on the \button{instance} button), you will see that the browser has modified it to include the instance variable \ct{cells}.

Let's look at this \ct{initialize} method.
The line \ct{| sampleCell width height n |}  declares 4 temporary variables. They are called temporary variables because their scope and lifetime are limited to this method.  Temporary variables with explanatory names are helpful in making code more readable.  Smalltalk has no special syntax to distinguish constants and variables, and in fact all four of these ``variables'' are really constants. 
Lines 4--7 define these constants.

How big should our game board be?  Big enough to hold some integral number of cells, and big enough to draw a border around them.
How many cells is the right number?  5? 10? 100? We don't know yet, and if we did, we would probably change our minds later.  So we delegate the responsibility for knowing that number to another method, which we will call \ct{cellsPerSide}, and which we will write in a minute or two.
It's because we are sending the \ct{cellsPerSide} message before we define a method with that name that \pharo asked us to ``confirm, correct, or cancel'' when we accepted the method body for \ct{initialize}.
Don't be put off by this: it is actually good practice to write in terms of other methods that we haven't yet defined.
Why?  Well, it wasn't until we started writing the \ct{initialize} method that we realized that we needed it, and at that point, we can give it a meaningful name, and move on, without interrupting our flow.
 
The fourth line uses this method: 
the Smalltalk \ct{self cellsPerSide} sends the message \ct{cellsPerSide} to \pvind{self}, i.e., to this very object.  
The response, which will be the number of cells per side of the game board, is assigned to \ct{n}.

The next three lines create a new \ct{LOCell} object, and assign its width and height to the appropriate temporary variables. 

%The eighth line sends the message \ct{bounds:} to \self.
%\ct{bounds:} is a method that we inherit from our superclass; it is used to define the space on the screen that this Morph will occupy.  
%The single colon (\ct{:}) at the end of the name says that \ct{bounds:} expects a single parameter, which should be a rectangle object.
Line 8 sets the \ct{bounds} of the new object.
Without worrying too much about the details just yet, just believe us that the expression in parentheses creates a square with its origin (\ie its top-left corner) at the point (5,5) and its bottom-right corner far enough away to allow space for the right number of cells.

The last line sets the \ct{LOGame} object's instance variable \ct{cells} to a newly created \clsind{Matrix} with the right number of rows and columns.   We do this by sending the message \ct{new:tabulate:} to the \ct{Matrix} class (classes are objects too, so we can send them messages).  We know that \mthind{Matrix class}{new:tabulate:} takes two arguments because it has two colons (\ct{:}) in its name.   The arguments go right after the colons.
If you are used to languages that put all of the arguments together inside parentheses, this may seem weird at first.  Don't panic, it's only syntax!
It turns out to be a very good syntax because the name of the method can be used to explain the roles of the arguments.  For example, it is pretty clear that \ct{Matrix rows: 5 columns: 2} has 5 rows and 2 columns, and not 2 rows and 5 columns.
\cmindex{Matrix class}{rows:columns:}

\ct{Matrix new: n tabulate: [ :i :j | self newCellAt: i at: j ]} creates a new \ct{n}{$\times$}\ct{n} matrix and initializes its elements.  The initial value of each element will depend on its coordinates.  The \ct{(i,j)}\textsuperscript{th} element will be initialized to the result of evaluating \ct{self newCellAt: i at: j}.  

%:===> Pretty-print is broken! (how to pretty-print?)

% \on{I think it is silly to copy paste from the pretty-print view to the normal view}

%That's \ct{initialize}.  When you accept this message body, you might want to take the opportunity to pretty-up the formatting.  You don't have to do this by hand: from the \actclick menu select \menu{more \ldots \go prettyprint}, and the browser will do it for you\damien{this didn't do anything to me}.  You have to \menu{accept} again after you have \subind{method}{pretty-print}{}ed a method, or of course you can \subind{keyboard shortcut}{cancel} 
%(\short{l}\,---\,that's a lower-case letter \emph{L}) if you don't like the result.
%Alternatively, you can set up the browser to use the pretty-printer automatically whenever it shows you code: use the the right-most button in the button bar to adjust the view.
%\seeindex{pretty-print}{method}

%If you find yourself using \menu{more\,\ldots} a lot, it's useful to know that you can hold down the {\sc shift} key when you click to directly bring up the \menu{more \ldots} menu.

%=================================================================
\section{Organizing methods into protocols}

Before we define any more methods, let's take a quick look at the third pane at the top of the browser.
In the same way that the first pane of the browser lets us categorize classes into packages so we are not overwhelmed by a very long list of class names in the second pane, so the third pane lets us categorize methods so that we are not overwhelmed by a very long list of method names in the fourth pane.   
These categories of methods are called ``protocols''.
\index{protocol}

If there are only a few methods in a class, the extra level of hierarchy provided by protocols is not really necessary.
This is why the browser also offers us the \prot{-{}-all-{}-} virtual protocol, which, you will not be surprised to learn, contains all of the methods in the class.
\protindex{all}

\begin{figure}[htbp]
   \centering
   \includegraphics[width=\textwidth]{Categorize} 
   \caption{Automatically categorize all uncategorized methods.\figlabel{categorize}}
\end{figure}

If you have followed along with this example, the third pane may well contain the protocol \protind{as yet unclassified}.

\dothis{\Actclick in the protocol pane and select \menu{various \go categorize automatically} to fix this, and move the \ct{initialize} methods to a new protocol called \protind{initialization}.}
How does \pharo{} know that this is the right protocol?  Well, in general \pharo{} can't know, but in this case there is also an \ct{initialize} method in a superclass, and \pharo assumes that our \ct{initialize} method should go in the same category as the one that it overrides.
\index{method!categorize}

%You may find that \pharo has already put your \ct{initialize} method into the \protind{initialization} protocol.
%If so, it's probably because you have loaded a package called \ct{AutomaticMethodCategorizer} into your image.

\paragraph{A typographic convention.} Smalltalkers frequently use the notation ``\verb|>>|'' to identify the class to which a method belongs, so, for example, the \ct{cellsPerSide} method in class \ct{LOGame} would be referred to as \ct{LOGame>>cellsPerSide}.
To indicate that this is \emph{not} \st syntax, we will use the special symbol \ct{>>>} instead, so this method will appear in the text as \ct{LOGame>>>cellsPerSide}
\cmindex{Behavior}{>>}

From now on, when we show a method in this book, we will write the name of the method in this form.  Of course, when you actually type the code into the browser, you don't have to type the class name or the \ct{>>>}; instead, you just make sure that the appropriate class is selected in the class pane.  

Now let's define the other two methods that are used by the \ct{LOGame>>>initialize} method. Both of them can go in the \prot{initialization} protocol.

\begin{method}[sbegamecellsperside]{A constant method.}
LOGame>>>cellsPerSide
   "The number of cells along each side of the game"
   ^ 10
\end{method}
\cmindex{LOGame}{cellsPerSide}
\index{constant methods}

This method could hardly be simpler: it answers the constant 10.  One advantage of representing constants as methods is that if the program evolves so that the constant then depends on some other features, the method can be changed to calculate this value.

\needlines{10}
\begin{method}[newCellAt:at:]{An initialization helper method}
LOGame>>>newCellAt: i at: j
   "Create a cell for position (i,j) and add it to my on-screen
   representation at the appropriate screen position.  Answer the new cell"
   | c origin |
   c := LOCell new.
   origin := self innerBounds origin.
   self addMorph: c.
   c position: ((i - 1) * c width) @ ((j - 1) * c height) + origin.
   c mouseAction: [self toggleNeighboursOfCellAt: i at: j]
\end{method}
\cmindex{LOGame}{newCellAt:at:}
%   ^ c      "omit this final line to create a bug"

\dothis{Add the methods \ct{LOGame>>>cellsPerSide} and \ct{LOGame>>>newCellAt:at:}.}
Confirm the spelling of the new selectors \ct{toggleNeighboursOfCellAt:at:} and \ct{mouseAction:}.

\Mthref{newCellAt:at:} answers a new LOCell, specialized to position \ct{(i, j)} in the \clsind{Matrix} of cells.
The last line defines the new cell's \ct{mouseAction} to be the \emph{block}
\mbox{\lct{[self toggleNeighboursOfCellAt: i at: j ]}.}
 In effect, this defines the callback behaviour to perform when the mouse is clicked.
The corresponding method also needs to be defined.

\begin{method}[toggleNeighboursOfCellAt:at:]{The callback method}
LOGame>>>toggleNeighboursOfCellAt: i at: j
   (i > 1) ifTrue: [ (cells at: i - 1 at: j ) toggleState].
   (i < self cellsPerSide) ifTrue: [ (cells at: i + 1 at: j) toggleState].
   (j > 1) ifTrue: [ (cells at: i  at: j - 1) toggleState].
   (j < self cellsPerSide) ifTrue: [ (cells at: i at: j + 1) toggleState].
\end{method}
\cmindex{LOGame}{toggleNeighboursOfCellAt:at:}

\Mthref{toggleNeighboursOfCellAt:at:} toggles the state of the four cells to the north, south, west and east of cell (\ct{i}, \ct{j}).  The only complication is that the board is finite, so we have to make sure that a neighboring cell exists before we toggle its state.

\dothis{Place this method in a new protocol called \prot{game logic}. (\Actclick in the protocol pane to add a new protocol.)}
To move the method, you can simply click on its name and drag it to the newly-created protocol (\figref{dragMethod}).

\begin{figure}[htbp]
   \centering
   \ifluluelse
		{\includegraphics[width=\textwidth]{DragMethod} }
		{\includegraphics[scale=0.7]{DragMethod} }
   \caption{Drag a method to a protocol.\figlabel{dragMethod}}
\end{figure}

To complete the Lights Out game, we need to define two more methods in class \ct{LOCell} to handle mouse events.
\begin{method}[mouseAction:]{A typical setter method}
LOCell>>>mouseAction: aBlock
   ^ mouseAction := aBlock
\end{method}
\cmindex{LOCell}{mouseAction:}

\Mthref{mouseAction:} does nothing more than set the cell's \ct{mouseAction} variable to the argument, and then answers the new value.
Any method that \emph{changes} the value of an instance variable in this way is called a \emph{setter method}; a method that \emph{answers} the current value of an instance variable is called a \emph{getter method}.
\seeindex{setter method}{accessor}
\seeindex{getter method}{accessor}

If you are used to getters and setters in other programming languages, you might expect these methods to be called \ct{setmouseAction} and \ct{getmouseAction}.
The \st convention is different.
A getter always has the same name as the variable it gets, and a setter is named similarly, but with a trailing ``\ct{:}'', hence \ct{mouseAction} and \ct{mouseAction:}.

Collectively, setters and getters are called  \emphind{accessor} methods, and by convention they should be placed in the \protind{accessing} protocol.
In Smalltalk, \emph{all} instance variables are private to the object that owns them, so the only way for another object to read or write those variables in the Smalltalk language is through accessor methods like this one\footnote{In fact, the instance variables can be accessed in subclasses too.}.

\dothis{Go to the class \ct{LOCell}, define \ct{LOCell>>>mouseAction:} and put it in the \prot{accessing} protocol.}

Finally, we need to define a method \ct{mouseUp:}; this will be called automatically by the GUI framework if the mouse button is released while the mouse is over this cell on the screen.

\begin{method}[sbecellmouseup]{An event handler}
LOCell>>>mouseUp: anEvent
   mouseAction value
\end{method}
\cmindex{LOCell}{mouseUp:}

\dothis{Add the method \ct{LOCell>>>mouseUp:} and then
  \menu{categorize automatically} methods.}
\index{method!categorize}

What this method does is to send the message \ct{value} to the object stored in the instance variable \ct{mouseAction}. 
Recall that in \ct{LOGame>>>newCellAt: i at: j} we assigned the following code fragment to \ct{mouseAction}:

\ct{[self toggleNeighboursOfCellAt: i at: j ]} 

\noindent
Sending the \ct{value} message causes this code fragment to be evaluated, and consequently the state of the cells will toggle.

%=================================================================
\section{Let's try our code}

That's it: the Lights Out game is complete!

If you have followed all of the steps, you should be able to play the game, consisting of just 2 classes and 7 methods.

\dothis{In a workspace, type \ct{LOGame new openInWorld} and \menu{do it}.}

The game will open, and you should be able to click on the cells and see how it works.

Well, so much for theory\ldots{}
When you click on a cell, a \emphind{notifier} window called the \clsind{PreDebugWindow}window appears with an error message!
As depicted in \figref{lightsOutError}, it says \ct{MessageNotUnderstood: LOGame>>>toggleState}.

\begin{figure}[ht]
\ifluluelse
	{\centerline{\includegraphics[width=\textwidth]{Error}}}
	{\centerline{\includegraphics[scale=0.7]{Error}}}
\caption{There is a bug in our game when a cell is clicked!
\figlabel{lightsOutError}}
\end{figure}

\noindent
What happened? To find out, let's use one of Smalltalk's more powerful tools: the \ind{debugger}.

\dothis{Click on the \menu{debug} button in the notifer window.}
The debugger will appear.
In the upper part of the debugger window you can see the execution stack, showing all the active methods; selecting any one of them will show, in the middle pane, the Smalltalk code being executed in that method, with the part that triggered the error highlighted.

\dothis{Click on the line labelled
\ct{LOGame>>>toggleNeighboursOfCellAt:at:} (near the top).}
The debugger will show you the \ind{execution context} within this method where the error occurred (\figref{debugToggle}).

\begin{figure}[ht]
\ifluluelse
	{\centerline {\includegraphics[width=\textwidth]{Debugger}}}
	{\centerline {\includegraphics[scale=0.7]{Debugger}}}
\caption{The debugger, with the method \ct{toggleNeighboursOfCell:at:}  selected.
\figlabel{debugToggle}}
\end{figure}

At the bottom of the debugger are two small inspector windows.  On the left, you can inspect the object that is the receiver of the message that caused the selected method to execute, so you can look here to see the values of the instance variables.
On the right you can inspect an object that represents the currently executing method itself, so you can look here to see the values of the method's parameters and temporary variables.

Using the debugger, you can execute code step by step, inspect objects in parameters and local variables, evaluate code just as you can in a workspace, and, most surprisingly to those used to other debuggers, change the code while it is being debugged! Some Smalltalkers program in the debugger almost all the time, rather than in the browser.  The advantage of this is that you see the method that you are writing as it will be executed, with real parameters in the actual execution context.

In this case we can see in the first line of the top panel that the \ct{toggleState} message has been sent to an instance of \ct{LOGame}, while it should clearly have been an instance of \lct{LOCell}.
The problem is most likely with the initialization of the \ct{cells} matrix.
Browsing the code of \cmind{LOGame}{initialize} shows that \ct{cells} is filled with the return values of \ct{newCellAt:at:}, but when we look at that method, we see that there is no return statement there!
By default, a method returns \ct{self}, which in the case of \ct{newCellAt:at:} is indeed an instance of \ct{LOGame}.
\index{method!returning self}

\dothis{Close the debugger window.
Add the expression ``\ct{^ c}'' to the end of the method \ct{LOGame>>>newCellAt:at:} so that it returns \ct{c}.
% It should now look as shown in \mthref{newCellAt:at:nobug}.}
(See \mthref{newCellAt:at:nobug}.)}

% \needlines{6}
\begin{method}[newCellAt:at:nobug]{Fixing the bug.}
LOGame>>>newCellAt: i at: j
   "Create a cell for position (i,j) and add it to my on-screen
   representation at the appropriate screen position.  Answer the new cell"
   | c origin |
   c := LOCell new.
   origin := self innerBounds origin.
   self addMorph: c.
   c position: ((i - 1) * c width) @ ((j - 1) * c height) + origin.
   c mouseAction: [self toggleNeighboursOfCellAt: i at: j].
   ^ c
\end{method}
\cmindex{LOGame}{newCellAt:at:}

\noindent
Recall from \charef{quick} that the construct to \ind{return} a \subind{method}{value} from a method in Smalltalk is \ct{^}, which you obtain by typing \verb|^|.
% \index{^@\verb|^|}
\index{^@{$\uparrow$}|see{return}}

Often, you can fix the code directly in the debugger window and click \menu{Proceed} to continue running the application.
In our case, because the bug was in the initialization of an object, rather than in the method that failed, the easiest thing to do is to close the debugger window, destroy the running instance of the game (with the \subind{Morphic}{halo}), and create a new one.

%Indeed, even in this case it would be possible to \menu{do} \ct{self initialize} and then \menu{Proceed} the \ct{toggleNeighboursOfCellAt:at:} method.
%\ab{St\'eph, did you try this?  It seems to me that it ought to work, but when I tried it, it messed up my image.}
% ON : It messed me up too!  Better not propose this.

\dothis{Do: \ct{LOGame new openInWorld} again.}

Now the game should work properly ... or nearly so.  If we happen to move the mouse between clicking and releasing, then the cell the mouse is over will also be toggled.  This turns out to be behavior that we inherit from \ct{SimpleSwitchMorph}. We can fix this simply by overriding \ct{mouseMove:} to do nothing:

% \needlines{6}
\begin{method}[mouseMove:]{Overriding mouse move actions.}
LOGame>>>mouseMove: anEvent
\end{method}

Finally we are done!

%\sd{It would be good to have a word about the debugger buttons into, step.... Or to have a separate chapter, we would use the material I wrote for my turtle book, please check it.}
%\on{I think that is too much for this chapter. It will come soon enough.}

%=================================================================
\section{Saving and sharing Smalltalk code}
\seclabel{Monticello}

Now that you have the Lights Out game working, you probably want to save it somewhere so that you can share it with your friends. Of course, you can save your whole \pharo image, and show off your first program by running it, but your friends probably have their own code in their images, and don't want to give that up to use your image.
What you need is a way of getting source code out of your \pharo image so that other programmers can bring it into theirs.

The simplest way of doing this is by \emph{filing out} the code.  The \actclick menu in the Package pane will give you the option to \menu{various \go{} file out} the whole of package \scat{PBE-LightsOut}.
The resulting file is more or less human readable, but is really intended for computers, not humans.
You can email this file to your friends, and they can file it into their own \pharo images using the file list browser.
\seeindex{saving code}{categories}
\seeindex{category!filing out}{file, filing out}
\seeindex{class!filing out}{file, filing out}
\seeindex{method!filing out}{file, filing out}
\index{file!filing out}

\dothis{\Actclick on the \scat{PBE-LightsOut} package and \menu{various \go{} file out} the contents.}
You should now find a file called ``PBE-LightsOut.st'' in the same folder on disk where your image is saved.
Have a look at this file with a text editor.

\dothis{Open a fresh \pharo image and use the File Browser tool (\menu{Tools \ldots {\go} File Browser}) to \menu{file in} the PBE-LightsOut.st fileout.
Verify that the game now works in the new image.}
\seeindex{category!filing in}{file, filing in}
\seeindex{class!filing in}{file, filing in}
\seeindex{method!filing in}{file, filing in}
\index{file!filing in}

\begin{figure}[ht]
\centerline {\includegraphics[width=\textwidth]{FileIn}}
\caption{Filing in \pharo source code.
\figlabel{filein}}
\end{figure}

\subsection{Monticello packages}
Although fileouts are a convenient way of making a snapshot of the code you have written, they are decidedly ``old school''.
Just as most open-source projects find it much more convenient to maintain their code in a repository using \ind{CVS}\footnote{\url{http://www.nongnu.org/cvs}} or \ind{Subversion}\footnote{\url{http://subversion.tigris.org}},
so \pharo programmers find it more convenient to manage their code using \ind{Monticello} packages. 
These packages are represented as files with names ending in \ct{.mcz}; they are actually zip-compressed bundles that contain the complete code of your \ind{package}.

Using the Monticello package browser, you can save packages to repositories on various types of server, including FTP and HTTP servers; you can also just write the packages to a repository in a local file system directory.
A copy of your package is also always cached on your local hard-disk in the \emph{package-cache} folder. 
Monticello lets you save multiple versions of your program, merge versions, go back to an old version, and browse the differences between versions. 
In fact, Monticello is a distributed revision control system; this means it allows developers to save their work on different places, not on a single repository as it is the case with CVS or Subversion.\damien{Mercurial, Git are examples of distributed revision control system; not sure it's worth mentioning them.}
\seeindex{package browser}{Monticello}

You can also send a \ct{.mcz} file by email. 
The recipient will have to place it in her \emph{package-cache} folder; she will then be able to use Monticello to browse and load it. 
%(It is also possible to load it using the file list, but there is a difference between loading a \ct{.mcz} file using a file list and using Monticello \sd{check}.)

\dothis{Open the Monticello browser from the \menu{World} menu.}
In the right-hand pane of the browser (see \figref{monticello1}) is a list of Monticello repositories, which will include all of the repositories from which code has been loaded into the image that you are using.  
%In addition to \sqsrc servers, Monticello repositories can live in a variety of other places, the simplest being a directory on your local disk.

\begin{figure}[hbt]
\ifluluelse
	{\centerline {\includegraphics[width=\textwidth]{MonticelloBrowser}}}
	{\centerline {\includegraphics[scale=0.7]{MonticelloBrowser}}}
\caption{The Monticello browser.
\figlabel{monticello1}}
\end{figure}

At the top of the list in the Monticello browser is a repository in a local directory called the \emphind{package cache}, which caches copies of the packages that you have loaded or published over the network. This local cache is really handy because it lets you keep your own local history; it also allows you to work in places where you do not have internet access, or where access is slow enough that you do not want to save to a remote repository very frequently.


\subsection{Saving and loading code with Monticello.}
On the left-hand side of the Monticello browser is a list of packages that have a version loaded into the image; packages that have been modified since they were loaded are marked with an asterisk.  (These are sometimes referred to as \subind{package}{dirty} packages.)  If you select a package, the list of repositories is restricted to just those repositories that contain a copy of the selected package.
\seeindex{*}{package, dirty}
\seeindex{dirty package}{package, dirty}

%What is a package?  For now, you can think of a package as a group of  class and method categories that share the same prefix.  Since we put all of the code for the Lights Out game into the category called \scat{PBE-LightsOut}, we can refer to it as the \ct{PBE-LightsOut} package.

\dothis{Add the \ct{PBE-LightsOut} package to your Monticello browser using the \button{+Package} button and type \ct{PBE-LightsOut}.}

\subsection{\ind{\sqsrc}: a \ind{SourceForge} for \pharo.} 
We think that the best way to save your code and share it is to create an account for your project on a \sqsrc server. 
\sqsrc is like SourceForge\footnote{\url{http://sourceforge.net}}: it is a web front-end to a HTTP Monticello server that lets you manage your projects.
There is a public \sqsrc server at \url{http://www.squeaksource.com}, and a copy of the code related to this book is stored there at \url{http://www.squeaksource.com/PharoByExample.html}. You can look at this project with a web browser, but it's a lot more productive to do so from inside \pharo, using the Monticello browser, which lets you manage your packages.

\dothis{Open a web browser to \url{http://www.squeaksource.com}.
Create an account for yourself and then create (\ie ``register'') a project for the Lights Out game.}
\sqsrc will show you the information that you should use when adding a repository using the Monticello browser. 

Once your project has been created on \sqsrc, you have to tell your \pharo system to use it. 

\dothis{With the \ct{PBE-LightsOut} package selected, click the \button{+Repository} button in the Monticello browser.}  You will see a list of the different types of Repository that are available; to add a \sqsrc repository select \menu{HTTP}. You will be presented with a dialog in which you can provide the necessary information about the server.
You should copy the presented template to identify your \sqsrc project, paste it into Monticello and supply your initials and password:

\needlines{5}
\begin{code}{}
MCHttpRepository 
    location: 'http://www.squeaksource.com/!\emph{YourProject}!'
    user: '!\emph{yourInitials}!' 
    password: '!\emph{yourPassword}!'
\end{code}   

\noindent
If you provide empty initials and password strings, you can still load the project, but you will not be able to update it:

\needlines{5}
\begin{code}{}
MCHttpRepository 
    location: 'http://www.squeaksource.com/!\emph{YourProject}!'
    user: '' 
    password: ''
\end{code}   

%You can then load the code in your image by selecting the version you want. You can browse the code without loading it, using the \button{Browse} button.
Once you have accepted this template, your new repository should be listed on the right-hand side of the Monticello browser.

\begin{figure}[hbt]
\ifluluelse
	{\centerline {\includegraphics[width=\textwidth]{BrowseRepository}}}
	{\centerline {\includegraphics[scale=0.7]{BrowseRepository}}}
\caption{Browsing a Monticello Repository
\figlabel{monticello3}}
\end{figure}

\dothis{Click on the \button{Save} button to save a first version of your Lights Out game on \sqsrc.}

To load a package into your image, you must first select a particular version.  You can do this in the repository browser, which you can open using the \button{Open} button or the \actclick menu.  Once you have selected a version, you can load it onto your image.

\dothis{Open the \ct{PBE-LightsOut} repository you have just saved.}

Monticello has many more capabilities, which will be discussed in depth in \charef{env}.
You can also look at the on-line documentation for Monticello at \url{http://www.wiresong.ca/Monticello/}.

%=================================================================
\section{Chapter summary}
In this chapter you have seen how to create categories, classes and methods.  You have see how to use the browser, the inspector, the debugger and the Monticello browser.

\begin{itemize}
  \item Categories are groups of related classes.
  \item A new class is created by sending a message to its superclass.
  \item Protocols are groups of related methods.
  \item A new method is created or modified by editing its definition in the browser and then \emph{accepting} the changes.
  \item The inspector offers a simple, general-purpose GUI for inspecting and interacting with arbitrary objects.
  \item The browser detects usage of undeclared methods and variables, and offers possible corrections.
  \item The \ct{initialize} method is automatically executed after an object is created in \pharo. You can put any initialization code there.
  \item The debugger provides a high-level GUI to inspect and modify the state of a running program.
  \item You can share source code \emph{filing out} a category.
  \item A better way to share code is to use Monticello to manage an external repository, for example defined as a \sqsrc project.
\end{itemize}

%=================================================================
\ifx\wholebook\relax\else\end{document}\fi
%=================================================================
%=================================================================
%%% Local Variables:
%%% coding: utf-8
%%% mode: latex
%%% TeX-master: t
%%% TeX-PDF-mode: t
%%% ispell-local-dictionary: "english"
%%% End:
