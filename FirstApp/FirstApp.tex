% $Author$
% $Date$
% $Revision$

% HISTORY:
% 2006-12-05 - Stef started
% 2006-12-30 - Andrew new material
% 2007-01-10 - Stef edit
% 2007-01-12 - Andrew edit
% 2007-06-07 - Oscar edit
% 2007-07-03 - Stef edit
% 2007-09-06 - Lukas review
% 2007-09-06 - Cassou corrections
% 2007-09-24 - Cassou review
% 2009-07-01 - Oscar migrated to Pharo

%=================================================================
\ifx\wholebook\relax\else
% --------------------------------------------
% Lulu:
	\documentclass[a4paper,10pt,twoside]{book}
	\usepackage[
		papersize={6.13in,9.21in},
		hmargin={.75in,.75in},
		vmargin={.75in,1in},
		ignoreheadfoot
	]{geometry}
	\input{../common.tex}
	\pagestyle{headings}
	\setboolean{lulu}{true}
% --------------------------------------------
% A4:
%	\documentclass[a4paper,11pt,twoside]{book}
%	\input{../common.tex}
%	\usepackage{a4wide}
% --------------------------------------------
    \graphicspath{{figures/} {../figures/}}
	\begin{document}
	% \renewcommand{\nnbb}[2]{} % Disable editorial comments
	\sloppy
\fi
%=================================================================
\chapter{最初のアプリケーション}
\chalabel{firstApp}

この章では、簡単なゲーム:\ind{Lights Out}\footnote{\url{http://en.wikipedia.org/wiki/Lights_Out_(game)}}を作成します。ゲームの作成を通じて、\pharo プログラマーがプログラミングやデバックに使用している多くのツールや、他の開発者とプログラムをやりとりする方法について体験していきます。
この章では、\ind{Browser}、オブジェクトinspector、debugger、\ind{Monticello} \ind{package} Browser を扱います。\st での開発はとても効率的です。\st はとても簡潔なプログラミング言語です。また、開発ツールが言語にとてもうまく統合されています。そのため、あなたは開発の手間に時間をかけることなく、コードを書くことに多くの時間を使うことができるでしょう。

%=================================================================
\section{Lights Outゲーム}

% DON'T USE WRAPFIGURE CLOSE TOO A PAGE BREAK!!! (ON)
%\begin{wrapfigure}[13]{r}{0.35\linewidth}%
%	\vskip -\baselineskip
%	\centerline{\includegraphics[width=.8\linewidth]{GameBoard}}
%	\caption{The Lights Out game board. The user has just clicked the mouse as shown by the cursor.
%	\figlabel{gameBoard}}
%\end{wrapfigure}

\begin{figure}[ht]
	\vskip -\baselineskip
	\centerline{\includegraphics[width=.3\linewidth]{GameBoard}}
	\caption{Lights Outゲームボード。ユーザーは単にボード上のcellをマウスでクリックするだけです。
	\figlabel{gameBoard}}
\end{figure}

\pharo の開発ツールの使い方を体験するために、これから\emph{Lights Out}と呼ばれる簡単なゲームを作っていきます。ゲームボードは\figref{gameBoard}のように淡黄色の\emph{cell}の長方形配列で構成されます。いずれかのcellをマウスでクリックすると、周囲の4つのcellが青色に変わります。もう一度クリックすると、それらはまた淡黄色に戻ります。このゲームの目的は、できるだけ多くのcellを青色に変えることです。
\figref{gameBoard}に示されているゲームはゲームボード本体と100個からなるcellの2種類のオブジェクトで構成されています。このゲームを実装するための\pharo のコードは2つのクラスから成ります。1つはゲームを表すクラス、もう1つはcellを表すクラスです。それでは、\pharo の開発ツールを使って、どのようにこれらのクラスを定義していくかを見ていくことにしましょう。

%=================================================================
\section{パッケージを作成する}

\charef{quick}のBrowserの使い方において、クラスやメソッドを見る方法や新たなメソッドを定義する方法を学びました。ここではパッケージ、カテゴリ、クラスの作り方を学びます。
\index{category!creating}
\index{package!creating}

\dothis{Browserを開きパッケージペインで\actclick 。その後\menu{create package}を選択します。\footnote{パッケージBrowserが標準Browserとしてインストールされているものとします。もし\figref{addPackage}で示されているようなBrowserを使っていない場合はデフォルトのBrowserを変更する必要があります。\faqref{packagebrowser}参照。}}

\begin{figure}[htb]
\begin{minipage}[b]{0.48\textwidth}
\ifluluelse
	{\centerline {\includegraphics[width=0.9\textwidth]{AddPackage}}}
	{\centerline {\includegraphics[scale=0.7]{AddPackage}}}
	\caption{パッケージを追加します。
	\figlabel{addPackage}}
\end{minipage}
\hfill
\begin{minipage}[b]{0.48\textwidth}
\ifluluelse
	{\centerline {\includegraphics[width=0.8\textwidth]{ClassTemplate}}}
	{\centerline {\includegraphics[scale=0.6]{ClassTemplate}}}
	\caption{クラステンプレート。
	\figlabel{classTemplate}}
\end{minipage}
\end{figure}

ダイアログボックスに新たなパッケージ名(\scat{PBE-LightsOut}とします)を入力し、\menu{accept}をクリックしてください(または単にリターンキーを押してください)。すると新たなパッケージが作成され、パッケージ一覧にアルファベット順に表示されます。

%=================================================================
\section{LOCellクラスを定義する}

当然ですが、新しいパッケージにはまだ何のクラスも定義されていません。しかし、メインの編集ペインには新たなクラスを作りやすいように、自動的にクラスのテンプレートが用意されます(\figref{classTemplate}参照)。

このテンプレートは、\ct{Object}クラスに\ct{NameOfSubClass}というサブクラスを新しく作成するようメッセージを送信する\st 式になっています。新しいクラスはいづれの変数も持たず、\scat{PBE-LightsOut}カテゴリーに属するようになっています。

\subsection{カテゴリーとパッケージについて}
\seclabel{categoriesPackages}

従来の\st の世界では\emph{カテゴリー}はよく知られていますが、パッケージは知られていません。この2つの違いは何でしょうか。カテゴリーとは\st のイメージファイル内に関係するクラスを単に集めたものです。\emph{パッケージ}とは関係するクラス\emph{と拡張メソッド}を集めたもので、バージョン管理ツール\ind{Monticello}でバージョン管理する単位などでまとめられます。慣習としてパッケージ名とカテゴリー名には同じ名前を使います。通常はこれら2つの違いを気にする必要はありません。しかし、これらの用語の違いは重要なため、本書では注意深く使い分けています。実際に\ind{Monticello}を使い始めるところで、これらの違いについてより詳しく知ることにしましょう。

\index{package}
\index{category}

\subsection{新しいクラスを作成する}

テンプレートを修正し、目的に合ったクラスを作ってみましょう。

\dothis{クラス作成のテンプレートを修正する手順は以下の通りです:}
\begin{itemize}
  \item \clsind{Object}を\clsind{SimpleSwitchMorph}に書き換えます。
  \item \ct{NameOfSubClass}を\clsind{LOCell}に書き換えます。
  \item インスタンス変数のリストに\ct{mouseAction}を追加します。
\end{itemize}
結果は\clsref{firstClassDef}の通りです。

\needlines{5}
\begin{classdef}[firstClassDef]{\ct| LOCell|クラスを定義する}
SimpleSwitchMorph subclass: #LOCell
   instanceVariableNames: 'mouseAction'
   classVariableNames: ''
   poolDictionaries: ''
   category: 'PBE-LightsOut'
\end{classdef}
\index{browser!defining a class}
\index{class!creation}
\index{Morphic}

この新しい定義は、\ct{SimpleSwitchMorph}という既存のクラスに対して\ct{LOCell}という名のサブクラスを作るようメッセージを送る\st 式になっています(実際には\ct{LOCell}クラスはまだ存在していないため、クラス名を表す\emphind{シンボル} \ct{#LOCell}を引数として渡しています)。
また、このメッセージには新しいクラスのインスタンスが\ct{mouseAction}インスタンス変数を持つことも含まれています。\ct{mouseAction}は、cell上でマウスがクリックされた際の振る舞いを定義するために使用するインスタンス変数になります。

\emph{この時点ではまだ何も作られてはいません。}クラステンプレートペインの境界線が赤に変わっていることに注目してください(\figref{acceptClassDef})。これはまだ\emph{変更が保存されていないこと}を意味しています。実際にこのメッセージを送るためには、この変更を\menu{accept}する必要があります。

\begin{figure}[h!t]
\ifluluelse
	{\centerline {\includegraphics[width=\textwidth]{AcceptClassDef}}}
	{\centerline {\includegraphics[scale=0.7]{AcceptClassDef}}}
\caption{クラス作成テンプレート。
\figlabel{acceptClassDef}}
\end{figure}

\dothis{新しいクラス定義をacceptします}
action-clickして\menu{accept}を選択、もしくはショートカットの\short{s} (save)を実行して下さい。\ct{SimpleSwitchMorph}にメッセージが送られ、新しいクラスがコンパイルされます。
クラス定義をacceptすると、すぐさまクラスが作成されてBrowserのクラスペインに表示されます(\figref{LOCell})。編集ペインにはクラス定義が表示され、その下にある小さなペインdでは簡潔にそのクラスの目的を説明することがうながされます。ここへ記述する内容は\emph{クラスコメント}と呼ばれ、他のプログラマーへクラスの大まかな目的を伝えるためのとても重要なものになります。\st プログラマーはコードの読みやすさを重視するため、メソッド内に詳細なコメントが書かれることは稀です。ここでの哲学は、コードの意図はコード自身に語らせよ、ということです(もしそうなっていない場合は、そうなるまでリファクタリングしましょう!)。

\index{keyboard shortcut!accept}

\subind{クラス}{コメント}には、クラスの詳細な説明ではなく、後進のプログラマーがこのクラスを調査する際に必要となる全体的な目的を簡潔に記述するようにしましょう。

\dothis{LOCellクラスのコメントを入力し、acceptして下さい。これは後に変更することができます。}

\begin{figure}[h!t]
\ifluluelse
	{\centerline {\includegraphics[width=\textwidth]{LOCell}}}
	{\centerline {\includegraphics[scale=0.7]{LOCell}}}
\caption{新しく作成された\ct{LOCell}クラス \figlabel{LOCell}}
\end{figure}

%=================================================================
\section{クラスにメソッドを追加する}

それではこのクラスにメソッドをいくつか追加してみましょう。

\dothis{プロトコルペインにある\prot{-{}-all-{}-}プロトコルを選択します。}

編集ペインにメソッドを作成するためのテンプレートが表示されています。
それを選択して、\mthref{scbecellinitialize}に示されている内容と置き換えてください。

\protindex{all}
\index{method!creation}
\index{browser!defining a method}

\needlines{10}
\begin{numMethod}[scbecellinitialize]{\ct{LOCell}のインスタンスを初期化する}
initialize
   super initialize.
   self label: ''.
   self borderWidth: 2.
   bounds := 0@0 corner: 16@16.
   offColor := Color paleYellow.
   onColor := Color paleBlue darker.
   self useSquareCorners.
   self turnOff
\end{numMethod}
\index{initialization}

\noindent
3行目\ct{''}の文字は、間に何の文字も挟んでいない2つの単一引用符です。二重引用符ではないことに注意してください。\ct{''}は空の文字列を表しています。

\dothis{このメソッドの定義をacceptします。}

上記のコードは何をしているのでしょうか?ここでは詳細については触れずに(本の残りの部分はそのためにあるのです!)、簡単に内容を見ていきます。それでは1行ずつ進んでいきましょう。

\mthind{LOCell}{initialize}と呼ばれるメソッドに注目してください。この名前はとても重要です。initializeと名付けられたメソッドは、慣習的にオブジェクトが生成された直後に呼び出されます。つまり、\ct{LOCell new}が評価されると\ct{initialize}メッセージが新しく作られたオブジェクトに自動的に送られます。\ct{initialize}メソッドはオブジェクトの状態、主にそのインスタンス変数、を設定するために使われます。
\seeindex{Object!initialization}{initialization}
\index{initialization}

このメソッドは、まず最初(2行目)にスーパークラスである\ct{SimpleSwitchMorph}の\ct{initialize}メソッドを実行しています。\lct{スーパークラスの}\ct{initialize}メソッドを実行すると継承された何らかの状態が適切に初期化されると考えてください。処理をする前には、必ずスーパークラスの\ct{initialize}メソッドを呼び出して継承された状態を初期化しておくとよいでしょう。\ct{SimpleSwitchMorph}の\ct{initialize}メソッドが実際に何をするかは知らなくとも構いません。しかし、あえて不明瞭な状態から処理を始めるよりは、いくつかのインスタンス変数に妥当な初期値が設定されることを期待して、スーパークラスの\ct{initialize}メソッドを呼び出すようにしましょう。

メソッドの残りの部分はこのオブジェクトの状態を設定しています。
例えば、\ct{self label: ''}をオブジェクトに送ることで、このオブジェクトのラベルに空の文字列を設定しています。
\pvindex{self}

\ct{0@0 corner: 16@16}については、恐らくいくつかの説明が必要でしょう。\lct{0@0}は、両方共0に設定された$x$,$y$座標の\clsind{Point}オブジェクトを表しています。実際には、\ct{0@0}は数値\ct{0}に\ct{@}メッセージを引数\ct{0}を付けて送るメッセージ式になっています。
% Yuck... the following should be \mthind{Number}{@} 
%%% THIS IS BROKEN -- don't do it! (on)
%\def\atsign{\textsf{@}}%
%{\makeatletter
%	\protected@write\@indexfile{}%
%    {\string\indexentry{\string\atsign|see{Number, \string\atsign}}{\thepage}}%
%	\protected@write\@indexfile{}%
%    {\string\indexentry{Number!\string\atsign|hyperpage}{\thepage}}%
%	\makeatother}
その結果、数値\ct{0}は\ct{Point}クラスに座標(0,0)で新たなインスタンスを作るよう指示します。ここでは新しく作られた\ct{Point}オブジェクトに\ct{corner: 16@16}メッセージを送って、角が\ct{0@0}と\ct{16@16}の\ct{Rectangle}オブジェクトを作っています。新しく作られた\ct{Rectangle}オブジェクトは、スーパークラスから継承された\ct{bounds}変数に割り当てられます。
\pharo の画面の原点は\emph{左上}であり、\emph{下に向かって} $y$座標の値が増えることに注意してください。

メソッドの残りの部分は説明の必要はないでしょう。良い\st のコードを書くコツの1つは、コードが英語の混成語のように読めるような良いメソッド名をつけることです。あなたはブジェクトが自分自身に語りかけ、``\lct{自分は直角の角を使う!}'', ``\lct{自分は電源を切る!}''と言っているイメージを持つことができるはずです。

%=================================================================
\section{オブジェクトをインスペクトする}

新しく\ct{LOCell}オブジェクトを作ってインスペクトすることによって、さきほど書いたコードの効果を確認することができます。

\dothis{workspaceを開いて、\ct{LOCell new}と打ち込み\menu{inspect it}して下さい。}

\begin{figure}[htbp]
   \centering
   \includegraphics[width=\textwidth]{LOCellInspector} 
   \caption{LOCellオブジェクトを調べるために使用したinspector。\figlabel{LOCellInspector}}
\end{figure}

\ind{inspector}の左側のペインには、インスペクトしたオブジェクトのインスタンス変数のリストが表示されます。その中の1つ(例えば\mbox{\ct{bounds}})を選んで見てください。すると、右側のペインにその\ind{インスタンス変数}の値が表示されます。

%  You can also use the inspector to change the value of an instance variable.
%\dothis{Change the value of \ct{bounds} to \ct{0@0 corner: 50@50} and \menu{accept} it.}
%\on{This does not work any more. I get:}
%\ct{OTNamedVariableNode(Object)>>doesNotUnderstand: #selectedClass}
%\on{should use the mini workspace instead to send bounds: ?}

inspectorの下側にあるペインはmini-workspaceです。このmini-workspaceは擬似変数\self が選択されたオブジェクトに束縛されているため、とても便利です。

\dothis{
inspectorウィンドウのルートにあるLOCellを選択してください。下部にあるペインに、\ct{self bounds: (200@200 corner: 250@250)}と打ち込み do it して下さい。inspector上で\ct{bounds}の値が変化します。次にmini-workspaceに\ct{self openInWorld}と打ち込み \menu{do it}して下さい。}

画面の左上の端にcellが表示されます。実際、\ct{bounds}メッセージで表示するよう指示した場所に正確に表示されます。cell上で\metaclick をすると\subind{Morphic}{halo}が現れます。cellを右上2番目の茶色いハンドルを使って移動したり、右下にある黄色のハンドルを使ってリサイズしてみましょう。bounds の値がどのように変化するか、inspectorを使って確認して下さい。(新たな bounds の値を見るために \menu{refresh} \actclick が必要かもしれません。)

\begin{figure}[htbp]
\centering
\ifluluelse
	{\includegraphics[width=\textwidth]{LOCellResize} }
	{\includegraphics[scale=0.7]{LOCellResize} }
\caption{cellのリサイズ。\figlabel{cellresize}}
\end{figure}

\dothis{ピンク色のハンドル\ct{x}をクリックすることでcellは消せます。}

%=================================================================
\section{LOGameクラスを定義する}

これから、\clsind{LOGame}と呼ばれるゲームを作るのに必要な他のクラスを作りましょう。

\dothis{Browserのメインウインドウで、クラス定義テンプレートを作ってください。}
これはパッケージ名をクリックすればできます。以下の通りにコードを書き、\menu{accept}して下さい。

\needlines{6}
\begin{classdef}[sbegame]{LOGameクラスを定義します。}
BorderedMorph subclass: #LOGame
   instanceVariableNames: ''
   classVariableNames: ''
   poolDictionaries: ''
   category: 'PBE-LightsOut'
\end{classdef}

ここでは\clsind{BorderedMorph}をサブクラス化しています。\clsind{Morph}は\pharo のグラフィック図形すべてのスーパークラスです。そして(驚くことに)、\ct{BorderedMorph}は境界線を持った\ct{Morph}です(意味不明 再度翻訳する)。2行目にある引用符の間にインスタンス変数の名前を入れることができますが、今はここは空のままにしておきましょう。

\ct{LOGame}の\mthind{LOGame}{initialize}メソッドを定義しましょう。

\dothis{以下の内容を\ct{LOGame}のメソッドとしてBrowserに書き込んで、\menu{accept}して下さい。}

\begin{numMethod}[sbegameinitialize]{ゲームを初期化する}
initialize
   | sampleCell width height n |
   super initialize.
   n := self cellsPerSide.
   sampleCell := LOCell new.
   width := sampleCell width.
   height := sampleCell height.
   self bounds: (5@5 extent: ((width*n) @(height*n)) + (2 * self borderWidth)).
   cells := Matrix new: n tabulate: [ :i :j | self newCellAt: i at: j ].
\end{numMethod}

%\sd{it would be nicer if we would not have to create an instance of LOCell for nothing}
%\on{yes}

\pharo は用語のいくつかが認識できないと警告をしてくるはずです。まずはじめに、\pharo は、\ct{cellsPerSide}メッセージが認識できないことと、スペルミスの場合に備えた修正案をいくつか提示してきます。

\begin{figure}[htb]
\begin{minipage}{0.48\textwidth}
	\centering
	\ifluluelse
		{\includegraphics[width=\textwidth]{UnknownSelector}}
		{\includegraphics[scale=0.7]{UnknownSelector}}
	\caption{\pharo が未知のセレクターを検出。\figlabel{unknownSelector}}
\end{minipage}
\hfill
\begin{minipage}{0.48\textwidth}
	\centering
	\ifluluelse
		{\includegraphics[width=\textwidth]{DeclareInstanceVar}}
		{\includegraphics[scale=0.7]{DeclareInstanceVar}}
	\caption{新しいインスタンス変数を宣言。\figlabel{declareInstance}}
\end{minipage}
\end{figure}

しかし、\ct{cellsPerSide}はスペルミスではありません。単にまだこのメソッドを定義していないだけです。これから1,2分の間にはこのメソッドを定義することにしましょう。

\dothis{メニューから最初のアイテムを選択し、\ct{cellsPerSide}がスペルミスなどではないことを確認します。}

次に\pharo は\ct{cells}が認識できないと警告してきます。\pharo はこの問題を解消するための案をいくつか提示してきます。

\dothis{\ct{cell}をインスタンス変数として扱いたいので、\menu{declare instance}を選択します。}

最後に\pharo は最終行で使われている\ct{newCellAt:at:}メッセージについて警告してきます。もちろんこれもスペルミスではないので、先ほどと同様に確認してください。

\index{on the fly variable definition}
\index{instance variable definition} 

さて、もう一度このクラスの定義を見てみましょう(\button{instance}ボタンをクリックすることでできます)。Browser上からインスタンス変数\ct{cells}が修正されていることがわかります。

この\ct{initialize}メソッドを見てみましょう。
\ct{| sampleCell width height n |}が記されている行において、4つの一時変数が宣言されています。これらの変数の影響範囲と生存期限がこのメソッドに限定されてしまうため、一時変数と呼ばれています。説明的な名前を持った一時変数を使うとコードがより読みやすくなります。\st では定数と変数を区別するための特別な記述方法はありません。実際これらの4つの"変数"は全て定数です
%(意味不明 再翻訳必要)
。4--7行目でこれらの中身を定義しています。

ゲームボードをどれくらいの大きさにするとよいでしょうか? 必要な数のcellとそれらの境界を表示するために十分な大きさが必要です。cellの数はどれぐらい必要でしょうか? 5? 10? 100? 今はまだどのくらい必要なのかわかりません。今の時点で数を決めてしまうと、後に考えを変えてしまうことになるかもしれません。そこで、その数を知るための責務を、これから1,2分で書き\ct{cellsPerSide}と呼ぶ別のメソッド委譲します。メソッドを定義する前に\ct{cellsPerSide}メッセージを送ることにより、\ct{initialize}のためのメソッド本体をacceptするとき、\pharo から``confirm, correct, or cancel''と警告されます。まだ定義していない他のメソッドを書くという点からは実際に良い練習となるのですが、これを避けてはいけません。そればなぜでしょう? 
\ct{initialize}メソッドを書き始めるまでは、それが必要だと認識していませんでした。またその時点において、流れを邪魔するものなしに、それに意味のある名前を与えることや移動すること、ができます。
%(上3行意味不明 後に再翻訳)

 
4行目においてこのメソッドが使われています。\st 式\ct{self cellsPerSide}は\ct{cellsPerSide}メッセージをを\pvind{self}つまりまさにそのObjectに送ります。ゲームボードの一辺に必要なcellの数を決める責務は\ct{n}にあてがわれました。

次の3行で、新たな\ct{LOCell}オブジェクトを作成し、ゲームボードの幅と高さを適切な一時変数に割り当てています。 

%The eighth line sends the message \ct{bounds:} to \self.
%\ct{bounds:} is a method that we inherit from our superclass; it is used to define the space on the screen that this Morph will occupy.  
%The single colon (\ct{:}) at the end of the name says that \ct{bounds:} expects a single parameter, which should be a rectangle object.
8行目では新しいオブジェクトの\ct{bounds}を設定します。詳細については今はまだそれほど気にはせず、括弧内の式で原点(\ie 左上隅)、(5,5)と右下隅からなる、適切な数のcellを置くことのに十分な数の長方形がつくられることを理解して下さい。次の行では、 \ct{LOGame}オブジェクトのインスタンス変数の\ct{cells}を、適切な数の行と列をもった、新たに作った\clsind{Matrix}に設定しています。上記のことを\ct{new:tabulate:}メッセージを \ct{Matrix}クラス(このクラスももちろんオブジェクトです。そのためメッセージを送ることができます。)に送ることにより実現しています。\mthind{Matrix class}{new:tabulate:}は2つのコロン(\ct{:})を持つので2つの引数を取ります。引数はコロンの後に書きます。もし今まで引数を全て一緒の括弧内に置く言語を使っている場合、最初のうちはこの書き方が異様に感じるかもしれません。慌てることはありません、単に記法の問題なのですから。この書き方は、メソッド名が引数の役割を説明してくれるので、とてもわかりやすい記法です。例えば次の式、\ct{Matrix rows: 5 columns: 2}により、Matrixは2行5列ではなく、5行2列であることは明白です。
\cmindex{Matrix class}{rows:columns:}

\ct{Matrix new: n tabulate: [ :i :j | self newCellAt: i at: j ]}式により\ct{n}{$\times$}\ct{n}の行列とその要素を初期化しています。 各要素の初期値はその座標に依存します。\ct{(i,j)}\textsuperscript{th}の位置にある要素は、\ct{self newCellAt: i at: j}式を評価した結果として初期化されます。

%:===> Pretty-print is broken! (how to pretty-print?)

% \on{I think it is silly to copy paste from the pretty-print view to the normal view}

%That's \ct{initialize}.  When you accept this message body, you might want to take the opportunity to pretty-up the formatting.  You don't have to do this by hand: from the \actclick menu select \menu{more \ldots \go prettyprint}, and the browser will do it for you\damien{this didn't do anything to me}.  You have to \menu{accept} again after you have \subind{method}{pretty-print}{}ed a method, or of course you can \subind{keyboard shortcut}{cancel} 
%(\short{l}\,---\,that's a lower-case letter \emph{L}) if you don't like the result.
%Alternatively, you can set up the browser to use the pretty-printer automatically whenever it shows you code: use the the right-most button in the button bar to adjust the view.
%\seeindex{pretty-print}{method}

%If you find yourself using \menu{more\,\ldots} a lot, it's useful to know that you can hold down the {\sc shift} key when you click to directly bring up the \menu{more \ldots} menu.

%=================================================================
\section{メソッドをプロトコルにまとめる}

メソッドを定義する前に、Browserの上部にある3つめのペインを少し見てみましょう。Browserの最初のペインと同様に、クラスをパッケージに分類することで、2つ目のペインにあるクラス名の非常に長いリストに圧倒されることはなく、3つめのペインでメソッドをカテゴリー化することにより4つ目のペインにあるメソッド名の非常に長いリストに圧倒されることもない。これらのメソッドの分類は``プロトコル''とよばれています。
\index{protocol}

1つのクラスにわずかのメソッドしかない場合、プロトコルにより提供される余分な階層のレベルは必要ではありません。このことは、Browserが仮想プロトコル\prot{-{}-all-{}-}を提供する理由であり、またクラスの全てのメソッドを含んでいることに驚くことはありません。
\protindex{all}

\begin{figure}[htbp]
   \centering
   \includegraphics[width=\textwidth]{Categorize} 
   \caption{すべての未カテゴリー化メソッドを自動的にカテゴリー化します。\figlabel{categorize}}
\end{figure}

この例に沿って作業をしている場合は、3つ目のペインには\protind{まだ分類されていない}プロトコルが含まれているかもしれません。

\dothis{プロトコルペインを\actclick して下さい。その後\menu{various \go categorize automatically}を選択したら、\ct{initialize}メソッドを新しいプロトコル\protind{initialization}に移動してください。}

\pharo はプロトコルが妥当であるかをどのようにして知るのでしょうか。一般に\pharo はそれを知ることはできません。しかしこのケースでは、スーパークラスにも\ct{initialize}メソッドが定義されているので、\pharo は\ct{initialize}メソッドをオーバーライド元のメソッドと同じプロトコルに分類するべきだと判断します。

\index{method!categorize}

%You may find that \pharo has already put your \ct{initialize} method into the \protind{initialization} protocol.
%If so, it's probably because you have loaded a package called \ct{AutomaticMethodCategorizer} into your image.

\paragraph{表記規則} Smalltalkerは、メソッドが属するクラスを識別するために``\verb|>>|''という表記をよく使用します。例えば\ct{LOGame}クラスのcellsPerSideメソッドは\ct{LOGame>>cellsPerSide}と表します。
この表記が\st 式では\emph{ない}ことを明示するため、本書ではその代わりに特殊なシンボル\ct{>>>}を使うことにします。さきほどのメソッドであれば、文中では\ct{LOGame>>>cellsPerSide}と表します。

\cmindex{Behavior}{>>}

以後、本書ではこの記法でメソッド名を記述します。もちろん、実際にBrowser上でコードを打ち込むときはクラス名や\ct{>>>}を打ち込む必要はありません。クラスペインで適切なクラスが選択されていることを確認するだけで十分です。

それでは、これから\ct{LOGame>>>initialize}メソッドで使われている他の2つのメソッドを定義していきます。両方とも\prot{initialization} プロトコルに入れることにします。

\begin{method}[sbegamecellsperside]{定数メソッド}
LOGame>>>cellsPerSide
   "ゲームの側面にあるcellの数"
   ^ 10
\end{method}
\cmindex{LOGame}{cellsPerSide}
\index{constant methods}

このメソッドは非常にシンプルで、単に10という値を返すだけです。このように値をメソッドとして表現しておくと、プログラムが成長してこの値が単純に定まらなくなった場合に、値を計算して返すようにメソッドを変更することで適応できるという利点があります。

\needlines{10}
\begin{method}[newCellAt:at:]{初期化補助メソッド}
LOGame>>>newCellAt: i at: j
   "位置(i,j)にあるcellを作成し、適切な画面の位置に追加。新しいcellを返す"
   | c origin |
   c := LOCell new.
   origin := self innerBounds origin.
   self addMorph: c.
   c position: ((i - 1) * c width) @ ((j - 1) * c height) + origin.
   c mouseAction: [self toggleNeighboursOfCellAt: i at: j]
\end{method}
\cmindex{LOGame}{newCellAt:at:}
%   ^ c      "omit this final line to create a bug"

\dothis{\ct{LOGame>>>cellsPerSide}メソッドと\ct{LOGame>>>newCellAt:at:}メソッドを追加してください。}

新しい\ct{toggleNeighboursOfCellAt:at:}セレクタと\ct{mouseAction:}セレクタのスペルをチェックしましょう。

\Mthref{newCellAt:at:} 新しい\ct{toggleNeighboursOfCellAt:at:}セレクタと\ct{mouseAction:}セレクタのスペルをチェックしましょう。
メソッド2.6において新しいLOCellは、cell\nlclsind{行列}{ぎょうれつ}の\ct{(i,j)}の位置を特定しています。最後の行では新たにcellの\ct{mouseAction}に\emph{block}
\mbox{\lct{[self toggleNeighboursOfCellAt: i at: j ]}.}を定義しています。これによりマウスをクリックしたときに呼び出される振る舞いを定義しています。それに対応する定義もする必要があります。

\begin{method}[toggleNeighboursOfCellAt:at:]{呼び出されるメソッド}
LOGame>>>toggleNeighboursOfCellAt: i at: j
   (i > 1) ifTrue: [ (cells at: i - 1 at: j ) toggleState].
   (i < self cellsPerSide) ifTrue: [ (cells at: i + 1 at: j) toggleState].
   (j > 1) ifTrue: [ (cells at: i  at: j - 1) toggleState].
   (j < self cellsPerSide) ifTrue: [ (cells at: i at: j + 1) toggleState].
\end{method}
\cmindex{LOGame}{toggleNeighboursOfCellAt:at:}

\Mthref{toggleNeighboursOfCellAt:at:} はcell(\ct{i}, \ct{j})の東西南北の位置にある4つのセルの状態を切り替えます。唯一の難点はボードが有限であることです。そのため隣接するcellが存在することを、状態を切り替える前に確認しなければなりません。

\dothis{このメソッドを\prot{game logic}と呼ばれる新たなプロトコルに置きます。(プロトコルペイン上で新たなプロトコルを\actclick して下さい)}
メソッドを移動するためには、たんにその名前をクリックし、新たに作ったプロトコルまで持って行ってください。 (\figref{dragMethod}).

\begin{figure}[htbp]
   \centering
   \ifluluelse
		{\includegraphics[width=\textwidth]{DragMethod} }
		{\includegraphics[scale=0.7]{DragMethod} }
   \caption{メソッドをプロトコルへ持って行く。\figlabel{dragMethod}}
\end{figure}

Lights Outゲームを完成するために、\ct{LOCell}クラスでマウスのイベントを操作する、もう2つのメソッドを定義する必要があります。
\begin{method}[mouseAction:]{典型的なセッターメソッド}
LOCell>>>mouseAction: aBlock
   ^ mouseAction := aBlock
\end{method}
\cmindex{LOCell}{mouseAction:}

\Mthref{mouseAction:} はcellの\ct{mouseAction}変数に引数を設定してその値を返す以上のことはしていません。このようにインスタンス変数の値を\emph{変更する}メソッドは\emph{セッターメソッド}と呼ばれています。また、インスタンス変数の現在の値を返すメソッドは\emph{ゲッターメソッド}と呼ばれています。
\seeindex{setter method}{accessor}
\seeindex{getter method}{accessor}

もし他のプログラミング言語でゲッターメソッドとセッターメソッドを使うことに慣れていたら、これらのメソッドを\ct{setmouseAction}や\ct{getmouseAction}と呼びたくなるかもしれません。しかし\st の習慣では異なります。ゲッターメソッド名は常にそのメソッドで得るインスタンス変数名と同じ名前にします。セッターメソッド名も同様ですが、\ct{mouseAction}と\ct{mouseAction:}のように後端に``\ct{:}''を付けます。

セッターメソッドとゲッターメソッドはあわせて\emphind{アクセサ}メソッドと呼ばれています。また、これらは慣習として\protind{accessing}プロトコルに置かれます。\st では全てのインスタンス変数はそれらを持つオブジェクトのものです。そのため、\st 言語で他のオブジェクトがそれらの変数を読み書きするには、このようにアクセサメソッドを用意して行う他ありません\footnote{実際にはインスタンス変数はサブクラスの中からも参照できます。}。

\dothis{\ct{LOCell}クラスを表示し、\ct{LOCell>>>mouseAction:}を定義してください。定義が終わったら、そのメソッドを\prot{accessing}プロトコルに置いてください。}

最後に\ct{mouseUp:}メソッドを定義する必要があります。このメソッドは画面上のcellの上にあるマウスのマウスボタンが離されたときに自動的に呼び出されます。

\begin{method}[sbecellmouseup]{イベントハンドラ}
LOCell>>>mouseUp: anEvent
   mouseAction value
\end{method}
\cmindex{LOCell}{mouseUp:}

\dothis{\ct{LOCell>>>mouseUp:}を追加し、\menu{categorize automatically}を実行してください。}

\index{method!categorize}

このメソッドは\lct{value}メッセージを\ct{mouseAction}インスタンス変数に保存されているオブジェクトに送ります。\ct{LOGame>>>newCellAt: i at: j}で以下のコード片を\ct{mouseAction:}に指定したことを思い出してください。

\ct{[self toggleNeighboursOfCellAt: i at: j ]}

\noindent
\lct{value}メッセージを送ると、このコード片が評価されてcellの状態が切り替わることになります。

%=================================================================
\section{コードを実行してみましょう}

以上でLights Outゲームは完成です!

手順通りに進んできたのなら、2つのクラスと7つのメソッドから構成されているこのゲームで遊ぶことができるはずです。

\dothis{workspace上で\ct{LOGame new openInWorld}と打ち込み、\menu{do it}して下さい。}

ゲームが開き、cellをクリックでき、またそれがどのように動作するかを見ることができるでしょう。

理論上は動くはずです\ldots{}
理論上は動くはずです…。cell上でクリックするとき、エラーメッセージと共に\clsind{PreDebugWindow}と呼ばれる\emphind{つうち@通知}ウインドウが表示されます。 \figref{lightsOutError}に描かれているように、このウインドウは、\ct{MessageNotUnderstood: LOGame>>>toggleState} と言ってきています。

\begin{figure}[ht]
\ifluluelse
	{\centerline{\includegraphics[width=\textwidth]{Error}}}
	{\centerline{\includegraphics[scale=0.7]{Error}}}
\caption{cellをクリックしたときにゲームでバグが発生
\figlabel{lightsOutError}}
\end{figure}

\noindent
何が起こったのでしょう? それを見つけるために、\st で使うとても役に立つツール、\ind{debugger}を使いましょう。

\dothis{通知ウインドウにある\menu{debug}ボタンをクリックしてください。}
debuggerが表示されます。
debuggerウインドウの上側には実行処理のスタックが表示されます。そこでは全ての有効なメソッドが表示されます。そのなかの1つを選んぶと真ん中のペイン上で\st コードがそのメソッドの中で実行され、エラーを引き起こした部分がハイライトされます。

\dothis{(上部付近の)\ct{LOGame>>>toggleNeighboursOfCellAt:at:}と表示されている行をクリックしてください。}
エラーの発生した、このメソッドにある実行コンテキストをdebuggerで見ることができます(\figref{debugToggle}).

\begin{figure}[ht]
\ifluluelse
	{\centerline {\includegraphics[width=\textwidth]{Debugger}}}
	{\centerline {\includegraphics[scale=0.7]{Debugger}}}
\caption{debugger上で\ct{toggleNeighboursOfCell:at:}メソッドを選択する。
\figlabel{debugToggle}}
\end{figure}

debuggerウインドウの下側には2つの小さなinspectorウインドウがあります。左側のウインドウでは、選んだコードを実行するメッセージを受けたオブジェクトの中の値を見ることができます。つまりこのインスタンス変数の値をここで見ることができます。右側のウインドウでは、 現在実行したメソッドそれ自体が表すオブジェクトの中の値を見ることができます。つまりメソッドのパラメーター値や一時変数の値などを見ることができます。

debuggerを使うことによって、workspace上で式を評価するのと同様に、オブジェクトの中のパラメータや局所変数の値を見ながら、コードを1行ずつ実行することができます。debuggerを使って一番驚くべきことは、デバッグ中にコードを書き換えることができることです。Smalltalkerの内の何人かはむしろBrowser上よりも、ほとんどの時間debugger上でプログラムをしています。debugger上でプログラムをする利点は、今書いているメソッドが意図した通りに動くかどうかを、実際の実行コンテキスト中のパラメーターと共に見ることができることです。

この場合、パネル上部にある最初の行に\ct{toggleState}メッセージが、明らかに\lct{LOCell}のインスタンスとなっているはずの、\ct{LOGame}インスタンスに送られるのが見えるでしょう。問題は\ct{cells}行列の初期化にあるように思われます。
\cmind{LOGame}{initialize}のコードをBrowserで見ると\ct{cells}には\ct{newCellAt:at:}の戻り値が入力されていることがわかります。しかしそのメソッドを見ると、そこには何も返っていないことがわかります。デフォルトではメソッドは \ct{self}を戻り値として返します。  \ct{newCellAt:at:}の場合は実際に\ct{LOGame}のインスタンスがそれに該当します。
\index{method!returning self}

\dothis{debuggerウインドウを閉じで下さい。
その後\ct{c}を返すために、``\ct{^ c}''式を \ct{LOGame>>>newCellAt:at:}メソッドの最後に追加して下さい。
% It should now look as shown in \mthref{newCellAt:at:nobug}.}
(\mthref{newCellAt:at:nobug}参照。)}

% \needlines{6}
\begin{method}[newCellAt:at:nobug]{バグの修正。}
LOGame>>>newCellAt: i at: j
   "位置(i,j)にあるcellを作成し、適切な画面の位置に追加。新しいcellを返す"
   | c origin |
   c := LOCell new.
   origin := self innerBounds origin.
   self addMorph: c.
   c position: ((i - 1) * c width) @ ((j - 1) * c height) + origin.
   c mouseAction: [self toggleNeighboursOfCellAt: i at: j].
   ^ c
\end{method}
\cmindex{LOGame}{newCellAt:at:}

\noindent
\charef{quick}に戻り、\st では\subind{method}{値}を\ind{返す}時には\ct{^}を使うこと、そのためには\verb|^|と打ち込むことを思い出してください。
% \index{^@\verb|^|}
\index{^@{$\uparrow$}|\ \hfill$\Longrightarrow$ {return}~~~}

debuggerウインドウ上で直接コードを修正したのち、\menu{Proceed}をクリックしアプリケーションを継続して実行することがしばしばあります。今回の場合はバグが間違ったメソッドではなく、オブジェクトの初期化処理にあるため、最も簡単な処理方法はdebuggerウインドウを閉じることです。debuggerウインドウを閉じることにより、(\subind{Morphic}{halo})で実行中のゲームのインスタンスを破棄し、新たにインスタンスを作成することができます。

%Indeed, even in this case it would be possible to \menu{do} \ct{self initialize} and then \menu{Proceed} the \ct{toggleNeighboursOfCellAt:at:} method.
%\ab{St\'eph, did you try this?  It seems to me that it ought to work, but when I tried it, it messed up my image.}
% ON : It messed me up too!  Better not propose this.

\dothis{もう一度\ct{LOGame new openInWorld} を実行してください。}

これでこのゲームは正常...あるいはそれに近い動作をするはずです。マウスをクリックしリリースする間にマウスを移動させると、マウス上のcellも切り替わります。このことから、この動作は \ct{SimpleSwitchMorph} を継承したクラスの振る舞いによるものであることがわかります。単に\ct{mouseMove:}をオーバーライドすることで、何もしないように修正しましょう。

% \needlines{6}
\begin{method}[mouseMove:]{マウスの動作をオーバーライドする。}
LOGame>>>mouseMove: anEvent
\end{method}

ついに完成しました!

%\sd{It would be good to have a word about the debugger buttons into, step.... Or to have a separate chapter, we would use the material I wrote for my turtle book, please check it.}
%\on{I think that is too much for this chapter. It will come soon enough.}

%=================================================================
\section{\st のコードの保存と共有}
\seclabel{Monticello}

現在、動作するLights Outゲームが有ります。このゲームを友人と共有するためにどこかに保存したいと思うかもしれません。もちろん、\pharo のイメージファイル全体を保存し、最初のプログラムを走って見せて回ることもできます。しかしその友人は自分のコードが入ったイメージファイルを持っていて、あなたのイメージファイルを欲しがらないかもしれません。この場合必要なのは他のプログラマーが自分のイメージファイルに取り込めるよう、\pharo のイメージファイルからソースコードを取り出すことです。

そのための最も簡単な方法はコードを\emph{File out}することです。パッケージペインのメニューを\actclick するとオプションで\scat{PBE-LightsOut}パッケージ全てを\menu{various \go  File out}できます。取り出されたファイルは多かれ少なかれ人の読める形式ですが、本来はコンピュターが読むための形式であり、人のためのものではありません。このファイルを友人にメールで送ることが出来ます。そしてその友人たちはファイルをfile list Browserを使うことで自分の\pharo イメージファイルに組み込むことができます。
\seeindex{saving code}{categories}
\seeindex{category!filing out}{file, filing out}
\seeindex{class!filing out}{file, filing out}
\seeindex{method!filing out}{file, filing out}
\index{file!filing out}

\dothis{\scat{PBE-LightsOut}パッケージを\actclick し、内容を\menu{various \go  File out}して下さい。}
``PBE-LightsOut.st''という名のファイルがイメージファイルがある同一フォルダ内にあるのを見つけることができます。テキストエディタでこのファイルを見てみてください。

\dothis{まっさらな\pharo のイメージファイルを開き、File Browserツールを使い、先ほど取り出したPBE-LightsOut.stを(\menu{Tools \ldots {\go} File Browser})から\menu{File in}して下さい。そしてまっさらなイメージファイル上のゲームが動作することを確認して下さい。}
\seeindex{category!filing in}{file, filing in}
\seeindex{class!filing in}{file, filing in}
\seeindex{method!filing in}{file, filing in}
\index{file!filing in}

\begin{figure}[ht]
\centerline {\includegraphics[width=\textwidth]{FileIn}}
\caption{\pharo にソースコードをFile inする。
\figlabel{filein}}
\end{figure}

\subsection{Monticelloパッケージ}
ファイル出力は書いたコードを保存するのには便利な方法ですが、これは明らかに古い方法です。
大部分のオープンソースプロジェクトでは、リポジトリの中のそれらのコードを保守するために役立つ\ind{CVS}\footnote{\url{http://www.nongnu.org/cvs}}や\ind{Subversion}\footnote{\url{http://subversion.tigris.org}}を使っています。\pharo プログラマーにはコードを管理するのにもっと役に立つ\ind{Monticello}パッケージがあります。これらのパッケージはファイル名の最後に \ct{.mcz}が付いています。これらのファイルの正体は\ind{パッケージ}のコード全てをまとめてZIPで圧縮したものです。

Monticello package Browserを使うことにより、FTPやHTTPなどを含む色々な種類のサーバーのリポジトリにパッケージを保存することができます。もちろん自分で管理しているローカルディレクトリのリポジトリにパッケージを保存することもできます。パッケージのコピーは常にローカルハードディスクの\emph{package-cache}フォルダにキャッシュされています。Monticelloにより複数のバージョンのプログラムを保存したり、マージしたり、古いバージョンに戻したり、バージョン間の違いを閲覧することができます。実際に\ind{Monticello}はリビジョン管理システムとして広まりました。このことは開発者の成果物をCVSやSubversionのように一箇所のリポジトリにではなく、いつくかの異なった場所に保管することができることを意味します。\damien{Mercurial, Git はリビジョン管理システムの例ではあるが、ここで触れる必要があるかは確かではない。}

もちろん\ct{.mcz}ファイルをEメールで送ることも出来ます。
受け取った人はそれを\emph{package-cache}フォルダーに置くことにより。そのファイルをMonticelloを使い閲覧したりロードしたりすることができます。
%(It is also possible to load it using the file list, but there is a difference between loading a \ct{.mcz} file using a file list and using Monticello \sd{check}.)

\dothis{\menu{World}メニューからMonticello Browserを開いてください。}
Browserの右側のペイン(\figref{monticello1}参照)にはMonticelloのリポジトリのリストがあります。このリストには、今使っているイメージにロードしたコードを含め全てのリポジトリを含んでいます。
 
%In addition to \sqsrc servers, Monticello repositories can live in a variety of other places, the simplest being a directory on your local disk.

\begin{figure}[hbt]
\ifluluelse
	{\centerline {\includegraphics[width=\textwidth]{MonticelloBrowser}}}
	{\centerline {\includegraphics[scale=0.7]{MonticelloBrowser}}}
\caption{Monticello Browser。
\figlabel{monticello1}}
\end{figure}

Monticello Browserのリスト上部には \emphind{package cache}と呼ばれるローカルディレクトリがあります。それはネットワーク越しにロードしたり、公開したパッケージのコピーをキャッシュしています。
ローカルキャッシュは自分のローカル履歴が保存できるため非常に便利です。つまりインターネットに接続できない場所や、回線が遅いために何度も保存したいとは思わないぐらい遠くのリポジトリでも作業することができることを意味します。


\subsection{Monticelloを使ったコードの保存と読み込み}
Monticello Browserの左側にはイメージ内に読み込まれたバージョンのパッケージリストが表示されます。そして読み込まれた後に修正されたパッケージにはアスタリスクで印がつけられています。(これらは時に\subind{package}{dirty}と呼ばれています。) パッケージを選択すると、リポジトリのリストは、選択したパッケージのコピーを含むそれらのリポジトリに制限されます。(意味不明 後に再翻訳)
\seeindex{*}{package, dirty}
\seeindex{dirty package}{package, dirty}

%What is a package?  For now, you can think of a package as a group of  class and method categories that share the same prefix.  Since we put all of the code for the Lights Out game into the category called \scat{PBE-LightsOut}, we can refer to it as the \ct{PBE-LightsOut} package.

\dothis{\button{+Package}ボタンと\ct{PBE-LightsOut}と打ち込むことにより、\ct{PBE-LightsOut}パッケージをMonticello Browserに追加します。}

\subsection{\ind{\sqsrc}: \pharo のための\ind{SourceForge}} 
コードを保存し、また共有する一番良い方法は\sqsrc サーバー上のでアカウントを取りプロジェクトを作ることだと考えます。\sqsrc はSourceForge\footnote{\url{http://sourceforge.net}}のようなものです。それはHTTPの\ind{Monticello}サーバーのwebフロントエンドでプロジェクトを管理するることができます。
\url{http://www.squeaksource.com}に公開されている\sqsrc サーバーがあり、この本に関係したコードのコピーが\url{http://www.squeaksource.com/PharoByExample.html}に保管されています。このプロジェクトは web browserから見ることができますが、Monticello Browser を使うことにより、パッケージを管理するためにより多くの生産的な作業が\pharo から行えます。

\dothis{web browserで\url{http://www.squeaksource.com}を開いてください。アカウントを作り、Lights Outゲームのためのプロジェクトを作ってください(つまり登録してください)。}
Monticello Browserを使いリポジトリに追加するとき、\sqsrc で使うべき情報を見ることでしょう。

一度\sqsrc にプロジェクトが作られると、\pharo システムにそれを使うよう伝える必要があります。

\dothis{\ct{PBE-LightsOut}パッケージを選択し、Monticello Browserの\button{+Repository}ボタンをクリックして下さい。}  利用可能なさまざまなタイプのリポジトリのリストが表示されます。\sqsrc リポジトリに追加するためには\menu{HTTP}を選択してください。サーバに関する必要な情報を提供できるのダイアログが表示されます。\sqsrc プロジェクトを識別するため、提示されたテンプレートをコピーし、Monticelloに貼り付け、自分のイニシャルとパスワードを提供する必要があります。 

\needlines{5}
\begin{code}{}
MCHttpRepository 
    location: 'http://www.squeaksource.com/!\emph{YourProject}!'
    user: '!\emph{yourInitials}!' 
    password: '!\emph{yourPassword}!'
\end{code}   

\noindent
もしイニシャルとパスワードを空にした場合、プロジェクトを読み込むことはでいますが更新することはできません。

\needlines{5}
\begin{code}{}
MCHttpRepository 
    location: 'http://www.squeaksource.com/!\emph{YourProject}!'
    user: '' 
    password: ''
\end{code}   

%You can then load the code in your image by selecting the version you want. You can browse the code without loading it, using the \button{Browse} button.
一度このテンプレートを受諾すると新たなリポジトリーがMonticello Browserの右側のリストに表示されます。

\begin{figure}[hbt]
\ifluluelse
	{\centerline {\includegraphics[width=\textwidth]{BrowseRepository}}}
	{\centerline {\includegraphics[scale=0.7]{BrowseRepository}}}
\caption{\ind{Monticello}リポジトリの閲覧
\figlabel{monticello3}}
\end{figure}

\dothis{\button{Save}ボタンを押し最初のバージョンのLight Outゲームを\sqsrc に保存してください。}

パッケージを自分のイメージファイルに読み込むためには、最初にバージョンを特定しなければいけません。Repository Browserを使うことでこれはできます。そのために\button{Open}ボタンを使うかメニューを\actclick してRepository Browserを開いてください。一度バージョンを選択すると、イメージファイルにパッケージを読み込むことができます。

\dothis{\ct{PBE-LightsOut}リポジトリを開き、単に保存してください。}

\charef{env}で詳細に述べますがMonticelloは色々なことができます。
またMonticelloについては\url{http://www.wiresong.ca/Monticello/}でオンラインドキュメントを読むことができます。

%=================================================================
\section{章のまとめ}
この章ではカテゴリ、クラス、そしてメソッドの作り方について学びました。
Browser、inspector、debugger、そしてMonticello Browserの使い方について学びました。

\begin{itemize}
  \item 関連するクラスのグループの分類
  \item スーパークラスへメッセージを送ることによる新たなクラスの作成
  \item プロトコルが、関連するメソッドをグループ化すること
  \item Browser上で定義し変更を\emph{acceptする}ことにより、新たなメソッドが作られたり修正されること
  \item inspectorは簡単に、任意のオブジェクトの中の値を見たり割り込んだりするための多目的GUIであること。
  \item Browserは宣言されていないメソッドや変数の使い方を検出し、利用可能なコレクションを提供すること。
  \item \pharo において\ct{initialize}メソッドはオブジェクトが作られた後自動的に実行される。どんな初期化コードもそこに書くことができる。
  \item debuggerは動作しているプログラムの状態を見たり修正するための高度なGUIを提供する。
  \item カテゴリーを\emph{File out}することでソースコードを共有できる。
  \item コードを共有するより良い方法としては、外部のリポジトリ例えば\sqsrc のプロジェクトとして定義し、Monticelloを使いを管理する。
\end{itemize}

%=================================================================
\ifx\wholebook\relax\else\end{document}\fi
%=================================================================
%=================================================================
%%% Local Variables:
%%% coding: utf-8
%%% mode: latex
%%% TeX-master: t
%%% TeX-PDF-mode: t
%%% ispell-local-dictionary: "english"
%%% End:
