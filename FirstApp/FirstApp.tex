% $Author$
% $Date$
% $Revision$

% HISTORY:
% 2006-12-05 - Stef started
% 2006-12-30 - Andrew new material
% 2007-01-10 - Stef edit
% 2007-01-12 - Andrew edit
% 2007-06-07 - Oscar edit
% 2007-07-03 - Stef edit
% 2007-09-06 - Lukas review
% 2007-09-06 - Cassou corrections
% 2007-09-24 - Cassou review
% 2009-07-01 - Oscar migrated to Pharo

%=================================================================
\ifx\wholebook\relax\else
% --------------------------------------------
% Lulu:
	\documentclass[a4paper,10pt,twoside]{book}
	\usepackage[
		papersize={6.13in,9.21in},
		hmargin={.75in,.75in},
		vmargin={.75in,1in},
		ignoreheadfoot
	]{geometry}
	\input{../common.tex}
	\pagestyle{headings}
	\setboolean{lulu}{true}
% --------------------------------------------
% A4:
%	\documentclass[a4paper,11pt,twoside]{book}
%	\input{../common.tex}
%	\usepackage{a4wide}
% --------------------------------------------
    \graphicspath{{figures/} {../figures/}}
	\begin{document}
	% \renewcommand{\nnbb}[2]{} % Disable editorial comments
	\sloppy
\fi
%=================================================================
\chapter{最初のアプリケーション}
\chalabel{firstApp}

この章では、簡単なゲーム:\ind{Lights Out}\footnote{\url{http://en.wikipedia.org/wiki/Lights_Out_(game)}}を作成します。ゲームの作成を通じて、プログラミングやデバックに\pharo プログラマーが使用するツール群や、プログラムを他の開発者とやりとりする方法について体験していきます。
この章では、\ind{Browser}、オブジェクト・インスペクタ、デバッガ、\ind{Monticello} \ind{package} ブラウザ を扱います。\st による開発はとても効率的です。\st はとても簡潔なプログラミング言語です。また、開発ツールが言語に非常によく統合されています。そのため、あなたは開発の手間に時間をかけることなく、コードを書くことに多くの時間を使うことができるでしょう。

%=================================================================
\section{Lights Outゲーム}

% DON'T USE WRAPFIGURE CLOSE TOO A PAGE BREAK!!! (ON)
%\begin{wrapfigure}[13]{r}{0.35\linewidth}%
%	\vskip -\baselineskip
%	\centerline{\includegraphics[width=.8\linewidth]{GameBoard}}
%	\caption{The Lights Out game board. The user has just clicked the mouse as shown by the cursor.
%	\figlabel{gameBoard}}
%\end{wrapfigure}

\begin{figure}[ht]
	\vskip -\baselineskip
	\centerline{\includegraphics[width=.3\linewidth]{GameBoard}}
	\caption{Lights Outゲームボード。ユーザーはボード上のセルをマウスでクリックするだけです。
	\figlabel{gameBoard}}
\end{figure}

\pharo の開発ツールの使い方を体験するために、これから\emph{Lights Out}と呼ばれる簡単なゲームを作っていきます。ゲームボードは\figref{gameBoard}のように淡黄色の\emph{セル}の長方形配列で構成されます。セルのどれかをマウスでクリックすると、周囲の4つのセルが青色に変わります。もう一度クリックすると、それらはまた淡黄色に戻ります。このゲームの目的は、できるだけ多くのセルを青色に変えることです。
\figref{gameBoard}に示されているゲームはゲームボード本体と、100個のセルといった2種類のオブジェクトで構成されています。\pharo のコードとしては、2つのクラスでゲームを実装しています。1つはゲームを表すクラス、もう1つはセルを表すクラスです。それでは、\pharo の開発ツールを使い、どのようにこれらのクラスを定義していくかを見ていくことにしましょう。

%=================================================================
\section{パッケージを作成する}

\charef{quick}のブラウザの紹介で、クラスやメソッドのブラウザの仕方や、新たなメソッドを定義する方法を学びました。ここではパッケージ、カテゴリ、クラスの作り方を学びます。
\index{category!creating}
\index{package!creating}

\dothis{ブラウザを開きパッケージペインで\actclick し、\menu{create package}を選択します。
\footnote{パッケージ・ブラウザが標準ブラウザとしてインストールされているものとします。もし\figref{addPackage}で示されているようなブラウザでない場合は、デフォルトのブラウザを変更する必要があります。\faqref{packagebrowser}参照。}}

\begin{figure}[htb]
\begin{minipage}[b]{0.48\textwidth}
\ifluluelse
	{\centerline {\includegraphics[width=0.9\textwidth]{AddPackage}}}
	{\centerline {\includegraphics[scale=0.7]{AddPackage}}}
	\caption{パッケージを追加します。
	\figlabel{addPackage}}
\end{minipage}
\hfill
\begin{minipage}[b]{0.48\textwidth}
\ifluluelse
	{\centerline {\includegraphics[width=0.8\textwidth]{ClassTemplate}}}
	{\centerline {\includegraphics[scale=0.6]{ClassTemplate}}}
	\caption{クラステンプレート。
	\figlabel{classTemplate}}
\end{minipage}
\end{figure}

ダイアログボックスに新たなパッケージ名(\scat{PBE-LightsOut}とします)を入力し、\menu{accept}をクリックしてください(または単にリターンキーを押してください)。すると新たなパッケージが作成され、パッケージ一覧にアルファベット順に整列されて表示されます。

%=================================================================
\section{LOCellクラスを定義する}

当然ですが、新しいパッケージにはまだ何のクラスも定義されていません。しかし、メインの編集ペインには新たなクラスを作りやすいように、自動的にクラスのテンプレートが用意されています(\figref{classTemplate}参照)。

このテンプレートは、\ct{Object}クラスに\ct{NameOfSubClass}というサブクラスを作成するメッセージを送信する\st 式になっています。新しいクラスはいずれの変数も持たず、\scat{PBE-LightsOut}カテゴリに属するようになっています。

\subsection{カテゴリとパッケージについて}
\seclabel{categoriesPackages}

従来から\st にあった\emph{カテゴリ}はよく知られていますが、パッケージは知られていません。この2つの違いは何でしょうか。カテゴリとは\st 関連するクラスをイメージファイル内で集約しているだけのものです。\emph{パッケージ}とは関係するクラス\emph{と拡張メソッド}を集めたもので、\ind{Monticello}のようなツールでバージョン管理されるものです。慣習としてパッケージ名とカテゴリ名には同じ名前を使います。通常は2つの違いを気にする必要はありません。しかし、これらの用語の違いが重要なこともあるため、本書では注意深く使い分けています。実際に\ind{Monticello}を使い始めるところで、より詳しく見ていくことにしましょう。

\index{package}
\index{category}

\subsection{新しいクラスを作成する}

テンプレートを修正し、目的に合ったクラスを作ってみましょう。

\dothis{クラス作成のテンプレートを修正する手順は以下の通りです:}
\begin{itemize}
  \item \clsind{Object}を\clsind{SimpleSwitchMorph}に書き換えます。
  \item \ct{NameOfSubClass}を\clsind{LOCell}に書き換えます。
  \item インスタンス変数のリストに\ct{mouseAction}を追加します。
\end{itemize}
結果は\clsref{firstClassDef}の通りです。

\needlines{5}
\begin{classdef}[firstClassDef]{\ct| LOCell|クラスを定義する}
SimpleSwitchMorph subclass: #LOCell
   instanceVariableNames: 'mouseAction'
   classVariableNames: ''
   poolDictionaries: ''
   category: 'PBE-LightsOut'
\end{classdef}
\index{browser!defining a class}
\index{class!creation}
\index{Morphic}

この新しい定義は、\ct{SimpleSwitchMorph}という既存のクラスに対し、\ct{LOCell}というサブクラスを作るようメッセージを送る\st 式になっています(実際には\ct{LOCell}クラスはまだ存在していないため、クラス名を表す\emphind{シンボル} \ct{#LOCell}を引数として渡しています)。
また、このメッセージには新しいクラスのインスタンスが\ct{mouseAction}インスタンス変数を持つことも含まれています。\ct{mouseAction}は、セル上でマウスがクリックされた時の振る舞いを定義するためのインスタンス変数となります。

\emph{この時点ではまだ何も作られてはいません。}クラステンプレートペインの境界線が赤に変わっていることに注目してください(\figref{acceptClassDef})。これはまだ\emph{変更が保存されていないこと}を意味しています。実際にこのメッセージを送るためには、この変更を\menu{accept}する必要があります。

\begin{figure}[h!t]
\ifluluelse
	{\centerline {\includegraphics[width=\textwidth]{AcceptClassDef}}}
	{\centerline {\includegraphics[scale=0.7]{AcceptClassDef}}}
\caption{クラス作成テンプレート。
\figlabel{acceptClassDef}}
\end{figure}

\dothis{新しいクラス定義を「了解(accept)」します}
action-clickして\menu{accept}を選択、もしくはショートカットの\short{s} (save)を実行して下さい。\ct{SimpleSwitchMorph}にメッセージが送られ、新しいクラスがコンパイルされます。
クラス定義をacceptすると、クラスが作成されてBrowserのクラスペインに表示されます(\figref{LOCell})。編集ペインにはクラス定義が表示され、その下にある小さなペインでクラスの目的を説明するようにうながされます\footnote{訳者が確認した環境(Pharo by Example Image および 1.3環境)ではそのようなペインはありません。代わりに、クラスブラウザ中にある「?」というボタンを押すことでクラスの目的を書くペインが表示されていました。また、クラスコメントが書かれていない場合には、その「?」の箇所が赤くなりクラスの目的を書くことをうながすような挙動となっています。}。ここへ記述する内容は\emph{クラスコメント}と呼ばれ、他のプログラマーへクラスの大まかな目的を伝えるための重要なものになります。\st プログラマーはコードの読みやすさを重視するため、メソッドに詳細なコメントが書くことは稀です。ここでの哲学は、コードの意図はコード自身に語らせよ、ということです(もしそうなっていない場合は、そうなるまでリファクタリングしましょう!)。

\index{keyboard shortcut!accept}

\subind{クラス}{コメント}には、クラスの詳細すぎる説明ではなく、後でプログラマーがこのクラスを調査する際に必要となる全体としての目的を簡潔に記述するようにしましょう。

\dothis{LOCellクラスのコメントを入力し、acceptして下さい。これは後に変更することができます。}

\begin{figure}[h!t]
\ifluluelse
	{\centerline {\includegraphics[width=\textwidth]{LOCell}}}
	{\centerline {\includegraphics[scale=0.7]{LOCell}}}
\caption{新しく作成された\ct{LOCell}クラス \figlabel{LOCell}}
\end{figure}

%=================================================================
\section{クラスにメソッドを追加する}

それではこのクラスにメソッドをいくつか追加してみましょう。

\dothis{プロトコルペインにある\prot{-{}-all-{}-}プロトコルを選択します。}

編集ペインにメソッドを作成するためのテンプレートが表示されています。
それを選択して、\mthref{scbecellinitialize}に示されている内容に書き換えてください。

\protindex{all}
\index{method!creation}
\index{browser!defining a method}

\needlines{10}
\begin{numMethod}[scbecellinitialize]{\ct{LOCell}のインスタンスを初期化する}
initialize
   super initialize.
   self label: ''.
   self borderWidth: 2.
   bounds := 0@0 corner: 16@16.
   offColor := Color paleYellow.
   onColor := Color paleBlue darker.
   self useSquareCorners.
   self turnOff
\end{numMethod}
\index{initialization}

\noindent
3行目\ct{''}の文字は、間に何の文字も挟まない引用符です。二重引用符ではないことに注意してください。\ct{''}は空の文字列を表しています。

\dothis{このメソッドの定義をacceptします。}

上記のコードは何をしているのでしょうか?ここでは詳細については触れずに(本の残りの部分はそのためにあるのです!)、簡単に内容を見ていきます。それでは1行ずつ進んでいきましょう。

\mthind{LOCell}{initialize}と呼ばれるメソッドに注目してください。この名前はとても重要です。initializeと名付けられたメソッドは、慣習的にオブジェクトが生成された直後に呼び出されます。つまり、\ct{LOCell new}が評価されると\ct{initialize}メッセージが新しく作られたオブジェクトに自動的に送られます。\ct{initialize}メソッドはオブジェクトの状態、主にそのインスタンス変数を設定するために使われます。
\seeindex{Object!initialization}{initialization}
\index{initialization}

このメソッドは、スーパークラスである\ct{SimpleSwitchMorph}の\ct{initialize}メソッドをまず実行しています(2行目)。\lct{スーパークラスの}\ct{initialize}メソッドを実行すると継承された状態が適切に初期化されると考えてください。処理をする前には、必ずスーパークラスの\ct{initialize}メソッドを呼び出して、継承された状態を初期化しておくとよいでしょう。\ct{SimpleSwitchMorph}の\ct{initialize}メソッドが実際に何をするかは知らなくとも構いません。しかし、あえて不明瞭な状態から処理を始めるよりは、いくつかのインスタンス変数に妥当な初期値が設定されることを期待して、スーパークラスの\ct{initialize}メソッドを呼び出すようにしましょう。

メソッドの残りの部分はこのオブジェクトの状態を設定しています。
例えば、\ct{self label: ''}をオブジェクトに送ることで、このオブジェクトのラベルに空の文字列を設定しています。
\pvindex{self}

\ct{0@0 corner: 16@16}については、おそらく説明が必要でしょう。\lct{0@0}は、$x$,$y$座標の両方が0に設定された\clsind{Point}オブジェクトを表しています。実際には、\ct{0@0}は数値\ct{0}に\ct{@}メッセージを引数\ct{0}で送るメッセージ式になっています。
% Yuck... the following should be \mthind{Number}{@}
%%% THIS IS BROKEN -- don't do it! (on)
%\def\atsign{\textsf{@}}%
%{\makeatletter
%	\protected@write\@indexfile{}%
%    {\string\indexentry{\string\atsign|see{Number, \string\atsign}}{\thepage}}%
%	\protected@write\@indexfile{}%
%    {\string\indexentry{Number!\string\atsign|hyperpage}{\thepage}}%
%	\makeatother}
その結果、数値\ct{0}は\ct{Point}クラスに、座標(0,0)の新たなインスタンスを作るよう指示します。新たに作られた\ct{Point}オブジェクトに\ct{corner: 16@16}メッセージが送られ、角が\ct{0@0}と\ct{16@16}の\ct{Rectangle}オブジェクトが作られます。この\ct{Rectangle}オブジェクトは、スーパークラスから継承された\ct{bounds}変数に代入されます。
\pharo の画面の座標系の原点は\emph{左上}で、\emph{下に向かって} $y$座標の値が増えることに注意してください。

メソッドの残りの部分について説明の必要はないでしょう。良い\st を書くコツの1つは、コードが英語のように読めるような良いメソッド名をつけることです。オブジェクトが自分自身に語りかけ、``\lct{自分は直角の角を使う!}'', ``\lct{自分はスィッチを切る!}''と言っているイメージを持つことができるはずです。

%ここまで(p32)%

%=================================================================
\section{オブジェクトをインスペクトする}

新しく\ct{LOCell}オブジェクトを作ってインスペクトすることによって、さきほど書いたコードの効果を確認することができます。

\dothis{workspaceを開いて、\ct{LOCell new}と打ち込み\menu{inspect it}してください。}

\begin{figure}[htbp]
   \centering
   \includegraphics[width=\textwidth]{LOCellInspector}
   \caption{LOCellオブジェクトを調べるために使用したinspector。\figlabel{LOCellInspector}}
\end{figure}

\ind{inspector}の左側のペインには、インスペクトしたオブジェクトのインスタンス変数のリストが表示されます。その中の1つ(例えば\mbox{\ct{bounds}})を選んで見てください。すると、右側のペインにその\ind{インスタンス変数}の値が表示されます。

%  You can also use the inspector to change the value of an instance variable.
%\dothis{Change the value of \ct{bounds} to \ct{0@0 corner: 50@50} and \menu{accept} it.}
%\on{This does not work any more. I get:}
%\ct{OTNamedVariableNode(Object)>>doesNotUnderstand: #selectedClass}
%\on{should use the mini workspace instead to send bounds: ?}

inspectorの下側にあるペインはmini-workspaceです。このmini-workspaceは擬似変数\self が選択されたオブジェクトに束縛されているため、とても便利です。

\dothis{
inspectorウィンドウのルートにあるLOCellを選択してください。下部にあるペインに、\ct{self bounds: (200@200 corner: 250@250)}と打ち込み \menu{do it}して下さい。inspector上で\ct{bounds}の値が変化します。次にmini-workspaceに\ct{self openInWorld}と打ち込み \menu{do it}して下さい。}

画面の左上の端、実際には\ct{bounds}メッセージで表示するよう指示した場所にセルが表示されます。セル上で\metaclick をすると\subind{Morphic}{halo}が現れます。右上の左隣にある茶色いハンドルを使ってセルを移動したり、右下にある黄色のハンドルを使ってセルをリサイズしてみましょう。boundsの値がどのように変化するか、inspectorを使って確認して下さい。(新たな boundsの値を見るために \menu{refresh} \actclick が必要かもしれません。)

\begin{figure}[htbp]
\centering
\ifluluelse
	{\includegraphics[width=\textwidth]{LOCellResize} }
	{\includegraphics[scale=0.7]{LOCellResize} }
\caption{セルのリサイズ。\figlabel{cellresize}}
\end{figure}

\dothis{ピンク色のハンドル\ct{x}をクリックしてセルを消してください。}

%=================================================================
\section{LOGameクラスを定義する}

ゲームを行うために必要なもう一つのクラスを作っていきます。クラス名は\clsind{LOGame}とします。

\dothis{Browserのメインウインドウで、クラス定義テンプレートを表示させてください。}

パッケージ名をクリックすることで、クラス定義テンプレートを表示させることができます。以下のようにコードを書いたら、それを\menu{accept}して下さい。

\needlines{6}
\begin{classdef}[sbegame]{LOGameクラスを定義します。}
BorderedMorph subclass: #LOGame
   instanceVariableNames: ''
   classVariableNames: ''
   poolDictionaries: ''
   category: 'PBE-LightsOut'
\end{classdef}

\ct{LOGame}は\clsind{BorderedMorph}のサブクラスとしました。\clsind{Morph}は\pharo のグラフィック図形すべてのスーパークラスで、\ct{BorderedMorph}は境界線を扱える\ct{Morph}になります。2行目にある引用符の間にはインスタンス変数の名前を入れることができますが、ここでは空のままにしておきましょう。

\ct{LOGame}の\mthind{LOGame}{initialize}メソッドを定義しましょう。

\dothis{以下の内容を\ct{LOGame}のメソッドとしてBrowserに書き込んで、\menu{accept}して下さい。}

\begin{numMethod}[sbegameinitialize]{ゲームを初期化する}
initialize
   | sampleCell width height n |
   super initialize.
   n := self cellsPerSide.
   sampleCell := LOCell new.
   width := sampleCell width.
   height := sampleCell height.
   self bounds: (5@5 extent: ((width*n) @(height*n)) + (2 * self borderWidth)).
   cells := Matrix new: n tabulate: [ :i :j | self newCellAt: i at: j ].
\end{numMethod}

%\sd{it would be nicer if we would not have to create an instance of LOCell for nothing}
%\on{yes}

\pharo は用語のいくつかが認識できないと警告をしてくるはずです。まずはじめに、\pharo は、\ct{cellsPerSide}メッセージが認識できないことと、スペルミスの場合に備えた修正案をいくつか提示してきます。

\begin{figure}[htb]
\begin{minipage}{0.48\textwidth}
	\centering
	\ifluluelse
		{\includegraphics[width=\textwidth]{UnknownSelector}}
		{\includegraphics[scale=0.7]{UnknownSelector}}
	\caption{\pharo が未知のセレクターを検出。\figlabel{unknownSelector}}
\end{minipage}
\hfill
\begin{minipage}{0.48\textwidth}
	\centering
	\ifluluelse
		{\includegraphics[width=\textwidth]{DeclareInstanceVar}}
		{\includegraphics[scale=0.7]{DeclareInstanceVar}}
	\caption{新しいインスタンス変数を宣言。\figlabel{declareInstance}}
\end{minipage}
\end{figure}

しかし、\ct{cellsPerSide}はスペルミスではありません。単にまだこのメソッドを定義していないだけです。これから1,2分の間にはこのメソッドを定義することにしましょう。

\dothis{メニューから最初のアイテムを選択し、\ct{cellsPerSide}がスペルミスなどではないことを確認します。}

次に\pharo は\ct{cells}が認識できないと警告してきます。\pharo はこの問題を解消するための案をいくつか提示してきます。

\dothis{\ct{cell}をインスタンス変数として扱いたいので、\menu{declare instance}を選択します。}

最後に\pharo は最終行で使われている\ct{newCellAt:at:}メッセージについて警告してきます。もちろんこれもスペルミスではないので、先ほどと同様に確認してください。

\index{on the fly variable definition}
\index{instance variable definition}

さて、もう一度\ct{LOGame}のクラス定義を見てみましょう(\button{instance}ボタンをクリックすることでできます)。クラス定義が更新され、インスタンス変数\ct{cells}が含まれるようになっていることがBrowser上から確認できるはずです。

それでは、この\ct{initialize}メソッドの中身を見ていきましょう。
\ct{| sampleCell width height n |}と書かれている行では、4つの一時変数を宣言しています。ここで宣言された変数は、スコープがこのメソッドに限定されてしまうため、一時変数と呼ばれています。説明的な名前を持った一時変数を使うことで、コードをより読みやすくすることができます。\st では定数と変数を区別するための特別な構文はありません。また、ここで宣言された4つの"変数"は、実際には定数と変わりありません。4〜7行目で一度だけ値が設定されています。

ゲームボードには必要な数のセルとそれらの境界を表示するために十分な大きさが必要ですが、セルの数がどれぐらい必要になるかは今はまだ判断できません。そこで、\ct{cellsPerSide}というメソッドをこの後で別に定義して、セルの数を決定する責務をそちらに委譲することにします。メソッドを定義する前に\ct{cellsPerSide}をメッセージ送信するようにしたため、\ct{initialize}メソッドをacceptするときに\pharo から``confirm, correct, or cancel''と警告されることになってしまいましたが、これを避けてはいけません。
こうしたやり方は、まだ定義していない他のメソッドを書くという観点でとても良い習慣です。なぜなら、\ct{initialize}メソッドを書き始めるまで私たちはそのメソッドが必要かどうかを認識することはできませんし、その時点にならないと意味のある名前や流れを邪魔しないインターフェイスを与えることはできないからです。

4行目でこのメソッドが使われています。\st 式\ct{self cellsPerSide}は\ct{cellsPerSide}メッセージを\pvind{self}、つまりオブジェクト自身に送ります。この結果、ゲームボードの一辺に必要なセルの数が\ct{n}に割り当てられました。

次の3行では、新たな\ct{LOCell}オブジェクトを作成し、ゲームボードの幅と高さを適切な一時変数に割り当てています。

%The eighth line sends the message \ct{bounds:} to \self.
%\ct{bounds:} is a method that we inherit from our superclass; it is used to define the space on the screen that this Morph will occupy.
%The single colon (\ct{:}) at the end of the name says that \ct{bounds:} expects a single parameter, which should be a rectangle object.
8行目では新しいオブジェクトの\ct{bounds}を設定しています。詳細は今は気にせず、括弧内の式で原点(\ie 左上隅)、(5,5)と右下隅からなる、適切な数のセルを置くのに十分な長方形がつくられると理解してください。

最後の行では、\ct{LOGame}オブジェクトのインスタンス変数\ct{cells}に、新しく作った適切な数の行と列をもつ\clsind{Matrix}オブジェクトを設定しています。\clsind{Matrix}オブジェクトの作成は、\ct{new:tabulate:}メッセージを\ct{Matrix}クラス(このクラスももちろんオブジェクトです。そのためメッセージを送ることができます。)に送ることで実現しています。\mthind{Matrix class}{new:tabulate:}は、2つのコロン(\ct{:})を持つことから2つの引数を取ることがわかります。引数はコロンの後に書きます。もし、引数を括弧内に一緒に書く言語しか使ったことがなければ、最初はこの書き方に戸惑うかもしれません。しかし、これはただの構文です。慌てることはありません。慣れれば、メソッド名が引数の役割を説明してくれる、とてもわかりやすい構文だとわかるでしょう。例えば次の式、\ct{Matrix rows: 5 columns: 2}なら、Matrixは2行5列ではなく5行2列であることは明白です。

\cmindex{Matrix class}{rows:columns:}

\ct{Matrix new: n tabulate: [ :i :j | self newCellAt: i at: j ]}式は\ct{n}{$\times$}\ct{n}の行列とその要素を初期化しています。各要素の初期値はその座標に依存します。\ct{(i,j)}\textsuperscript{th}の位置にある要素は\ct{self newCellAt: i at: j}式を評価した結果で初期化されます。

%:===> Pretty-print is broken! (how to pretty-print?)

% \on{I think it is silly to copy paste from the pretty-print view to the normal view}

%That's \ct{initialize}.  When you accept this message body, you might want to take the opportunity to pretty-up the formatting.  You don't have to do this by hand: from the \actclick menu select \menu{more \ldots \go prettyprint}, and the browser will do it for you\damien{this didn't do anything to me}.  You have to \menu{accept} again after you have \subind{method}{pretty-print}{}ed a method, or of course you can \subind{keyboard shortcut}{cancel}
%(\short{l}\,---\,that's a lower-case letter \emph{L}) if you don't like the result.
%Alternatively, you can set up the browser to use the pretty-printer automatically whenever it shows you code: use the the right-most button in the button bar to adjust the view.
%\seeindex{pretty-print}{method}

%If you find yourself using \menu{more\,\ldots} a lot, it's useful to know that you can hold down the {\sc shift} key when you click to directly bring up the \menu{more \ldots} menu.

%=================================================================
\section{メソッドをプロトコルにまとめる}

メソッドの定義を続ける前に、Browserの上部にある3つめのペインを少し見てみましょう。3つめのペインを使うことで、1つめのペインを使ってクラスをパッケージに分類したようにメソッドを分類化することができます。これによって4つめのペインが大量のメソッドで埋もれてしまうのを防ぐことができます。こうしたメソッドの分類は``プロトコル''とよばれています。

\index{protocol}

クラスにわずかのメソッドしかない場合、プロトコルによる余分な階層のレベルは必要ありません。そうしたクラスのために、Browserはクラスの全てのメソッドを含む仮想プロトコル\prot{-{}-all-{}-}を提供しています。

\protindex{all}

\begin{figure}[htbp]
   \centering
   \includegraphics[width=\textwidth]{Categorize}
   \caption{分類されていないすべてのメソッドを自動的に分類します。\figlabel{categorize}}
\end{figure}

この例に沿って作業をしている場合は、3つ目のペインには\protind{まだ分類されていない}プロトコルが含まれているかもしれません。

\dothis{プロトコルペインを\actclick して下さい。その後\menu{various \go categorize automatically}を選択したら、\ct{initialize}メソッドを新しいプロトコル\protind{initialization}に移動してください。}

\pharo はプロトコルが妥当であるかをどのようにして知るのでしょうか。一般に\pharo はそれを知ることはできません。しかしこのケースでは、スーパークラスにも\ct{initialize}メソッドが定義されているので、\pharo は\ct{initialize}メソッドをオーバーライド元のメソッドと同じプロトコルに分類するべきだと判断します。

\index{method!categorize}

%You may find that \pharo has already put your \ct{initialize} method into the \protind{initialization} protocol.
%If so, it's probably because you have loaded a package called \ct{AutomaticMethodCategorizer} into your image.

\paragraph{表記規則} Smalltalkerは、メソッドが属するクラスを識別するために``\verb|>>|''という表記をよく使用します。例えば\ct{LOGame}クラスのcellsPerSideメソッドは\ct{LOGame>>cellsPerSide}と表します。
この表記が\st 式では\emph{ない}ことを明示するため、本書ではその代わりに特殊なシンボル\ct{>>>}を使うことにします。さきほどのメソッドであれば、文中では\ct{LOGame>>>cellsPerSide}と表します。

\cmindex{Behavior}{>>}

以後、本書ではこの記法でメソッド名を記述します。もちろん、実際にBrowser上でコードを打ち込むときはクラス名や\ct{>>>}を打ち込む必要はありません。クラスペインで適切なクラスが選択されていることを確認するだけで十分です。

それでは、これから\ct{LOGame>>>initialize}メソッドで使われている他の2つのメソッドを定義していきます。両方とも\prot{initialization} プロトコルに入れることにします。

\begin{method}[sbegamecellsperside]{定数メソッド}
LOGame>>>cellsPerSide
   "ゲームの側面にあるセルの数"
   ^ 10
\end{method}
\cmindex{LOGame}{cellsPerSide}
\index{constant methods}

このメソッドは非常にシンプルで、単に10という値を返すだけです。このように値をメソッドとして表現しておくと、プログラムが成長してこの値が単純に定まらなくなった場合に、値を計算して返すようにメソッドを変更することで適応できるという利点があります。

\needlines{10}
\begin{method}[newCellAt:at:]{初期化補助メソッド}
LOGame>>>newCellAt: i at: j
   "位置(i,j)にあるセルを作成し、適切な画面の位置に追加。新しいセルを返す"
   | c origin |
   c := LOCell new.
   origin := self innerBounds origin.
   self addMorph: c.
   c position: ((i - 1) * c width) @ ((j - 1) * c height) + origin.
   c mouseAction: [self toggleNeighboursOfCellAt: i at: j]
\end{method}
\cmindex{LOGame}{newCellAt:at:}
%   ^ c      "omit this final line to create a bug"

\dothis{\ct{LOGame>>>cellsPerSide}メソッドと\ct{LOGame>>>newCellAt:at:}メソッドを追加してください。}

新しい\ct{toggleNeighboursOfCellAt:at:}セレクタと\ct{mouseAction:}セレクタのスペルをチェックしましょう。

\Mthref{newCellAt:at:} において新しいLOCellは、セル\nlclsind{行列}{ぎょうれつ}の\ct{(i,j)}の位置を特定しています。最後の行では新たにcellの\ct{mouseAction}に\emph{block}\mbox{\lct{[self toggleNeighboursOfCellAt: i at: j ]}.}を定義しています。これによってマウスをクリックしたときに呼び出される振る舞いを定義しています。あわせて、対応するメソッドも定義する必要があります。

\begin{method}[toggleNeighboursOfCellAt:at:]{呼び出されるメソッド}
LOGame>>>toggleNeighboursOfCellAt: i at: j
   (i > 1) ifTrue: [ (cells at: i - 1 at: j ) toggleState].
   (i < self cellsPerSide) ifTrue: [ (cells at: i + 1 at: j) toggleState].
   (j > 1) ifTrue: [ (cells at: i  at: j - 1) toggleState].
   (j < self cellsPerSide) ifTrue: [ (cells at: i at: j + 1) toggleState].
\end{method}
\cmindex{LOGame}{toggleNeighboursOfCellAt:at:}

\Mthref{toggleNeighboursOfCellAt:at:} はcell(\ct{i}, \ct{j})の東西南北の位置にある4つのセルの状態を切り替えます。唯一の難点はボードが有限であることです。そのため、隣接するcellが存在することを状態を切り替える前に確認しなければなりません。

\dothis{このメソッドを\prot{game logic}と呼ばれる新たなプロトコルに置きます。(プロトコルペイン上で新たなプロトコルを\actclick して下さい)}

メソッドを移動するには、単にその名前をクリックして新たに作ったプロトコルまでドラッグするだけです(\figref{dragMethod})。

\begin{figure}[htbp]
   \centering
   \ifluluelse
		{\includegraphics[width=\textwidth]{DragMethod} }
		{\includegraphics[scale=0.7]{DragMethod} }
   \caption{メソッドをプロトコルへ持って行く。\figlabel{dragMethod}}
\end{figure}

Lights Outゲームを完成させるには、マウスのイベントを操作するためのメソッドを\ct{LOCell}クラスにあと2つ\ct{LOCell}定義する必要があります。

\begin{method}[mouseAction:]{典型的なセッターメソッド}
LOCell>>>mouseAction: aBlock
   ^ mouseAction := aBlock
\end{method}
\cmindex{LOCell}{mouseAction:}

\Mthref{mouseAction:} はcellの\ct{mouseAction}変数に引数を設定してその値を返す以上のことはしていません。このようにインスタンス変数の値を\emph{変更する}メソッドは\emph{セッターメソッド}と呼ばれています。また、インスタンス変数の現在の値を返すメソッドは\emph{ゲッターメソッド}と呼ばれています。
\seeindex{setter method}{accessor}
\seeindex{getter method}{accessor}

もし他のプログラミング言語でゲッターメソッドとセッターメソッドを使うことに慣れていたら、これらのメソッドを\ct{setmouseAction}や\ct{getmouseAction}と呼びたくなるかもしれません。しかし\st の習慣では異なります。ゲッターメソッド名は常にそのメソッドで得るインスタンス変数名と同じ名前にします。セッターメソッド名も同様ですが、\ct{mouseAction}と\ct{mouseAction:}のように後端に``\ct{:}''を付けます。

セッターメソッドとゲッターメソッドはあわせて\emphind{アクセサ}メソッドと呼ばれています。また、これらは慣習として\protind{accessing}プロトコルに置かれます。\st では全てのインスタンス変数はそれらを持つオブジェクトのものです。そのため、\st 言語で他のオブジェクトがそれらの変数を読み書きするには、このようにアクセサメソッドを用意して行う他ありません\footnote{実際にはインスタンス変数はサブクラスの中からも参照できます。}。

\dothis{\ct{LOCell}クラスを表示し、\ct{LOCell>>>mouseAction:}を定義してください。定義が終わったら、そのメソッドを\prot{accessing}プロトコルに置いてください。}

最後に\ct{mouseUp:}メソッドを定義する必要があります。このメソッドは画面上のcellの上にあるマウスのマウスボタンが離されたときに自動的に呼び出されます。

\begin{method}[sbecellmouseup]{イベントハンドラ}
LOCell>>>mouseUp: anEvent
   mouseAction value
\end{method}
\cmindex{LOCell}{mouseUp:}

\dothis{\ct{LOCell>>>mouseUp:}を追加し、\menu{categorize automatically}を実行してください。}

\index{method!categorize}

このメソッドは\lct{value}メッセージを\ct{mouseAction}インスタンス変数に保存されているオブジェクトに送ります。\ct{LOGame>>>newCellAt: i at: j}で以下のコード片を\ct{mouseAction:}に指定したことを思い出してください。

\ct{[self toggleNeighboursOfCellAt: i at: j ]}

\noindent
\lct{value}メッセージを送ると、このコード片が評価されてcellの状態が切り替わることになります。

%=================================================================
\section{コードを実行してみましょう}

以上でLights Outゲームは完成です!

手順通りに進んできたのなら、2つのクラスと7つのメソッドから構成されているこのゲームで遊ぶことができるはずです。

\dothis{workspace上で\ct{LOGame new openInWorld}と打ち込み、\menu{do it}して下さい。}

ゲームが始まって、cellをクリックでき、またそれがどの動作するかを見ることができるでしょう。

cellをクリックするとエラーメッセージと共に\clsind{PreDebugWindow}と呼ばれる\emphind{つうち@通知}ウインドウが表示されました。理論上は動くはずなのですが\ldots{}。\figref{lightsOutError}に描かれているように、このウインドウは\ct{MessageNotUnderstood: LOGame>>>toggleState} と言ってきています。

\begin{figure}[ht]
\ifluluelse
	{\centerline{\includegraphics[width=\textwidth]{Error}}}
	{\centerline{\includegraphics[scale=0.7]{Error}}}
\caption{cellをクリックしたときにゲームでバグが発生
\figlabel{lightsOutError}}
\end{figure}

\noindent
一体何が起こったのでしょうか? 原因を見つけるために、\st のとても役に立つツールの一つ、\ind{debugger}を使いましょう。

\dothis{通知ウインドウにある\menu{debug}ボタンをクリックしてください。}

debuggerが表示されるはずです。debuggerウインドウの上部には実行スタックが表示されます。実行スタックには全ての有効なメソッドが表示されています。そのなかの1つを選ぶと、真ん中のペインにメソッド中の実行された\st コードが表示され、エラーを引き起こした部分がハイライトされます。

\dothis{(上部付近の)\ct{LOGame>>>toggleNeighboursOfCellAt:at:}と表示されている行をクリックしてください。}

debugger上でメソッド中のエラーが発生した実行コンテキストを確認することができます(\figref{debugToggle})。

\begin{figure}[ht]
\ifluluelse
	{\centerline {\includegraphics[width=\textwidth]{Debugger}}}
	{\centerline {\includegraphics[scale=0.7]{Debugger}}}
\caption{debugger上で\ct{toggleNeighboursOfCell:at:}メソッドを選択する。
\figlabel{debugToggle}}
\end{figure}

debuggerウインドウの下側には2つの小さなinspectorウインドウが表示されています。左側のウインドウでは、メッセージのレシーバーである、選択されたメソッドを実行しているオブジェクトの中身をインスペクトすることができます。つまりインスタンス変数の値などをここで確認することができます。右側のウインドウではメソッドの実行状態を表すオブジェクトの中身をインスペクトすることができます。つまりメソッドの引数や一時変数の値などを確認することができます。

debuggerを使うことで、workspace上で式を評価するのと同じように、オブジェクトの中のパラメータや局所変数の値を見ながらコードを1行ずつ実行していくことができます。他のデバッガと比べて最も驚くべきことは、デバッグ中にコードを書き換えることができるということです。Browser上よりもdebugger上でほとんどの時間プログラムをしているSmalltalkerもいるくらいです。debugger上でプログラムをする利点は、作成中のメソッドが意図通りに動くかどうかを実際の実行コンテキスト中の引数と一緒に見ることができることにあります。

今回の場合では、\ct{LOGame}インスタンスに送られた\ct{toggleState}メッセージが上部パネルの最初の行に表示されているはずです。これは\lct{LOCell}のインスタンスへ送られるはずのメッセージです。問題は\ct{cells}行列の初期化にあるように思われます。
\cmind{LOGame}{initialize}のコードをBrowserで見ると\ct{cells}には\ct{newCellAt:at:}の戻り値が設定されていることがわかります。しかしメソッドの中身を確認すると、戻り値として何も返していません!メソッドはデフォルトでは\ct{self}を戻り値として返します。\ct{newCellAt:at:}のケースでは\ct{LOGame}のインスタンスが返されることになります。
\index{method!returning self}

\dothis{debuggerウインドウを閉じで下さい。
その後、\ct{c}を返すために``\ct{^ c}''式を \ct{LOGame>>>newCellAt:at:}メソッドの最後に追加して下さい。
% It should now look as shown in \mthref{newCellAt:at:nobug}.}
(\mthref{newCellAt:at:nobug}参照。)}

% \needlines{6}
\begin{method}[newCellAt:at:nobug]{バグの修正。}
LOGame>>>newCellAt: i at: j
   "位置(i,j)にあるcellを作成し、適切な画面の位置に追加。新しいcellを返す"
   | c origin |
   c := LOCell new.
   origin := self innerBounds origin.
   self addMorph: c.
   c position: ((i - 1) * c width) @ ((j - 1) * c height) + origin.
   c mouseAction: [self toggleNeighboursOfCellAt: i at: j].
   ^ c
\end{method}
\cmindex{LOGame}{newCellAt:at:}

\noindent
\charef{quick}でやった、\st では\subind{method}{値}を\ind{返す}時には\ct{^}を使うこと、そのために\verb|^|と打ち込むことを思い出してください。
% \index{^@\verb|^|}
\index{^@{$\uparrow$}|\ \hfill$\Longrightarrow$ {return}~~~}

debuggerウインドウ上で直接コードを修正した後で、\menu{Proceed}をクリックしアプリケーションを継続して実行することがしばしばあります。今回の場合は、バグは実行に失敗したメソッドにではなくオブジェクトの初期化処理にあったので、最も簡単な処理方法は新たにやり直すことです。debuggerウインドウを閉じ、(\subind{Morphic}{halo})で実行中のゲームのインスタンスを破棄し、新たにインスタンスを作成します。

%Indeed, even in this case it would be possible to \menu{do} \ct{self initialize} and then \menu{Proceed} the \ct{toggleNeighboursOfCellAt:at:} method.
%\ab{St\'eph, did you try this?  It seems to me that it ought to work, but when I tried it, it messed up my image.}
% ON : It messed me up too!  Better not propose this.

\dothis{もう一度\ct{LOGame new openInWorld} を実行してください。}

これでこのゲームは正常...あるいはそれに近い動作をするはずです。マウスをクリックしてから離すまでの間にマウスを移動させると、マウス上のcellも切り替わってしまいます。この動作は\ct{SimpleSwitchMorph}を継承したことによる振る舞いです。\ct{mouseMove:}をオーバーライドして、何もしないように修正しましょう。

% \needlines{6}
\begin{method}[mouseMove:]{マウスの動作をオーバーライドする。}
LOGame>>>mouseMove: anEvent
\end{method}

ついに完成しました!

%\sd{It would be good to have a word about the debugger buttons into, step.... Or to have a separate chapter, we would use the material I wrote for my turtle book, please check it.}
%\on{I think that is too much for this chapter. It will come soon enough.}

%=================================================================
\section{\st のコードの保存と共有}
\seclabel{Monticello}

現在、手元には動作するLights Outゲームがあります。あなたはこのゲームを友人と共有するためにどこかに保存したいと思うかもしれません。もちろん、\pharo のイメージファイル全体を保存し、最初のプログラムを走って見せて回ることもできます。しかしもし友人が自分のコードが入ったイメージファイルを持っていたら、あなたのイメージファイルを欲しがらないかもしれません。この場合に必要なのは、他のプログラマーがあなたのソースコードを自分のイメージファイルに取り込めるよう、\pharo のイメージファイルからソースコードを取り出すことです。

最も簡単なやり方はコードを\emph{File out}することです。パッケージペインのメニューを\actclick すると、オプションで\scat{PBE-LightsOut}パッケージ全てを\menu{various \go  File out}できます。取り出されたファイルはいちおう読むことも可能ですが、本来はコンピュターが読むためのものであり人のためのものではありません。このファイルを友人にメールで送れば、友人たちはfile list Browserを使うことでソースコードを自分の\pharo イメージファイルに組み込むことができます。

\seeindex{saving code}{categories}
\seeindex{category!filing out}{file, filing out}
\seeindex{class!filing out}{file, filing out}
\seeindex{method!filing out}{file, filing out}
\index{file!filing out}

\dothis{\scat{PBE-LightsOut}パッケージを\actclick し、内容を\menu{various \go  File out}して下さい。}
``PBE-LightsOut.st''という名のファイルがイメージファイルがある同一フォルダ内にあるのを見つけることができます。テキストエディタでこのファイルを見てみてください。

\dothis{まっさらな\pharo のイメージファイルを開き、File Browserツールを使い、先ほど取り出したPBE-LightsOut.stを(\menu{Tools \ldots {\go} File Browser})から\menu{File in}して下さい。そしてそのイメージファイル上でゲームが動作することを確認して下さい。}
\seeindex{category!filing in}{file, filing in}
\seeindex{class!filing in}{file, filing in}
\seeindex{method!filing in}{file, filing in}
\index{file!filing in}

\begin{figure}[ht]
\centerline {\includegraphics[width=\textwidth]{FileIn}}
\caption{\pharo にソースコードをFile inする。
\figlabel{filein}}
\end{figure}

\subsection{Monticelloパッケージ}
\emph{File out}は書いたコードを保存するのには便利な方法ですが、これは明らかに古い方法です。
ほとんどのオープンソースプロジェクトでは、コードを保守するために役立つ\ind{CVS}\footnote{\url{http://www.nongnu.org/cvs}}や\ind{Subversion}\footnote{\url{http://subversion.tigris.org}}などのリポジトリを使っています。\pharo プログラマーにはコードを管理するのにもっと役に立つ\ind{Monticello}パッケージがあります。これらのパッケージにはファイル名の最後に \ct{.mcz}が付いています。このファイルの正体は\ind{パッケージ}のコード全てをまとめてZIPで圧縮したものです。

Monticello package Browserを使うことで、FTPやHTTPなどを含む色々な種類のサーバーのリポジトリにパッケージを保存することができます。もちろん自分で管理しているローカルディレクトリのリポジトリにパッケージを保存することもできます。パッケージのコピーは常にローカルハードディスクの\emph{package-cache}フォルダにキャッシュされています。Monticelloにより複数のバージョンのプログラムを保存したり、マージしたり、古いバージョンに戻したり、バージョン間の違いを閲覧することができます。実際、\ind{Monticello}は分散バージョン管理システムとして広まりました。このことは開発者の成果物をCVSやSubversionのように一箇所のリポジトリにではなく、いつくかの異なった場所に保管することができることを意味します。\damien{Mercurial, Git はバージョン管理システムの例ではあるが、ここで触れる必要があるかは確かではない。}

もちろん\ct{.mcz}ファイルをEメールで送ることも出来ます。
受け取った人はそれを\emph{package-cache}フォルダーに置くことにより。そのファイルをMonticelloを使い閲覧したりロードしたりすることができます。
%(It is also possible to load it using the file list, but there is a difference between loading a \ct{.mcz} file using a file list and using Monticello \sd{check}.)

\dothis{\menu{World}メニューからMonticello Browserを開いてください。}
Browserの右側のペイン(\figref{monticello1}参照)にはMonticelloのリポジトリのリストがあります。このリストには、今使っているイメージにロードしたコードを含めて全てのリポジトリを含んでいます。

%In addition to \sqsrc servers, Monticello repositories can live in a variety of other places, the simplest being a directory on your local disk.

\begin{figure}[hbt]
\ifluluelse
	{\centerline {\includegraphics[width=\textwidth]{MonticelloBrowser}}}
	{\centerline {\includegraphics[scale=0.7]{MonticelloBrowser}}}
\caption{Monticello Browser。
\figlabel{monticello1}}
\end{figure}

Monticello Browserのリストの上部はローカルディレクトリにある \emphind{package cache}と呼ばれるリポジトリです。ここにはネットワーク越しにロードしたり、公開したパッケージのコピーがキャッシュされます。
ローカルキャッシュは自分のローカル履歴が保存できるため非常に便利です。つまりインターネットに接続できない場所や、回線が遅いために何度も保存したいとは思わないぐらい遠くのリポジトリでも作業することができることを意味します。


\subsection{Monticelloを使ったコードの保存と読み込み}
Monticello Browserの左側にはイメージ内に読み込まれたバージョンを持つパッケージのリストが表示されます。そして読み込まれた後に修正されたパッケージにはアスタリスクで印がつけられています。(これらは時に\subind{package}{dirty}と呼ばれています。) パッケージを選択すると、リポジトリのリストは選択したパッケージのコピーを含むリポジトリに限定されます。
\seeindex{*}{package, dirty}
\seeindex{dirty package}{package, dirty}

%What is a package?  For now, you can think of a package as a group of  class and method categories that share the same prefix.  Since we put all of the code for the Lights Out game into the category called \scat{PBE-LightsOut}, we can refer to it as the \ct{PBE-LightsOut} package.

\dothis{\button{+Package}ボタンと\ct{PBE-LightsOut}と打ち込み、\ct{PBE-LightsOut}パッケージをMonticello Browserに追加します。}

\subsection{\ind{\sqsrc}: \pharo のための\ind{SourceForge}}

コードを保存、共有する一番良い方法は\sqsrc サーバー上のでアカウントを取りプロジェクトを作ることです。\sqsrc はSourceForge\footnote{\url{http://sourceforge.net}}のようなものです。\sqsrc はHTTPの\ind{Monticello}サーバーのwebフロントエンドでプロジェクトを管理するることができます。
\url{http://www.squeaksource.com}に公開されている\sqsrc サーバーがあります。本書に関係したコードのコピーは\url{http://www.squeaksource.com/PharoByExample.html}に保管されています。ウェブブラウザからプロジェクトを見ることもできますが、Monticello Browser を使うことでパッケージを管理するためにより多くの生産的な作業が\pharo から行えます。

\dothis{web browserで\url{http://www.squeaksource.com}を開いてください。アカウントを作り、Lights Outゲームのためのプロジェクトを作ってください(つまり登録してください)。}

\sqsrc はリポジトリを追加する際にMonticello Browserを使うことを薦めてくるでしょう。

\sqsrc にプロジェクトを作成したら、\pharo システムにそれを使うよう伝える必要があります。

\dothis{\ct{PBE-LightsOut}パッケージを選択し、Monticello Browserの\button{+Repository}ボタンをクリックして下さい。}
利用可能なさまざまなタイプのリポジトリのリストが表示されます。\sqsrc リポジトリに追加するためには\menu{HTTP}を選択してください。サーバに関する必要な情報を提供できるダイアログが表示されます。\sqsrc プロジェクトを識別するため、提示されたテンプレートをコピーし、Monticelloに貼り付け、自分のイニシャルとパスワードを提供する必要があります。

\needlines{5}
\begin{code}{}
MCHttpRepository
    location: 'http://www.squeaksource.com/!\emph{YourProject}!'
    user: '!\emph{yourInitials}!'
    password: '!\emph{yourPassword}!'
\end{code}

\noindent
もしイニシャルとパスワードを空にした場合、プロジェクトを読み込むことはできますが更新することはできません。

\needlines{5}
\begin{code}{}
MCHttpRepository
    location: 'http://www.squeaksource.com/!\emph{YourProject}!'
    user: ''
    password: ''
\end{code}

%You can then load the code in your image by selecting the version you want. You can browse the code without loading it, using the \button{Browse} button.
一度このテンプレートを受諾すると新たなリポジトリがMonticello Browserの右側のリストに表示されます。

\begin{figure}[hbt]
\ifluluelse
	{\centerline {\includegraphics[width=\textwidth]{BrowseRepository}}}
	{\centerline {\includegraphics[scale=0.7]{BrowseRepository}}}
\caption{\ind{Monticello}リポジトリの閲覧
\figlabel{monticello3}}
\end{figure}

\dothis{\button{Save}ボタンを押し最初のバージョンのLight Outゲームを\sqsrc に保存してください。}

パッケージを自分のイメージファイルに読み込むためには、最初にバージョンを特定しなければいけません。Repository Browserを使うことでにバージョンを特定することができます。\button{Open}ボタンを使うかメニューを\actclick してRepository Browserを開いてください。一度バージョンを選択すると、イメージファイルにパッケージを読み込むことができます。

\dothis{保存した\ct{PBE-LightsOut}リポジトリを開いてください。}

詳細は\charef{env}で述べますがMonticelloはとても多くのことができます。
Monticelloについては\url{http://www.wiresong.ca/Monticello/}でオンラインドキュメントを読むこともできます。

%=================================================================
\section{章のまとめ}
この章ではカテゴリ、クラス、そしてメソッドの作り方について学びました。
また、Browser、inspector、debugger、そしてMonticello Browserの使い方について学びました。

\begin{itemize}
  \item カテゴリは関連するクラスのグループ化したものです。
  \item スーパークラスへメッセージを送ることで新たにクラスを定義することができます。
  \item プロトコルは関連するメソッドをグループ化したものです。
  \item Browser上で定義し編集し\emph{acceptする}ことで新たなメソッドを作ったり修正することができます。
  \item inspectorは任意のオブジェクトの中の値を簡単に見たり割り込んだりするための多目的GUIです。
  \item Browserは宣言されたメソッドや変数の使い方を見つたり、可能な訂正を受け入れるためのものです。
  \item \pharo では\ct{initialize}メソッドはオブジェクトが作られた直後に自動的に実行されます。どんな初期化コードもそこに書くことができます。
  \item debuggerは動作しているプログラムの状態を見たり修正するための高度なGUIを提供します。
  \item カテゴリを\emph{File out}することでソースコードを共有することができます。
  \item コードを共有するより良い方法は、Monticelloを使って外部のレポジトリ、例えば\sqsrc のプロジェクトとして定義したリポジトリを管理することです。
\end{itemize}

%=================================================================
\ifx\wholebook\relax\else\end{document}\fi
%=================================================================
%=================================================================
%%% Local Variables:
%%% coding: utf-8
%%% mode: latex
%%% TeX-master: t
%%% TeX-PDF-mode: t
%%% ispell-local-dictionary: "english"
%%% End:
