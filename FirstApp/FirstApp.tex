% $Author$
% $Date$
% $Revision$

% HISTORY:
% 2006-12-05 - Stef started
% 2006-12-30 - Andrew new material
% 2007-01-10 - Stef edit
% 2007-01-12 - Andrew edit
% 2007-06-07 - Oscar edit
% 2007-07-03 - Stef edit
% 2007-09-06 - Lukas review
% 2007-09-06 - Cassou corrections
% 2007-09-24 - Cassou review
% 2009-07-01 - Oscar migrated to Pharo

%=================================================================
\ifx\wholebook\relax\else
% --------------------------------------------
% Lulu:
	\documentclass[a4paper,10pt,twoside]{book}
	\usepackage[
		papersize={6.13in,9.21in},
		hmargin={.75in,.75in},
		vmargin={.75in,1in},
		ignoreheadfoot
	]{geometry}
	\input{../common.tex}
	\pagestyle{headings}
	\setboolean{lulu}{true}
% --------------------------------------------
% A4:
%	\documentclass[a4paper,11pt,twoside]{book}
%	\input{../common.tex}
%	\usepackage{a4wide}
% --------------------------------------------
    \graphicspath{{figures/} {../figures/}}
	\begin{document}
	% \renewcommand{\nnbb}[2]{} % Disable editorial comments
	\sloppy
\fi
%=================================================================
\chapter{最初のアプリケーション}
\chalabel{firstApp}

この章では、簡単なゲーム:\ind{Lights Out}\footnote{\url{http://en.wikipedia.org/wiki/Lights_Out_(game)}}を作成します。ゲームの作成を通じて、\pharo プログラマーがプログラミングやデバックに使用するツール群や、他の開発者とプログラムをやりとりする方法について体験していきます。
この章では、\ind{ブラウザ}、オブジェクト・インスペクタ、デバッガ、\ind{Monticello} \ind{パッケージ}ブラウザ を扱います。\st による開発はとても効率的です。\st はとても簡潔なプログラミング言語です。また、開発ツールが言語と非常によく統合されています。そのため、開発のプロセスに手間取ることなく、コードを書くことに多くの時間を使うことができるでしょう。

%=================================================================
\section{Lights Outゲーム}

% DON'T USE WRAPFIGURE CLOSE TOO A PAGE BREAK!!! (ON)
%\begin{wrapfigure}[13]{r}{0.35\linewidth}%
%	\vskip -\baselineskip
%	\centerline{\includegraphics[width=.8\linewidth]{GameBoard}}
%	\caption{The Lights Out game board. The user has just clicked the mouse as shown by the cursor.
%	\figlabel{gameBoard}}
%\end{wrapfigure}

\begin{figure}[ht]
	\vskip -\baselineskip
	\centerline{\includegraphics[width=.3\linewidth]{GameBoard}}
	\caption{Lights Outのゲーム盤。ユーザーは盤上のセルをマウスでクリックするだけです。
	\figlabel{gameBoard}}
\end{figure}

\pharo の開発ツールの使い方を体験するために、これから\emph{Lights Out}という簡単なゲームを作っていきます。ゲーム盤は\figref{gameBoard}のように、淡黄色をした\emph{セル}の四角形の配列で構成されます。セルをマウスでクリックすると、周囲の4つのセルが青色に変わります。もう一度クリックすると、それらはまた淡黄色に戻ります。このゲームの目的は、できるだけ多くのセルを青色に変えることです。
\figref{gameBoard}で示すように、ゲームはゲーム盤と、100個のセルから成る2種類のオブジェクトで構成されています。\pharo のコードとしては、クラス2つでゲームを実装しています。1つはゲームを表すクラス、もう1つはセルを表すクラスです。それでは、\pharo の開発ツールを使い、どのようにこれらのクラスを定義していくか、見ていくことにしましょう。

%=================================================================
\section{パッケージを作成する}

\charef{quick}のブラウザの紹介で、クラスやメソッドのブラウザの仕方や、新たなメソッドを定義する方法を学びました。ここではパッケージ、カテゴリ、クラスの作り方を学びます。
\index{category!creating}
\index{package!creating}

\dothis{ブラウザを開きパッケージペインで\actclick し、\menu{create package}を選択します。
\footnote{パッケージ・ブラウザが標準ブラウザとしてインストールされているものとします。もし\figref{addPackage}で示したブラウザでない場合は、デフォルトのブラウザを変更する必要があります。\faqref{packagebrowser}参照。}}

\begin{figure}[htb]
\begin{minipage}[b]{0.48\textwidth}
\ifluluelse
	{\centerline {\includegraphics[width=0.9\textwidth]{AddPackage}}}
	{\centerline {\includegraphics[scale=0.7]{AddPackage}}}
	\caption{パッケージを追加します。
	\figlabel{addPackage}}
\end{minipage}
\hfill
\begin{minipage}[b]{0.48\textwidth}
\ifluluelse
	{\centerline {\includegraphics[width=0.8\textwidth]{ClassTemplate}}}
	{\centerline {\includegraphics[scale=0.6]{ClassTemplate}}}
	\caption{クラステンプレート。
	\figlabel{classTemplate}}
\end{minipage}
\end{figure}

ダイアログボックスに新たなパッケージ名(\scat{PBE-LightsOut}とします)を入力し、\menu{accept}をクリックしてください(または単にリターンキーを押してください)。すると新たなパッケージができあがり、パッケージ一覧にアルファベット順となって表示されます。

%=================================================================
\section{LOCellクラスを定義する}

当然ですが、新しいパッケージにはまだ何のクラスもありません。しかし、新たなクラスを作りやすいように、クラスのテンプレートが、メインの編集ペインに自動的に表示されます(\figref{classTemplate}参照)。

このテンプレートは、\ct{Object}クラスに\ct{NameOfSubClass}というサブクラスを作成するという、\st のメッセージ式になっています。新しいクラスは何の変数も持たず、\scat{PBE-LightsOut}カテゴリに属するようになっています。

\subsection{カテゴリとパッケージについて}
\seclabel{categoriesPackages}

従来の\st では\emph{カテゴリ}はありましたが、パッケージというものはありませんでした。この2つの違いは何でしょうか。カテゴリとは\st 関連するクラスをイメージファイル内で集約しているだけのものです。\emph{パッケージ}とは関係するクラス\emph{と拡張メソッド}を集めたもので、\ind{Monticello}のようなツールでバージョン管理できるものです。慣習としてパッケージ名とカテゴリ名には同じ名前を使います。通常は2つの違いを気にする必要はありません。しかし、これらの用語の違いが重要なこともあるため、本書では注意深く使い分けています。実際に\ind{Monticello}を使い始めるところで、より詳しく見ていくことにしましょう。

\index{package}
\index{category}

\subsection{新しいクラスを作成する}

テンプレートを修正し、目的に合ったクラスを作ってみましょう。

\dothis{クラス作成のテンプレートを修正する手順は以下の通りです:}
\begin{itemize}
  \item \clsind{Object}を\clsind{SimpleSwitchMorph}に書き換えます。
  \item \ct{NameOfSubClass}を\clsind{LOCell}に書き換えます。
  \item インスタンス変数のリストに\ct{mouseAction}を追加します。
\end{itemize}
結果は\clsref{firstClassDef}の通りです。

\needlines{5}
\begin{classdef}[firstClassDef]{\ct| LOCell|クラスを定義する}
SimpleSwitchMorph subclass: #LOCell
   instanceVariableNames: 'mouseAction'
   classVariableNames: ''
   poolDictionaries: ''
   category: 'PBE-LightsOut'
\end{classdef}
\index{browser!defining a class}
\index{class!creation}
\index{Morphic}

この新しいクラス定義は、\ct{SimpleSwitchMorph}という既存のクラスに対し、\ct{LOCell}というサブクラスを作るようにメッセージを送る\st の式になっています(実際には\ct{LOCell}クラスはまだ存在していないため、クラス名を表す\emphind{シンボル} \ct{#LOCell}を引数で渡しています)。
また、このメッセージには新たなクラスのインスタンスが\ct{mouseAction}インスタンス変数を持つという指定も含まれています。\ct{mouseAction}は、セル上でマウスがクリックされた時の振る舞いを定義するための、インスタンス変数となります。

\emph{この時点ではまだ何も作られてはいません。}クラステンプレートペインの境界線が赤に変わっていることに注目してください(\figref{acceptClassDef})。これはまだ\emph{変更が保存されていないこと}を意味しています。実際にこのメッセージを送るには、変更を\menu{accept}する必要があります。

\begin{figure}[h!t]
\ifluluelse
	{\centerline {\includegraphics[width=\textwidth]{AcceptClassDef}}}
	{\centerline {\includegraphics[scale=0.7]{AcceptClassDef}}}
\caption{クラス作成テンプレート。
\figlabel{acceptClassDef}}
\end{figure}

\dothis{新しいクラス定義を「了解(accept)」します}
action-clickして\menu{accept}を選択、もしくはショートカットの\short{s} (save)を実行してください。\ct{SimpleSwitchMorph}にメッセージが送られ、新しいクラスがコンパイルされます。
クラス定義をacceptすると、クラスができてブラウザのクラスペインに表示されます(\figref{LOCell})。編集ペインにはクラス定義が表示され、その下にある小さなペインでクラスの目的を説明するように促されます\footnote{訳者が確認した環境(Pharo by Example Image および 1.3環境)ではそのようなペインはありません。代わりに、クラスブラウザ中にある「?」ボタンを押すことでクラスの目的を書くペインが表示されます。また、クラスコメントが書かれていない場合には、その「?」の箇所が赤くなりクラスの目的を書くことを促す挙動となっています。}。ここへ記述する内容は\emph{クラスコメント}と呼ばれ、他のプログラマーにクラスの大まかな目的を伝える重要な手段となります。\st プログラマーはコードそのものの読みやすさを重視するため、メソッド中に詳細なコメントを書くことは稀です。ここでの哲学は、コードの意図はコード自身に語らせよということです(もしそうなっていない場合は、そうなるまでリファクタリングしましょう!)。

\index{keyboard shortcut!accept}

\subind{クラス}{コメント}には、クラスの詳細すぎる説明ではなく、後でプログラマーがこのクラスを見ていくべきかを判断できる、大まかな目的を簡潔に記述するようにしましょう。

\dothis{LOCellクラスのコメントを入力し、acceptしてください。後に内容を変更することもできます。}

\begin{figure}[h!t]
\ifluluelse
	{\centerline {\includegraphics[width=\textwidth]{LOCell}}}
	{\centerline {\includegraphics[scale=0.7]{LOCell}}}
\caption{新しくできた\ct{LOCell}クラス \figlabel{LOCell}}
\end{figure}

%=================================================================
\section{クラスにメソッドを追加する}

それではこのクラスにメソッドをいくつか追加してみましょう。

\dothis{プロトコルペインにある\prot{-{}-all-{}-}プロトコルを選択します。}

編集ペインにメソッド作成用のテンプレートが表示されます。
それを選択し、\mthref{scbecellinitialize}で示した内容に書き換えてください。

\protindex{all}
\index{method!creation}
\index{browser!defining a method}

\needlines{10}
\begin{numMethod}[scbecellinitialize]{\ct{LOCell}のインスタンスを初期化する}
initialize
   super initialize.
   self label: ''.
   self borderWidth: 2.
   bounds := 0@0 corner: 16@16.
   offColor := Color paleYellow.
   onColor := Color paleBlue darker.
   self useSquareCorners.
   self turnOff
\end{numMethod}
\index{initialization}

\noindent
3行目\ct{''}の文字は、間に何の文字も挟まない引用符です。二重引用符ではないことに注意してください。\ct{''}は空の文字列を表します。

\dothis{このメソッドの定義をacceptします。}

上記のコードは何をしているのでしょうか?ここでは詳細については触れずに(本の残りの部分はそのためにあるのです!)、簡単に内容を見ていきます。それでは1行ずつ進んでいきましょう。

\mthind{LOCell}{initialize}というメソッドに注目してください。この名前はとても重要です。initializeというメソッドは、慣習的にオブジェクトが生成された直後に呼び出されます。つまり、\ct{LOCell new}が評価されると\ct{initialize}メッセージが新しく作られたオブジェクトに自動的に送られます。\ct{initialize}メソッドはオブジェクトの状態、主にそのインスタンス変数を設定するために使われます。
\seeindex{Object!initialization}{initialization}
\index{initialization}

このメソッドでは、スーパークラスである\ct{SimpleSwitchMorph}の\ct{initialize}メソッドをまず実行しています(2行目)。\lct{スーパークラスの}\ct{initialize}メソッドの実行により、継承した状態が適切に初期化されるということを期待しています。何かの処理をする前には、スーパークラスの\ct{initialize}メソッドを呼び出して、継承した状態を初期化しておくとよいでしょう。\ct{SimpleSwitchMorph}の\ct{initialize}メソッドが実際に何をするかは知らなくとも構いません。不明瞭な状態のまま処理を始める危険を冒すよりは、スーパークラスの\ct{initialize}メソッドを呼び出すことで、インスタンス変数に妥当な初期値が設定されるようにしたほうが良いのです。

メソッドの残りの部分でこのオブジェクトの状態をさらに設定しています。
例えば、\ct{self label: ''}をオブジェクトに送ることで、オブジェクトのラベルに空文字列を設定しています。
\pvindex{self}

\ct{0@0 corner: 16@16}については、おそらく説明が必要でしょう。\lct{0@0}は、$x$,$y$座標の両方が0に設定された\clsind{Point}オブジェクトを表しています。実際には、\ct{0@0}は、数値\ct{0}に\ct{@}メッセージを引数\ct{0}で送るという、メッセージ送信になっています。
% Yuck... the following should be \mthind{Number}{@}
%%% THIS IS BROKEN -- don't do it! (on)
%\def\atsign{\textsf{@}}%
%{\makeatletter
%	\protected@write\@indexfile{}%
%    {\string\indexentry{\string\atsign|see{Number, \string\atsign}}{\thepage}}%
%	\protected@write\@indexfile{}%
%    {\string\indexentry{Number!\string\atsign|hyperpage}{\thepage}}%
%	\makeatother}
その結果、数値\ct{0}は、\ct{Point}クラスに座標(0,0)のインスタンスを作るよう指示します。新たに作られた\ct{Point}オブジェクトに\ct{corner: 16@16}メッセージが送られ、\ct{0@0}と\ct{16@16}の角を持つ\ct{Rectangle}オブジェクトが作られます。この\ct{Rectangle}オブジェクトは、スーパークラスから継承された\ct{bounds}変数に代入されます。
\pharo の画面の座標系の原点は\emph{左上}で、\emph{下に向かって} $y$座標の値が増えることに注意してください。

メソッドの残りの部分について説明の必要はないでしょう。\st をうまく書く秘訣は、コードが英語のように読めるように、良いメソッド名を付けることです。オブジェクトが自分自身に語りかけ、``\lct{自分は直角を使う!}'', ``\lct{自分はスイッチを切る!}''と言っているイメージを持つことができるはずです。

%=================================================================
\section{オブジェクトをインスペクトする}

\ct{LOCell}オブジェクトを作ってインスペクトすることで、さきほど書いたコードの効果を確認できます。

\dothis{ワークスペースを開いて、\ct{LOCell new}と打ち、\menu{inspect it}してください。}

\begin{figure}[htbp]
   \centering
   \includegraphics[width=\textwidth]{LOCellInspector}
   \caption{LOCellオブジェクトを調べるために使用したインスペクタ。\figlabel{LOCellInspector}}
\end{figure}

\ind{inspector}の左側のペインには、インスペクトしたオブジェクトのインスタンス変数のリストが表示されます。どれか1つ(例えば\mbox{\ct{bounds}})を選んでみてください。右側のペインにその\ind{インスタンス変数}の値が表示されます。

%  You can also use the inspector to change the value of an instance variable.
%\dothis{Change the value of \ct{bounds} to \ct{0@0 corner: 50@50} and \menu{accept} it.}
%\on{This does not work any more. I get:}
%\ct{OTNamedVariableNode(Object)>>doesNotUnderstand: #selectedClass}
%\on{should use the mini workspace instead to send bounds: ?}

インスペクタの下側にあるペインは小さなワークスペースです。このワークスペースでは、選択中のオブジェクトが擬似変数\self で参照できるため、とても便利です。

\dothis{
インスペクタウィンドウのルートにあるLOCellを選択してください。下部のペインに、\ct{self bounds: (200@200 corner: 250@250)}と打ち \menu{do it}しましょう。インスペクタ上で\ct{bounds}の値が変化します。次にワークスペース部分に\ct{self openInWorld}と打ち込み \menu{do it}してください。}

画面の左上の端、つまり\ct{bounds}が示している位置にセルが表示されます。セル上で\metaclick をすると\subind{Morphic}{halo}が現れます。右上の左隣にある茶色いハンドルでセルを移動したり、右下にある黄色のハンドルでセルをリサイズしたりしてみましょう。boundsの値の変化を、インスペクタから確認できます。(新たな boundsの値を見るために \menu{refresh} \actclick が必要かもしれません。)

\begin{figure}[htbp]
\centering
\ifluluelse
	{\includegraphics[width=\textwidth]{LOCellResize} }
	{\includegraphics[scale=0.7]{LOCellResize} }
\caption{セルのリサイズ。\figlabel{cellresize}}
\end{figure}

\dothis{ピンク色のハンドル\ct{x}をクリックしてセルを消してください。}

%=================================================================
\section{LOGameクラスを定義する}

ゲームに必要となる、もう一つのクラスを作っていきます。クラス名は\clsind{LOGame}とします。

\dothis{ブラウザのメインウインドウに、クラス定義テンプレートを表示させます。}

パッケージ名をクリックすることで、クラス定義テンプレートを表示させることができます。以下のようにコードを書いたら、\menu{accept}してください。

\needlines{6}
\begin{classdef}[sbegame]{LOGameクラスの定義}
BorderedMorph subclass: #LOGame
   instanceVariableNames: ''
   classVariableNames: ''
   poolDictionaries: ''
   category: 'PBE-LightsOut'
\end{classdef}

\ct{LOGame}は\clsind{BorderedMorph}のサブクラスとしました。\clsind{Morph}は\pharo のグラフィック図形すべてのスーパークラスで、\ct{BorderedMorph}は境界線を扱える\ct{Morph}になります。2行目にある引用符の間にはインスタンス変数の名前を入れることができますが、ここでは空のままにしておきましょう。

\ct{LOGame}の\mthind{LOGame}{initialize}メソッドを定義しましょう。

\dothis{以下の内容を\ct{LOGame}のメソッドとしてBrowserに書き込み、\menu{accept}してください。}

\begin{numMethod}[sbegameinitialize]{ゲームの初期化}
initialize
   | sampleCell width height n |
   super initialize.
   n := self cellsPerSide.
   sampleCell := LOCell new.
   width := sampleCell width.
   height := sampleCell height.
   self bounds: (5@5 extent: ((width*n) @(height*n)) + (2 * self borderWidth)).
   cells := Matrix new: n tabulate: [ :i :j | self newCellAt: i at: j ].
\end{numMethod}

%\sd{it would be nicer if we would not have to create an instance of LOCell for nothing}
%\on{yes}

\pharo はいくつかの用語が認識できないと警告をしてくるはずです。まず\pharo は、\ct{cellsPerSide}メッセージが認識できないため、スペルミスとして修正案をいくつか提示してきます。

\begin{figure}[htb]
\begin{minipage}{0.48\textwidth}
	\centering
	\ifluluelse
		{\includegraphics[width=\textwidth]{UnknownSelector}}
		{\includegraphics[scale=0.7]{UnknownSelector}}
	\caption{\pharo が未知のセレクタを検出。\figlabel{unknownSelector}}
\end{minipage}
\hfill
\begin{minipage}{0.48\textwidth}
	\centering
	\ifluluelse
		{\includegraphics[width=\textwidth]{DeclareInstanceVar}}
		{\includegraphics[scale=0.7]{DeclareInstanceVar}}
	\caption{新しいインスタンス変数を宣言。\figlabel{declareInstance}}
\end{minipage}
\end{figure}

しかし、\ct{cellsPerSide}はスペルミスではありません。単にまだメソッドを定義していないだけです。メソッドはこのすぐ後で定義することにしましょう。

\dothis{メニューから最初のアイテムを選択し、\ct{cellsPerSide}がスペルミスではないことを確認します。}

次に\pharo は\ct{cells}が認識できないと警告してきます。\pharo はこの問題を解消するための案をいくつか提示してきます。

\dothis{\ct{cell}をインスタンス変数として扱いたいので、\menu{declare instance}を選択します。}

最後に\pharo は、最終行で使われている\ct{newCellAt:at:}メッセージについて警告してきます。もちろんこれもスペルミスではないので、先ほどと同様に確認してください。

\index{on the fly variable definition}
\index{instance variable definition}

さて、もう一度\ct{LOGame}のクラス定義を見てみましょう(\button{instance}ボタンをクリックします)。クラス定義が更新され、インスタンス変数\ct{cells}が含まれるようになったことを、ブラウザで確認できるはずです。

それでは、この\ct{initialize}メソッドの中身を見ていきましょう。
\ct{| sampleCell width height n |}と書かれた行では、4つの一時変数を宣言しています。ここで宣言された変数は、スコープがこのメソッドに限定されるため、一時変数と呼ばれます。説明的な名前を持った一時変数を使うことで、コードをより読みやすくできます。\st では定数と変数を区別するための特別な構文はありません。また、ここで宣言された4つの"変数"は、実際には定数と変わりありません。4〜7行目で一度だけ値が設定されています。

ゲーム盤には必要な数だけのセルと、それらの境界を表示するための十分な大きさが必要ですが、セルの数がどれぐらいになるかは今はまだ判断できません。そこで、\ct{cellsPerSide}というメソッドを別に定義して、セルの数を決定する責務をそちらに委譲することにします。メソッドを定義する前に\ct{cellsPerSide}をメッセージ送信するようにしたため、\ct{initialize}メソッドをacceptするときに\pharo から``confirm, correct, or cancel''と警告されることになってしまいましたが、ここで避けてはいけません。
まだ定義していないメソッドを使いながらメソッドを書いていくというのは、実のところとても良い習慣です。なぜなら、\ct{initialize}メソッドを書くまで私たちはそうしたメソッドが必要となるかわかりませんでしたし、その時点になってはじめて、意味のある名前をメソッドに付けることができ、思考を中断せずに進んでいくことができるからです。

4行目でこのメソッドが使われています。\st 式\ct{self cellsPerSide}は\ct{cellsPerSide}メッセージを\pvind{self}、つまりオブジェクト自身に送ります。この結果、ゲーム盤の一辺に必要なセルの数が\ct{n}に割り当てられました。

次の3行では、新たな\ct{LOCell}オブジェクトを作成し、ゲーム盤の幅と高さを、適切な一時変数に代入しています。

%The eighth line sends the message \ct{bounds:} to \self.
%\ct{bounds:} is a method that we inherit from our superclass; it is used to define the space on the screen that this Morph will occupy.
%The single colon (\ct{:}) at the end of the name says that \ct{bounds:} expects a single parameter, which should be a rectangle object.
8行目では自身の\ct{bounds}を設定しています。詳細は今は気にせず、括弧内の式で原点(\ie 左上隅(5,5))と、右下隅からなる、指定した数のセルを置くのに十分な長方形が作られると理解してください。

最後の行では、\ct{LOGame}オブジェクトのインスタンス変数\ct{cells}に、適切な数の行と列をもつ\clsind{Matrix}オブジェクトを生成して代入しています。\clsind{Matrix}オブジェクトの生成は、\ct{new:tabulate:}メッセージを\ct{Matrix}クラス(このクラスももちろんオブジェクトです。そのためメッセージを送ることができます。)に送ることで実現しています。\mthind{Matrix class}{new:tabulate:}は、2つのコロン(\ct{:})を持つことから2つの引数を取ることがわかります。引数はコロンの後に書きます。もし、引数を括弧内に一緒に書く言語しか使ったことがなければ、最初はこの書き方に戸惑うかもしれません。しかし、これはただの構文です。あわてることはありません。そのうちに、メソッド名が引数の役割を説明してくれるわかりやすい構文だということがわかるでしょう。例えば次の式、\ct{Matrix rows: 5 columns: 2}なら、Matrixは2行5列ではなく5行2列であることは明白です。

\cmindex{Matrix class}{rows:columns:}

\ct{Matrix new: n tabulate: [ :i :j | self newCellAt: i at: j ]}によって\ct{n}{$\times$}\ct{n}の行列とその要素を初期化しています。各要素の初期値はその座標に依存します。\ct{(i,j)}\textsuperscript{th}の位置にある要素は、\ct{self newCellAt: i at: j}式を評価した結果で初期化されます。

%:===> Pretty-print is broken! (how to pretty-print?)

% \on{I think it is silly to copy paste from the pretty-print view to the normal view}

%That's \ct{initialize}.  When you accept this message body, you might want to take the opportunity to pretty-up the formatting.  You don't have to do this by hand: from the \actclick menu select \menu{more \ldots \go prettyprint}, and the browser will do it for you\damien{this didn't do anything to me}.  You have to \menu{accept} again after you have \subind{method}{pretty-print}{}ed a method, or of course you can \subind{keyboard shortcut}{cancel}
%(\short{l}\,---\,that's a lower-case letter \emph{L}) if you don't like the result.
%Alternatively, you can set up the browser to use the pretty-printer automatically whenever it shows you code: use the the right-most button in the button bar to adjust the view.
%\seeindex{pretty-print}{method}

%If you find yourself using \menu{more\,\ldots} a lot, it's useful to know that you can hold down the {\sc shift} key when you click to directly bring up the \menu{more \ldots} menu.

%=================================================================
\section{メソッドをプロトコルにまとめる}

メソッドの定義を続ける前に、ブラウザ上部にある3つめのペインを見てみましょう。1つめのペインでクラスをパッケージに分類したように、3つめのペインではメソッドの分類ができます。これにより4つめのペインが大量のメソッドで埋もれてしまうのを防ぐことができます。こうしたメソッドの分類は「プロトコル」と呼ばれます。

\index{protocol}

クラスにメソッドが少ししかない時は、プロトコルによる階層は必要ありません。そうしたクラスのために、ブラウザにはすべてのメソッドが含まれる仮想プロトコル\prot{-{}-all-{}-}が用意されています。

\protindex{all}

\begin{figure}[htbp]
   \centering
   \includegraphics[width=\textwidth]{Categorize}
   \caption{分類されていないメソッドを自動的に分類。 \figlabel{categorize}}
\end{figure}

本書の例に従って作業をしているなら、3つ目のペインには\protind{as yet unclassified(まだ分類されていない)}というプロトコルが含まれているかもしれません。

\dothis{プロトコルペインを\actclick してください。\menu{various \go categorize automatically}を選択すると、\protind{initialization}プロトコルが新しくでき、\ct{initialize}メソッドが移ります。}

\pharo はどうやってプロトコルを定めるのでしょうか。一般的には\pharo 側で判断することはできません。しかしこの例では、スーパークラスにも\ct{initialize}メソッドが定義されているため、\pharo は\ct{initialize}メソッドを、オーバーライド元のメソッドと同じプロトコルに分類します。

\index{method!categorize}

%You may find that \pharo has already put your \ct{initialize} method into the \protind{initialization} protocol.
%If so, it's probably because you have loaded a package called \ct{AutomaticMethodCategorizer} into your image.

\paragraph{表記規則} Smalltalkerは、メソッドが属するクラスを識別するために``\verb|>>|''という表記をよく使用します。例えば\ct{LOGame}クラスのcellsPerSideメソッドは\ct{LOGame>>cellsPerSide}と表します。
この表記が\st 式では\emph{ない}ことを明示するため、本書では代わりに特殊なシンボル\ct{>>>}を使うことにします。先ほどのメソッドであれば、文中では\ct{LOGame>>>cellsPerSide}と表します。

\cmindex{Behavior}{>>}

以後、本書ではこの記法でメソッド名を記述します。もちろん、実際にBrowser上でコードを打ち込むときはクラス名や\ct{>>>}を打つ必要はありません。クラスペインで適切なクラスが選択されていることを確認するだけで十分です。

それでは、\ct{LOGame>>>initialize}メソッドで使われている他の2つのメソッドを定義していきましょう。両方とも\prot{initialization} プロトコルに入れることにします。

\begin{method}[sbegamecellsperside]{定数メソッド}
LOGame>>>cellsPerSide
   "ゲーム盤の辺に並ぶセルの数"
   ^ 10
\end{method}
\cmindex{LOGame}{cellsPerSide}
\index{constant methods}

このメソッドは非常にシンプルで、単に10という値を返すだけです。このように値をメソッドとして表現しておくと、プログラムが成長してこの値が単純に定まらなくなった場合に、値を計算して返すようにメソッドを変更して対処できるという利点があります。

\needlines{10}
\begin{method}[newCellAt:at:]{初期化補助メソッド}
LOGame>>>newCellAt: i at: j
   "位置(i,j)にセルを作成し、適切な画面の位置に追加。新しいセルを返す"
   | c origin |
   c := LOCell new.
   origin := self innerBounds origin.
   self addMorph: c.
   c position: ((i - 1) * c width) @ ((j - 1) * c height) + origin.
   c mouseAction: [self toggleNeighboursOfCellAt: i at: j]
\end{method}
\cmindex{LOGame}{newCellAt:at:}
%   ^ c      "omit this final line to create a bug"

\dothis{\ct{LOGame>>>cellsPerSide}メソッドと\ct{LOGame>>>newCellAt:at:}メソッドを追加してください。}

新たに出てきた\ct{toggleNeighboursOfCellAt:at:}セレクタと\ct{mouseAction:}セレクタのスペルをチェックしましょう。

\Mthref{newCellAt:at:} では、\nlclsind{行列}{ぎょうれつ}の\ct{(i,j)}の位置に新しいLOCellを生成しています。最後の行ではセルの\ct{mouseAction}として\emph{block}\mbox{\lct{[self toggleNeighboursOfCellAt: i at: j ]}.}を設定しています。これはマウスをクリックしたときのコールバックとなります。対応するメソッドを定義しましょう。

\begin{method}[toggleNeighboursOfCellAt:at:]{コールバックメソッド}
LOGame>>>toggleNeighboursOfCellAt: i at: j
   (i > 1) ifTrue: [ (cells at: i - 1 at: j ) toggleState].
   (i < self cellsPerSide) ifTrue: [ (cells at: i + 1 at: j) toggleState].
   (j > 1) ifTrue: [ (cells at: i  at: j - 1) toggleState].
   (j < self cellsPerSide) ifTrue: [ (cells at: i at: j + 1) toggleState].
\end{method}
\cmindex{LOGame}{toggleNeighboursOfCellAt:at:}

\Mthref{toggleNeighboursOfCellAt:at:} では、セルの(\ct{i}, \ct{j})の東西南北の位置に隣接する4つのセルの状態を切り替えます。ここで唯一難しいのはゲーム盤が有限ということです。そのため、隣接するセルが存在するかどうかを、状態を切り替える前に確認しなければなりません。

\dothis{このメソッドを\prot{game logic}という新たなプロトコルに置きます。(プロトコルペイン上で、\actclick して新たなプロトコルを作ってください)}

メソッドを移動するには、名前の上でクリックし、新たに作成したプロトコルまでドラッグします(\figref{dragMethod})。

\begin{figure}[htbp]
   \centering
   \ifluluelse
		{\includegraphics[width=\textwidth]{DragMethod} }
		{\includegraphics[scale=0.7]{DragMethod} }
   \caption{メソッドをプロトコルへ持って行く。\figlabel{dragMethod}}
\end{figure}

Lights Outゲームを完成させるには、さらにマウスイベントを操作するための2つのメソッドを、\ct{LOCell}クラスに定義する必要があります。

\begin{method}[mouseAction:]{典型的なセッターメソッド}
LOCell>>>mouseAction: aBlock
   ^ mouseAction := aBlock
\end{method}
\cmindex{LOCell}{mouseAction:}

\Mthref{mouseAction:} ではセルの\ct{mouseAction}変数に引数の値を設定して、値を返しているだけです。このようにインスタンス変数の値を\emph{変更する}メソッドは\emph{セッターメソッド}と呼ばれています。また、インスタンス変数の現在の値を返すメソッドは\emph{ゲッターメソッド}と呼ばれます。
\seeindex{setter method}{accessor}
\seeindex{getter method}{accessor}

もし他のプログラミング言語でゲッターメソッドとセッターメソッドを使うことに慣れていたら、メソッド名を\ct{setmouseAction}や\ct{getmouseAction}としたくなるかもしれません。しかし\st では違い流儀です。ゲッターメソッドの名前は常に取得するインスタンス変数と同じ名前にします。セッターメソッド名も同じですが、\ct{mouseAction}に対して\ct{mouseAction:}のように末尾を``\ct{:}''とします。

セッターメソッドとゲッターメソッドをあわせて\emphind{アクセサ}メソッドと呼びます。また、これらは慣習として\protind{accessing}プロトコルに置かれます。\st ではすべてのインスタンス変数はプライベートなものとしてオブジェクトが保持します。そのため、\st で、他のオブジェクトの変数を読み書きするには、アクセサメソッドを介するしかありません\footnote{厳密にはインスタンス変数はサブクラスからもアクセスできます。}。

\dothis{\ct{LOCell}クラスに移動し、\ct{LOCell>>>mouseAction:}を定義してください。定義が終わったら、そのメソッドを\prot{accessing}プロトコルに置いてください。}

最後に\ct{mouseUp:}メソッドを定義する必要があります。このメソッドは画面のセル上でマウスボタンを離したときに、GUIフレームワークから自動的に呼び出されます。

\begin{method}[sbecellmouseup]{イベントハンドラ}
LOCell>>>mouseUp: anEvent
   mouseAction value
\end{method}
\cmindex{LOCell}{mouseUp:}

\dothis{\ct{LOCell>>>mouseUp:}を追加し、\menu{categorize automatically}を実行してください。}

\index{method!categorize}

このメソッドは\lct{value}メッセージを\ct{mouseAction}インスタンス変数にバインドされたオブジェクトに送ります。\ct{LOGame>>>newCellAt: i at: j}で以下のコードを\ct{mouseAction:}に指定したことを思い出してください。

\ct{[self toggleNeighboursOfCellAt: i at: j ]}

\noindent
\lct{value}メッセージを送ると、この部分が評価されてセルの状態が切り替わることになります。

%=================================================================
\section{コードを実行してみましょう}

これでLights Outゲームができあがりました!

手順に沿って進んできたのなら、2つのクラスと7つのメソッドで構成された、このゲームで遊ぶことができるはずです。

\dothis{workspace上で\ct{LOGame new openInWorld}と打ち、\menu{do it}してください。}

ゲームが始まります。セルをクリックすると動作を確認できるでしょう。

ところがクリックすると、\clsind{PreDebugWindow}と呼ばれる\emphind{ノーティファイア}が、エラーメッセージと共に表示されました。理論上は動くはずなのですが\ldots{}... \figref{lightsOutError}のように、このウインドウは\ct{MessageNotUnderstood: LOGame>>>toggleState} と言ってきています。

\begin{figure}[ht]
\ifluluelse
	{\centerline{\includegraphics[width=\textwidth]{Error}}}
	{\centerline{\includegraphics[scale=0.7]{Error}}}
\caption{cellをクリックしたときにゲームでバグが発生
\figlabel{lightsOutError}}
\end{figure}

\noindent
何が起こったのでしょうか? 原因を探るため、\st ではさらに強力なツールとなっている、\ind{debugger}を使いましょう。

\dothis{通知ウインドウにある\menu{debug}ボタンをクリックしてください。}

デバッガが表示されるはずです。ウインドウの上部には実行スタックが表示されます。実行スタックには実行されたメソッドが表示されています。その中の一つを選ぶと、真ん中のペインにメソッド内の\st コードが表示され、エラーを引き起こした部分がハイライトされます。

\dothis{(上部の)\ct{LOGame>>>toggleNeighboursOfCellAt:at:}と表示されている行をクリックしてください。}

デバッガ上で、エラーが発生したメソッドの実行コンテキストを確認することができます(\figref{debugToggle})。

\begin{figure}[ht]
\ifluluelse
	{\centerline {\includegraphics[width=\textwidth]{Debugger}}}
	{\centerline {\includegraphics[scale=0.7]{Debugger}}}
\caption{デバッガ上で\ct{toggleNeighboursOfCell:at:}メソッドを選択する。
\figlabel{debugToggle}}
\end{figure}

デバッガの下側には二つの小さなインスペクタが表示されています。左側のペインでは、レシーバ、つまり選択中のメソッドを実行するためのメッセージを受けとったオブジェクトの中身をインスペクトできます。インスタンス変数の値は、ここで確認できます。右側のペインでは現在のメソッドの実行状態を示すオブジェクトをインスペクトできます。メソッドの引数や一時変数の値などはこちらで確認します。

デバッガを使うことで、コードを1行ずつ実行していくことができます。また、引数や一時変数にバインドされたオブジェクを見たり、ワークスペース同様に式を評価したりすることも可能です。通常のデバッガと比べて最も驚くのは、デバッグ中にコードを書き換えられることです。ほとんどの時間をブラウザよりもデバッガ上でプログラミングするSmalltalkerもいるくらいです。デバッガ上でプログラミングをする利点は、作成中のメソッドが意図通りに動くかを、実行コンテキストの中で、実際の引数値を使って確認できることにあります。

今回の場合、\ct{toggleState}メッセージが、\ct{LOGame}インスタンスに送られたと、上部パネルの最初の行に表示されています。これは本来は\lct{LOCell}のインスタンスへ送るべきメッセージです。問題は\ct{cells}の初期化にあるように思われます。
\cmind{LOGame}{initialize}のコードをブラウザで見ると\ct{cells}には\ct{newCellAt:at:}の戻り値が設定されていることがわかります。しかしメソッドの中身を確認すると、戻り値として何も指定していません! メソッドはデフォルトでは\ct{self}を戻り値として返します。従って\ct{newCellAt:at:}の場合は、\ct{LOGame}のインスタンスが返ることになります。
\index{method!returning self}

\dothis{デバッガウィンドウを閉じてください。
その後で、``\ct{^ c}''と \ct{LOGame>>>newCellAt:at:}メソッドの最後に追加して、\ct{c}を返すようにします。
% It should now look as shown in \mthref{newCellAt:at:nobug}.}
(\mthref{newCellAt:at:nobug}参照。)}

% \needlines{6}
\begin{method}[newCellAt:at:nobug]{バグの修正。}
LOGame>>>newCellAt: i at: j
   "位置(i,j)にセルを作成し、適切な画面の位置に追加。新しいセルを返す"
   | c origin |
   c := LOCell new.
   origin := self innerBounds origin.
   self addMorph: c.
   c position: ((i - 1) * c width) @ ((j - 1) * c height) + origin.
   c mouseAction: [self toggleNeighboursOfCellAt: i at: j].
   ^ c
\end{method}
\cmindex{LOGame}{newCellAt:at:}

\noindent
\charef{quick}で、\st では\subind{method}{値}を\ind{返す}時には\ct{^}を使うと習ったのを思い出して下さい。\verb|^|をタイプします。
% \index{^@\verb|^|}
\index{^@{$\uparrow$}|\ \hfill$\Longrightarrow$ {return}~~~}

デバッガ上で直接コードを修正した後、\menu{Proceed}をクリックして、アプリケーションを継続実行することもよくあります。今回の場合は、バグは、失敗したメソッド中というより、オブジェクトの初期化処理にあったので、最も簡単なのは、単に初めからやり直すことです。デバッガウインドウを閉じ、(\subind{Morphic}{halo})で実行中のゲームのインスタンスを破棄し、新たにインスタンスを生成します。

%Indeed, even in this case it would be possible to \menu{do} \ct{self initialize} and then \menu{Proceed} the \ct{toggleNeighboursOfCellAt:at:} method.
%\ab{St\'eph, did you try this?  It seems to me that it ought to work, but when I tried it, it messed up my image.}
% ON : It messed me up too!  Better not propose this.

\dothis{もう一度\ct{LOGame new openInWorld} を実行してください。}

これでこのゲームはほぼ正常動作をするはずです。ただしマウスをクリックしてから離すまでの間にマウスを移動させると、マウス上のセルも切り替わってしまいます。これは\ct{SimpleSwitchMorph}を継承したことによる振る舞いです。\ct{mouseMove:}をオーバーライドして、何もしないように修正しましょう。

% \needlines{6}
\begin{method}[mouseMove:]{マウスの動作をオーバーライドする。}
LOGame>>>mouseMove: anEvent
\end{method}

ついに完成しました!

%\sd{It would be good to have a word about the debugger buttons into, step.... Or to have a separate chapter, we would use the material I wrote for my turtle book, please check it.}
%\on{I think that is too much for this chapter. It will come soon enough.}

%=================================================================
\section{\st コードの保存と共有}
\seclabel{Monticello}

現在、手元には動作するLights Outゲームがあります。このゲームを友人と共有するため、どこかに保存したいと思ったとします。もちろん、\pharo のイメージファイルを丸ごと保存し、初めてのプログラムとして動かしてみせることもできます。しかしもし友人が既に自分でコーディングしているイメージファイルを持っていたとしたらら、イメージファイル全体は欲しがらないかもしれません。そうした時に必要なのは、\pharo のイメージファイルからソースコードを取り出し、他のプログラマーが自身のイメージファイル内に取り込めるようにすることです。

最も簡単なやり方はコードを\emph{File out}することです。パッケージペインのメニューを\actclick すると、\scat{PBE-LightsOut}全体をファイルアウトするメニュー項目\menu{various \go  File out}が出てきます。書き出されたファイルは人間が読むことも可能ですが、本来はコンピュターが読むためのものです。このファイルを友人にメールで送れば、ファイルブラウザを使って、ソースコードを自分の\pharo イメージに取り込むことができます。

\seeindex{saving code}{categories}
\seeindex{category!filing out}{file, filing out}
\seeindex{class!filing out}{file, filing out}
\seeindex{method!filing out}{file, filing out}
\index{file!filing out}

\dothis{\scat{PBE-LightsOut}パッケージを\actclick し、\menu{various \go  File out}でファイルアウトしてください。}
``PBE-LightsOut.st''という名のファイルがイメージファイルと同じフォルダ内にできます。テキストエディタでこのファイルを見てみてください。

\dothis{まっさらな\pharo のイメージファイルを開き、先ほど出力したPBE-LightsOut.stを、ファイルブラウザを使って(\menu{Tools \ldots {\go} File Browser})、ファイルイン \menu{File in}します。そしてそのイメージ上でゲームが動作することを確認してください。}
\seeindex{category!filing in}{file, filing in}
\seeindex{class!filing in}{file, filing in}
\seeindex{method!filing in}{file, filing in}
\index{file!filing in}

\begin{figure}[ht]
\centerline {\includegraphics[width=\textwidth]{FileIn}}
\caption{\pharo にソースコードをファイルインする。
\figlabel{filein}}
\end{figure}

\subsection{Monticelloパッケージ}
\emph{ファイルアウト}はコードを保存するには便利ですが、これは明らかに古いやり方です。
ほとんどのオープンソースのプロジェクトでは、コードをリポジトリで保守することができる\ind{CVS}\footnote{\url{http://www.nongnu.org/cvs}}や\ind{Subversion}\footnote{\url{http://subversion.tigris.org}}といったツールを使っています。\pharo プログラマーも同様に、\ind{Monticello}パッケージによるコードの管理のほうが、より便利だと思うことでしょう。Monticellonoパッケージではファイル名の末尾に \ct{.mcz}が付きます。このファイルは\ind{パッケージ}内のコードを単にZIPで圧縮したものです。

Monticelloブラウザを使うことで、FTPやHTTPなどのさまざまなサーバーのリポジトリに、パッケージを保存できます。もちろん自分で管理しているローカルディレクトリのリポジトリに、パッケージを保存することもできます。パッケージのコピーは常にローカルのハードディスク上の\emph{package-cache}フォルダにキャッシュされます。Monticelloを使うと、複数のバージョンのプログラムを保存したり、マージしたり、古いバージョンに戻したり、バージョン間の違いを比較したりすることができます。実際のところ、\ind{Monticello}は分散バージョン管理システムです。つまり開発者の成果物は、CVSやSubversionのように一箇所のリポジトリに置かれるのではなく、複数の異なった場所に保管されます。\damien{Mercurial, Git はバージョン管理システムの例ではあるが、ここで触れる必要があるかは確かではない。}

もちろん\ct{.mcz}ファイルをEメールで送ることもできます。
受け取った人はそれを\emph{package-cache}フォルダーに置くことで、Monticelloで閲覧したりロードしたりすることができます。
%(It is also possible to load it using the file list, but there is a difference between loading a \ct{.mcz} file using a file list and using Monticello \sd{check}.)

\dothis{\menu{World}メニューからMonticelloブラウザを開いてください。}
右側のペイン(\figref{monticello1}参照)はMonticelloリポジトリのリストです。イメージ内にロードしたコードについてのリポジトリが、すべて並びます。

%In addition to \sqsrc servers, Monticello repositories can live in a variety of other places, the simplest being a directory on your local disk.

\begin{figure}[hbt]
\ifluluelse
	{\centerline {\includegraphics[width=\textwidth]{MonticelloBrowser}}}
	{\centerline {\includegraphics[scale=0.7]{MonticelloBrowser}}}
\caption{Monticello Browser。
\figlabel{monticello1}}
\end{figure}

Monticelloブラウザのリポジトリリストの一番上には、ローカルディレクトリの \emphind{package cache}というリポジトリが表示されます。ここにはネットワークを使ってロード・公開したパッケージのコピーがキャッシュされます。
ローカルキャッシュにはローカルでの自分の履歴が保存できるので非常に便利です。つまりインターネットに接続できない場所や、回線が遅くて頻繁にリモートに保存したくない状況でも作業できるということです。

\subsection{Monticelloを使ったコードの保存と読み込み}
Monticelloブラウザの左側には、イメージ内にロードしたパッケージ一覧が、バージョンと共に表示されます。ロード後に修正されたパッケージには、アスタリスクで印が付きます。(これらは時にダーティパッケージ\subind{package}{dirty}と呼ばれます。) パッケージを選択すると、リポジトリリストは、選択したパッケージのコピーを含むリポジトリのみを表示します。
\seeindex{*}{package, dirty}
\seeindex{dirty package}{package, dirty}

%What is a package?  For now, you can think of a package as a group of  class and method categories that share the same prefix.  Since we put all of the code for the Lights Out game into the category called \scat{PBE-LightsOut}, we can refer to it as the \ct{PBE-LightsOut} package.

Monticelloブラウザの\dothis{\button{+Package}ボタンを押して、\ct{PBE-LightsOut}と打ち、\ct{PBE-LightsOut}パッケージを追加します。}

\subsection{\ind{\sqsrc}: \pharo のための\ind{SourceForge}}

コードを保存、共有するのに一番良い方法は、プロジェクト用に\sqsrc サーバーのアカウントを取ることです。\sqsrc はSourceForge\footnote{\url{http://sourceforge.net}}のようなものです。\sqsrc は\ind{Monticello}のHTTPリポジトリのwebフロントエンドであり、各種のプロジェクトを管理できます。
\url{http://www.squeaksource.com}に、公開\sqsrc サーバーがあります。本書に関係したコードのコピーは\url{http://www.squeaksource.com/PharoByExample.html}に保管されています。ウェブブラウザからプロジェクトを見ることもできますが、\pharo からMonticelloブラウザを使ってパッケージを管理するほうが効率よく作業できるでしょう。

\dothis{ウェブブラウザで\url{http://www.squeaksource.com}を開いてください。アカウントを作り、Lights Outゲーム用のプロジェクトを作ってください(つまり登録してください)。}

\sqsrc は、Monticelloブラウザ上にリポジトリを追加するための情報を表示します。

\sqsrc にプロジェクトを作成したら、\pharo にその設定情報を伝える必要があるのです。

\dothis{\ct{PBE-LightsOut}パッケージを選択し、Monticelloブラウザの\button{+Repository}ボタンをクリックしてください。}
利用可能なさまざまなリポジトリタイプが表示されます。\sqsrc のリポジトリを追加するには\menu{HTTP}を選択してください。サーバ情報を入力するためのダイアログが表示されます。\sqsrc プロジェクトを識別するため、サーバから提示されたテンプレートをコピーし、Monticelloのダイアログに貼りつけ、自分のイニシャルとパスワードを記入する必要があります。

\needlines{5}
\begin{code}{}
MCHttpRepository
    location: 'http://www.squeaksource.com/!\emph{YourProject}!'
    user: '!\emph{yourInitials}!'
    password: '!\emph{yourPassword}!'
\end{code}

\noindent
イニシャルとパスワードを空にした場合、プロジェクトを読むことはできても更新はできません。

\needlines{5}
\begin{code}{}
MCHttpRepository
    location: 'http://www.squeaksource.com/!\emph{YourProject}!'
    user: ''
    password: ''
\end{code}

%You can then load the code in your image by selecting the version you want. You can browse the code without loading it, using the \button{Browse} button.
このテンプレートを入力すると新たなリポジトリがMonticelloブラウザの右側のリストに表示されます。

\begin{figure}[hbt]
\ifluluelse
	{\centerline {\includegraphics[width=\textwidth]{BrowseRepository}}}
	{\centerline {\includegraphics[scale=0.7]{BrowseRepository}}}
\caption{\ind{Monticello}リポジトリの閲覧
\figlabel{monticello3}}
\end{figure}

\dothis{\button{Save}ボタンを押して、Light Outゲームの最初のバージョンを\sqsrc に保存してください。}

パッケージを自分のイメージファイルにロードするには、まずバージョンを指定しなければいけません。リポジトリブラウザを使うことで、特定のバージョンを選ぶことができます。\button{Open}ボタンかメニューの\actclick でリポジトリブラウザを開いてください。バージョン選択後、イメージファイルにパッケージを読み込むことができます。

\dothis{先ほど保存した\ct{PBE-LightsOut}リポジトリを開いてください。}

Monticelloでは非常に多くのことができますが、詳細は\charef{env}で述べます。
\url{http://www.wiresong.ca/Monticello/}でオンラインドキュメントを読むこともできます。

%=================================================================
\section{章のまとめ}
この章ではカテゴリ、クラス、そしてメソッドの作成方法について学びました。
また、ブラウザ、インスペクタ、デバッガ、そしてMonticelloブラウザの使い方についても学びました。

\begin{itemize}
  \item カテゴリは関連するクラスをグループ化したものです。
  \item 既存のスーパークラスへメッセージを送ることで、新たなクラスを定義できます。
  \item プロトコルは関連するメソッドをグループ化したものです。
  \item ブラウザでメソッド定義を編集して\emph{accept}することで、新たなメソッドを作ったり修正できます。
  \item インスペクタは、シンプルで汎用的なUIを提供し、ユーザは任意のオブジェクトの値を見てやりとりすることができます。
  \item ブラウザは定義されていないメソッドや変数をチェックし、修正候補を出してきます。
  \item \pharo では\ct{initialize}メソッドはオブジェクトが作られた直後に自動的に実行されます。初期化コードはそこに書きます。
  \item デバッガは動作中のプログラムの状態を見たり修正するための高度なGUIを提供します。
  \item カテゴリを\emph{File out}することでソースコードを共有することができます。
  \item Monticelloを使って、\sqsrc のような外部のレポジトリ上でコードを共有するのが、より望ましい方法です。
\end{itemize}

%=================================================================
\ifx\wholebook\relax\else\end{document}\fi
%=================================================================
%=================================================================
%%% Local Variables:
%%% coding: utf-8
%%% mode: latex
%%% TeX-master: t
%%% TeX-PDF-mode: t
%%% ispell-local-dictionary: "english"
%%% End:
