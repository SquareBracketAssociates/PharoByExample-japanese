% $Author$
% $Date$
% $Revision$

% HISTORY:
% 2006-12-07 - Stef starts
% 2007-01-26 - Andrew updates
% 2007-05-22 - Andrew first draft
% 2007-06-24 - Oscar edit
% 2009-07-06 - Oscar migrate to pharo

%=================================================================
\ifx\wholebook\relax\else
% --------------------------------------------
% Lulu:
	\documentclass[a4paper,10pt,twoside]{book}
	\usepackage[
		papersize={6.13in,9.21in},
		hmargin={.75in,.75in},
		vmargin={.75in,1in},
		ignoreheadfoot
	]{geometry}
	\input{../common.tex}
	\pagestyle{headings}
	\setboolean{lulu}{true}
% --------------------------------------------
% A4:
%	\documentclass[a4paper,11pt,twoside]{book}
%	\input{../common.tex}
%	\usepackage{a4wide}
% --------------------------------------------
    \graphicspath{{figures/} {../figures/}}
	\begin{document}
	% \renewcommand{\nnbb}[2]{} % Disable editorial comments
	\sloppy
\fi
%=================================================================
\chapter{SUnit}
\chalabel{SUnit}

%=================================================================
%\section{Introduction}

\section{はじめに}

\on{Would be nice to have an example of test-driven development with SUnit from beginning to end. Perhaps this is for another chapter?}

\if0
\indmain{SUnit} is a minimal yet powerful framework that supports the
creation and deployment of tests.
As might be guessed from its name, the design of \sunit focussed on \emph{Unit Tests}, 
but in fact it can be used for integration tests and functional tests as well.
\sunit was
originally developed by Kent Beck and subsequently extended by Joseph
Pelrine and others to incorporate the
notion of a resource, which we will describe in \secref{resource}.
\fi

\indmain{SUnit} は小さいながらも強力な、テストの作成と運用のためのフレームワークです。その名前からわかるように、 \emph{ユニットテスト} を対象として設計されていますが、統合テストや機能テストにも使えます。 \sunit は Kent Beck が開発し、 Joseph Pelrine を始めとする開発者達がテストリソース (\secref{resource}) の概念を取り入れました。

\index{Beck, Kent}
\index{Pelrine, Joseph}
\seeindex{resource}{test, resource}

\if0
The interest in testing and \ind{Test Driven Development}
is not limited to \pharo or \st.  
Automated testing has become a hallmark of the \ind{Agile software development} movement, and any
software developer concerned with improving software quality would do well to adopt it.
Indeed, developers in many languages have come to appreciate
the power of unit testing, and versions of
\xUnit{}  now exist for many languages, including \ind{Java}, \ind{Python}, \ind{Perl}, .Net and \ind{Oracle}.
\seeindex{Matrix!free will}{Oracle} % sorry, couldn't resist
%  OSCAR: There was a broken citation here for the xprogramming web site
% I could not figure out what it was supposed to refer to.
This chapter describes \SUnit~3.3 (the current version as of this writing); the official web site of \sunit is
\url{sunit.sourceforge.net}, where updates can be found.
\index{xUnit}
\index{Net@.Net}
\fi

テストと\ind{テスト駆動開発}のメリットは \pharo や \st に留まりません。
テスト自動化は\ind{アジャイルソフトウェア開発}の特徴であり、ソフトウェアの品質を上げたい開発者なら採用するでしょう。 
ユニットテストの効果を評価する開発者は多く、今では多くの言語に \xUnit{} ライブラリがあります (\ind{Java} 、 \ind{Python} 、 \ind{Perl} 、 .Net 、\ind{Oracle} など) 。
\seeindex{Matrix!free will}{Oracle} % sorry, couldn't resist
この章では \SUnit~3.3 (原文の執筆時のバージョン) について解説します。
\sunit の最新版は \url{sunit.sourceforge.net} で確認できます。
\index{xUnit}
\index{Net@.Net}

\if0
Neither testing, nor the building of test suites, is new:  everybody knows that
tests are a good way to catch errors.
\mbox{\ind{eXtreme Programming},} by making testing a
core practice and by emphasizing \emph{automated} tests, 
has helped to make testing productive and fun, rather than a 
chore that programmers dislike.
The \st community has a long tradition of
testing because of the incremental style of development supported by its
programming environment.  
In traditional \st development, the programmer would write tests in a workspace
as soon as a method was finished.
Sometimes a test would be incorporated as a comment at the head of the method that it exercised, or tests that needed some set up would be included as example methods in the class.
The problem with these practices is that tests in a workspace are not available to other programmers who modify the code; comments and example methods are better in this respect, 
but there is still no easy way to keep track of them and
to run them automatically.
Tests that are not run do not help you to find bugs!
Moreover, an example method does not
inform the reader of the expected result:
you can run the example and see the\,---\,perhaps surprising\,---\,result, 
but you will not know if the observed behaviour is correct.
\fi

テストは新しい概念ではありません。
誰でもテストはエラーを発見するいい方法だと知っています。
\mbox{\ind{エクストリーム・プログラミング}} (eXtreme Programming) ではテストの実践方法が考案され、テストの\emph{自動化}が重視されました。
その結果テストの価値が上がり、プログラマにとってテストは面倒な作業から楽しい作業に変わりました。
\st のプログラミング環境がインクリメンタルな開発を支えることから、 \st コミュニティは長らくテストを重視してきました。
\st の伝統的な開発では、実装したメソッドをすぐにワークスペースでテストします。
メソッドのコメントにテストを書いたり、テストの準備がいる場合はサンプルメソッドとして実装する場合もあります。
ワークスペースにテストを書くと (\st イメージを共有しない限り) 他のプログラマから使えないので、その点ではコメントやサンプルメソッドのほうがまだ便利です。
ただ、メソッドの開発に合わせてテストを見直し、さらに自動的に実行させるのは簡単ではありません。
テストが実行されなければ、バグも見つけられません!
また、サンプルメソッドが一見正しい結果を出したとしても、本当に正しい処理が行われた証拠にはなりません。

\if0
\sunit is valuable because it allows us to write tests that are self-checking:
the test itself defines what the correct result should be.
It also helps us to
organize tests into groups, to describe the context in which the tests must run, and to
run a group of tests
automatically.  In less than two minutes you can write tests using
\sunit, so instead of writing small code snippets in a workspace, we encourage you
to use \sunit and get all the
advantages of stored and automatically executable tests.
\fi

\sunit には十分な価値があります。
テストの結果が正しいかどうかをテスト自身で検証できます。
テストをグループ分けしたり、テストに必要な環境の準備をしたり、グループ単位でテストを実行できます。
もうワークスペースで小さなコードを書かなくても、二分もあれば \sunit でテストを書けるようになります。
\sunit を使ってテストの自動化をしましょう。

\if0
In this chapter we start by discussing why we test, and what makes a good test. We then present a series of small 
examples showing how to use \sunit.
Finally, we look at the implementation of \sunit, so that you can understand how
\st uses the power of \ind{reflection} in supporting its tools. 
\fi

この章では、テストを行う理由とよりよいテストを書く方法の議論から始めます。
小さな例を挙げながら \sunit の使い方を紹介し、最終的には \sunit の内部実装を見ていきます。
\sunit を支えている \st の \ind{リフレクション} の力がわかるはずです。

%=================================================================
%\section{Why testing is important}
\section{なぜテストは重要か}
\seclabel{why}

\if0
Unfortunately, many developers believe that tests are a waste of their time.  
After all, \emph{they} do not write bugs\,---\,only \emph{other} programmers do that.
Most of us have said, at some time or other:
``I would write tests if I had more time.''
If you never write a bug, and if your code will never be changed in the future,
then indeed tests are a waste of your time.
However, this most likely
also means that your application is trivial, or that it is not used by you or anyone else.  
Think of tests as an investment for the future: having a
suite of tests will be quite useful now, but it will be \emph{extremely} useful when
your application, or the environment in which it executes, changes in the future.
\fi

残念ながら、テストを時間の無駄と思い込んでいる開発者もたくさんいます。
\emph{彼ら}はバグを作りません\,---\,\emph{他の}プログラマだけがバグを作るのです。
誰しも一度くらいこう言ってしまった経験があります: 「時間があればテストを書きますよ」。
絶対にバグのないコードを書けて、絶対にそのコードが将来も変更されないなら、確かにテストは時間の無駄です。
しかし、そんなコードはごく小さいか誰も使わないようなものがほとんどです。
テストを将来への投資と考えてください。
テストはすぐにでも役に立ちますが、将来アプリケーションや実行環境が変わっても\emph{絶対に}役に立ちます。

\if0
Tests play several roles. First, they provide documentation of the functionality that they cover.  
Moreover, the documentation is active: watching the tests pass tells you that the documentation is up-to-date.
Second,
tests help developers to confirm that some changes that they have just made to a package
have not broken anything else in the system\,---\,and to find the parts that break when that confidence turns out to be misplaced.
Finally, writing tests at the same time as\,---\,or even
before\,---\,programming forces you to think about the functionality
that you want to design, \emph{and how it should appear to the client}, 
rather than about how to implement it.
By writing the tests first\,---\,before the code\,---\,you are compelled to state
the context in which your functionality will run, the way it will
interact with the client code, and the expected results.  
Your code will improve: try it.
\fi

テストにはいくつかの役割があります。
一つは生きたドキュメントです。
すべてのテストに成功すれば、ドキュメントは最新版です。
もう一つは、変更したコードが無事動くという確信です。
もしシステムの一部が壊れたとしても、すぐに原因を見つけられます。
最後に、実装方法よりもアプリケーションの設計と\emph{ユーザとのインターフェース}を考えるようになります。
コードを書く前にテストを書いておけば、テストが成功するようにコードを書くしかありません。
ぜひ試してコードの品質を向上させてください。

%The culture of tests has always been present in the \st
%community because after writing a method, we would write a small
%expression to test it.  This practice supports the extremely tight
%incremental development cycle promoted by \st.  However, doing
%so does not bring the maximum benefit from testing because the tests
%are not saved and run automatically.  Moreover it often happens that
%the context of the tests is left unspecified so the reader has to
%interpret the results and assess if they are right or wrong.

\if0
We cannot test all aspects of any realistic application.
Covering a complete application is simply impossible and should not be the
goal of testing. 
Even with a good test suite
some bugs will still creep into the application, where they can lay dormant
waiting for an opportunity to damage your system.  
If you find that this has happened, take advantage of it!
As soon as you uncover the bug, write a test that exposes it, run the test, and watch it fail.
Now you can start to fix the bug: the test will tell you when you are done.
\fi
%=================================================================
\section{What makes a good test?}

Writing good tests is a skill that can be learned most easily by
practicing.  Let us look at the properties that tests should have to
get a maximum benefit.

\begin{enumerate}
\item Tests should be repeatable.  You should be able to run a test
  as often as you want, and always get the same answer.

\item Tests should run without human intervention.  You should even be
  able to run them during the night.

\item Tests should tell a story.  Each test should cover one aspect of a 
  piece of code.  A test should act as a scenario that you or someone else can
  read to understand a piece of functionality. \label{prop:oneAspect}

\item Tests should have a change frequency lower than that of the
  functionality they cover:  you do not want to have to change all your
  tests every time you modify your application.  One way to achieve
  this is to write tests based on the public interfaces of the
  class that you are testing.  
  It is OK to write a test for a private ``helper'' method if you feel that the method
  is complicated enough to need the test, but you should be aware that such a test 
  may have to be changed, or thrown away entirely, when you think of a better
  implementation.
\end{enumerate}

A consequence of property (\ref{prop:oneAspect}) is that 
the number of tests should be somewhat proportional to the number of
functions to be tested: changing one aspect of the
system should not break all the tests but only a limited
number.  This is important because having 100 tests fail should send a
much stronger message than having 10 tests fail.
However, it is not always possible to achieve this ideal: 
in particular, if a change breaks the initialization of an object, or the
set-up of a test, it is likely to cause all of the tests to fail. 

\ind{eXtreme Programming} advocates writing tests before writing code.  This may seem to go
against our deep instincts as software developers.  
All we can say is: go ahead and try it.
We have found that writing the tests before the code helps us
to know what we want to code, helps us know when we are done,
and helps us conceptualize the functionality of a class and to
design its interface.
Moreover, test-first development gives us the courage to go fast, because we are not afraid that we will forget something important.  

% \on{I cannot understand this without some explanation!}

%Writing tests is not difficult in itself. What is more difficult is choosing what to test.
%The pragmatic programmers\footnote{\url{www.pragmaticprogrammer.com}} offer the right-BICEP principle. It stands for: 
%\begin{itemize}
%\item Right -- Are the results right?
%\item B -- Are all the boundary conditions correct?
%\item I -- Can you check inverse relationships?
%\item C -- Can you cross-check results using other means?
%\item E -- Can you force error conditions to happen?
%\item P -- Are performance characteristics within bounds?
%\end{itemize}


% Now let's write our first test, and show you the benefits of using \SUnit.
%=================================================================
%\section{\sunit by example}

\section{\sunit by example}

Before going into the details of \SUnit, we will show a step by step
example.  We use an example that tests the class \ct{Set}.  Try
entering the code as we go along.

\SUnit の詳細に触れる前に、一つずつ例を挙げていきます。 \ct{Set} クラスのテストを例に進めますので、例に従ってコードを入力してください。
%---------------------------------------------------------
%\subsection{Step 1: create the test class}

\subsection{ステップ 1: テストクラスを作る}

\if0
\dothis{First you should create a new subclass of \clsind{TestCase} called
\ct{ExampleSetTest}.   Add two instance variables so that your new
class looks like this:}
\fi

\dothis{まず、 \clsind{TestCase} のサブクラス \ct{ExampleSetSet} を作ります。このクラスに二つのインスタンス変数 \ct{full} と \ct{empty} を追加します。}

%\begin{classdef}[exampleSetTest]{An Example Set Test class}
\begin{classdef}[exampleSetTest]{ExampleSetTest クラス}
TestCase subclass: #ExampleSetTest
	instanceVariableNames: 'full empty'
	classVariableNames: ''
	poolDictionaries: ''
	category: 'MySetTest'
\end{classdef}

\if0
We will use the class \ct{ExampleSetTest} to group all the tests related to
the class \ct{Set}.  It defines the context in which the tests
will run.  Here the context is described by
the two instance variables \ct{full} and \ct{empty}
that we will use to represent a full and an empty set.
\fi

\ct{ExampleSetTest} クラスに \ct{Set} クラスのテストをまとめます。このクラスにはテスト実行時のコンテキスト (テストの実行に必要な状況、状態) を定義します。コンテキストは二つのインスタンス変数 \ct{full} と \ct{empty} で表します。 \ct{full} は要素を持つセット、 \ct{empty} は空のセットを表します。

\if0
The name of the class is not critical, but by convention it should end in \ct{Test}.
If you define a class called \ct{Pattern} and call the corresponding test class \ct{PatternTest}, the two classes will be alphabetized together in the browser (assuming that they are in the same category).  It \emph{is} critical that your class be a subclass of \ct{TestCase}.
\fi

クラスの名前は重要ではありませんが、 \ct{Test} で終わる名前にすべきです。 \ct{Pattern} クラスのテストクラスなら \ct{PatternTest} とすれば、ブラウザで見たときに (どちらも同じカテゴリであれば) 両方のクラスがアルファベット順に並びます。テストクラスが \ct{TestCase} のサブクラスであるほうが重要です。

%---------------------------------------------------------
%\subsection{Step 2: initialize the test context}

\subsection{ステップ 2: テスト環境を整える}

\if0
The method \mthind{TestCase}{setUp} defines the context in which the tests will run, a bit like an initialize method.
\ct{setUp} is invoked before the execution of each test
method defined in the test class.
\index{SUnit!set up method}
\seeindex{testing}{SUnit}
\fi

\mthind{TestCase}{setUp} メソッドで、テスト実行時の状態を定義します。ちょっと初期化メソッドに似ていますね。 \ct{setUp} は (テストクラスに実装した) 各テストメソッドの実行前に実行されます。

\index{SUnit!set up method}
\seeindex{testing}{SUnit}

\if0
\dothis{Define the \ct{setUp} method as follows, to initialize the \ct{empty} variable to refer to an empty set and the \ct{full} variable to
refer to a set containing two elements. }
\if

\dothis{\ct{setUp} メソッドを次のように定義してください。インスタンス変数 \ct{empty} に空のセットを代入し、 \ct{full} に二つの要素を持つセットを代入します。

\needlines{3}
%\begin{method}[setupExampleSetTest]{Setting up a fixture}
\begin{method}[setupExampleSetTest]{フィクスチャを設定する}
ExampleSetTest>>>setUp
	empty := Set new.
	full := Set with: 5 with: 6
\end{method}

\noindent
\if0
In testing jargon the context is called the \emph{fixture} for the
test.
\fi

このテストではコンテキストのことを \emph{フィクスチャ} と呼びます。テスト用語です。

\index{SUnit!fixture}
\seeindex{fixture}{SUnit, fixture}

%---------------------------------------------------------
%\subsection{Step 3: write some test methods}
\subsection{ステップ 3: テストメソッドを書く}

\if0
Let's create some tests by defining some methods in the class
\ct{ExampleSetTest}.  
Each method represents one test; 
the name of the method should start with the string `\ct{test}' so that \sunit
will collect them into test suites.
Test methods take no arguments.
\fi

それではテストを書いてみましょう。
\ct{ExampleSetTest} にテスト用のメソッドを定義します。
一つのメソッドで一つのテストを表します。
テストメソッドは名前を `\ct{test}' で始め、引数を取りません。
\sunit はテストメソッドを集めてテストスイートとしてまとめます。

\if0
\dothis{Define the following test methods.}
The first test, named \ct{testIncludes}, tests the
\ct{includes:} method of \ct{Set}.  The test says that sending the
message \ct{includes: 5} to a set containing 5 should return
\ct{true}.  Clearly, this test relies on the fact
that the \ct{setUp} method has already run.
\fi

\dothis{次のテストメソッドを定義します。}

最初のテスト \ct{testIncludes} は、 \ct{Set} の \ct{includes:} メソッドをテストします。
\ct{5} を要素に持つ \ct{Set} に \ct{includes: 5} のメッセージを送ると \ct{true} が返ると仮定します。
明らかに \ct{setUp} メソッドが先に実行されることが前提になっています。

%\begin{method}[testIncludes]{Testing set membership}
\begin{method}[testIncludes]{\ct{Set} の要素のテスト}
ExampleSetTest>>>testIncludes
	self assert: (full includes: 5).
	self assert: (full includes: 6)
\end{method}

\if0
The second test, named \ct{testOccurrences}, verifies that the
number of occurrences of~5 in \ct{full} set is equal to one, even if we
add another element~5 to the set.
\fi

二つ目のテスト \ct{testOccurrences} は、 \ct{Set} に同じ要素が一つしか含まれないことを検証します。 \ct{full} にはすでに \ct{5} が含まれていますが、さらに \ct{5} を追加しても \ct{5} の数は変わりません。

\needlines{6}
%\begin{method}[testOccurrences]{Testing occurrences}
\begin{method}[testOccurrences]{同じ要素の数のテスト}
ExampleSetTest>>>testOccurrences
	self assert: (empty occurrencesOf: 0) = 0.
	self assert: (full occurrencesOf: 5) = 1.
	full add: 5.
	self assert: (full occurrencesOf: 5) = 1
\end{method}

\if0
Finally, we test that the set no
longer contains the element 5 after we have removed it.
\fi

最後に \ct{5} を削除し、 \ct{Set} が \ct{5} を含まないことをテストします。

%\begin{method}[testRemove]{Testing removal}
\begin{method}[testRemove]{要素を削除するテスト}
ExampleSetTest>>>testRemove
	full remove: 5.
	self assert: (full includes: 6).
	self deny: (full includes: 5)
\end{method}

\if0
\noindent
Note the use of the method \mthind{TestCase}{deny:} to assert something that should not be true.
\ct{aTest deny: anExpression} is equivalent to \ct{aTest assert: anExpression not}, but is much more readable.
\fi

\noindent
\ct{true} を返すべきではないテストに \mthind{TestCase}{deny:} メソッドを使っているのに注意してください。
\ct{aTest deny: anExpression} は \ct{aTest assert: anExpression not} と同じ意味ですが、よりわかりやすくなります。

%---------------------------------------------------------
%\subsection{Step 4: run the tests}
\subsection{ステップ 4: テストを実行する}

\if0
The easiest way to run the tests is directly from the browser.  Simply \actclick on the package, class name, or on an individual test method, and select \menu{run the tests (t)}.
The test methods will be flagged red or green, depending on whether they pass or not, and the class will be flagged fully or partially green or red depending on whether all, some or none of the tests pass.
\fi

テストの実行にはブラウザを使うと簡単です。
パッケージ、クラス名、テストメソッドのいずれかを選択し、メニューから \menu{run the tests (t)} を実行します。
テストの成功したメソッドには緑のマークが、失敗したメソッドには赤のマークがつきます。
すべてのテストに成功するとクラスに緑のマークが、一つでも失敗すると赤のマークがつきます。

\begin{figure}[tbh]
  \begin{center}
	\includegraphics[width=\linewidth]{browser-tests}
%	\caption{Running \sunit tests from the browser}
	\caption{ブラウザから \sunit のテストを実行する}
	\figlabel{browser-tests}
  \end{center}
\end{figure}

% 最新版の Pharo ではデザインが多少変わっています。

\if0
You can also select sets of test suites to run, and obtain a more detailed log of the results using the \sunit \emphind{Test Runner}, which you can open by selecting \menu{World \go Test Runner}.
The \emph{TestRunner}, shown in \figref{test-runner}, is designed to make it easy  to execute groups of tests.
The left-most pane lists all of the categories that contain test classes (\ie  subclasses of \ct{TestCase}); when some of these categories are selected, the test classes that they contain appear in the pane to the right.
Abstract classes are italicized, and the test class hierarchy is shown by indentation, so subclasses of \ct{ClassTestCase} are indented more than subclasses of \ct{TestCase}.
\fi

\menu{World} メニューから \menu{Test Runner} を選択すると開く \sunit \emphind{テストランナー} (図 \figref{test-runner}) を使えば、テストの詳細な内容と実行結果を見られます。
\emph{テストランナー} はまとまったテストを簡単に実行できるように設計されています。
最も左にあるペインでは、テストクラス (\ie \ct{TestCase} のサブクラス) を含むすべてのカテゴリを表示しています。カテゴリを選択すると、そのカテゴリに含まれるテストクラスのクラス階層が隣のペインに表示されます。
イタリック体のフォントで表示されるクラスは抽象クラス (テストメソッドを含まないテストクラス) です。

% ここまで読み進めた読者には "the test class hierarchy is shown by indentation, ..." のくだりは冗長と考えて省略しました。
% また、 TestCase が表示されるのは "SUnit-Kernel" カテゴリを選んだ場合か、どのカテゴリも選ばない場合です。

\begin{figure}[tbh]
  \begin{center}
	\includegraphics[width=\linewidth]{test-runner}
%	\caption{The \pharo \sunit Test Runner}
	\caption{\pharo \sunit テストランナー}
	\figlabel{test-runner}
  \end{center}
\end{figure}

%\dothis{Open a Test Runner, select the category \menu{MySetTest}, and click the \button{Run Selected} button.}

\dothis{テストランナーを開いて \menu{MySetTest} カテゴリを選択し、 \button{Run Selected} ボタンをクリックします。}

% ON: With OB, you don't need this.
%You can also run a single test by executing a \menu{print it} on the following code: \ct{ExampleSetTest run: #testRemove}.  
%We usually include an executable comment in our test methods that allows us to run them
%with a \menu{do it} from the browser, as shown in \mthref{ExampleSetTestTestRemoveii}.

%\needlines{6}
%\begin{method}[ExampleSetTestTestRemoveii]{Executable comments in test methods}
%ExampleSetTest>>>testRemove
%	"self run: #testRemove"
%	full remove: 5.
%	self assert: (full includes: 6).
%	self deny: (full includes: 5)
%\end{method}

%\dothis{Introduce a bug in \ct{ExampleSetTest>>>testRemove} and run the tests again. For example, change \ct{5} to \ct{4}.}
\dothis{\ct{ExampleSetTest>>>testRemove} にバグを入れて、再度テストを実行してみましょう。例えば、 \ct{5} を \ct{4} にします。}

\if0
The tests that did not pass (if any) are listed in the right-hand panes of
the \emph{Test Runner};
if you want to debug one, to see why it failed, just click on the name.
\fi

失敗したテストは \emph{テストランナー} の右のペインに表示されます。
失敗の原因を知りたければ、表示されたテストメソッドをクリックしてください。
ここからデバッグもできます。

%Alternatively, you can execute the following expressions:
%\begin{code}{}
%(ExampleSetTest selector: #testRemove) debug
%\end{code}
%or
%\begin{code}{}
%ExampleSetTest debug: #testRemove
%\end{code}

%---------------------------------------------------------
%\subsection{Step 5: interpret the results}
\subsection{ステップ 5: 結果を分析する}

\if0
The method \mthind{TestCase}{assert:}\,, which is defined in the class
\ct{TestCase}, expects a boolean argument, usually the value of a tested expression.  When the argument is
true, the
test passes; when the argument is false, the test fails.
\fi

\ct{TestCase} クラスの \mthind{TestCase}{assert:}\ メソッドは論理値の引数を一つ
とります。
通常はこの値にテストする式の戻り値を指定します。
引数が真ならテストに成功、偽なら失敗とみなします。

\if0
There are actually three possible outcomes of a test.
The outcome that we hope for is that all of the assertions in the test are true, in which case the test passes.
In the test runner, when all of the tests pass, the bar at the top turns green.
However, there are also two kinds of thing that can go wrong when
you run a test.
Most obviously, one of the assertions can be false, causing the test to \emph{fail}.
However, it is also possible that some kind of error occurs during the execution of the test, such as a \emph{message not understood} error or an \emph{index out of bounds} error.
If an error occurs, the assertions in the test method may not have been executed at all,
so we can't say that the test has failed; nevertheless, something is clearly wrong!
In the {test runner}, failing tests cause the bar at the top to turn yellow, and are listed in the middle pane on the right, whereas erroneous tests cause the bar to turn red, and are listed in the bottom pane on the right.
\fi

% 原文が回りくどいので構成を変えました。

テストの実行結果は 3 通りあります。
一つはすべてのテストに成功する場合。
すべてのテストに成功すると、{テストランナー}の右上のバーが緑に変わります。
次に、一つでもテストに\emph{失敗}する場合。
{テストランナー}のバーが黄色に変わり、その下のペインに失敗したテストが表示されます。
最後に、テストの実行中にエラー (例外) が発生する場合。
例えば \emph{message not understood} や \emph{index out of bounds} といったエラーえす。
エラーが発生するとバーが赤に変わり、エラーの発生したテストが表示されます。

%\dothis{Modify your tests to provoke both errors and failures.}
\dothis{エラーしたり失敗するようにテストを変更してみましょう。}

%=================================================================
%\section{The \SUnit cook book}

\section{\SUnit クックブック}

\if0
This section will give you more details on how to use \SUnit.  If you
have used another testing framework such as \JUnit\footnote{\url{http://junit.org}},
much of this will be familiar, since all these frameworks have their roots in \SUnit.
Normally you will use \SUnit's GUI to run tests, but there are
situations where you may not want to use it.
\fi

この章では \SUnit の詳しい使い方を解説します。
\JUnit\footnote{\url{http://junit.org}} などの他のテスティングフレームワークの経験があるなら、 \SUnit は親しみやすいでしょう。どのフレームワークも \SUnit にルーツがあります。
通常 \SUnit では GUI でテストを実行しますが、 GUI を使いたくない場合もあるでしょう。

%---------------------------------------------------------
%\subsection{Other assertions}
\subsection{様々なアサーション}

\if0
In addition to \ct{assert:} and \ct{deny:}, there are several other methods that can be used to make assertions.
\fi

\ct{assert:} や \ct{deny:} の他にもアサーション用のメソッドがあります。

\if0
First, \mthind{TestCase}{assert:description:} and \mthind{TestCase}{deny:description:} take a second argument which is a message string that can be used to describe the reason for the failure, if it is not obvious from the test itself.   These methods are described in~\secref{descriptionStrings}.
\fi

一つ目は \mthind{TestCase}{assert:description:} と \mthind{TestCase}{deny:description:} です。
二つ目の引数には、テストの失敗時に表示する理由を示す文字列を指定します。
失敗の原因がわかりにくい場合に使うといいでしょう。
~\secref{descriptionStrings}でこれらのメソッドについて触れます。

\if0
Next, \sunit provides two additional methods, \mthind{TestCase}{should:raise:} and
\mthind{TestCase}{shouldnt:raise:} for testing exception propagation.
For example, you would use 
\ct{(self should: aBlock raise: anException)} to test that a particular exception is raised during the execution of \ct{aBlock}.  \Mthref{ESTtestIllegal}
illustrates the use of \mbox{\ct{should:raise:}.}
\fi

二つ目は \mthind{TestCase}{should:raise:} と \mthind{TestCase}{shouldnt:raise:} です。
テストの実行時に適切な例外が発生するかどうかをテストします。
例えば \ct{(self should: aBlock raise: anException)} を実行すると、 \ct{aBlock} の実行中に例外 \ct{anException} が発生すればテストは成功です。
\Mthref{ESTtestIllegal} に \mbox{\ct{should:raise:}} を使った例を示します。

%\dothis{Try running this test.}

\dothis{このテストを実行してみましょう。}

\if0
Note that the first argument of the \ct{should:} and \ct{shouldnt:} methods is a \emphind{block} that \emph{contains} the expression to be evaluated.   
\fi

\ct{should:} と \ct{shouldnt:} の最初の引数は、評価する式を\emph{含む}\emphind{ブロック}です。

%\begin{method}[ESTtestIllegal]{Testing error raising}
\begin{method}[ESTtestIllegal]{エラーの発生をテストする}
ExampleSetTest>>>testIllegal
	self should: [empty at: 5] raise: Error.
	self should: [empty at: 5 put: #zork] raise: Error
\end{method}

\if0
\sunit is portable: it can be used from all dialects of \st.  To make
\sunit portable, its developers factored-out the dialect-dependent aspects.  The class method \cmind{TestResult class}{error} answers
the system's error class in a dialect-independent fashion. 
You can take advantage of this: if you want to write tests that will work in any dialect of \st, instead of 
\mthref{ESTtestIllegal} you would write:
\fi

\sunit では処理系依存のコードをできるだけ避けて移植性を高めており、どの \st 処理系でも使えるようになっています。
例えばシステムエラーのクラスは処理系によって異なるため、 \cmind{TestResult class}{error} メソッドで返すようにしています。
処理系に依存しないようにするには、 \mthref{ESTtestIllegal} のコードを次のように書き換えます。

\needlines{4}
%\begin{method}[portabletestillegal]{Portable error handling}
\begin{method}[portabletestillegal]{移植性の高いエラー捕捉方法}
ExampleSetTest>>>testIllegal
	self should: [empty at: 5] raise: TestResult error.
	self should: [empty at: 5 put: #zork] raise: TestResult error
\end{method}

%\dothis{Give it a try.}
\dothis{試してみましょう。}

%---------------------------------------------------------
\subsection{メニューを使わずにテストランナーを開く}

ワークスペースなどで \ct{TestRunner open} を実行 (コンテキストメニューから \menu{do it} を選択) します。

% この内容は次の "Running a single test" で書かれていますが、
% あまり関係がないため独立させました。

%---------------------------------------------------------
%\subsection{Running a single test}

\subsection{テストを一つだけ実行する}

\if0
Normally, you will run your tests using the Test Runner.
If you don't want to launch the Test Runner from the \menu{open\,\ldots} menu, you can execute \ct{TestRunner open} as a \menu{print it}.

You can run a single test as follows.
\fi

テストランナーを開かずにテストを一つだけ実行したい場合、次のコードを実行します。

\begin{code}{}
ExampleSetTest run: #testRemove --> 1 run, 1 passed, 0 failed, 0 errors
\end{code}

%---------------------------------------------------------
%\subsection{Running all the tests in a test class}
\subsection{テストクラスに定義されたすべてのテストを実行する}

\if0
Any subclass of \ct{TestCase} responds to the message \ct{suite}, which will build a test suite that contains all the
methods in the class whose names start with the string ``\ct{test}''.
To run the tests in the suite, send it the message \ct{run}.
For example:
\fi

\ct{TestCase} のサブクラスに \ct{suite} メッセージを送ると、名前が ``\ct{test}'' で始まるメソッドを集めてテストスイートを組み立てます。
このテストスイートに \ct{run} メッセージを送ってテストを実行します。
次に例を示します。

\begin{code}{}
ExampleSetTest suite run --> 5 run, 5 passed, 0 failed, 0 errors
\end{code}

%---------------------------------------------------------
%\subsection{Must I subclass TestCase?}

\subsection{必ず TestCase クラスを継承しなければならない?}

\if0
In \JUnit{} you can build a \clsind{TestSuite} from an arbitrary class
containing \ct{test*} methods.  In \st you can do the same
but you will then have to create a suite by hand and your class will
have to implement all the essential \ct{TestCase} methods like \ct{assert:}.
We recommend that you not try to do this.  The framework is there: use it.
\fi

\ct{TestCase} クラスを継承しなくても \clsind{TestSuite} は組み立てられます。
ただし、自分で \ct{TestCase} クラスの主要なメソッド (\ct{assert:} など) を実装したり、テストスイートを自分で作る必要が出てきます。
\ct{TestCase} クラスを継承しましょう。

% JUnit の説明が少し違います。
% JUnit 3.x 系では TestCase クラスを継承して test* メソッドを定義しますが、
% 4.x 系ではアノテーションを使って任意のメソッドをテストに指定します。
% JUnit の説明のほうが多くなるので割愛しました。

%=================================================================
%\section{The SUnit framework}
\section{SUnit フレームワーク}

\if0
\sunit consists of four main classes: \clsind{TestCase},
\clsind{TestSuite}, \clsind{TestResult}, and \clsind{TestResource}, as shown in \figref{sunit-classes}.
The notion of a \emph{test resource} was introduced in \sunit 3.1 to represent a resource that is expensive to set-up but which can be used by
a whole series of tests.  A \ct{TestResource}
specifies a \ct{setUp} method that is executed just once before a suite of tests;
this is in distinction to the \ct{TestCase>>>setUp} method, which is executed before
each test.
\fi

SUnit は次の 4 つのクラスから構成されます: \clsind{TestCase} 、
\clsind{TestSuite} 、 \clsind{TestResult} 、 \clsind{TestResource} (\figref{sunit-classes}) 。
\emph{テストリソース}は \sunit 3.1 で導入された概念で、コストの高いセットアップに使われます。
\ct{TestResource} の \ct{setUp} メソッドは \ct{TestCase>>>setUp} と異なり、テストスイートの実行前に一度だけ実行されます。 \ct{TestCase>>>setUp} は各テストの実行前に setUp メソッドが実行されます。

\begin{figure}[htb]
  \begin{center}
		{\includegraphics[width=0.8\textwidth]{sunit-classes}}
%	\caption{The four classes representing the core of \SUnit}
	\caption{\SUnit を構成する 4 つのクラス}
	\figlabel{sunit-classes}
  \end{center}
\end{figure}


%---------------------------------------------------------
\subsection{TestCase}

\if0
\clsindmain{TestCase} is an abstract class that is designed to be subclassed; each of its subclasses represents a group of tests that share a common context (that is, a test suite).
Each test is run by creating a new instance of a subclass of \ct{TestCase},
running \mthind{TestCase}{setUp}, running the test method itself, and then running \mthind{TestCase}{tearDown}.
\fi

\clsindmain{TestCase} は抽象クラスです。サブクラスにテストを記述し、テスト間で実行時の状態を共有します (テストスイート) 。
各テストは、それぞれ \ct{TestCase} のサブクラスのインスタンスを生成してから実行されます。 \mthind{TestCase}{setUp} を実行し、テストメソッドを実行し、最後に \mthind{TestCase}{tearDown} を実行します。

\if0
The context is specified
by instance variables of the subclass
and by the specialization of the method
\ct{setUp}, which initializes those instance variables.
Subclasses of \ct{TestCase} can also override method
\ct{tearDown}, which is invoked after the execution of each test,
and can be used to release any objects
allocated during \ct{setUp}.
\fi

実行時の状態をインスタンス変数で表し、オーバーライドした \ct{setUp} メソッドで初期化します。 \ct{setUp} で初期化したオブジェクトの終了処理を行うには、 \ct{tearDown} メソッドをオーバーライドします。

%---------------------------------------------------------
\subsection{TestSuite}

\if0
Instances of the class \clsindmain{TestSuite} contain a collection of test cases.  An
instance of \ct{TestSuite} contains tests, and other test suites.
That is, a test suite contains sub-instances of
\ct{TestCase} and \ct{TestSuite}.
Both individual \lct{TestCase}s and \lct{TestSuite}s understand the same protocol, so they can be treated in the same way; for example, both can be \ct{run}.
This is in fact an application of the composite
pattern in which \ct{TestSuite} is the composite and the
\ct{TestCase}s are the leaves\,---\,see \textit{Design Patterns} for more information on this pattern\cite{Gamm95a}.
\fi

\clsindmain{TestSuite} はテストケース (テストとテストスイート) の集合を含みます。つまり、 \ct{TestSuite} は \ct{TestCase} と \ct{TestSuite} のインスタンスを含みます。
\lct{TestCase} と \lct{TestSuite} の間に継承関係はありませんが、どちらも同じプロトコルを実装しているので同じように扱えます。
例えば、どちらも \ct{run} メソッドが定義されています。
\ct{TestSuite} を枝、 \ct{TestCase} を葉として Composite パターンを適用しています (詳しくは\textit{デザインパターン}を参照してください) 。

%---------------------------------------------------------
\subsection{TestResult}

\if0
The class \clsindmain{TestResult} represents the results of a
\ct{TestSuite} execution.  It records the number of tests passed,
the number of tests failed, and the number of errors signalled.
\fi

\clsindmain{TestResult} は \ct{TestSuite} の実行結果を扱います。
成功したテストの数、失敗したテストの数、エラーの発生したテストの数を記録します。

%---------------------------------------------------------
\subsection{TestResource}
\seclabel{resource}

\if0
One of the important  features of a suite of tests is that they should be independent of each other: the failure of one test should not cause an avalanche of failures of other tests that depend upon it, nor should the order in which the tests are run matter.
Performing \ct{setUp} before each test and \ct{tearDown} afterwards helps to reinforce this independence. 
\fi

テストでは、各テストが互いに独立していることが重要です。
一つのテストの失敗が他のテストに波及しないように、テストの実行順序が結果に影響ようにすべきです。
\ct{setUp} と \ct{tearDown} で調整しましょう。

\if0
However, there are occasions where setting up the necessary context is just too time-consuming for it to be practical to do once before the execution of each test.
Moreover, if it is known that the test cases do not disrupt the resources used by the tests, then it is wasteful to set them up afresh for each test; it is sufficient to set them up once for each suite of tests.
Suppose, for example, that a suite of tests needs to query a database, or do some analysis on some compiled code.
In such cases, it may make sense to set up the database and open a connection to it, or to compile some source code, before any of the tests start to run.
\fi

しかし、テストのたびに環境を用意していては時間がかかり過ぎる場合があります。
さらに環境を壊さずに済むテストなら、一度作った環境を使い回せば十分です。
例えばデータベースに問い合わせをするテストなら、事前に用意しておいたデータベースに接続するだけで準備できます。
コンパイルされたコードを分析するテストなら、同様に事前にソースコードをコンパイルしておけば十分です。

\if0
Where should we cache these resources, so that they can be shared by a suite of tests?
The instance variables of a particular \ct{TestCase} sub-instance won't do, because such an instance persists only for the duration of a single test.
A global variable would work, but using too many global variables pollutes the name space, and the binding between the global and the tests that depend on it will not be explicit.
A better solution is to put the necessary resources in a singleton object of some class.
The class \clsindmain{TestResource} exists to be subclassed by such resource classes.
Each subclass of \lct{TestResource} understands the message  \ct{current}, which will answer a singleton instance of that subclass.
Methods \ct{setUp} and \ct{tearDown} should be overridden in the subclass to ensure that the resource is initialized and finalized.
\fi

リソースをキャッシュさえすれば、テスト間で共有できるようになるでしょうか?
\ct{TestCase} のインスタンス変数はキャッシュとして使えません。
\ct{TestCase} クラスのインスタンスは一つのテストでしか使われないからです。
グローバル変数ならできますが、多用するとシステムがグローバル変数であふれてしまいますし、各テストとグローバル変数の関係がわかりにくくなります。
必要なリソースを含むシングルトンオブジェクトをクラスに持たせるのがいい方法です。
そのための抽象クラスとして \clsindmain{TestResource} が用意されています。
\lct{TestResource} のサブクラスに \ct{current} メッセージを送ると、サブクラスのシングルトンインスタンスを得られます。
リソースの初期化と終了処理を行うには、 \ct{setUp} と \ct{tearDown} をオーバーライドしてください。

\if0
One thing remains: somehow, \sunit has to be told which resources are associated with which test suite.
A resource is associated
with a particular subclass of \ct{TestCase} 
by overriding the \emph{class} method \ct{resources}.
\ab{The set of resources attributed to each test is actually the closure of these resources under the resources message, but I think that we don't want to say that!}
By default, the resources of 
a \ct{TestSuite} are
the union of the resources of
the \ct{TestCase}s that it contains.
\fi

もう一つ、どうにかしてリソースとテストスイートを関連付ける必要があります。
\ct{TestCase} のサブクラスで \ct{resources} \emph{クラスメソッド} をオーバーライドし、テストクラスで使うリソースを返すようにします。
\ab{}
デフォルトでは、 \ct{TestSuite} のリソースはすべての \ct{TestCase} のリソースの集合です。

\if0
Here is an example. 
We define a subclass of \ct{TestResource} called
\ct{MyTestResource} and we associate it with \ct{MyTestCase}
by specializing the class method \ct{resources} to return an array
of the test classes that it will use.
\fi

次にリソースの例を示します。
\ct{TestResource} のサブクラス \ct{MyTestResource} を定義します。
次に \ct{MyTestCase} のクラスメソッド \ct{resources} で、このクラスのテストで使うリソースの配列を返します。

\needlines{8}
%\begin{classdef}[mytestresource]{An example of a TestResource subclass}
\begin{classdef}[mytestresource]{TestResource のサブクラスの使用例}
TestResource subclass: #MyTestResource
	instanceVariableNames: ''

MyTestCase class>>>resources
%	"associate the resource with this class of test cases"
	"このクラスのテストで使うリソースを返す"
	^{ MyTestResource }
\end{classdef}

%\needlines{10}
%\begin{classdef}[mytestresource]{An example of a TestResource subclass}
%TestResource subclass: #MyTestResource
%	instanceVariableNames: ''

%MyTestResource>>>setUp
%	"Set up resources here."

%MyTestResource>>>tearDown
%	"Tear down resources here."

%MyTestCase class>>>resources
%	"associate the resource with this class of test cases"
%	^{ MyTestResource }
%\end{classdef}

% \on{Do we really need the empty setUp and tearDown methods here?}

%=================================================================
\section{Advanced features of SUnit}
In addition to \ct{TestResource}, the current version of \sunit contains assertion
description strings, logging support, and resumable test failures.

%---------------------------------------------------------
\subsection{Assertion description strings}
\seclabel{descriptionStrings}

The \ct{TestCase} assertion protocol includes a
number of methods that allow the programmer to supply a description of the assertion.  The description is a \ct{String}; if the test case
fails, this string will be displayed by the test runner.  Of
course, this string can be constructed dynamically.
\begin{code}{}
| e |
e := 42.
self assert: e = 23
	description: 'expected 23, got ', e printString
\end{code}

The relevant methods in \ct{TestCase} are:
\begin{code}{}
#assert:description:
#deny:description:
#should:description:
#shouldnt:description:
\end{code}
\cmindex{TestCase}{assert:description:}
\cmindex{TestCase}{deny:description:}
\cmindex{TestCase}{should:description:}
\cmindex{TestCase}{shouldnt:description:}

%---------------------------------------------------------
\subsection{Logging support}
The description strings described above may also be logged to a
\ct{Stream} such as the \ct{Transcript}, or a file stream.
You can choose whether to log by overriding
\cmind{TestCase}{isLogging} in your test class; you must also choose where
to log by overriding \cmind{TestCase}{failureLog} to answer an appropriate stream.

%---------------------------------------------------------
\subsection{Continuing after a failure}
\sunit also allows us to specify whether or not a test should continue after a failure.  This is a really
powerful feature that uses the exception mechanisms offered
by \st.  To see what this can be used for, let's look at an
example. Consider the following test expression:
\begin{code}{}
aCollection do: [ :each | self assert: each even]
\end{code}
In this case, as soon as the test finds the first element of the collection that isn't
\ct{even}, the test stops. 
However, we would usually like to
continue, and see both how many elements, and which elements, aren't
\ct{even}, and maybe also log this information.  You can do this
as follows:
\begin{code}{}
aCollection do:
	[:each |
	self
		assert: each even
		description: each printString , ' is not even'
		resumable: true]
\end{code}
This will print out a message on your logging stream for each element
that fails.  It doesn't accumulate failures, \ie if the assertion
fails 10~times in your test method, you'll still only see one failure.
All the other assertion methods that we have seen are not resumable;
\ct{assert: p description: s} is equivalent to \ct{assert: p description: s resumable: false}.
\cmindex{Collection}{do:}
%=================================================================
\section{The implementation of SUnit}

The implementation of \sunit makes an interesting case study of a \st framework.
Let's look at some key aspects of the implementation by following the
execution of a test.
%---------------------------------------------------------
\subsection{Running one test}

To execute one test, we evaluate the expression
\ct{(aTestClass selector: aSymbol) run.}


\begin{figure}[tbh]
  \begin{center}
		{\includegraphics[width=0.7\textwidth]{sunit-scenario}}
	\caption{Running one test}
	\figlabel{sunit-scenario}
  \end{center}
\end{figure}

The method \cmind{TestCase}{run} creates an instance of
\clsind{TestResult} that will accumulate the results of the
tests, then it sends itself the message \mthind{TestCase}{run:}.
(See \figref{sunit-scenario}.)

\needlines{6}
\begin{method}[tastecaserun]{Running a test case}
TestCase>>>run
	| result |
	result := TestResult new.
	self run: result.
	^result
\end{method}

% Note that in a future release, the class of the \ct{TestResult} to
% be created will be returned by a method so that new
%\ct{TestResult} can be introduced. }

The method
\cmind{TestCase}{run:} sends the message
\mthind{TestResult}{runCase:} to the test result:

\begin{method}[testcaserun:]{Passing the test case to the test result}
TestCase>>>run: aResult
	aResult runCase: self
\end{method}
The method \ct{TestResult>>>runCase:} sends
the message \mthind{TestCase}{runCase} to an individual test, to execute the test.
\ct{TestResult>>>runCase} deals with 
any exceptions that may be raised during the
execution of a test, runs a \ct{TestCase} by sending it the
message \ct{runCase}, and counts the errors, failures
and passes.
\begin{method}[testresultruncase]{Catching test case errors and failures}
TestResult>>>runCase: aTestCase
	| testCasePassed |
	testCasePassed := true.
	[[aTestCase runCase] 
			on: self class failure
			do: 
				[:signal | 
				failures add: aTestCase.
				testCasePassed := false.
				signal return: false]]
					on: self class error
					do:
						[:signal |
						errors add: aTestCase.
						testCasePassed := false.
						signal return: false].
	testCasePassed ifTrue: [passed add: aTestCase]
\end{method}

The method \ct{TestCase>>>runCase} sends the messages
\mthind{TestCase}{setUp} and \mthind{TestCase}{tearDown} as shown below.
\needlines{3}
\begin{method}[testcaseruncase]{Test case template method}
TestCase>>>runCase
	[self setUp.
	self performTest] ensure: [self tearDown]
\end{method}

%---------------------------------------------------------
\subsection{Running a \lct{TestSuite}}

To run more than one test, we send the message
\ct{run} to a \ct{TestSuite} that contains the relevant tests. 
\ct{TestCase class} provides some functionality to build a test suite from
its methods.  The expression \ct{MyTestCase buildSuiteFromSelectors} returns a suite containing all the tests defined in the {\ct{MyTestCase} class.
The core of this process is
\begin{method}[testcasetestselectors]{Auto-building the test suite}
TestCase class>>>testSelectors 
	^self selectors asSortedCollection asOrderedCollection select: [:each | 
		('test*' match: each) and: [each numArgs isZero]]
\end{method}
\cmindex{MyTestCase class}{buildSuiteFromSelectors}

The method \cmind{TestSuite}{run} creates an instance of
\ct{TestResult}, verifies that all the resources are available, and then sends itself
the message \mthind{TestSuite}{run:}, which runs all
the tests in the suite.  All the resources are then
released.
\begin{method}[testsuiterun]{Running a test suite}
TestSuite>>>run
	| result |
 	result := TestResult new.
	self resources do: [ :res |
		res isAvailable ifFalse: [^res signalInitializationError]].
	[self run: result] ensure: [self resources do: [:each | each reset]].
	^result
\end{method}

\begin{method}[testsuiterun:]{Passing the test result to the test suite}
TestSuite>>>run: aResult
	self tests do: [:each | 
		self changed: each.
		each run: aResult].
\end{method}
The class \clsind{TestResource} and its subclasses keep track of
their currently created instances (one per class) that can be accessed
and created using the class method \mthind{TestResource class}{current}.  This instance is
cleared when the tests have finished running and the resources are
reset.

The resource availability check makes it possible for the resource to be 
re-created if needed, as shown in the class method \cmind{TestResource class}{isAvailable}.  During the \ct{TestResource}
instance creation, it is initialized and the method \mthind{TestResource}{setUp} is
invoked.

%(Note it may happen that your version of \sunit 3.0 does
%not correctly initialize the resource.  A version with this bug
%circulated a lot.  Verify that \ct{TestResource}
%\ct{class>>>new} calls the method \ct{initialize}).

\needlines{4}
\begin{method}[testresourceisavailable]{Test resource availability}
TestResource class>>>isAvailable
	^self current notNil and: [self current isAvailable]
\end{method}
\begin{method}[testresourcecurrent]{Test resource creation}
TestResource class>>>current
	current isNil ifTrue: [current := self new].
	^current
\end{method}
\begin{method}[restresourceinitialize]{Test resource initialization}
TestResource>>>initialize
	super initialize.
	self setUp
\end{method}
%=================================================================
%\section{Some advice on testing}

\section{テストのポイント}

\if0
While the mechanics of testing are easy, writing good tests is not.
Here is some advice on how to design tests.
\fi

テストの仕組みは簡単ですが、いいテストを書くのは簡単ではありません。いくつかテストのポイントを紹介します。

\begin{description}
%\item[Self-contained tests.] You do not
%  want to have to change your tests  each time you change your code, so try to write the tests
%  so that they are self-contained.  This can be difficult, but pays off in the
%  long term.  Writing tests in terms of stable interfaces supports
%  self-contained tests.
%  \on{I have no idea what you are trying to tell me.
%  What specifically should I do or not do?
%  Give an example!}

%\item[Do not over-test.] Try to build your tests so that they do not
%  overlap.  It is annoying to have many tests covering the same
%  functionality, because one bug in the code will then break many tests at the same time.
%  This is covered by Black's rule, below.

\if0
\index{Feathers, Michael}
\item[Feathers' Rules for Unit tests.]
  Michael Feathers, an  agile process consultant and author, writes:\footnote{See \url{http://www.artima.com/weblogs/viewpost.jsp?thread=126923}. 9 September 2005} 
  \begin{quotation}
  \noindent
  {\it
  A test is not a unit test if:
  \begin{itemize}
	\item it talks to the database,
	\item it communicates across the network,
	\item it touches the file system,
	\item it can't run at the same time as any of your other unit tests, or
	\item you have to do special things to your environment (such as editing config files) to run it.
 \end{itemize}
Tests that do these things aren't bad. Often they are worth writing, and they can be written in a unit test harness. However, it is important to be able to separate them from true unit tests so that we can keep a set of tests that we can run fast whenever we make our changes.
 }
  \end{quotation}
Never get yourself into a situation where you don't want to run your unit test suite because it takes too long.   
\fi

\index{Feathers, Michael}
\item[Feathers のユニットテストの法則。]
  アジャイル開発コンサルタントでもあり、「レガシーコード改善ガイド」の著者でもある Michael Feathers は次のように書いています。 \footnote{\url{http://www.artima.com/weblogs/viewpost.jsp?thread=126923} を参照。 2005/9/9 。}
  \begin{quotation}
  \noindent
  {\it
  次のテストはユニットテストではありません:
  \begin{itemize}
	\item データベースと対話したり、
	\item インターネットに接続したり、
	\item ファイルシステムに触ったり、
	\item 他のユニットテストと並行して実行できなかったり、
	\item 特別な環境 (設定ファイルの編集が必要など) を準備しなければならない。
 \end{itemize}
\if0
Tests that do these things aren't bad. Often they are worth writing, and they can be written in a unit test harness. However, it is important to be able to separate them from true unit tests so that we can keep a set of tests that we can run fast whenever we make our changes.
\fi

このようなテストも悪くはありません。
書く価値はありますし、ユニットテストの補助にもなります。
しかしコードを変更したらすぐにテストできるように、本当のユニットテストとは分けておくべきです。

 }
  \end{quotation}

\if0
Never get yourself into a situation where you don't want to run your unit test suite because it takes too long.   
\f0

決して、時間がかかるという理由でテストを避けてはいけません。

\if0
 \item[Unit Tests \textit{vs.}\ Acceptance Tests.] Unit tests capture one piece of
  functionality, and as such make it easier to identify bugs in that functionality.
  As far as
  possible try to have unit tests for each method that could possibly fail, and group them per class.
  However,
  for certain deeply recursive or complex setup situations, it is
  easier to write tests that represent a scenario in the larger application; these are called acceptance 
  tests or functional tests.
  Tests that break Feathers' rules may make good acceptance tests.
  Group acceptance tests according to the functionality that they test.
  For example, if you are writing a compiler, you might write acceptance tests that make 
  assertions about the code generated for each possible source language statement.
  Such tests might exercise many classes, and might take a long time to run because they touch the 
  file system.
  You can write them using \sunit, but you won't want to run them each time you make a small change,
  so they should be separated from the true unit tests.
\fi
 
 \item[ユニットテスト \textit{vs.}\ 承認テスト。] ユニットテストは機能の一部を単体でテストし、機能に関するバグを容易に見つけられます。
  各テストを失敗するまで可能な限り実行し、クラス単位で結果をまとめます。
  しかし、複雑な環境のテストや再帰的なテストが必要ならば、シナリオに沿ってテストを書くほうが簡単です。
  このようなテストは承認テストまたは機能テストと呼ばれます。
  Feathers の法則を守らないテストは、いい承認テストになるかもしれません。
  テストする機能に沿って承認テストをまとめましょう。
  例えばコンパイラを書く場合、ソースコードから生成するコードの承認テストを書くでしょう。
  たくさんのテストが実行されるでしょうが、ファイルシステムにアクセスするテストなら時間がかかるかもしれません。
  わずかなコードの変更にもテストをしたければ、ユニットテストと承認テストを分けるべきです。

%\item[Black's Rule of Testing.]
\item[Black のテストの法則。]

% Rex Black のこと?

\if0
  For every test in the system, you should be able to identify some property for which
  the test increases your confidence.
  It's obvious that there should be no important property that you are not testing.
  This rule states the less obvious fact that there should be
  no test that does not add value to the system by increasing your confidence that a useful property
  holds.
  For example, several tests of the same property do no good. 
  In fact, they do harm in two ways.
  First, they make it harder to infer the behaviour of the class by reading the tests. 
  Second, 
  because one bug in the code might then break many tests, they make it harder to estimate how many bugs remain in the code.
  So, have a property in mind when you write a test.
\fi

  どのテストも、テスト内容がどうシステムの信頼性につながるのかわかるようにすべきです。
  テストの必要がないと判断した機能を重要なものと考えるべきではありません。
  システムの信頼性が高まるような内容でなければ、テストをする必要はありません。
  例えば同じ目的のテストを複数書いても、かえって二つのデメリットがあります。
  一つは、テストを読んでもクラスの挙動が理解しにくくなります。
  もう一つは、コードにいくつバグが残っているのか見積もりにくくなります。
  たった一つでもバグがあれば、多くのテストが失敗するからです。
  テストを書くときは常に目的を考えましょう。

% そのまま訳すとわかりにくくなりそうでだいぶ訳を変えましたが、おかしいかも。

\end{description}

%\section{Extending \SUnit}
%\seclabel{extending}

%In this section we will explain how to extend \sunit so that it uses
%a \ct{setUp} and \ct{tearDown} that are shared by all of the
%tests in a \ct{TestCase} subclass.  We will define a new sublass
%of \ct{TestCase} called \ct{SharingSetUpTestCase}, and a
%subclass of \ct{SharingSetUpTestCase} called \ct{SharedOne}.
%We will also need to define a new subclass of \ct{TestSuite}
%called \ct{SharedSetUpTestSuite}, and we will make some minor
%adjustments to \ct{TestCase}.

%Our tests will be in \ct{SharedOne}.  When we execute
%\begin{script}
%Transcript clear.
%SharedOne suite run
%\end{script}
%we will obtain the following trace.
%\begin{code}{}
%SharedOne>>>setUp
%SharedOne class>>>sharedSetUp
%SharedOne>>>testOne
%SharedOne>>>tearDown
%SharedOne>>>setUp
%SharedOne>>>testTwo
%SharedOne>>>tearDown
%SharedOne class>>>sharedTearDown
%2 run, 2 passed, 0 failed, 0 errors
%\end{code}
%You can see that the shared code is executed just once for both
%tests.

%\subsection{\ct{SharedSetUpTestCase}}

%The extension of the \sunit framework is based on the introduction
%of two classes: \ct{SharedSetUpTestCase} and
%\ct{SharedSetUpTestSuite}.  The basic idea is to use a flag that
%is flushed (cleared) after a certain number of tests have been run.
%The class \ct{SharedSetUpTestCase} defines one instance variable
%that indicates whether each test is run individually or in the context
%of a shared \ct{setUp} and \ct{tearDown}.  There are also two
%class instance variables.  One indicates the number of tests for which
%the shared \ct{setUp} should be in effect, and the other indicates
%whether the shared \ct{setUp} is in effect.
%\begin{method}[xxx]{xxx}
%SharedSetUpTestCase
%	superclass: TestCase
%	instanceVariableNames: 'runIndividually '
%	classInstanceVariableNames: 'numberOfTestsToTearDown
%								 sharedSetUp '
%\end{method}
%\ct{suiteClass} is used by \ct{TestCase} to determine the
%suite that is running.
%\begin{method}[xxx]{xxx}
%SharedSetUpTestCase class>>>suiteClass
%	^SharedSetUpTestSuite
%\end{method}
%\begin{method}[xxx]{xxx}
%SharedSetUpTestCase class>>>sharedSetUp
%	"A subclass should only override this hook to define
%	 a sharedSetUp"
%\end{method}
%\begin{method}[xxx]{xxx}
%SharedSetUpTestCase class>>>sharedTearDown
%	"Here we specify the teardown of the shared setup"
%\end{method}
%\begin{method}[xxx]{xxx}
%SharedSetUpTestCase class>>>flushSharedSetUp
%	sharedSetUp := nil
%\end{method}
%The \ct{SharedSetUpTestCase} class is initialized with the number
%of tests for which the shared \ct{setUp} should be in effect.
%\begin{method}[xxx]{xxx}
%SharedSetUpTestCase class>>>armTestsToTearDown: aNumber
%	self flushSharedSetUp.
%	numberOfTestsToTearDown := aNumber.
%\end{method}
%Every time a test is run, the method \ct{anothertestHasBeenRun} is
%invoked.  Once the specified number of tests is reached the
%\ct{sharedSetUp} is flushed and the \ct{sharedTearDown} is
%executed.
%\begin{method}[xxx]{xxx}
%SharedSetUpTestCase class>>>anotherTestHasBeenRun
%	"Everytimes a test is run this method is called,
%	 once all the tests of the suite
%	 are run the shared setup is reset"
%	numberOfTestsToTearDown := numberOfTestsToTearDown - 1.
%	numberOfTestsToTearDown isZero
%		ifTrue:
%			[self flushSharedSetUp.
%			self sharedTearDown]
%\end{method}
%When a test is run its \ct{setUp} is executed and it then it calls
%the class method \ct{privateSharedSetUp}.  This method will only
%invoke the \ct{sharedSetUp} if the \ct{sharedSetUp} test
%indicates that it hasn't been done yet.
%\begin{method}[xxx]{xxx}
%SharedSetUpTestCase class>>>privateSharedSetUp
%	sharedSetUp isNil
%		ifTrue:
%			[sharedSetUp := 1.
%			self sharedSetUp]
%\end{method}
%\begin{method}[xxx]{xxx}
%SharedSetUpTestCase>>>setUp
%	self class privateSharedSetUp
%\end{method}
%\begin{method}[xxx]{xxx}
%SharedSetUpTestCase>>>tearDown
%	self class anotherTestHasBeenRun
%\end{method}
%When a test case is created we assume that it will be run once.  We
%can change this later by invoking the method
%\ct{executedFromASuite}.
%\begin{method}[xxx]{xxx}
%SharedSetUpTestCase>>>setTestSelector: aSymbol
%	"Must do it this way because there is no initialize"

%	runIndividually := true.
%	super setTestSelector: aSymbol
%\end{method}
%\begin{method}[xxx]{xxx}
%SharedSetUpTestCase>>>executedFromASuite
%	runIndividually := false
%\end{method}
%The methods responsible for test execution are then specialized as
%follows.
%\begin{method}[xxx]{xxx}
%runIndividually
%	^runIndividually
%\end{method}
%\begin{method}[xxx]{xxx}
%SharedSetUpTestCase>>>armTearDownCounter
%	self runIndividually
%		ifTrue: [self class armTestsToTearDown: 1]
%\end{method}
%\begin{method}[xxx]{xxx}
%SharedSetUpTestCase>>>runCaseAsFailure
%	self armTearDownCounter.
%	super runCaseAsFailure
%\end{method}
%\begin{method}[xxx]{xxx}
%SharedSetUpTestCase>>>runCase
%	self armTearDownCounter.
%	super runCase
%\end{method}

%\subsection{\ct{SharedOne}}

%\ct{SharedOne} is a new class which inherits from
%\ct{SharingSetUpTestCase} as follows.  We define two simple tests
%\ct{testOne} and \ct{testTwo}.
%\begin{method}[xxx]{xxx}
%SharedOne
%	superclass: SharingSetUpTestCase
%\end{method}
%\begin{method}[xxx]{xxx}
%SharedOne>>>testOne
%	Transcript
%		show: 'SharedOne>>>testOne';
%		cr
%\end{method}
%\begin{method}[xxx]{xxx}
%SharedOne>>>testTwo
%	Transcript
%		show: 'SharedOne>>>testTwo';
%		cr
%\end{method}
%Then we define the methods \ct{setUp} and \ct{tearDown} that
%will be executed before and after the execution of the tests exactly
%in the same way as with non sharing tests.  Note however, the fact
%that with the solution we will present we have to explicitly invoke
%the \ct{setUp} method and \ct{tearDown} of the superclass.
%\begin{method}[xxx]{xxx}
%SharedOne>>>setUp
%	Transcript
%		show: 'SharedOne>>>setUp';
%		cr.
%	super setUp
%\end{method}
%\begin{method}[xxx]{xxx}
%SharedOne>>>tearDown
%	Transcript
%		show: 'SharedOne>>>tearDown';
%		cr.
%	super tearDown
%\end{method}
%Finally, we define the methods \ct{sharedSetUp} and
%\ct{sharedTearDown} that will be only executed once for the two
%tests.  Note that this solution assumes that the tests are not
%destructive to the shared fixture, but just query it.
%\begin{method}[xxx]{xxx}
%SharedOne class>>>sharedSetUp
%	Transcript
%		show: 'SharedOne class>>>sharedSetUp';
%		cr
%	"My set up here."
%\end{method}
%\begin{method}[xxx]{xxx}
%SharedOne class>>>sharedTearDown
%	Transcript
%		show: 'SharedOne class>>>sharedTearDown';
%		cr
%	"My tear down here."
%\end{method}

%\subsection{\ct{SharedSetUpTestSuite}}

%The \ct{SharedSetUpTestSuite} defines just one instance variable
%\ct{testCaseClass} and redefines the two methods necessary to run
%the test suite \ct{run:} and \ct{run}.
%\ct{checkAndArmSharedSetUp} initializes the number of tests to run
%before the shared \ct{tearDown} is executed.
%\begin{method}[xxx]{xxx}
%SharedSetUpTestSuite
%	superclass: TestSuite
%	instanceVariableNames: 'testCaseClass'
%\end{method}
%\begin{method}[xxx]{xxx}
%SharedSetUpTestSuite>>>checkAndArmSharedSetUp
%	self tests isEmpty
%		ifFalse: [self tests first class
%				 armTestsToTearDown: self tests size]
%\end{method}
%\begin{method}[xxx]{xxx}
%SharedSetUpTestSuite>>>run: aResult
%	self checkAndArmSharedSetUp.
%	^super run: aResult
%\end{method}
%\begin{method}[xxx]{xxx}
%SharedSetUpTestSuite>>>run
%	self checkAndArmSharedSetUp.
%	^super run
%\end{method}
%Finally the method \ct{addTest:} is specialized so that it marks
%all its tests with the fact that they are executed in a
%\ct{TestSuite} and checks whether all its tests are from the same
%class to avoid inconsistency.
%\begin{method}[xxx]{xxx}
%SharedSetUpTestSuite>>>addTest: aTest
%	"Sharing a setup only works if the test case
%	composing the test suite are from
%	the same class so we test it"

%	aTest executedFromASuite.
%	testCaseClass isNil
%		ifTrue: [testCaseClass := aTest class.
%				super addTest: aTest ]
%		ifFalse: [aTest class == testCaseClass
%				  ifFalse: [self error:
%						   'you cannot have test case of
%							different classes in
%							a SharingSetUpTestSuite'.]
%				  ifTrue: [super addTest: aTest]]
%\end{method}

%\subsection{Changes to \ct{TestCase}}

%In order for the above changes to work, you must make
%\ct{TestCase} aware of your new test suite.
%\begin{method}[xxx]{xxx}
%TestCase class>>>buildSuite
%	| suite |
%	^self isAbstract
%		ifTrue:
%			[suite := self suiteClass new.
%			suite name: self name asString.
%			self allSubclasses
%				do: [:each |
%					each isAbstract
%						ifFalse: [suite addTest:
%						  each buildSuiteFromSelectors]].
%			suite]
%		ifFalse: [self buildSuiteFromSelectors]
%\end{method}
%\begin{method}[xxx]{xxx}
%TestCase class>>>buildSuiteFromMethods: testMethods
%	^testMethods
%		inject: ((self suiteClass new)
%				name: self name asString;
%				yourself)
%		into:
%			[:suite :selector |
%			suite
%				addTest: (self selector: selector);
%				yourself]
%\end{method}
%If you have made all the changes correctly, you should be able to run
%your tests and see the results shown in section~\ref{sec:extending}.
%
%\section{Exercise}

%The previous section was designed to give you some insight into the
%workings of \SUnit.  You can obtain the same effect by using \SUnit's
%resources.

%Create new classes \ct{MyTestResource} and \ct{MyTestCase}
%which are subclasses of \ct{TestResource} and \ct{TestCase}
%respectively.  Add the appropriate methods so that the following
%messages are written to the \ct{Transcript} when you run your
%tests.

%\begin{method}[xxx]{xxx}
%MyTestResource>>>setUp has run.
%MyTestCase>>>setUp has run.
%MyTestCase>>>testOne has run.
%MyTestCase>>>tearDown has run.
%MyTestCase>>>setUp has run.
%MyTestCase>>>testTwo has run.
%MyTestCase>>>tearDown has run.
%MyTestResource>>>tearDown has run.
%\end{method}

%% You need to write the following six methods.

%% MyTestCase>>>setUp
%%	 Transcript
%%		 show: 'MyTestCase>>>setUp has run.';
%%		 cr

%% MyTestCase>>>tearDown
%%	 Transcript
%%		 show: 'MyTestCase>>>tearDown has run.';
%%		 cr

%% MyTestCase>>>testOne
%%	 Transcript
%%		 show: 'MyTestCase>>>testOne has run.';
%%		 cr

%% MyTestCase>>>testTwo
%%	 Transcript
%%		 show: 'MyTestCase>>>testTwo has run.';
%%		 cr

%% MyTestCase class>>>resources
%%	 ^Array with: MyTestResource

%% MyTestResource>>>setUp
%%	 Transcript
%%		 show: 'MyTestResource>>>setUp has run';
%%		 cr

%% MyTestResource>>>tearDown
%%	 Transcript
%%		 show: 'MyTestResource>>>tearDown has run.';
%%		 cr
%=================================================================
%\section{Chapter summary}
\section{まとめ}

\if0
This chapter explained why tests are an important investment in 
the future of your code.  
We explained in a step-by-step fashion how
to define a few tests for the class \ct{Set}.
Then we gave an overview of the core of the \sunit framework by presenting
the classes \ct{TestCase}, \ct{TestResult}, \ct{TestSuite}
and \lct{TestResources}.  Finally we looked deep inside \sunit by
following the execution of a test and a test suite.
\fi

この章では、なぜテストが重要な投資であるのかを説明しました。
\ct{Set} クラスのテストを一つずつ定義しながら説明しました。
\sunit フレームワークの中心となるクラス群、 \ct{TestCase} 、 \ct{TestResult} 、 \ct{TestSuite} 、 \ct{TestResource} の全体像をお見せしました。
テストとテストスイートの実行過程の深層を見てきました。

\if0
\begin{itemize}
  \item To maximize their potential, unit tests should be fast, repeatable, independent of any direct human interaction and cover a single unit of functionality.
  \item Tests for a class called \ct{MyClass} belong in a class classed \ct{MyClassTest}, which should be introduced as a subclass of \lct{TestCase}.
  \item Initialize your test data in a \ct{setUp} method.
  \item Each test method should start with the word ``test''.
  \item Use the \ct{TestCase} methods \ct{assert:}, \ct{deny:} and others to make assertions.
  \item Run tests using the SUnit test runner tool (in the tool bar).
\end{itemize}
\fi

\begin{itemize}
  \item ユニットテストの効果を最大限に引き出すポイント: 高速、繰り返し可能、 途中で人の操作を要求しない、一つの機能のみをテストする。

  \item テストクラス名は「元のクラス名 + \ct{Test} 」にします (\ct{MyClass} のテストクラスなら \ct{MyClassTest}) 。テストクラスは \ct{TestCase} のサブクラスとして定義すべきです。

  \item \ct{setUp} メソッドでテストデータを初期化してください。

  \item 各テストメソッド名は ``test'' で始めるべきです。

  \item \ct{TestCase} の \ct{assert:} や \ct{deny:} などのメソッドでアサーションを定義します。

  \item SUnit のテストランナーを使ってテストを実行しましょう。テストランナーはツールバーにあります。

\end{itemize}

%=============================================================
\ifx\wholebook\relax\else
   \bibliographystyle{jurabib}
   \nobibliography{scg}
   \end{document}
\fi
%=============================================================
%%% Local Variables:
%%% coding: utf-8
%%% mode: latex
%%% TeX-master: t
%%% TeX-PDF-mode: t
%%% ispell-local-dictionary: "english"
%%% End:
