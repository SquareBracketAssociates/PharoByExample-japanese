% $Author$
% $Date$
% $Revision$

% HISTORY:
% 2006-12-07 - Stef starts
% 2007-01-26 - Andrew updates
% 2007-05-22 - Andrew first draft
% 2007-06-24 - Oscar edit
% 2009-07-06 - Oscar migrate to pharo

%=================================================================
\ifx\wholebook\relax\else
% --------------------------------------------
% Lulu:
	\documentclass[a4paper,10pt,twoside]{book}
	\usepackage[
		papersize={6.13in,9.21in},
		hmargin={.75in,.75in},
		vmargin={.75in,1in},
		ignoreheadfoot
	]{geometry}
	\input{../common.tex}
	\pagestyle{headings}
	\setboolean{lulu}{true}
% --------------------------------------------
% A4:
%	\documentclass[a4paper,11pt,twoside]{book}
%	\input{../common.tex}
%	\usepackage{a4wide}
% --------------------------------------------
    \graphicspath{{figures/} {../figures/}}
	\begin{document}
	% \renewcommand{\nnbb}[2]{} % Disable editorial comments
	\sloppy
\fi
%=================================================================
\chapter{SUnit}
\chalabel{SUnit}

%=================================================================

\section{はじめに}

\indmain{SUnit} は小さいながらも強力な、テストの作成と運用のためのフレームワークです。その名前からわかるように、 \emph{ユニットテスト} を対象として設計されていますが、統合テストや機能テストにも使えます。 \sunit は Kent Beck が開発し、 Joseph Pelrine を始めとする開発者達がテストリソース (\secref{resource}) の概念を取り入れました。

\index{Beck, Kent}
\index{Pelrine, Joseph}
\seeindex{resource}{test, resource}

テストと\ind{テスト駆動開発}のメリットは \pharo や \st に留まりません。
テスト自動化は\ind{アジャイルソフトウェア開発}の特徴であり、ソフトウェアの品質を上げたい開発者なら採用するでしょう。 
ユニットテストの効果を評価する開発者は多く、今では多くの言語に \xUnit{} ライブラリがあります (\ind{Java} 、 \ind{Python} 、 \ind{Perl} 、 .Net 、\ind{Oracle} など) 。
\seeindex{Matrix!free will}{Oracle} % sorry, couldn't resist
この章では \SUnit~3.3 (原文の執筆時のバージョン) について解説します。
\sunit の最新版は \url{sunit.sourceforge.net} で確認できます。
\index{xUnit}
\index{Net@.Net}

テストは新しい概念ではありません。
誰でもテストはエラーを発見するいい方法だと知っています。
\mbox{\ind{エクストリーム・プログラミング}} (eXtreme Programming) ではテストの実践方法が考案され、テストの\emph{自動化}が重視されました。
その結果テストの価値が上がり、プログラマにとってテストは面倒な作業から楽しい作業に変わりました。
\st のプログラミング環境がインクリメンタルな開発を支えることから、 \st コミュニティは長らくテストを重視してきました。
\st の伝統的な開発では、実装したメソッドをすぐにワークスペースでテストします。
メソッドのコメントにテストを書いたり、テストの準備がいる場合はサンプルメソッドとして実装する場合もあります。
ワークスペースにテストを書くと (\st イメージを共有しない限り) 他のプログラマから使えないので、その点ではコメントやサンプルメソッドのほうがまだ便利です。
ただ、メソッドの開発に合わせてテストを見直し、さらに自動的に実行させるのは簡単ではありません。
テストが実行されなければ、バグも見つけられません!
また、サンプルメソッドが一見正しい結果を出したとしても、本当に正しい処理が行われた証拠にはなりません。

\sunit には十分な価値があります。
テストの結果が正しいかどうかをテスト自身で検証できます。
テストをグループ分けしたり、テストに必要な環境の準備をしたり、グループ単位でテストを実行できます。
もうワークスペースで小さなコードを書かなくても、二分もあれば \sunit でテストを書けるようになります。
\sunit を使ってテストの自動化をしましょう。

この章では、テストを行う理由とよりよいテストを書く方法の議論から始めます。
小さな例を挙げながら \sunit の使い方を紹介し、最終的には \sunit の内部実装を見ていきます。
\sunit を支えている \st の \ind{リフレクション} の力がわかるはずです。

\section{なぜテストは重要か}
\seclabel{why}

残念ながら、テストを時間の無駄と思い込んでいる開発者もたくさんいます。
\emph{彼ら}はバグを作りません\,---\,\emph{他の}プログラマだけがバグを作るのです。
誰しも一度くらいこう言ってしまった経験があります: 「時間があればテストを書きますよ」。
絶対にバグのないコードを書けて、絶対にそのコードが将来も変更されないなら、確かにテストは時間の無駄です。
しかし、そんなコードはごく小さいか誰も使わないようなものがほとんどです。
テストを将来への投資と考えてください。
テストはすぐにでも役に立ちますが、将来アプリケーションや実行環境が変わっても\emph{絶対に}役に立ちます。

テストにはいくつかの役割があります。
一つは生きたドキュメントです。
すべてのテストに成功すれば、ドキュメントは最新版です。
もう一つは、変更したコードが無事動くという確信です。
もしシステムの一部が壊れたとしても、すぐに原因を見つけられます。
最後に、実装方法よりもアプリケーションの設計と\emph{ユーザとのインターフェース}を考えるようになります。
コードを書く前にテストを書いておけば、テストが成功するようにコードを書くしかありません。
ぜひ試してコードの品質を向上させてください。

アプリケーションのあらゆる面をすべてテストするのは不可能です。
いくら良質のテストを書いても、システムを壊す機会を窺っているバグが入り込みます。
バグを見つけたらすぐにあぶりだすテストを書き、実行し、失敗を確認し、修正を始めましょう。
そのテストが作業の終わりを知らせてくれます。

\section{いいテストを書くには?}

いいテストを書けるようになるには、練習するのが一番です。
テストのメリットを最大限に活かすヒントを見ていきましょう。

\begin{enumerate}
\item テストは繰り返し実行可能であるべきです。
  実行したいときに実行できて、常に同じ結果を出すべきです。

\item テストは人の手を借りずに実行するべきです。
  例え夜中でも実行できるようにすべきです。

\item テストは物語であるべきです。各テストの目的を一つに絞るべきです。
  テストをシナリオと考え、読めばアプリケーションの機能を理解できるようにすべきです。 \label{prop:oneAspect}

\item テストの変更はなるべく控えるべきです。
  アプリケーションを変更するたびにテストを変更する必要が出ては大変です。
  テストを変更せずに済ませるには、公開するインターフェースを主に使ってテストを書きましょう。
  公開するインターフェースが内部で複雑な処理を行うなら、プライベートなヘルパーメソッドのテストを書いてもいいでしょう。
  ただし、インターフェースの内部実装が変わったときには、そのテストを変更するか捨てる必要があるかもしれません。

\end{enumerate}

テストの目的を絞り (\ref{prop:oneAspect}) 、テストすべき機能とテストの数のバランスをとりましょう。
システムの一部を変更したときに動かなくなるテストを少なくするべきです。
100 のテストに失敗するより、 10 のテストの失敗で済むほうが対処しやすいのですから。
ただし、いつもこの通りにできるとは限りません。
例えば変更によってオブジェクトの初期化やテストのセットアップに問題が出れば、すべてのテストが失敗してしまいます。

\ind{エクストリーム・プログラミング} では、コードを書く前にテストを書く (テストファースト) ことを提案しています。
これはソフトウェア開発者の本能と相反するように見えますが、我々は自信を持って言えます: 「さあ進め、さあ試せ」。 
テストを先に書くことで、どんなコードを書くべきか、何をすべきか、どう要件をクラスに落とせばいいか、どんな API にすればいいのかを考える手助けになります。
さらに、開発速度を上げる自信にもつながります。開発を急ぐあまりに肝心な点を見落とすかもしれない、といった心配をする必要はありません。

\section{\sunit by example}

\SUnit の詳細に触れる前に、一つずつ例を挙げていきます。 \ct{Set} クラスのテストを例に進めますので、例に従ってコードを入力してください。

\subsection{ステップ 1: テストクラスを作る}

\dothis{まず、 \clsind{TestCase} のサブクラス \ct{ExampleSetSet} を作ります。このクラスに二つのインスタンス変数 \ct{full} と \ct{empty} を追加します。}

\begin{classdef}[exampleSetTest]{ExampleSetTest クラス}
TestCase subclass: #ExampleSetTest
	instanceVariableNames: 'full empty'
	classVariableNames: ''
	poolDictionaries: ''
	category: 'MySetTest'
\end{classdef}

\ct{ExampleSetTest} クラスに \ct{Set} クラスのテストをまとめます。このクラスにはテスト実行時のコンテキスト (テストの実行に必要な状況、状態) を定義します。コンテキストは二つのインスタンス変数 \ct{full} と \ct{empty} で表します。 \ct{full} は要素を持つセット、 \ct{empty} は空のセットを表します。

クラスの名前は重要ではありませんが、 \ct{Test} で終わる名前にすべきです。 \ct{Pattern} クラスのテストクラスなら \ct{PatternTest} とすれば、ブラウザで見たときに (どちらも同じカテゴリであれば) 両方のクラスがアルファベット順に並びます。テストクラスが \ct{TestCase} のサブクラスであるほうが重要です。

\subsection{ステップ 2: テスト環境を整える}

\mthind{TestCase}{setUp} メソッドで、テスト実行時の状態を定義します。ちょっと初期化メソッドに似ていますね。 \ct{setUp} は (テストクラスに実装した) 各テストメソッドの実行前に実行されます。

\index{SUnit!set up method}
\seeindex{testing}{SUnit}

\dothis{\ct{setUp} メソッドを次のように定義してください。インスタンス変数 \ct{empty} に空のセットを代入し、 \ct{full} に二つの要素を持つセットを代入します。}

\needlines{3}
\begin{method}[setupExampleSetTest]{フィクスチャを設定する}
ExampleSetTest>>>setUp
	empty := Set new.
	full := Set with: 5 with: 6
\end{method}

\noindent

このテストではコンテキストのことを \emph{フィクスチャ} と呼びます。テスト用語です。

\index{SUnit!fixture}
\seeindex{fixture}{SUnit, fixture}

\subsection{ステップ 3: テストメソッドを書く}

それではテストを書いてみましょう。
\ct{ExampleSetTest} にテスト用のメソッドを定義します。
一つのメソッドで一つのテストを表します。
テストメソッドは名前を `\ct{test}' で始め、引数を取りません。
\sunit はテストメソッドを集めてテストスイートとしてまとめます。

\dothis{次のテストメソッドを定義します。}

最初のテスト \ct{testIncludes} は、 \ct{Set} の \ct{includes:} メソッドをテストします。
\ct{5} を要素に持つ \ct{Set} に \ct{includes: 5} のメッセージを送ると \ct{true} が返ると仮定します。
明らかに \ct{setUp} メソッドが先に実行されることが前提になっています。

\begin{method}[testIncludes]{\ct{Set} の要素のテスト}
ExampleSetTest>>>testIncludes
	self assert: (full includes: 5).
	self assert: (full includes: 6)
\end{method}

二つ目のテスト \ct{testOccurrences} は、 \ct{Set} に同じ要素が一つしか含まれないことを検証します。 \ct{full} にはすでに \ct{5} が含まれていますが、さらに \ct{5} を追加しても \ct{5} の数は変わりません。

\needlines{6}
\begin{method}[testOccurrences]{同じ要素の数のテスト}
ExampleSetTest>>>testOccurrences
	self assert: (empty occurrencesOf: 0) = 0.
	self assert: (full occurrencesOf: 5) = 1.
	full add: 5.
	self assert: (full occurrencesOf: 5) = 1
\end{method}

最後に \ct{5} を削除し、 \ct{Set} が \ct{5} を含まないことをテストします。

\begin{method}[testRemove]{要素を削除するテスト}
ExampleSetTest>>>testRemove
	full remove: 5.
	self assert: (full includes: 6).
	self deny: (full includes: 5)
\end{method}

\noindent
\ct{true} を返すべきではないテストに \mthind{TestCase}{deny:} メソッドを使っているのに注意してください。
\ct{aTest deny: anExpression} は \ct{aTest assert: anExpression not} と同じ意味ですが、よりわかりやすくなります。

\subsection{ステップ 4: テストを実行する}

テストの実行にはブラウザを使うと簡単です。
パッケージ、クラス名、テストメソッドのいずれかを選択し、メニューから \menu{run the tests (t)} を実行します。
テストの成功したメソッドには緑のマークが、失敗したメソッドには赤のマークがつきます。
すべてのテストに成功するとクラスに緑のマークが、一つでも失敗すると赤のマークがつきます。

\begin{figure}[tbh]
  \begin{center}
	\includegraphics[width=\linewidth]{browser-tests}
	\caption{ブラウザから \sunit のテストを実行する}
	\figlabel{browser-tests}
  \end{center}
\end{figure}

% 最新版の Pharo ではデザインが多少変わっています。

\menu{World} メニューから \menu{Test Runner} を選択すると開く \sunit \emphind{テストランナー} (図 \figref{test-runner}) を使えば、テストの詳細な内容と実行結果を見られます。
\emph{テストランナー} はまとまったテストを簡単に実行できるように設計されています。
最も左にあるペインでは、テストクラス (\ie \ct{TestCase} のサブクラス) を含むすべてのカテゴリを表示しています。カテゴリを選択すると、そのカテゴリに含まれるテストクラスのクラス階層が隣のペインに表示されます。
イタリック体のフォントで表示されるクラスは抽象クラス (テストメソッドを含まないテストクラス) です。

% ここまで読み進めた読者には "the test class hierarchy is shown by indentation, ..." のくだりは冗長と考えて省略しました。
% また、 TestCase が表示されるのは "SUnit-Kernel" カテゴリを選んだ場合か、どのカテゴリも選ばない場合です。

\begin{figure}[tbh]
  \begin{center}
	\includegraphics[width=\linewidth]{test-runner}
	\caption{\pharo \sunit テストランナー}
	\figlabel{test-runner}
  \end{center}
\end{figure}

\dothis{テストランナーを開いて \menu{MySetTest} カテゴリを選択し、 \button{Run Selected} ボタンをクリックします。}

\dothis{\ct{ExampleSetTest>>>testRemove} にバグを入れて、再度テストを実行してみましょう。例えば、 \ct{5} を \ct{4} にします。}

失敗したテストは \emph{テストランナー} の右のペインに表示されます。
失敗の原因を知りたければ、表示されたテストメソッドをクリックしてください。
ここからデバッグもできます。

\subsection{ステップ 5: 結果を分析する}

\ct{TestCase} クラスの \mthind{TestCase}{assert:}\ メソッドは論理値の引数を一つ
とります。
通常はこの値にテストする式の戻り値を指定します。
引数が真ならテストに成功、偽なら失敗とみなします。

% 原文が回りくどいので構成を変えました。

テストの実行結果は 3 通りあります。
一つはすべてのテストに成功する場合。
すべてのテストに成功すると、{テストランナー}の右上のバーが緑に変わります。
次に、一つでもテストに\emph{失敗}する場合。
{テストランナー}のバーが黄色に変わり、その下のペインに失敗したテストが表示されます。
最後に、テストの実行中にエラー (例外) が発生する場合。
例えば \emph{message not understood} や \emph{index out of bounds} といったエラーえす。
エラーが発生するとバーが赤に変わり、エラーの発生したテストが表示されます。

\dothis{エラーしたり失敗するようにテストを変更してみましょう。}

\section{\SUnit クックブック}

この章では \SUnit の詳しい使い方を解説します。
\JUnit\footnote{\url{http://junit.org}} などの他のテスティングフレームワークの経験があるなら、 \SUnit は親しみやすいでしょう。どのフレームワークも \SUnit にルーツがあります。
通常 \SUnit では GUI でテストを実行しますが、 GUI を使いたくない場合もあるでしょう。

\subsection{様々なアサーション}

\ct{assert:} や \ct{deny:} の他にもアサーション用のメソッドがあります。

一つ目は \mthind{TestCase}{assert:description:} と \mthind{TestCase}{deny:description:} です。
二つ目の引数には、テストの失敗時に表示する理由を示す文字列を指定します。
失敗の原因がわかりにくい場合に使うといいでしょう。
~\secref{descriptionStrings}でこれらのメソッドについて触れます。

二つ目は \mthind{TestCase}{should:raise:} と \mthind{TestCase}{shouldnt:raise:} です。
テストの実行時に適切な例外が発生するかどうかをテストします。
例えば \ct{(self should: aBlock raise: anException)} を実行すると、 \ct{aBlock} の実行中に例外 \ct{anException} が発生すればテストは成功です。
\Mthref{ESTtestIllegal} に \mbox{\ct{should:raise:}} を使った例を示します。

\dothis{このテストを実行してみましょう。}

\ct{should:} と \ct{shouldnt:} の最初の引数は、評価する式を\emph{含む}\emphind{ブロック}です。

\begin{method}[ESTtestIllegal]{エラーの発生をテストする}
ExampleSetTest>>>testIllegal
	self should: [empty at: 5] raise: Error.
	self should: [empty at: 5 put: #zork] raise: Error
\end{method}

\sunit では処理系依存のコードをできるだけ避けて移植性を高めており、どの \st 処理系でも使えるようになっています。
例えばシステムエラーのクラスは処理系によって異なるため、 \cmind{TestResult class}{error} メソッドで返すようにしています。
処理系に依存しないようにするには、 \mthref{ESTtestIllegal} のコードを次のように書き換えます。

\needlines{4}
\begin{method}[portabletestillegal]{移植性の高いエラー捕捉方法}
ExampleSetTest>>>testIllegal
	self should: [empty at: 5] raise: TestResult error.
	self should: [empty at: 5 put: #zork] raise: TestResult error
\end{method}

\dothis{試してみましょう。}

\subsection{メニューを使わずにテストランナーを開く}

ワークスペースなどで \ct{TestRunner open} を実行 (コンテキストメニューから \menu{do it} を選択) します。

% この内容は次の "Running a single test" で書かれていますが、
% あまり関係がないため独立させました。

\subsection{テストを一つだけ実行する}

テストランナーを開かずにテストを一つだけ実行したい場合、次のコードを実行します。

\begin{code}{}
ExampleSetTest run: #testRemove --> 1 run, 1 passed, 0 failed, 0 errors
\end{code}

\subsection{テストクラスに定義されたすべてのテストを実行する}

\ct{TestCase} のサブクラスに \ct{suite} メッセージを送ると、名前が ``\ct{test}'' で始まるメソッドを集めてテストスイートを組み立てます。
このテストスイートに \ct{run} メッセージを送ってテストを実行します。
次に例を示します。

\begin{code}{}
ExampleSetTest suite run --> 5 run, 5 passed, 0 failed, 0 errors
\end{code}

\subsection{必ず TestCase クラスを継承しなければならない?}

\ct{TestCase} クラスを継承しなくても \clsind{TestSuite} は組み立てられます。
ただし、自分で \ct{TestCase} クラスの主要なメソッド (\ct{assert:} など) を実装したり、テストスイートを自分で作る必要が出てきます。
\ct{TestCase} クラスを継承しましょう。

% JUnit の説明が少し違います。
% JUnit 3.x 系では TestCase クラスを継承して test* メソッドを定義しますが、
% 4.x 系ではアノテーションを使って任意のメソッドをテストに指定します。
% JUnit の説明のほうが多くなるので割愛しました。

\section{SUnit フレームワーク}

SUnit は次の 4 つのクラスから構成されます: \clsind{TestCase} 、
\clsind{TestSuite} 、 \clsind{TestResult} 、 \clsind{TestResource} (\figref{sunit-classes}) 。
\emph{テストリソース}は \sunit 3.1 で導入された概念で、コストの高いセットアップに使われます。
\ct{TestResource} の \ct{setUp} メソッドは \ct{TestCase>>>setUp} と異なり、テストスイートの実行前に一度だけ実行されます。 \ct{TestCase>>>setUp} は各テストの実行前に setUp メソッドが実行されます。

\begin{figure}[htb]
  \begin{center}
		{\includegraphics[width=0.8\textwidth]{sunit-classes}}
	\caption{\SUnit を構成する 4 つのクラス}
	\figlabel{sunit-classes}
  \end{center}
\end{figure}

\subsection{TestCase}

\clsindmain{TestCase} は抽象クラスです。サブクラスにテストを記述し、テスト間で実行時の状態を共有します (テストスイート) 。
各テストは、それぞれ \ct{TestCase} のサブクラスのインスタンスを生成してから実行されます。 \mthind{TestCase}{setUp} を実行し、テストメソッドを実行し、最後に \mthind{TestCase}{tearDown} を実行します。

実行時の状態をインスタンス変数で表し、オーバーライドした \ct{setUp} メソッドで初期化します。 \ct{setUp} で初期化したオブジェクトの終了処理を行うには、 \ct{tearDown} メソッドをオーバーライドします。

\subsection{TestSuite}

\clsindmain{TestSuite} はテストケース (テストとテストスイート) の集合を含みます。つまり、 \ct{TestSuite} は \ct{TestCase} と \ct{TestSuite} のインスタンスを含みます。
\lct{TestCase} と \lct{TestSuite} の間に継承関係はありませんが、どちらも同じプロトコルを実装しているので同じように扱えます。
例えば、どちらも \ct{run} メソッドが定義されています。
\ct{TestSuite} を枝、 \ct{TestCase} を葉として Composite パターンを適用しています (詳しくは\textit{デザインパターン}を参照してください) 。

\subsection{TestResult}

\clsindmain{TestResult} は \ct{TestSuite} の実行結果を扱います。
成功したテストの数、失敗したテストの数、エラーの発生したテストの数を記録します。

\subsection{TestResource}
\seclabel{resource}

テストでは、各テストが互いに独立していることが重要です。
一つのテストの失敗が他のテストに波及しないように、テストの実行順序が結果に影響ようにすべきです。
\ct{setUp} と \ct{tearDown} で調整しましょう。

しかし、テストのたびに環境を用意していては時間がかかり過ぎる場合があります。
さらに環境を壊さずに済むテストなら、一度作った環境を使い回せば十分です。
例えばデータベースに問い合わせをするテストなら、事前に用意しておいたデータベースに接続するだけで準備できます。
コンパイルされたコードを分析するテストなら、同様に事前にソースコードをコンパイルしておけば十分です。

リソースをキャッシュさえすれば、テスト間で共有できるようになるでしょうか?
\ct{TestCase} のインスタンス変数はキャッシュとして使えません。
\ct{TestCase} クラスのインスタンスは一つのテストでしか使われないからです。
グローバル変数ならできますが、多用するとシステムがグローバル変数であふれてしまいますし、各テストとグローバル変数の関係がわかりにくくなります。
必要なリソースを含むシングルトンオブジェクトをクラスに持たせるのがいい方法です。
そのための抽象クラスとして \clsindmain{TestResource} が用意されています。
\lct{TestResource} のサブクラスに \ct{current} メッセージを送ると、サブクラスのシングルトンインスタンスを得られます。
リソースの初期化と終了処理を行うには、 \ct{setUp} と \ct{tearDown} をオーバーライドしてください。

もう一つ、どうにかしてリソースとテストスイートを関連付ける必要があります。
\ct{TestCase} のサブクラスで \ct{resources} \emph{クラスメソッド} をオーバーライドし、テストクラスで使うリソースを返すようにします。
\ab{}
デフォルトでは、 \ct{TestSuite} のリソースはすべての \ct{TestCase} のリソースの集合です。

次にリソースの例を示します。
\ct{TestResource} のサブクラス \ct{MyTestResource} を定義します。
次に \ct{MyTestCase} のクラスメソッド \ct{resources} で、このクラスのテストで使うリソースの配列を返します。

\needlines{8}
\begin{classdef}[mytestresource]{TestResource のサブクラスの使用例}
TestResource subclass: #MyTestResource
	instanceVariableNames: ''

MyTestCase class>>>resources
	"このクラスのテストで使うリソースを返す"
	^{ MyTestResource }
\end{classdef}

\section{SUnit のさらなる機能}

現在の \sunit は \ct{TestResource} に加えて、アサーションの詳細、ロギング、失敗したテストの実行再開に対応しています。

\subsection{アサーションの詳細}
\seclabel{descriptionStrings}

\ct{TestCase} のアサーションプロトコルには、アサーションの詳細を文字列 (\ct{String}) で知らせるメソッドが含まれています。
テストに失敗すると、テストランナーにアサーションの詳細が表示されます。
もちろん、詳細を表す文字列を動的に生成しても大丈夫です。

\begin{code}{}
| e |
e := 42.
self assert: e = 23
	description: '期待した値は 23 、実際の値は ', e printString
\end{code}

\ct{TestCase} 関連のメソッドを次に示します:
\begin{code}{}
#assert:description:
#deny:description:
#should:description:
#shouldnt:description:
\end{code}
\cmindex{TestCase}{assert:description:}
\cmindex{TestCase}{deny:description:}
\cmindex{TestCase}{should:description:}
\cmindex{TestCase}{shouldnt:description:}

\subsection{ロギング}

前節で触れたアサーションの詳細は、 \ct{Transcript} などのストリーム (\ct{Stream}) やファイルシステムにも出力できます。
出力の有無を設定するには、テストクラスの \cmind{TestCase}{isLogging} をオーバーライドして論理値を返すようにします。
出力するなら真を、しないなら偽を指定します。
出力先の指定は \cmind{TestCase}{failureLog} をオーバーライドし、出力したいストリームを返すようにします。
デフォルトでは \ct{Transcript} が指定されています。

\subsection{テストの失敗後も実行を続ける}

\sunit では、テストの失敗後に実行を続けるかどうかを指定できます。
\st の例外処理の仕組みを利用した、非常に強力な機能です。
どのように動くのか、例を見てみましょう。
まず、次のテストコードを見てください。

\begin{code}{}
aCollection do: [ :each | self assert: each even]
\end{code}

このコードではコレクションの要素を順に調べ、偶数 (\ct{even}) ではないオブジェクトがあれば、すぐにテストが失敗します。
できれば偶数ではないオブジェクトがいくつあるのか把握したいので、このコードを次の要に変更します。

\begin{code}{}
aCollection do:
	[:each |
	self
		assert: each even
		description: each printString , ' は偶数ではありません'
		resumable: true]
\end{code}

これで、失敗した要素についてのメッセージをログ出力できます。
実行を続けても失敗回数は増えないので、例え 10 回以上アサーションに失敗しても、テストの失敗は 1 回と見なされます。
これまで見てきたアサーション用のメソッドは、失敗後に実行を続けないアサーションと同じ意味です。
つまり、 \ct{assert: p description: s} なら \ct{assert: p description: s resumable: false} と同じです。

\cmindex{Collection}{do:}

\section{SUnit の内部実装}

\sunit の内部実装は、 \st のフレームワークのいい例でもあります。
テストの実行過程を通して、内部実装の鍵となる面を見てみましょう。
(訳者注: ここで紹介されている実装と最新版の実装は異なりますが、各メソッドの目的は同じです。)

\subsection{テストを一つ実行する}

テストを一つ実行するには、 \ct{(aTestClass selector: aSymbol) run.} を評価します。

\begin{figure}[tbh]
  \begin{center}
		{\includegraphics[width=0.7\textwidth]{sunit-scenario}}
	\caption{テストを一つ実行する}
	\figlabel{sunit-scenario}
  \end{center}
\end{figure}

\cmind{TestCase}{run} メソッドは、テスト結果を集計する \clsind{TestResult} のインスタンスを生成し、それを引数として自身に \mthind{TestCase}{run:} メッセージを送信します。
(\figref{sunit-scenario} を参照)

\needlines{6}
\begin{method}[tastecaserun]{テストケースを実行する}
TestCase>>>run
	| result |
	result := TestResult new.
	self run: result.
	^result
\end{method}

\cmind{TestCase}{run:} メソッドは \ct{TestResult} に \mthind{TestResult}{runCase:} メッセージを送信します。

\begin{method}[testcaserun:]{\ct{TestCase} を \ct{TestResult} に渡す}
TestCase>>>run: aResult
	aResult runCase: self
\end{method}

\ct{TestResult>>>runCase:} メソッドは、テストを実行するために各 \ct{TestCase} に \mthind{TestCase}{runCase} メッセージを送信します。
\ct{TestResult>>>runCase} はテストの実行中に発生した例外を捕捉し、エラーとして数えます。
さらにテストに失敗した回数と成功した回数を数えます。

\begin{method}[testresultruncase]{\ct{TestCase} のエラーと失敗をキャッシュする}
TestResult>>>runCase: aTestCase
	| testCasePassed |
	testCasePassed := true.
	[[aTestCase runCase] 
			on: self class failure
			do: 
				[:signal | 
				failures add: aTestCase.
				testCasePassed := false.
				signal return: false]]
					on: self class error
					do:
						[:signal |
						errors add: aTestCase.
						testCasePassed := false.
						signal return: false].
	testCasePassed ifTrue: [passed add: aTestCase]
\end{method}

\ct{TestCase>>>runCase} は、自身に \mthind{TestCase}{setUp} メッセージと \mthind{TestCase}{tearDown} を送ります。

\needlines{3}
\begin{method}[testcaseruncase]{\ct{TestCase>>runCase} の基本的な実装}
TestCase>>>runCase
	[self setUp.
	self performTest] ensure: [self tearDown]
\end{method}

\subsection{\lct{TestSuite} の実行}

複数のテストを実行するには、 \ct{TestSuite} に \ct{run} メッセージを送信します。
\ct{TestCase} のクラスメソッドには、 \ct{TestCase} のインスタンスメソッドから \ct{TestSuite} を組み立てるためのメソッドが用意されています。
\ct{MyTestCase buildSuiteFromSelectors} を評価すると、 \ct{MyTestCase} に定義したすべてのテスト (以下の \ct{TestCase class>>testSelectors} メソッドで取得できます) を含む \ct{TestSuite} を返します。

\begin{method}[testcasetestselectors]{\ct{TestSuite} の組み立て}
TestCase class>>>testSelectors 
	^self selectors asSortedCollection asOrderedCollection select: [:each | 
		('test*' match: each) and: [each numArgs isZero]]
\end{method}
\cmindex{MyTestCase class}{buildSuiteFromSelectors}

\cmind{TestSuite}{run} は \ct{TestResult} のインスタンスを生成し、すべてのリソースが問題なく使えるかどうかを検証します。
続けて自身に \cmind{TestSuite}{run:} を送り、自身の管理するすべてのテストを実行します。
リソースはテストの実行後に解放されます。

\begin{method}[testsuiterun]{\ct{TestSuite} の実行}
TestSuite>>>run
	| result |
 	result := TestResult new.
	self resources do: [ :res |
		res isAvailable ifFalse: [^res signalInitializationError]].
	[self run: result] ensure: [self resources do: [:each | each reset]].
	^result
\end{method}

\begin{method}[testsuiterun:]{\ct{TestResult} を \ct{TestSuite} に渡す}
TestSuite>>>run: aResult
	self tests do: [:each | 
		self changed: each.
		each run: aResult].
\end{method}

\clsind{TestResource} クラスとそのサブクラスは、生成したインスタンス (クラスに一つずつ) を保持しており、 \mthind{TestResource class}{current} で取得できます。
このインスタンスはテストの終了後に消され、リソースはリセットされます。

リソースが使えるかどうかを知るには \cmind{TestResource class}{isAvailable} クラスメソッドを使います。
リソースが有効であれば、必要に応じてリソースを再構築、つまり \ct{TestResource} のインスタンスを再生成できます。
インスタンスを生成するたびに \mthind{TestResource}{setUp} メソッドが実行されます。

\needlines{4}
\begin{method}[testresourceisavailable]{\ct{TestResource} が有効かの確認}
TestResource class>>>isAvailable
	^self current notNil and: [self current isAvailable]
\end{method}

\begin{method}[testresourcecurrent]{\ct{TestResource} のインスタンスの生成}
TestResource class>>>current
	current isNil ifTrue: [current := self new].
	^current
\end{method}

\begin{method}[restresourceinitialize]{\ct{TestResource} の初期化}
TestResource>>>initialize
	super initialize.
	self setUp
\end{method}

\section{テストのポイント}

テストの仕組みは簡単ですが、いいテストを書くのは簡単ではありません。いくつかテストのポイントを紹介します。

\begin{description}
\index{Feathers, Michael}
\item[Feathers のユニットテストの法則。]
  アジャイル開発コンサルタントでもあり、「レガシーコード改善ガイド」の著者でもある Michael Feathers は次のように書いています。 \footnote{\url{http://www.artima.com/weblogs/viewpost.jsp?thread=126923} を参照。 2005/9/9 。}
  \begin{quotation}
  \noindent
  {\it
  次のテストはユニットテストではありません:
  \begin{itemize}
	\item データベースと対話したり、
	\item インターネットに接続したり、
	\item ファイルシステムに触ったり、
	\item 他のユニットテストと並行して実行できなかったり、
	\item 特別な環境 (設定ファイルの編集が必要など) を準備しなければならない。
 \end{itemize}

このようなテストも悪くはありません。
書く価値はありますし、ユニットテストの補助にもなります。
しかしコードを変更したらすぐにテストできるように、本当のユニットテストとは分けておくべきです。

 }
  \end{quotation}

決して、時間がかかるという理由でテストを避けてはいけません。

 \item[ユニットテスト \textit{vs.}\ 承認テスト。] ユニットテストは機能の一部を単体でテストし、機能に関するバグを容易に見つけられます。
  各テストを失敗するまで可能な限り実行し、クラス単位で結果をまとめます。
  しかし、複雑な環境のテストや再帰的なテストが必要ならば、シナリオに沿ってテストを書くほうが簡単です。
  このようなテストは承認テストまたは機能テストと呼ばれます。
  Feathers の法則を守らないテストは、いい承認テストになるかもしれません。
  テストする機能に沿って承認テストをまとめましょう。
  例えばコンパイラを書く場合、ソースコードから生成するコードの承認テストを書くでしょう。
  たくさんのテストが実行されるでしょうが、ファイルシステムにアクセスするテストなら時間がかかるかもしれません。
  わずかなコードの変更にもテストをしたければ、ユニットテストと承認テストを分けるべきです。

\item[Black のテストの法則。]

% Rex Black のこと?

  どのテストも、テスト内容がどうシステムの信頼性につながるのかわかるようにすべきです。
  テストの必要がないと判断した機能を重要なものと考えるべきではありません。
  システムの信頼性が高まるような内容でなければ、テストをする必要はありません。
  例えば同じ目的のテストを複数書いても、かえって二つのデメリットがあります。
  一つは、テストを読んでもクラスの挙動が理解しにくくなります。
  もう一つは、コードにいくつバグが残っているのか見積もりにくくなります。
  たった一つでもバグがあれば、多くのテストが失敗するからです。
  テストを書くときは常に目的を考えましょう。

% そのまま訳すとわかりにくくなりそうでだいぶ訳を変えましたが、おかしいかも。

\end{description}

\section{まとめ}

この章では、なぜテストが重要な投資であるのかを説明しました。
\ct{Set} クラスのテストを一つずつ定義しながら説明しました。
\sunit フレームワークの中心となるクラス群、 \ct{TestCase} 、 \ct{TestResult} 、 \ct{TestSuite} 、 \ct{TestResource} の全体像をお見せしました。
テストとテストスイートの実行過程の深層を見てきました。

\begin{itemize}
  \item ユニットテストの効果を最大限に引き出すポイント: 高速、繰り返し可能、 途中で人の操作を要求しない、一つの機能のみをテストする。

  \item テストクラス名は「元のクラス名 + \ct{Test} 」にします (\ct{MyClass} のテストクラスなら \ct{MyClassTest}) 。テストクラスは \ct{TestCase} のサブクラスとして定義すべきです。

  \item \ct{setUp} メソッドでテストデータを初期化してください。

  \item 各テストメソッド名は ``test'' で始めるべきです。

  \item \ct{TestCase} の \ct{assert:} や \ct{deny:} などのメソッドでアサーションを定義します。

  \item SUnit のテストランナーを使ってテストを実行しましょう。テストランナーはツールバーにあります。

\end{itemize}

%=============================================================
\ifx\wholebook\relax\else
   \bibliographystyle{jurabib}
   \nobibliography{scg}
   \end{document}
\fi
%=============================================================
%%% Local Variables:
%%% coding: utf-8
%%% mode: latex
%%% TeX-master: t
%%% TeX-PDF-mode: t
%%% ispell-local-dictionary: "english"
%%% End:
