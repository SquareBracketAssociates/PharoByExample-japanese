% $Author$
% $Date$
% $Revision$

% HISTORY:
% 2006-10-24 - Stef started
% 2006-10-25 - Stef first draft
% 2006-12-07 - Andrew edit
% 2007-06-13 - Andrew revised
% 2007-06-21 - Oscar edit
% 2007-07-26 - Stef review
% 2007-08-23 - Oscar review
% 2007-08-29 - Andrew corrections

%=================================================================
\ifx\wholebook\relax\else
% --------------------------------------------
% Lulu:
	\documentclass[a4paper,10pt,twoside]{book}
	\usepackage[
		papersize={6.13in,9.21in},
		hmargin={.75in,.75in},
		vmargin={.75in,1in},
		ignoreheadfoot
	]{geometry}
	\input{../common.tex}
	\pagestyle{headings}
	\setboolean{lulu}{true}
% --------------------------------------------
% A4:
%	\documentclass[a4paper,11pt,twoside]{book}
%	\input{../common.tex}
%	\usepackage{a4wide}
% --------------------------------------------
    \graphicspath{{figures/} {../figures/}}
	\begin{document}
	\renewcommand{\nnbb}[2]{} % Disable editorial comments
	\sloppy
\fi
%=================================================================
\chapter{\st オブジェクトモデル}
\chalabel{model}

\st のプログラミングモデルは簡潔で一貫しています: すべてはオブジェクトで、オブジェクト間のやりとりはメッセージを送りあうことによってのみ行われます。一方で、この簡潔さと一貫性が、他の言語に慣れたプログラマにとって難しさの元になっています。この章では、\st のオブジェクトモデルの中心となる概念を示します。特に、クラスをオブジェクトとして表現することが何をもたらすのかを議論します。

%=========================================================
\section{モデルの規則}
\seclabel{rules}

\st のオブジェクトモデルは、統一的に適用できるいくつかの簡潔な規則に基づいています。

\begin{enumerate}[label={\textbf{Rule \arabic{*}}.}, ref={Rule \arabic{*}}, leftmargin=*]
\item{} \rulelabel{everything}
	すべてのものはオブジェクトである。

\item{} \rulelabel{instance}
	すべてのオブジェクトはクラスのインスタンスである。

\item{}  \rulelabel{inheritance}
	すべてのクラスにはスーパークラスがある。

\item{}  \rulelabel{message}
	すべてはメッセージ送信で起こる。

\item{}  \rulelabel{lookup}
	メソッド探索は継承の連鎖をたどっていく。

\end{enumerate}

\noindent
では、これらの規則を一つ一つ詳細に見ていきましょう。


%=========================================================
\section{すべてのものはオブジェクトである}

%\ruleref{everything}

「すべてのものはオブジェクトである」という信念は、感染力が高いものです。ほんの短い間でも \st を使えば、この規則がいかに物事を簡単にするかに、驚くことでしょう。例えば、整数も本物のオブジェクトで、他のあらゆるオブジェクトと同様、メッセージを送ることができます。

\begin{code}{@TEST}
3 + 4            --> 7    "'+ 4'を3に送ることで, 結果として7が得られます"
20 factorial  --> 2432902008176640000   "factorialを送ると, 大きな数が得られます"
\end{code}

\ct{20 factorial}の内部表現と\ct{7}の内部表現は実際には異なっていますが、それらはどちらもオブジェクトなので、コード上では --- \ct{factorial}の実装すら --- そうしたことを知る必要はありません。

\needlines{3}
また、この規則によって次の最も重要な帰結が生じます。
\important{クラスもオブジェクトである。}
もっと言えば、クラスはセカンドクラスオブジェクトではありません。クラスは正真正銘のファーストクラスオブジェクトで、メッセージを送ったり、インスペクトしたりできます。つまり\pharo は本当の意味でリフレクティブなシステムだということです。このことが開発者に強力な表現力を与えています。

\st の実装の深層部には、3種類のオブジェクトがあります。(1)参照渡しされるインスタンス変数を持つ、通常のオブジェクト、(2)値渡しされる\emph{小さな整数(SmallInteger)}、(3)配列のように連続したメモリ領域を持つインデックス付きオブジェクトです。\st の美しさは、これら3種類のオブジェクトの違いについて通常は気にする必要がないというところにあります。

%=========================================================
\section{すべてのオブジェクトはクラスのインスタンスである}

%\ruleref{instance}

すべてのオブジェクトはクラスを持っています。オブジェクトに\ct{class}メッセージを送るとわかります。

\begin{code}{@TEST}
1 class                 --> SmallInteger
20 factorial class --> LargePositiveInteger
'hello' class          --> ByteString
#(1 2 3) class       --> Array
(4@5) class         --> Point
Object new class --> Object
\end{code}

クラスはインスタンス変数を通じてインスタンスの\emph{構造}を定義します。そしてメソッドよってインスタンスの\emph{振る舞い}を定義します。各メソッドの名前は\emphsubind{メソッド}{セレクタ}と呼ばれます。名前はそのクラスの中でユニークです。

\emph{クラスはオブジェクト}で、{すべてのオブジェクトはクラスのインスタンスである}ことから、クラスもまた、あるクラスのインスタンスということになります。クラスをインスタンスとするクラスは\emphind{メタクラス}と呼ばれます。クラスを定義するたびに、システムは自動的にメタクラスを生成します。メタクラスは、インスタンスとしてのクラスの構造と振る舞いを定義します。
99\% \damien{どなたか、本当に99\%だと確かめましたか?大抵のものは'99\%'よりもよいかもしれませんよ}は、メタクラスについて考える必要はないでしょう。当面は無視しても大丈夫です。
(\charef{metaclasses}でメタクラスについて詳しく見ていきます。)
%---------------------------------------------------------
\subsection{インスタンス変数}

\st でのインスタンス変数は、その\emph{インスタンス}にとってのプライベートな変数です。これは、\ind{Java}や\ind{C++}のインスタンス変数(``フィールド''とか``メンバ変数''とも呼ばれています)が同じクラスの他のインスタンスからのアクセスを許しているのとは対照的です。JavaやC++ではオブジェクトの\jaemphind{カプセルかのきょうかい}{カプセル化の境界}がクラスであるのに対し、\st ではインスタンスになっていると言えます。

\seeindex{Henn!instance}{インスタンス変数}
\seeindex{field}{インスタンス変数}
\seeindex{attribute}{インスタンス変数}
\seeindex{slot}{インスタンス変数}
\index{インスタンスへんすう@インスタンス変数}

\st では、``\ind{アクセサ}メソッド''を定義しない限り、同じクラスの二つのインスタンスはお互いのインスタンス変数にアクセスすることができません。他のオブジェクトのインスタンス変数に直接アクセスするための言語構文がないのです。
(実際には\ind{リフレクション}という仕組みによって、他のオブジェクトからインスタンス変数の値にアクセスすることもできます。こうしたメタプログラミングによって、他のオブジェクトの中身を見ることを目的にしたツールであるオブジェクト\ind{インスペクタ}のようなツールを記述することができます。)

インスタンス変数は、その変数を定義しているクラスやサブクラスのインスタンスメソッドから、名前でアクセスすることができます。このことから、\st のインスタンス変数はC++やJavaの\emph{プロテクト}変数に似ていると言えます。しかしながら、やはりプライベートと見た方が好ましいでしょう。というのも、インスタンス変数をサブクラスからアクセスすることは、\st では良くないスタイルと考えられているからです。

\subsubsection{例}
\cmind{Point}{dist:} (\mthref{dist:})メソッドは、レシーバと引数で与えた点(aPoint)との距離を計算します。メソッド内から、レシーバのインスタンス変数\ct{x}と\ct{y}に、直接アクセスしています。しかし、もう一つの点であるaPointのインスタンス変数については、\ct{x}メッセージや\ct{y}メッセージを送ってアクセスしなければなりません。

\needlines{7}
\begin{method}[dist:]{the distance between two points}
Point>>>dist: aPoint 
	"aPointとレシーバの距離を答える。"  
	| dx dy |
	dx := aPoint x - x.
	dy :=  aPoint y - y.
	^ ((dx * dx) + (dy * dy)) sqrt
\end{method}

\begin{code}{@TEST}
1@1 dist: 4@5 --> 5.0
\end{code}

クラスでのカプセル化ではなく、インスタンスでのカプセル化を選択する決め手となったのは、インスタンスでのカプセル化が同じ抽象化の異なる実装を共存させることができるからです。例えば、\ct{Point>>dist:}メソッドは、引数\ct{aPoint}がレシーバと同じクラスなのかどうか知る必要もありませんし、気にする必要もありません。引数オブジェクトは極座標かもしれませんし、データベースのレコードかもしれませんし、分散システムの他の計算機上にあるかもしれません。\ct{x}メッセージや\ct{y}メッセージに応えてくれる限り、\mthref{dist:}のコードはちゃんと動作するでしょう。

%---------------------------------------------------------
\subsection{メソッド}

すべてのメソッドは\subind{メソッド}{パブリック}です。\footnote{正確には、ほとんどすべて、です。\pharo では、\ct{pvt}で始まるセレクタを持つメソッドはプライベートで、\ct{pvt}メッセージは\self に\emph{のみ}送ることができます。しかし、\ct{pvt}メソッドはあまり使われていません。}メソッドは、その目的に応じてプロトコルにグループ分けされています。慣例により、よく使われるプロトコルというものがあります。例えば、\protind{accessing}はアクセサメソッド、\protind{initialization}はオブジェクト初期化のメソッドという具合になっています。\protind{private}プロトコルは、外から見るべきでないメソッドをグループ化するのに使われます。しかし、そういった``private''メソッドがあるからといって、実際にメッセージを送ることを防ぐものではありません。

メソッドはそのオブジェクトのすべてのインスタンス変数にアクセスできます。\st 開発者の中には、アクセサを通してのみインスタンス変数にアクセスすることを好む人もいます。このプラクティスは価値あるものですが、クラスのインターフェースを乱雑にする側面があり、下手をすると外部の世界にプライベートな状態を晒すことにもなります。

%---------------------------------------------------------
\subsection{インスタンス側とクラス側}

クラスはオブジェクトなので、クラス自身のインスタンス変数やメソッドを持つことができます。これらは\emph{クラスインスタンス変数}や\emph{クラスメソッド}と呼ばれていますが、実際のところ普通のインスタンス変数やメソッドと何も違いません。クラスインスタンス変数は単にメタクラスで定義されたインスタンス変数にすぎませんし、クラスメソッドは単に\ind{メタクラス}で定義されたメソッドにすぎません。
\index{クラス!インスタンスへんすう@インスタンス変数}
\seeindex{へんすう@変数!クラスインスタンス}{クラス, インスタンス変数}
\index{クラス!メソッド}

クラスは\ind{メタクラス}のインスタンスですが、クラスとそのメタクラスは別々のクラスです。しかし、プログラマであるあなたにはおよそ関係ない話です。プログラマが気にかけることは、オブジェクトの振る舞いやそれを生成するクラスを定義することなのです。

\begin{figure}[htb]
\begin{center}
\includegraphics[width=\textwidth]{Color-Buttons}
\caption{クラスとそのメタクラスをブラウズする。
% \sd{Do we use Key everywhere in the picture as a legend indicator?}
% \on{sure, wherever appropriate}
\figlabel{Buttons}}
\end{center}
\end{figure}

こうしたことから、ブラウザ\index{ブラウザ}は、クラスとメタクラスの両方を、一つのものの二つの側面として見ることができるようになっています。\figref{Buttons}に示されている、``\jasubind{インスタンスがわ}{ブラウザ}{インスタンス側}''と\jasubind{クラスがわ}{ブラウザ}{クラス側}''です。図は\ct{Color}クラスをブラウズしているところですが、\button{instance}ボタンをクリックすると、\ct{Color}のインスタンス(例えば青色)にメッセージが送られたときに実行されるメソッド一覧が表示されます。\button{class}ボタンを押すと、\ct{Color class}クラスのブラウズになり、\ct{Color}クラス自体にメッセージが送られたときに実行されるメソッドを見ることになります。例えば、\ct{Color blue}は\clsind{Color}クラスに\ct{blue}メッセージを送ります。ということは、この\ct{blue}メッセージは\ct{Color}のクラス側に定義されているはずです。インスタンス側ではありません。
\seeindex{class side}{browser!class side}
\seeindex{instance side}{browser!instance side}

\needlines{5}
\begin{code}{@TEST | aColor |}
aColor := Color blue.               "クラス側のblueメソッド"
aColor        --> Color blue
aColor red  --> 0.0         "インスタンス側のアクセサメソッドred"
aColor blue --> 1.0        "インスタンス側のアクセサメソッドblue"
\end{code}

新たなクラスの定義は、\jasubind{インスタンスがわ}{ブラウザ}{インスタンス側}で提供されるテンプレートを埋めることで行えます。このテンプレートをアクセプトすると、システムはクラスだけでなく、対応するメタクラスも生成します。\button{class}ボタンをクリックするとメタクラスをブラウズできます。メタクラス生成テンプレートで編集するのは、クラスインスタンス変数のリストだけです。

いったんクラスが生成されると、\button{instance}ボタンのクリックで、そのクラス(およびそのサブクラス)のインスタンス用のメソッドを編集したりブラウズしたりできます。例えば、\figref{Buttons}を見ると、\ct{Color}クラスのインスタンスに\ct{hue}メソッドが定義されています。対して、\button{class}ボタンを押すと、メタクラス(この場合、\ct{Color class})のメソッドのブラウズ、定義に切り替わります

%---------------------------------------------------------
\subsection{クラスメソッド} 

クラスメソッドは非常に便利なものです。\ct{Color class}をブラウズすると良い例が見つかります。\subind{クラス}{メソッド}には大きく分けて二種類あります。\cmind{Color class}{blue}のように、そのクラスのインスタンスを作るものと、\cmind{Color class}{showColorCube}のように、ユーティリティ的な機能を提供するものです。他の用途に使われているクラスメソッドを見かけることもあるでしょうが、この二つが典型的な用例です。

\jasubind{クラスがわ}{ブラウザ}{クラス側}にユーティリティ的なメソッドを置くと、余計なインスタンスを生成することなく実行できるので便利です。実際、そういったユーティリティメソッドの多くは、簡単に実行できるコメントを含んでいます。

\dothis{ \ct{Color class>>>showColorCube}メソッドをブラウズして, コメント\ct{"Color showColorCube"}の引用符の内側をダブルクリックして、\short{d}を押してみてください。}

このメソッドが実行されたことがわかるでしょう。(元に戻すには \menu{World \go \ind{restore display}~(r)}を実行してください。)

\ind{Java}や\ind{C++}に慣れている人には、クラスメソッドは静的メソッドに似ているように見えるかもしれません。
しかし、\st の一貫性の点で、いくぶんの違いがあります。Javaの静的メソッドは単に静的に名前解決される手続きに過ぎないのですが、\st のクラスメソッドは動的に探索されるメソッドです。つまり、\st ではクラスメソッドを継承したりオーバーライドしたりsuperに送信したりできますが、Javaの静的メソッドではできません。

%---------------------------------------------------------
\subsection{クラスインスタンス変数}
通常のインスタンス変数では、あるクラスのインスタンスは全て同じ名前のインスタンス変数を持っていて、サブクラスのインスタンスもそれらの名前を継承します。しかし、各インスタンスのインスタンス変数はそれぞれ個別の値を持ちます。\jasubind{インスタンスへんすう}{クラス}{インスタンス変数}{}もまったく同様です。各クラスはそれぞれ個別のクラスインスタンス変数を持っています。サブクラスはクラスインスタンス変数を継承しますが、\emph{サブクラスはプライベートなコピーとしてクラスインスタンス変数を保持することになります。} オブジェクトがインスタンス変数を共有しないのと同様に、クラスとサブクラスもクラスインスタンス変数を共有しません。

あるクラスが生成したインスタンスの数を管理するために\ct{count}というクラスインスタンス変数を使うことはできますが、そのサブクラスはどれも自分自身の\ct{count}変数を持つため、サブクラスによって生成されるインスタンスの数は別々に数えられることになります。

\paragraph{例: クラスインスタンス変数はサブクラスと共有されない。}
二つのクラス\ct{Dog}(犬)と\ct{Hyena}(ハイエナ)を定義します。\ct{Hyena}は\ct{Dog}からクラスインスタンス変数\ct{count}を継承します。

\begin{classdef}[dog]{犬クラスとハイエナクラス}
Object subclass: #Dog
	instanceVariableNames: ''
	classVariableNames: ''
	poolDictionaries: ''
	category: 'PBE-CIV'

Dog class
	instanceVariableNames: 'count'

Dog subclass: #Hyena
	instanceVariableNames: ''
	classVariableNames: ''
	poolDictionaries: ''
	category: 'PBE-CIV'
\end{classdef}

\ct{Dog}の\ct{count}を\ct{0}に初期化するクラスメソッドと、新しいインスタンスが生成されたらそれに1を足すクラスメソッドを定義しましょう:

\begin{method}[dogcount]{新しい犬の数を数える}
Dog class>>>initialize
	super initialize.
	count := 0.

Dog class>>>new
	count := count +1.
	^ super new

Dog class>>>count
	^ count
\end{method}

これで、新しい\ct{Dogのインスタンス}(犬)が生成されるたびに、countに1が足されていきます。\ct{Hyena}(ハイエナ)も同様ですが、別々に数えられていきます:
\begin{code}{}
Dog initialize.
Hyena initialize.
Dog count     --> 0
Hyena count --> 0
Dog new.
Dog count     --> 1
Dog new.
Dog count     --> 2
Hyena new.
Hyena count --> 1
\end{code}
% ON: In order to make this a test, I need the previous code to be part of the setup. Bleh.

インスタンス変数がインスタンスごとにプライベートなのと同様に、クラスインスタンス変数もクラスごとにプライベートだということにも注意してください。クラスとそのインスタンスは別々のオブジェクトなので、以下のことが言えます。

\important{クラスはそのインスタンスのインスタンス変数にアクセスできない。}
\important{インスタンスは、そのクラスのクラスインスタンス変数にアクセスできない。}
このことから、インスタンスを初期化するメソッドは常に\jasubind{インスタンスがわ}{ブラウザ}{インスタンス側}で定義しなければなりません --- \jasubind{クラスがわ}{ブラウザ}{クラス側}からはインスタンス変数にアクセスできないので、初期化できないのです。クラスができることは、インスタンスを新しく生成するときに、アクセッサなどを用いて\jaind{しょきか}{初期化}用のメッセージを送ることだけです。

同様にインスタンスがクラスインスタンス変数にアクセスするには、クラスにアクセサメッセージを送り間接的に行うしかありません。

\ind{Java}にはクラスインスタンス変数に相当するものはありません。Javaと\ind{C++}の静的変数は、\st ではどちらかというと\secref{classVars}で説明するクラス変数に似ています。すべてのサブクラスとインスタンスが同一の静的変数を共有するというものだからです。

\paragraph{例: シングルトンを定義する。}
\ind{シングルトンパターン}~\cite{Alpe98a}はクラスインスタンス変数とクラスメソッドを使う典型例です。 ct{WebServer}クラスを実装してみましょう。シングルトンパターンを使い、\ct{WebServer}クラスには一つしかインスタンスができないようにします。

ブラウザの\button{instance}ボタンをクリックして、\clsind{WebServer}クラスを以下のように定義します。(\clsref{singleton})

\begin{classdef}[singleton]{シングルトンクラス}
Object subclass: #WebServer
	instanceVariableNames: 'sessions' 	
	classVariableNames: '' 	
	poolDictionaries: '' 	
	category: 'Web'
\end{classdef}

そして、\button{class}ボタンをクリックして、\jasubind{クラスがわ}{ブラウザ}{クラス側}でクラスインスタンス変数\ct{uniqueInstance}を追加します。

\begin{classdef}[webserver]{シングルトンクラスのクラス側}
WebServer class 	
	instanceVariableNames: 'uniqueInstance'
\end{classdef}

この結果、\ct{WebServer}クラスは、新たなインスタンス変数を追加で持つことになります。\ct{superclass}や\ct{methodDict}といった変数もスーパークラスから継承しているからです。

さて、これで\subind{クラス}{メソッド}\ct{uniqueInstance}を定義できます。\mthref{uniqueInstance}を見てください。このメソッドではまず、\ct{uniqueInstance}が初期化されているかをチェックします。初期化されていない場合は、インスタンスを生成し、クラスインスタンス変数\ct{uniqueInstance}に代入します。最後に、\ct{uniqueInstance}の値を返しています。\ct{uniqueInstance}はクラスインスタンス変数なので、このクラスメソッド内で直接アクセスすることができます。
    
\begin{method}[uniqueInstance]{uniqueInstance (クラス側)}
WebServer class>>>uniqueInstance
     uniqueInstance ifNil: [uniqueInstance := self new].
     ^ uniqueInstance
\end{method}

\ct{Webserver uniqueInstance}が最初に実行されるときには、\ct{WebServer}クラスのインスタンスが生成され、変数\ct{uniqueInstance}に代入されます。以後は、新しいインスタンスを生成することなしに、前回生成されたインスタンスが返されます。

\mthref{uniqueInstance}の条件分岐の内側にあるインスタンス生成のコードが\ct{self new}という形であって、\ct{WebServer new}ではないことに留意してください。何が違うのでしょうか?
\ct{uniqueInstance}メソッドは\ct{WebServer class}に定義されているので、同じことだと思うかもしれません。
\ct{WebServer}のサブクラスが作られるまでは確かに同じなのです。しかし、もし\lct{WebServer}のサブクラスとして\ct{ReliableWebServer}が作られ、\ct{uniqueInstance}メソッドを継承していたらどうなるでしょうか。普通は\ct{ReliableWebServer uniqueInstance}の結果は\lct{ReliableWebServer}のインスタンスとなってほしいはずです。こうしたときに、\self を使うことで、そのようにすることができます。\self はメッセージを受け取るクラス自身だからです。なお、\ct{WebServer}と{ReliableWebServer}はそれぞれ自分のクラスインスタンス変数\ct{uniqueInstance}を持っていることにも留意してください。それぞれの変数は、もちろん、別々の値になります。

%=========================================================
\section{すべてのクラスにはスーパークラスがある}

%\ruleref{inheritance}

\st では、クラスは単一の\emphind{スーパークラス}から、振る舞いおよび構造の記述を継承します。
つまり\st は単一\jaind{けいしょう}{継承}です。

\needlines{2}
\begin{code}{@TEST}
SmallInteger superclass --> Integer
Integer superclass          --> Number
Number superclass        --> Magnitude
Magnitude superclass    --> Object
Object superclass           --> ProtoObject
ProtoObject superclass  --> nil
\end{code}

伝統的には、\st の継承階層のルートは\clsind{Object}クラスです。(なぜなら、すべてのものはオブジェクトだからです。)
\pharo では、ルートは\clsind{ProtoObject}と呼ばれるクラスになっています。しかし、通常はこのクラスを気にする必要はありません。 \ct{ProtoObject}はすべてのオブジェクトが持って\emph{いなければならない}最低限のメッセージをカプセル化しています。しかしながら、たいていのクラスは\ct{Object}からの継承となっています。\ct{Object}はほとんどすべてのオブジェクトが理解し対応すべきメッセージを追加でたくさん定義しています。特別な理由がない限り、アプリケーションのクラスを生成する際には、\ct{Object}や、そのサブクラスを継承すべきです。

\dothis{新しいクラスを生成するには、既存のクラスに \ct{subclass: instanceVariableNames: ...} というメッセージを送ります。他にもいくつかクラスを生成するためのメソッドがあります。\prot{Kernel-Classes \go Class \go subclass creation}プロトコルを見て、どのようなメソッドがあるか見てみてください。}
\scatindex{Kernel-Classes}
\protindex{creation}

%There is no special syntax for creating abstract classes in \st.
%An abstract class is an ordinary class in which the implementation of some methods is deferred to a subclass.
%This is repeated in the next section

\pharo には多重継承はありませんが、お互いに関係のないクラスで同じ振る舞いを共有するための\emphind{トレイト}{}というメカニズムをサポートしています。トレイトはメソッドの集合で、継承関係にないクラス間で再利用されます。トレイトを使うと、同じコードを繰り返し定義することなく、別々のクラス間で共有することができます。

%---------------------------------------------------------
\subsection{抽象メソッドと抽象クラス}

\jasubind{ちゅうしょう}{クラス}{抽象}クラスは、インスタンスを生成するためではなく、サブクラスを定義するために存在するクラスです。抽象クラスは通常は不完全なもので、メソッドの中身すべてが実装されてはいるわけではありません。``空の''メソッド---他のメソッドがそのメソッドがあることを想定しているのに、実装が定義されていないメソッド---を\jasubind{ちゅうしょう}{メソッド}{抽象}メソッドと呼びます。
\seeindex{ちゅうしょうクラス@抽象クラス}{クラス, 抽象}
\seeindex{ちゅうしょうメソッド@抽象メソッド}{メソッド, 抽象}

\st には、抽象メソッドや抽象クラスを示す特別な構文はありません。コード規約として、抽象メソッドの本文に\mbox{\ct{self subclassResponsibility}}と書きます。これはいわゆる``マーカーメソッド''で、サブクラスでそのメソッドの具体的な実装を定義する責任があることを示しています。\ct{self subclassResponsibility}を書いたメソッドは必ずオーバーライドしなければならず、直接実行してはならないものです。もしオーバーライドし忘れたら、メソッド実行の結果、例外が発生してしまいます。
\cmindex{Object}{subclassResponsibility}

抽象メソッドを持つクラスが抽象クラスです。実際には抽象クラスのインスタンスを生成しないようにする仕組みはありません。抽象メソッドが実行されるまでは、普通に動作します。

\subsubsection{例: \ct{Magnitude}クラス}
\clsind{Magnitude}は互いを比較するクラスを定義するのに役立つ抽象クラスです。\ct{Magnitude}のサブクラスは\ct{<}および\ct{=}、\ct{hash}の3つのメソッドを定義しなければなりません。\ct{Magnitude}では、それらのメッセージを使って\ct{>}および\ct{>=}、\ct{<=}、\ct{max:}、\ct{min:}、\ct{between:and:}等のメソッドを定義しています。こうしたメソッドはサブクラスで継承されて使われます。抽象メソッド\mthind{Magnitude}{<}は、\mthref{MagnitudeLessThan}のようになっています。

\begin{method}[MagnitudeLessThan]{\ct{Magnitude>>><}}
Magnitude>>>< aMagnitude 
	"レシーバが引数より小さいかどうかを答える。"
	^self subclassResponsibility
\end{method}

\noindent
対照的に、\mthind{Magnitude}{>=}は具象メソッドで、\ct{<}を使って定義されています。

\begin{method}[Magnitude>=]{\ct{Magnitude>>>>=}}
>= aMagnitude 
	"レシーバが引数と等しいかそれ以上かどうかを答える。"
	^(self < aMagnitude) not
\end{method}
他の比較メソッドも同様です。

\clsind{Character}は\ct{Magnitude}のサブクラスで、\mthind{Object}{subclassResponsibility}となっていた\ct{<}メソッドをオーバーライドし、メソッドを再定義しています(\mthref{CharacterLessThan}参照)。\ct{Character}は\ct{=}メソッドおよび\ct{hash}メソッドも定義していて、\ct{>=}および\ct{<=}、\ct{~=}等のメソッドを\ct{Magnitude}から継承しています。

\begin{method}[CharacterLessThan]{\ct{Character>>><}}
Character>>>< aCharacter 
	"レシーバがaCharacterより小さければ、trueと答える。"
	^self asciiValue < aCharacter asciiValue
\end{method}

%---------------------------------------------------------
\subsection{トレイト}
\emphind{トレイト}は、継承なしにクラスの振る舞いに取り込むことができるメソッドの集合です。これによって、単一継承の容易さを保ちながら、複数のクラスにとって有用なメソッドを共通化することができます。

新しいトレイトを定義する方法は、単にサブクラス定義のテンプレートを、\clsind{Trait}用のメッセージに置き換えるだけです。

\needspace{5\baselineskip}
\begin{classdef}[tauthor]{新しいトレイトを定義する}
Trait named: #TAuthor
	uses: { }
	category: 'PBE-LightsOut'
\end{classdef}

\noindent
ここでは、\scat{PBE-LightsOut}カテゴリに\ct{TAuthor}トレイトを定義しています。このトレイトでは既存のトレイトを\emph{使って}いません。一般には、\ct{uses:}の引数として、他のトレイトの\emph{トレイト合成式}を使うことができます。ここでは単に空の配列にしています。

トレイトはメソッドを持つことができますが、インスタンス変数は持ちません。例として\ct{author}メソッドをクラス階層のどこに位置するかに関係なくいろいろなクラスに追加することを考えてみましょう。以下のように書きます。

\begin{method}[author]{authorメソッド}
TAuthor>>>author
    "著者の頭文字を返す。"
	^ 'on'    "oscar nierstraszの頭文字"
\end{method}

\noindent
クラスに既存のスーパークラスがあっても、このトレイトを使うことができます。\charef{firstApp}で定義した\ct{LOGame}クラスを例にします。単に、\ct{LOGame}クラスを定義するテンプレートを書き変え、\ct{uses:}の引数で\ct{TAuthor}を使うように指定するだけです。

\begin{classdef}[sbegamewithtrait]{トレイトを使う}
BorderedMorph subclass: #LOGame
	uses: TAuthor
	instanceVariableNames: 'cells'
	classVariableNames: ''
	poolDictionaries: ''
	category: 'PBE-LightsOut'
\end{classdef}

\ct{LOGame}のインスタンスを生成すると、期待通り\ct{author}メッセージに答えてくれるようになります。

\begin{code}{}
LOGame new author --> 'on'
\end{code}

トレイト合成式では複数のトレイトを\ct{+}演算子でつなげることができます。もし複数のトレイトで同じ名前のメソッドを定義しているなどの不整合があっても、\ct{-}演算子でそうしたメソッドを明示的に除外したり、新規のクラスやトレイト側で再定義したりすることで、解消することができます。\ct{@}演算子でメソッドに\emph{エイリアス}として別名を付けることも可能です。

トレイトはシステムの中核部分で使われています。\mbox{\clsind{Behavior}}クラスが良い例です。

\needlines{8}
\begin{classdef}[behaviorwithtraits]{\ct{Behavior}の定義で使われているトレイト}
Object subclass: #Behavior
	uses: TPureBehavior @ {#basicAddTraitSelector:withMethod:->#addTraitSelector:withMethod:}
	instanceVariableNames: 'superclass methodDict format'
	classVariableNames: 'ObsoleteSubclasses'
	poolDictionaries: ''
	category: 'Kernel-Classes'
\end{classdef}
\noindent
ここでは、\ct{TPureBehavior}の\ct{addTraitSelector:withMethod:}メソッドに、\ct{basicAddTraitSelector:withMethod:}として別名を付けています。
現在、トレイトをサポートする機能がブラウザに追加されているところです。

%PBE.image(1.0)のソースを見ても上記のようにはなっていない

%=========================================================
\section{すべてはメッセージ送信で起こる}

%\ruleref{message}

この規則は、\st プログラミングの本質を捉えています。

手続き的プログラミングでは、手続きが呼ばれたときにどのコードが使われるかは、呼び出し側が決めます。呼び出し側が、名前で、\emph{静的に}、手続きや関数を選んで実行します。

オブジェクト指向プログラミングでは、``メソッドを呼び出し''\emph{ません}。``メッセージを\jasubind{そうしん}{メッセージ}{送信}''します。この用語の違いは重要です。オブジェクトにはそれぞれの責任があります。オブジェクトが何をすべきか、その手続き\emph{指示する}ようなことはしません。そうではなく、オブジェクトにメッセージを送ることで、何かをするように\emph{お願い}するのです。メッセージはコードでは\emph{ありません}。単なる名前と引数の組なのです。メッセージを受けたレシーバは、頼まれたことを実行するため、\emph{メソッド}を自ら選択して、どう対応するのかを決めます。オブジェクトが異なれば、同じメッセージに対して反応するためのメソッドも異なっていてよいのです。つまりメソッドはメッセージを受け取ったときに\emph{動的に}選択されることになります。

\begin{code}{@TEST}
3 + 4         --> 7          "整数「3」にメッセージ「+」と引数「4」を送信する"
(1@2) + 4 --> 5@6    "点「(1@2)」にメッセージ「+」と引数「4」を送信する"
\end{code}

\noindent
結果として\emph{同じメッセージ}を異なるオブジェクトに送信することができ、オブジェクトはそのメッセージに応えるために\emph{各自でメソッド}を持つことができます。\ct{+ 4}というメッセージにどう反応するのかを\ct{SmallInteger} \ct{3}や\ct{Point} \ct{1@2}に直接指示しないのです。それぞれが\ct{+}メソッドを持っていて、そのメソッドに従って\ct{+ 4}に反応するのです。

\st のメッセージ送信モデルの帰結として言えることがあります。オブジェクトは非常に小さなメソッドのみを持ち、他のオブジェクトに仕事を移譲する傾向となり、巨大で手続き的なメソッドを保持し、過大な責任をかかえることを避けるようになります。ジョセフ・ペルリン(Joseph Pelrine)はこの原則を次のように簡潔に表現しています。
\ab{Citation?}
\on{sorry, just personal communication and my own lecture notes!}
\important{誰かに任せられる仕事を自分でやらないこと。}
\index{Pelrine, Joseph}

多くのオブジェクト指向言語は、オブジェクトに対して静的・動的両方の操作方法を提供していますが、\st には動的なメッセージ送信しかありません。静的な操作の代わりに、例えばクラスがオブジェクトとなっており、単にクラスにメッセージを送ればよいのです。

\st では、\emph{ほとんど}すべてのことはメッセージ送信で起こります。しかし時にはメッセージ送信ではない処理も発生します。

\begin{itemize}
  \item \emph{変数宣言}はメッセージ送信ではありません。
		実際、変数\jasubind{せんげん}{へんすう@変数}{宣言}{}は実行可能ですらありません。
		変数宣言は、オブジェクトの参照を格納する領域を確保する意味しか持ちません。
  \item \emph{代入}はメッセージ送信ではありません。
		変数への\jaind{だいにゅう}{代入}はそれを定義したスコープ内で名前を束縛します。
  \item \emph{リターン}はメッセージ送信ではありません。
		\ind{リターン}は計算結果をメッセージの送り手に返します。
  \item \emph{プリミティブ}はメッセージ送信ではありません。
		プリミティブは\jaind{かそうマシン}{仮想マシン}側で実装されています。
		\index{プリミティブ}
\end{itemize}

これらのわずかな例外を除き、ほとんどすべてはメッセージ送信で起こります。特に、\st には``パブリックなフィールド''はないので、他のオブジェクトの\jaind{インスタンスへんすう}{インスタンス変数}の値を変えるには、メッセージを送ってそのオブジェクト自身のフィールドを更新するようお願いするしかありません。もちろん、すべてのインスタンス変数についてセッターやゲッターを定義するのは、オブジェクト指向として良いスタイルとは言えません。ジョセフ・ペルリンは次のようにも言っています。

\important{自身のデータを他人にいじらせないこと。}


%=========================================================
\section{メソッド探索は継承の連鎖をたどっていく}

%\ruleref{lookup}

オブジェクトがメッセージを受け取ると、何が起こるのでしょう?

その過程はとてもシンプルです。メッセージを受けたレシーバのクラスは、メッセージを処理するのに必要なメソッドを探索します。もしクラスがそのメソッドを持っていなければ、自身の\ind{スーパークラス}に尋ねます。\jaind{けいしょう}{継承}の連鎖をたどって、メソッドを探していくのです。メソッドが見つかれば、渡した引数の値はそのメソッドの引数に束縛され、\jaind{かそうマシン}{仮想マシン}がメソッドを実行します。
\index{メソッド!たんさく@探索}

メソッド探索は基本的には簡単なことなのですが、いくつか注意しなければならない点もあります。

\begin{itemize}
  \item \emph{メソッドが明示的に値を返していない場合は、どうなるのでしょうか?}
  \item \emph{あるクラスがスーパークラスのメソッドを再実装している場合は、どうなるのでしょうか?}
  \item \emph{\pvind{self}送信と\pvind{super}送信は何が違うのでしょうか?}
  \item \emph{メソッドが見つからない場合には、どうなるのでしょうか?}
\end{itemize}

ここでは、メソッド探索の概念的な規則を示します。仮想マシンを実装する人はありとあらゆるトリックや最適化を駆使してメソッド探索を高速化するのですが、それは彼らに任せておきましょう。ここで示す規則から外れているとはわからないように、やっているはずです。

まず、探索の基本戦略から見ていきましょう。その上で、上記の一連の疑問について考察します。

%---------------------------------------------------------
\subsection{メソッド探索}
\ct{EllipseMorph}のインスタンスを生成してみます。
\begin{code}{@TEST | anEllipse |}
anEllipse := EllipseMorph new.
\end{code}
\noindent
ここで、このオブジェクトに\ct{defaultColor}メッセージを送ると、\ct{Color yellow}という結果が得られます。
\begin{code}{@TEST | anEllipse | anEllipse := EllipseMorph new.}
anEllipse defaultColor --> Color yellow
\end{code}
\noindent
\ct{EllipseMorph}クラスは\ct{defaultColor}を実装しているので、適切なメソッドが直ちに見つかります。

\begin{method}[defaultColor]{レシーバのクラス自体に実装されたメソッド}
EllipseMorph>>>defaultColor
	"レシーバのデフォルトの色/塗り方を答える"
	^ Color yellow
\end{method}
\cmindex{EllipseMorph}{defaultColor}

一方、\ct{anEllipse}に\ct{openInWorld}メッセージを送ると、メソッドはすぐには見つかりません。\ct{anEllipse}クラスは\ct{openInWorld}を実装していないからです。そのため、スーパークラスの\lct{BorderedMorph}を探索していきます。探索は\ct{openInWorld}メソッドが\ct{Morph}クラスで見つかるまで続きます。

\begin{method}[openInWorld]{継承されたメソッド}
Morph>>>openInWorld
	"このモーフをワールドに追加する。"

	self openInWorld: self currentWorld
\end{method}
\cmindex{Morph}{openInWorld}

\begin{figure}[htb]
\begin{center}
	{\includegraphics[width=0.8\textwidth]{openInWorldLookup}}
\caption{メソッド探索は継承の連鎖をたどっていく\figlabel{openInWorldLookup}}
\end{center}
\end{figure}

%---------------------------------------------------------
\subsection{selfを返す} 

\ct{EllipseMorph>>>defaultColor} (\mthref{defaultColor})は\ct{Color yellow}を明示的に返している一方で、\ct{Morph>>>openInWorld} (\mthref{openInWorld})には戻り値を示すものがないことに注目してください。

実際にはメソッドは\emph{常に}値を返します。もちろん、値というのはオブジェクトです。答える値はメソッド中の\ct{^}で決まることもありますが、\ct{^}を実行することなくメソッドの終わりまで来てしまったとしても、やはりメソッドは値を返します。その場合、メソッドは、そのメッセージを受け取ったオブジェクト自身を返すのです。これを、``\self を返す''と言います。Smalltalkでは擬似変数\self はメッセージを受けたレシーバそのものを指します。\ind{Java}の\ct{this}のようなものです。
\index{へんすう@変数!ぎじ@疑似}
\index{return}
\seeindex{caret}{return}

つまり、\mthref{openInWorld}は\mthref{openInWorldReturnSelf}と同じことです。

\needlines{5}
\begin{method}[openInWorldReturnSelf]{明示的にselfを返す}
Morph>>>openInWorld
	"このモーフをワールドに追加する。"
	
    self openInWorld: self currentWorld
	^ self		"不要なので、通常はわざわざ書かないこと!"
\end{method}

なぜ上のように\ct{^ self}と書かない方がよいのかというと、明示的に値を返す書き方は、メッセージを送った側に何か意味あるものを返していることを示すものだからです。\self を明示的に返す場合、センダ側がその戻り値を使うことを期待しているといった意味になるのです。上記のコードではそういう意図はないので、\self を返すことを明示しない方がよいでしょう。

これは\st では日常的なイディオムで、ケント・ベックは``興味を引くような戻り値(Interesting return value)'' \cite{Beck97a} と呼んでいます。
\index{Beck, Kent}

\important{センダにその値を使ってほしいときにだけ、戻り値を示すこと。}

%---------------------------------------------------------
\subsection{オーバーライドと拡張}

\figref{openInWorldLookup}の\ct{EllipseMorph}クラスの階層を見ると、\ct{Morph}クラスと\mbox{\ct{EllipseMorph}}クラスのどちらも\ct{defaultColor}を実装しています。実際、\ct{Morph new openInWorld}で新しいモーフを開くと、青い色になっています。楕円の場合には黄色になります。
\index{メソッド!オーバーライド}
\index{メソッド!かくちょう@拡張}
\seeindex{オーバーライド}{メソッド, オーバーライド}
\seeindex{かくちょう@拡張}{メソッド, 拡張}

これを、\ct{EllipseMorph}が\ct{Morph}から継承した\ct{defaultColor}メソッドを\emph{オーバーライド}している、と言います。継承したメソッドは、\ct{anEllipse}からは見えなくなります。

継承したメソッドをオーバーライドするのではなく、機能を追加して\emph{拡張}したいこともあります。つまり、オーバーライドされるメソッドに、サブクラスで新たな機能を\emph{付け加えて}実行したいということです。\st では、\super に対するメッセージ送信でこれを実現しています。単一継承をサポートする多くのオブジェクト指向言語で典型的な方法です。

\ct{initialize}メソッドは、この仕組みの最も重要な例です。あるクラスの新しいインスタンスを初期化するときには、スーパークラスから継承したインスタンス変数も必ず初期化しなければなりません。しかし、その初期の仕方は、継承の連鎖の中で各スーパークラスの\ct{initialize}メソッドに既に記述されています。サブクラスは継承したインスタンス変数の初期化には本来関わりがありません。

以上のことから、\ct{initialize}メソッドを実装するときには、まず\ct{super initialize}を送信してから、他の変数の\jaind{しょきか}{初期化}を書くようにするのが良いプラクティスです。

\index{super!initialize}

\needlines{6}
\begin{method}[morphinit]{Super initialize}
BorderedMorph>>>initialize
	"レシーバの状態を初期化する"
	super initialize.
	self borderInitialize
\end{method}

\important{ \ct{initialize}メソッドは、\ct{super initialize}で始めること。}

%---------------------------------------------------------
\subsection{self、superへのメッセージ送信}

\super\jasubind{へのメッセージそうしん}{super}{へのメッセージ送信}{}は、オーバーライドされてしまう振る舞いをサブクラス側で組み入れるのに使われます。継承されたものであろうとなかろうと、複数のメソッドを組み合わせるには、通常\self\jasubind{へのメッセージそうしん}{self}{へのメッセージ送信}{}を用います。

\self 送信と\super 送信の違いは何でしょうか? \self と同様に、\super はメッセージを受けたレシーバ自身を表します。違うのは、\jaind{メソッドたんさく}{メソッド探索}だけです。\super 送信があると、レシーバのクラスから探索を始めるのではなく、superへの送信を行っているメソッドのクラスのスーパークラスから探索を始めます。

\super はスーパークラス\emph{ではない}ことに注意してください! スーパークラスと思うのは、よく見られる勘違いです。また、レシーバのスーパークラスから探索を始めるというのも、間違いです。どういうことか、以下に例を示します。

\ct{constructorString}というメッセージを例にしましょう。これは、任意のモーフに送信することができます。
\begin{code}{@TEST | anEllipse | anEllipse := EllipseMorph new.}
anEllipse constructorString --> '(EllipseMorph newBounds: (0@0 corner: 50@40) color: Color yellow)'
\end{code}
戻り値は、評価することでモーフを復元できる文字列です。

\self 送信と\super 送信をどう組み合わせれば、この文字列が得られるのでしょうか? まず、\ct{anEllipse constructorString}が実行されると、\ct{Morph}クラスの\ct{constructorString}が見つかります。\figref{constructorStringLookup}を見てください。

\begin{figure}[htb]
\begin{center}
\ifluluelse
	{\includegraphics[width=\textwidth]{constructorStringLookup}}
	{\includegraphics[width=0.8\textwidth]{constructorStringLookup}}
\caption{\self 送信と\super 送信\figlabel{constructorStringLookup}}
\end{center}
\end{figure}

\needlines{2}
\begin{method}[constructorString]{\self 送信}
Morph>>>constructorString
	^ String streamContents: [:s | self printConstructorOn: s indent: 0].
\end{method}
\cmind{Morph}{constructorString}メソッドは\lct{printConstructorOn:indent:}を\self に送っています。このメッセージもまた探索されますが、探索は\lct{EllipseMorph}から始まって、\ct{Morph}で見つかります。そのメソッドの内部で、さらに\lct{printConstructorOn:indent:nodeDict:}が\self に送られます。それが\ct{fullPrintOn:}を\self に送ります。\ct{fullPrintOn:}は\ct{EllipseMorph}から探索を始め、\mthind{BorderedMorph}{fullPrintOn:}が\ct{BorderedMorph}で見つかります。(もう一度、\figref{constructorStringLookup}を見てください)。重要なのは、\self への送信ではレシーバのクラス、この場合には\ct{anEllipse}のクラスから、メソッド探索が始まるということです。

\important{\self 送信は、\emph{動的な}メソッド探索を、レシーバのクラスから開始します。}

\needlines{4}
\begin{method}[fullPrintOn]{\super 送信と\self 送信を組み合わせる}
BorderedMorph>>>fullPrintOn: aStream
	aStream nextPutAll: '('.
	!\textbf{super fullPrintOn: aStream.}!
	aStream nextPutAll: ') setBorderWidth: '; print: borderWidth;
		nextPutAll: ' borderColor: ' , (self colorString: borderColor)
\end{method}
一方、\ct{BorderedMorph>>>fullPrintOn:}では、\super へメッセージ送信をして、スーパークラスから継承した\ct{fullPrintOn:}の振る舞いを拡張しています。\super への送信なので、メソッド探索は\super 送信を行うクラスのスーパークラスである\ct{Morph}クラスから始まります。すると、\ct{Morph>>>fullPrintOn:}が即座に見つかるので、それが評価されます。

\super 送信の探索はレシーバのスーパークラスから開始しているわけではない、ということに注意してください。もしレシーバのスーパークラスから探索を開始すると、\ct{BorderedMorph}から探索を始めるということになり、無限ループになってしまいます!

\important{\super 送信は、\emph{静的な}メソッド探索を、\super 送信を行っているメソッドを定義しているクラスのスーパークラスから開始します。}

\super 送信の解説を踏まえ、\figref{constructorStringLookup}を見ると、\super は静的束縛だということがわかります。\super 送信をしているコードがどのクラスにあるかにより、すべて決まります。対照的に、\self 送信は動的です。今実行しているメッセージのレシーバを表しています。つまり、\self に送られた\emph{すべての}メッセージは、そのレシーバのクラスから探索を始めます。

%---------------------------------------------------------
\subsection{理解されないメッセージ}

探索してもメソッドが見つからなかったら、どうなるのでしょうか?
\index{メッセージ!りかいされない@理解されない}

楕円に\ct{foo}メッセージを送ってみましょう。まず、\ct{foo}メソッドを探して\clsind{Object}(正確には\clsind{ProtoObject})まで継承の連鎖を一通りたどる、通常の探索が開始されます。メソッドが見つからない場合、\jaind{かそうマシン}{仮想マシン}からそのオブジェクトに\ct{self doesNotUnderstand: #foo}メッセージが送信されます。(\figref{fooNotFound}を参照)

\begin{figure}[htb]
\begin{center}
\ifluluelse
	{\includegraphics[width=\textwidth]{fooNotFound}}
	{\includegraphics[width=0.8\textwidth]{fooNotFound}}
\caption{Message \lct{foo} is not understood\figlabel{fooNotFound}}
\end{center}
\end{figure}

これも通常の動的なメッセージ送信なので、また\ct{EllipseMorph}から探索が始まります。ただし、今度は\ct{doesNotUnderstand:}メソッドを探します。結局のところ、\ct{Object}が\ct{doesNotUnderstand:}を実装しています。
このメソッドでは、\ct{MessageNotUnderstood}オブジェクトが生成されます。\ct{MessageNotUnderstood}オブジェクトは、現在実行しているコンテキストでデバッガを立ち上げることができます。

なぜ明らかなエラーに対してこんなに複雑なことをするのかというと、こうすると開発者がエラーを捕まえて別のアクションをさせることが容易になるからです。\ct{Object}の任意のサブクラスは\mthind{Object}{doesNotUnderstand:}をオーバーライドすることで、別のやり方でエラー処理を行うことができます。

実際、あるオブジェクトから別のオブジェクトに自動的にメッセージを移譲するように簡単に実装できます。\ct{Delegator}オブジェクトとして、実装していないメッセージを別のオブジェクトに移譲して改めてメッセージを処理させたり、エラーを発生させたりすることも可能です。

%=========================================================
\section{共有変数}

ここでは\st の特徴の中で、5つの規則ではカバーされていないものを見ていきましょう。\jasubind{きょうゆう}{へんすう@変数}{共有}変数のことです。

\st には3種類の共有変数があります:(1) \empty{グローバル}な共有変数 (2) インスタンスとクラスの間で共有される変数(\emph{クラス変数}) (3) クラスのグループの間で共有される変数(\emph{プール変数}) です。これらの共有変数の名前は大文字で始まります。これは、変数が複数のオブジェクトの間で共有されているということの警告となっています。
\index{へんすう@変数!グローバル}
\index{クラス!へんすう@変数}
\index{へんすう@変数!プール}

%---------------------------------------------------------
\subsection{グローバル変数}
\pharo では、すべてのグローバル変数は\glbind{Smalltalk}という名前空間に格納されます。\glbind{Smalltalk}は\clsind{SystemDictionary}クラスのインスタンスとして実装されています。グローバル変数はどこからでもアクセス可能です。各クラスはグローバル変数で名付けられています。特別なオブジェクトや共通して使われるオブジェクトもグローバル変数で名前が付いています。

変数\glbind{Transcript}(トランスクリプト)は\clsind{TranscriptStream}のインスタンスを指しています。clsind{TranscriptStream}はスクロール付きのウィンドウに書き込むストリームです。以下のコードは、\ct{Transcript}に情報を表示して改行します。

\begin{code}{}
Transcript show: 'Pharoは楽しくてパワフル!' ; cr
\end{code}

\noindent
\menu{do it}する前に、\menu{World \go Tools \ldots \go Transcript}でトランスクリプトを開いておいてください。

\hint{トランスクリプトへの書き込みは遅いです。トランスクリプトウィンドウが開いているときはなおさらです。トランスクリプトに書き込んでいて遅いと思ったときは最小化してみるのも一つの手です。}

\subsubsection{他の便利なグローバル変数}

\begin{itemize}
\item
\ct{Smalltalk}は\ct{SystemDictionary}のインスタンスです。\ct{Smalltalk}自身を含むすべてのグローバル変数を定義してします。この辞書のキーはシンボルで、\st のコード内でグローバル変数を示すのに使われます。例えば、
\begin{code}{@TEST}
Smalltalk at: #Boolean --> Boolean
\end{code}
\ct{Smalltalk}はそれ自身がグローバル変数なので、
\begin{code}{}
Smalltalk at: #Smalltalk-->a SystemDictionary(lots of globals)
\end{code} 
さらに、
\begin{code}{@TEST}
(Smalltalk at: #Smalltalk) == Smalltalk --> true
\end{code}
となります。

\item \clsind{Sensor}は\clsind{EventSensor}のインスタンスで、\pharo への入力を表してます。例えば、\lct{Sensor keyboard}はキーボードからの次の入力を答えます。そして、\ct{Sensor leftShiftDown}は、左シフトキーが押されているときに\ct{true}と答えます。さらに、\ct{Sensor cursorPoint}は現在のマウスの位置を示す\ct{Point}を返します。

%Sensor mousePointとあったが、cursorPointの誤り (ume)

\item \glbind{World}は\clsind{PasteUpMorph}のインスタンスで、スクリーンを表します。\ct{World bounds}はスクリーン全体の範囲を示す矩形を答えます。すべてのモーフは\ct{World}のサブモーフです。
\index{Morphic}

\item 
\glbind{ActiveHand}は\clsind{HandMorph}の現在のインスタンスで、カーソルのグラフィックを表します。マウスでモーフをドラッグしている間は\ct{ActiveHand}のサブモーフになっています。

\item
\glbind{Undeclared}も辞書です。宣言されていない変数群を保持しています。宣言されていない変数を参照するメソッドを書くと、通常はブラウザが、グローバル変数やインスタンス変数として宣言するよう促します。それでも、変数の宣言を後で削除してしまった場合には、コードは未宣言の変数を参照することになります。\ct{Undeclared}をインスペクトすることで、奇妙な振る舞いの原因がわかることもあるでしょう。

\item
\glbind{SystemOrganization}は\clsind{SystemOrganizer}のインスタンスです。クラスの構成情報をまとめて保持しています。より正確に言えば、\emph{クラス名}をカテゴリに分類します。
\end{itemize}

\begin{code}{@TEST}
SystemOrganization categoryOfElement: #Magnitude --> #'Kernel-Numbers'
\end{code}

グローバル変数は極力使わないというのが現在のプラクティスとなっています。通常はクラスインスタンス変数やクラス変数を使い、クラスメソッドでアクセスする方が望ましいです。もし\pharo を一から再実装するようなことがあれば、グローバル変数は、クラス群を除いては、ほとんどシングルトンに置き換えられるでしょう。

通常、グローバル変数を定義するには、\menu{do it}で、大文字で始まる未宣言の識別子に代入するだけです。すると、パーサがグローバル変数を定義するよう促します。プログラムでグローバル変数を定義するには、\ct{Smalltalk at: #AGlobalName put: nil}を実行します。削除するには、\ct{Smalltalk removeKey: #AGlobalName}です。
\glbindex{Smalltalk}

%---------------------------------------------------------
\subsection{クラス変数}
\seclabel{classVars}

クラスおよびクラスのすべてのインスタンスで、データを共有したいときがあります。これは\emph{クラス変数}で可能です。クラス\jasubind{へんすう}{クラス}{変数}という用語は、変数のライフタイムがクラスと同じということを示しています。しかし、その用語の字面からは、クラス変数はクラスの間だけでなく、インスタンスとも共有されるということはわかりません。\figref{privateSharedVar}を見てください。実際のところ、\emph{共有変数}という名称の方が適切だったでしょう。役割がより明らかになりますし、使う際(特に変更する場合)の危険性も伝わります。

\begin{figure}[htb]
\begin{center}
\ifluluelse
	{\includegraphics[width=\textwidth]{privateSharedVarColor}}
	{\includegraphics[width=0.7\textwidth]{privateSharedVarColor}}
\caption{インスタンスメソッドとクラスメソッドが別々の変数にアクセスしている。\figlabel{privateSharedVar}}
\end{center}
\end{figure}

\figref{privateSharedVar}では、\ct{rgb}と\ct{cachedDepth}は\ct{Color}のインスタンス変数であり、そのため、\clsind{Color}のインスタンスからアクセス可能です。また、\lct{superclass}や\lct{subclass}、\lct{methodDict}等はクラスインスタンス変数なので、\lct{Color}クラスからのみアクセスできます。

新しく出てきたのは\ct{ColorNames}と\ct{CachedColormaps}の\emph{クラス変数}で、\ct{Color}に定義されています。変数が大文字であることが、共有されているというヒントを与えてくれます。実際、\ct{Color}の全インスタンスからこれらの共有変数にアクセスでき、さらに\ct{Color}クラス自身と\emph{そのすべてのサブクラスからもアクセス可能です}。インスタンスメソッドとクラスメソッドの両方からこれらの共有変数にアクセスできるのです。

%\begin{figure}
%\begin{center}\includegraphics[width=6cm]{dateToday}\caption{A date is an object that  represents only anumber of days; all the information about month names, day names, etc.\ is shared among all the instances \figlabel{dateToday}}\end{center}.
%\end{figure}

クラス\jasubind{へんすう}{クラス}{変数}はクラス定義テンプレートの中で宣言されます。例えば、\ct{Color}クラスでは大量のクラス変数を定義しており、色オブジェクトの生成を高速化しています。その定義を下図(\clsref{Color})に示します。
\needlines{5}
\begin{classdef}[Color]{Colorとクラス変数}
Object subclass: #Color 	
        instanceVariableNames: 'rgb cachedDepth cachedBitPattern'
        classVariableNames: 'Black Blue BlueShift Brown CachedColormaps ColorChart ColorNames ComponentMask ComponentMax Cyan DarkGray Gray GrayToIndexMap Green GreenShift HalfComponentMask HighLightBitmaps IndexedColors LightBlue LightBrown LightCyan LightGray LightGreen LightMagenta LightOrange LightRed LightYellow Magenta MaskingMap Orange PaleBlue PaleBuff PaleGreen PaleMagenta PaleOrange PalePeach PaleRed PaleTan PaleYellow PureBlue PureCyan PureGreen PureMagenta PureRed PureYellow RandomStream Red RedShift TranslucentPatterns Transparent VeryDarkGray VeryLightGray VeryPaleRed VeryVeryDarkGray VeryVeryLightGray White Yellow'
        poolDictionaries: '' 	
        category: 'Graphics-Primitives'
\end{classdef}

クラス変数\cvind{ColorNames}は、よく使われる色の名前を持つ配列です。この配列は\ct{Color}とそのサブクラス\clsind{TranslucentColor}の、すべてのインスタンスで共有されていて、インスタンスメソッドやクラスメソッドからアクセス可能です。% (see \figref{ClassVarAccess2}).

\ct{ColorNames}は\cmind{Color class}{initializeNames}で一度だけ初期化され、\ct{Color}のインスタンスからアクセスされます。\cmind{Color}{name}メソッドで、色の名前を探すためにこの変数が使われています。このようになっているのは、ほとんどの色は名前を持たないため、インスタンス変数\ct{name}を色ごとに持つのは不適切と考えたからです。

\subsubsection{クラスの初期化}

クラス変数があることで、どうやってそれを初期化するかが問題になります。一つの解は、\jasubind{しょきか}{クラス}{初期化}を遅延させることです。これは、クラス変数へのアクセサメソッドを実装し、その中で変数が初期化されていなければ初期化を行うように書くことで実現できます。この方法の場合、常にアクセサを使い、クラス変数には絶対に直接アクセスしないようにすることになります。また、アクセサ実行の際に、初期化の必要性をチェックするコストが上乗せされます。さらに言えば、アクセサを使った場合、変数は事実上もはや共有されていないことになるので、クラス変数を使う意味も薄れます。

\begin{method}[colorclasscolornames]{Color class>>colorNames}
Color class>>>colorNames	
	ColorNames ifNil: [self initializeNames].
	^ ColorNames
\end{method}	
\cmindex{Color class}{colorNames}

もう一つの解は、クラスメソッド\ct{initialize}をオーバーライドすることです。

\needlines{3}
\begin{method}[colorclassinit]{Color class>>initialize}
Color class>>>initialize	
	!\ldots!
	self initializeNames
\end{method}	
\cmindex{Color class}{initialize}

\noindent
この方法の場合、\ct{initialize}メソッドを定義したら、\ct{Color initialize}といった具合に明示的に実行しておくのを忘れないようにします。\jasubind{クラスがわ}{ブラウザ}{クラス側}の\ct{initialize}メソッドは、コードをメモリーにロードしたときには自動的に実行されますが、ブラウザで入力してコンパイルしたときや編集して再コンパイルしたときには、自動的には実行\emph{されない}のです。

%---------------------------------------------------------
\subsection{プール変数}
プール変数は、直接継承関係にないクラスの間でも共有される変数です。プール変数は元々はプール辞書に格納されていましたが、今では専用のクラス(SharedPoolのサブクラス)のクラス変数として定義しなければなりません。ここで忠告しておきますが、なるべく使うのを避けてください。プール変数が必要になることは非常にまれであり、特定の状況下だけです。以後は、\subind{へんすう@変数}{プール}変数を使ったコードを読むのに必要十分な説明にとどめておきます。

プール変数にアクセスするにはクラス定義でそのプールを示さなければなりません。例えば、\clsind{Text}クラスではプール辞書\ct{TextConstants}を使うことを示しています。プール辞書\ct{TextConstants}は\glbind{CR}や\glbind{LF}といった、テキストに関するすべての定数を保持しています。\ct{#CR}がキーで、値として\ct{Character cr}、つまり改行文字が結び付けられています。
\cmindex{Character class}{cr}

\begin{classdef}[textpooldict]{\ct{Text}クラスのプール辞書}
ArrayedCollection subclass: #Text
        instanceVariableNames: 'string runs' 	
        classVariableNames: '' 	
        !\textbf{poolDictionaries: 'TextConstants'}!
        category: 'Collections-Text'
\end{classdef}
   
こうすることで、\ct{Text}クラスのメソッドは、メソッド本文で\emph{直接}プール辞書のキーにアクセスすることができます。つまり、明示的に辞書に聞かずとも、変数を参照する構文でアクセスできます。例えば、以下のようにしてメソッドを書けます。
  
\begin{method}[texttestcr]{Text>>testCR}
Text>>>testCR 	
      ^ CR == Character cr
\end{method}

もう一度繰り返します。プール変数やプール辞書はなるべく使わないことをお奨めします。

%=========================================================
\section{章のまとめ}

\pharo のオブジェクトモデルはシンプルで一貫したものです。すべてはオブジェクトで、ほとんどすべてのことはメッセージ送信で起こります。

\begin{itemize}
  \item すべてのものはオブジェクトである。
  整数のようなプリミティブな存在もオブジェクトであり、クラスもファーストクラスのオブジェクトです。

  \item すべてのオブジェクトはクラスのインスタンスである。
  クラスは、\emph{プライベートな}インスタンス変数によってオブジェクトの構造を定義します。そして、\emph{パブリックな}メソッドによって振る舞いを定義します。各クラスはそれぞれのメタクラスの唯一のインスタンスです。クラス変数はクラスとそのすべてのインスタンスに共有されるプライベートな変数です。クラスはそのインスタンスのインスタンス変数に直接アクセスすることはできません。そしてインスタンスもクラスインスタンス変数にアクセスできません。そういったアクセスが必要ならば、アクセサを定義しなければなりません。

  \item すべてのクラスにはスーパークラスがある。
  単一継承の階層のルートは\lct{ProtoObject}です。しかし、クラスを定義するときには、通常は\ct{Object}クラスかそのサブクラスから継承します。抽象クラスを定義する特別な構文はありません。単に抽象メソッドを持てば抽象クラスとみなされます。抽象メソッドは\ct{self subclassResponsibility}のみからなるメソッドです。\pharo は単一継承のみをサポートしていますが、共有したいメソッド群を\emph{トレイト}としてパッケージングすることで、それらを簡単に共有できます。

  \item すべてはメッセージ送信で起こる。
	「メソッドを呼び出す」のではなく、「メッセージを送信」します。するとレシーバがそのメッセージに対応するメソッドを選択します。

  \item メソッド探索は継承の連鎖をたどっていく。
  \self 送信は動的で、レシーバのクラスからメソッド探索を始めます。一方、\super 送信は静的で、\super 送信を記述しているクラスのスーパークラスから探索を始めます。
  
  \item 3種類の共有変数がある。
	グローバル変数は、システムのどこからもアクセス可能です。クラス変数はクラスとそのサブクラスとインスタンス間とで共有されます。プール変数は指定したクラス間で共有されます。共有変数の使用は、できるだけ避けるべきです。

\end{itemize}

%=========================================================
\ifx\wholebook\relax\else
   \bibliographystyle{jurabib}
   \nobibliography{scg}
   \end{document}
\fi
%=========================================================

%=================================================================
%%% Local Variables:
%%% coding: utf-8
%%% mode: latex
%%% TeX-master: t
%%% TeX-PDF-mode: t
%%% ispell-local-dictionary: "english"
%%% End:

%---------------------------------------------------------
