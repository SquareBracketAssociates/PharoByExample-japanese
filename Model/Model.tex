% $Author$
% $Date$
% $Revision$

% HISTORY:
% 2006-10-24 - Stef started
% 2006-10-25 - Stef first draft
% 2006-12-07 - Andrew edit
% 2007-06-13 - Andrew revised
% 2007-06-21 - Oscar edit
% 2007-07-26 - Stef review
% 2007-08-23 - Oscar review
% 2007-08-29 - Andrew corrections

%=================================================================
\ifx\wholebook\relax\else
% --------------------------------------------
% Lulu:
	\documentclass[a4paper,10pt,twoside]{book}
	\usepackage[
		papersize={6.13in,9.21in},
		hmargin={.75in,.75in},
		vmargin={.75in,1in},
		ignoreheadfoot
	]{geometry}
	\input{../common.tex}
	\pagestyle{headings}
	\setboolean{lulu}{true}
% --------------------------------------------
% A4:
%	\documentclass[a4paper,11pt,twoside]{book}
%	\input{../common.tex}
%	\usepackage{a4wide}
% --------------------------------------------
    \graphicspath{{figures/} {../figures/}}
	\begin{document}
	\renewcommand{\nnbb}[2]{} % Disable editorial comments
	\sloppy
\fi
%=================================================================
\chapter{\stオブジェクトモデル}
\chalabel{model}

\stのプログラミングモデルは簡潔で一貫しています: 全てのものはオブジェクトで、オブジェクトはメッセージをお互いに送ることによってのみ、やりとりをします。
しかしながら、この簡潔さと一貫性が、他の言語に慣れたプログラマにとって難しさの元になっています。
この章では、\stのオブジェクトモデルの中心概念を示します。特に、クラスをオブジェクトとして表現することが何をもたらすのかを議論します。

%=========================================================
\section{モデルの規則}
\seclabel{rules}

\stのオブジェクトモデルは、以下の一連の簡潔かつ一貫して適用される規則に基いています。

\begin{enumerate}[label={\textbf{Rule \arabic{*}}.}, ref={Rule \arabic{*}}, leftmargin=*]
\item{} \rulelabel{everything}
	全てのものはオブジェクトである。

\item{} \rulelabel{instance}
	全てのオブジェクトはクラスのインスタンスである。

\item{}  \rulelabel{inheritance}
	全てのクラスはスーパークラスを持っている。

\item{}  \rulelabel{message}
	全てのことは、メッセージを送ることで起こる。

\item{}  \rulelabel{lookup}
	メソッド探索は継承の連鎖を辿る。

\end{enumerate}

\noindent
では、これらの規則を1つ1つ詳細に見ていきましょう。


%=========================================================
\section{全てのものはオブジェクトである}

%\ruleref{everything}

「全てのものはオブジェクトである」という信念は、感染力が高いものです。
ほんの短い間\stを使っただけで、この規則がいかに物事を簡単にするのか、驚くことでしょう。
例えば、整数も本物のオブジェクトで、他のあらゆるオブジェクトと同様、整数にもメッセージを送ることができます。

\begin{code}{@TEST}
3 + 4            --> 7    "'+ 4'を3に送ることで, 結果として7が得られます"
20 factorial  --> 2432902008176640000   "factorialを送ると, 大きな数が得られます"
\end{code}

\ct{20 factorial}の内部表現と\ct{7}の内部表現は実際には異なっていますが、それらはどちらもオブジェクトなので、実際のコード --- \ct{factorial}の実装ですら --- その内部表現を知る必要はありません。

\needlines{3}
おそらくこの規則の一番根本的な帰結は以下のものです。
\important{クラスもオブジェクトである。}
もっと言えば、クラスは第二級のオブジェクトではありません。クラスは本当にファーストクラスオブジェクトで、メッセージを投げたりインスペクトしたりすることができます。
つまり\pharoは本当の意味でリフレクティブなシステムだということで、このことが開発者に非常に強力な表現力を与えています。

\stの実装の深いところでは、3つの種類のオブジェクトがあります。(1)インスタンス変数を持ち、参照で渡される通常のオブジェクト、(2)値で渡される\emph{小さな整数}、(3)配列のように連続したメモリ領域を持つインデックス付オブジェクトです。\stの美しさは、これら3種類のオブジェクトの違いについて通常は気にする必要がないところです。

%=========================================================
\section{全てのオブジェクトはクラスのインスタンスである}

%\ruleref{instance}

全てのオブジェクトはクラスを持っています。オブジェクトに\ct{class}メッセージを送ることで見ることができます。

\begin{code}{@TEST}
1 class                 --> SmallInteger
20 factorial class --> LargePositiveInteger
'hello' class          --> ByteString
#(1 2 3) class       --> Array
(4@5) class         --> Point
Object new class --> Object
\end{code}

クラスはインスタンス変数を通してインスタンスの\emph{構造}を定義します。
そしてメソッドを通してインスタンスの\emph{振る舞い}を定義します。
各メソッドは名前を持っていて、\emphsubind{メソッド}{セレクタ}と呼ばれています。名前はそのクラスの中でユニークです。

\emph{クラスはオブジェクトである}ことから、そして、{全てのオブジェクトはクラスのインスタンスである}ことから、クラスもまた、クラスのインスタンスであるということになります。
クラスをインスタンスとするクラスは\emphind{メタクラス}と呼ばれます。
クラスを定義するたびに、システムは自動的にメタクラスを生成します。
メタクラスは、インスタンスであるクラスの構造と振る舞いを定義します。
99\% \damien{どなたか、本当に99\%だと確かめましたか?大抵のものは'99\%'よりも良いかもしれませんよ}の場合は、メタクラスについて考える必要はないでしょう。その場合は無視してくださって結構です。
(\charef{metaclasses}でメタクラスについて詳しく見てみましょう。)
%---------------------------------------------------------
\subsection{インスタンス変数}

\stではインスタンス変数は、その\emph{インスタンス}でプライベートです。
これは、\ind{Java}や\ind{C++}と対照的です。それらの言語では、インスタンス変数(``フィールド''とか``メンバ変数''とも呼ばれていますね)は同じクラスの他のインスタンスからのアクセスを許しています。
オブジェクトの\emphind{カプセル化の境界}がJavaやC++ではクラス、\stではインスタンスである、と言います。

\seeindex{Henn!instance}{インスタンス変数}
\seeindex{field}{インスタンス変数}
\seeindex{attribute}{インスタンス変数}
\seeindex{slot}{インスタンス変数}
\index{インスタンス変数}

\stでは、``\ind{アクセサ}メソッド''を定義しない限り、同じクラスの2つのインスタンスはお互いのインスタンス変数にアクセスすることはできません。
他のオブジェクトのインスタンス変数に直接アクセスするための言語構文がないのです。
(実のところ、\charef{metaprog}で扱われる\ind{リフレクション}と呼ばれる機構が他のオブジェクトにインスタンス変数の値へのアクセスを提供しています。
メタプログラミングはオブジェクト\ind{インスペクタ}のようなツールを記述することを意図しています。インスペクタはオブジェクトの中身を見ることを目的にしたツールです。)
インスタンス変数は、その変数を定義しているクラスのインスタンスメソッドから、名前でアクセスすることができます。
このことから、\stのインスタンス変数はC++やJavaの\emph{プロテクト}変数に似ていると言えます。しかしながら、やはりプライベートと読んだほうが好ましいでしょう。というのも、インスタンス変数をサブクラスからアクセスすることは、\stでは良くないスタイルと考えられているからです。

\subsubsection{例題}
\cmind{Point}{dist:} (\mthref{dist:})メソッドは、レシーバともう1つの点との間の距離を計算します。
レシーバのインスタンス変数\ct{x}と\ct{y}は、メソッドから直接アクセスされています。
しかし、もう1つの点は、\ct{x}メッセージや\ct{y}メッセージを送ることでアクセスしなければなりません。

\needlines{7}
\begin{method}[dist:]{the distance between two points}
Point>>>dist: aPoint 
	"Answer the distance between aPoint and the receiver."  
	| dx dy |
	dx := aPoint x - x.
	dy :=  aPoint y - y.
	^ ((dx * dx) + (dy * dy)) sqrt
\end{method}

\begin{code}{@TEST}
1@1 dist: 4@5 --> 5.0
\end{code}

クラスベースのカプセル化ではなく、インスタンスベースのカプセル化を選択する決め手となったのは、インスタンスベースのカプセル化は同じ抽象の異なる実装を共存させることができるからです。
例えば、\ct{point>>dist:}メソッドは、引数\ct{aPoint}がレシーバと同じクラスなのかどうか知る必要もありませんし、気にする必要もないからです。
引数オブジェクトは極座標かもしれませんし、データベースのレコードかもしれませんし、分散システムの他の計算機上にあるかもしれません。
\ct{x}メッセージや\ct{y}メッセージに答えてくれる限り、\mthref{dist}のコードはちゃんと動作するでしょう。

%---------------------------------------------------------
\subsection{メソッド}

全てのメソッドは\subind{メソッド}{パブリック}です。\footnote{正確には、ほぼ全て、です。\pharoでは、\ct{pvt}で始まるセレクタを持つメソッドはプライベートで、\ct{pvt}メッセージは\selfに\emph{のみ}送ることができます。しかし、\ct{pvt}メソッドはあまり使われていません。}
メソッドは、その目的に応じてプロトコルにグループ分けされています。
共通して使われているプロトコルはコンベンションによって確立されてきました。例えば、\protind{accessing}はアクセサメソッドで、\protind{initialization}はオブジェクトの一貫した初期状態を確立するメソッドです。
\protind{private}プロトコルは、外から見られるべきでないメソッドとしてグループ分けするのに使われます。
しかし、そういった``private''メソッドがあるからといって、実際にメッセージを送ることを防ぐことはできません。
メソッドはそのオブジェクトの全てのインスタンス変数にアクセスすることができます。
\st開発者の中には、アクセサを通してのみインスタンス変数にアクセスすることを好む人もいます。
この取り組みには一定の価値がありますが、クラスのインターフェースを乱雑にしたり、もっと悪いことに、外部の世界にプライベートな状態を晒すことにもなります。

%---------------------------------------------------------
\subsection{インスタンス側とクラス側}

クラスはオブジェクトなので、クラス自身のインスタンス変数やメソッドを持つことができます。
これらは\emph{クラスインスタンス変数}や\emph{クラスメソッド}と呼ばれていますが、実際のところ普通のインスタンス変数やメソッドと何も違いません。
クラスインスタンス変数は単にメタクラスで定義されたインスタンス変数にすぎませんし、クラスメソッドは単に\ind{メタクラス}で定義されたメソッドにすぎません。
\index{クラス!インスタンス変数}
\seeindex{変数!クラスインスタンス}{クラス, インスタンス変数}
\index{クラス!メソッド}

クラスは\ind{メタクラス}のインスタンスですが、クラスとそのメタクラスは別々のクラスです。
しかし、プログラマであるあなたにはおよそ関係ない話です。プログラマが気にかけることは、オブジェクトやそれを生成するクラスの振る舞いを定義することなのです。

\begin{figure}[htb]
\begin{center}
\includegraphics[width=\textwidth]{Color-Buttons}
\caption{クラスとそのメタクラスをブラウズする。
% \sd{Do we use Key everywhere in the picture as a legend indicator?}
% \on{sure, wherever appropriate}
\figlabel{Buttons}}
\end{center}
\end{figure}

このことから、ブラウザ\index{ブラウザ}は、クラスとメタクラスがあたかも1つのものの2つの側面であるかのように、クラスとメタクラスの両方をブラウズできるようになっています。\figref{Buttons}に示されている、``\subind{ブラウザ}{インスタンス側}''と\subind{ブラウザ}{クラス側}''です。
\button{instance}ボタンをクリックすると、\ct{Color}クラスをブラウズします。つまり、例えば青色のような\ct{Color}のインスタンスにメッセージが送られた時に実行されるメソッドをブラウズします。
\button{class}ボタンを推すと、\ct{Color class}クラスをブラウズします。つまり、\ct{Color}クラス自体にメッセージが送られた時に実行されるメソッドを見ることになります。
例えば、\ct{Color blue}は\ct{blue}メッセージを\clsind{Color}クラスに送ります。
したがって、この\ct{blue}メッセージは\ct{Color}のクラス側に定義されています。インスタンス側ではありません。
\seeindex{class side}{browser!class side}
\seeindex{instance side}{browser!instance side}

\needlines{5}
\begin{code}{@TEST | aColor |}
aColor := Color blue.               "クラス側のblueメソッド"
aColor        --> Color blue
aColor red  --> 0.0         "インスタンス側のアクセサメソッドred"
aColor blue --> 1.0        "インスタンス側のアクセサメソッドblue"
\end{code}

\subind{ブラウザ}{インスタンス側}で提供されるテンプレートを埋めることでクラスを定義します。
このテンプレートをアクセプトすると、システムはそのクラスだけでなく、対応するメタクラスも生成します。
\button{class}ボタンをクリックするとメタクラスをブラウズできます。
メタクラス生成テンプレートで編集するのは、インスタンス変数のリストだけです。

一旦クラスが生成されると、\button{instance}ボタンをクリックすると、そのクラス(およびそのサブクラス)が持つことになるメソッドを編集したりブラウズしたりできます。例えば、\figref{Buttons}を見ると、\ct{hue}メソッドが\ct{Color}クラスのインスタンスに定義されています。
逆に、\button{class}ボタンでは、そのメタクラス(この場合、\ct{Color class})を

%---------------------------------------------------------
\subsection{クラスメソッド} 

クラスメソッドはとても便利なことがあります。\ct{Color class}をブラウズすると良い例がいくつか見つかります。
クラスに定義された\subind{クラス}{メソッド}には2種類あります。\cmind{Color class}{blue}のように、そのクラスのインスタンスをつくるものと、\cmind{Color class}{showColorCube}のように、ユーティリティ的な機能を提供するものです。

他の用途に使われているクラスメソッドを見かけることもあるでしょうが、これが典型的な用法です。

余計なインスタンスを生成することなく実行できるので、\subind{ブラウザ}{クラス側}にユーティリティ的なメソッドを置くと便利です。
実際、そういったユーティリティメソッドの多くは、簡単に実行できるコメントを含んでいます。

\dothis{ \ct{Color class>>>showColorCube}メソッドをブラウズして, コメント\ct{"Color showColorCube"}の引用符の内側をダブルクリックして、\short{d}を押してみてください。}
このメソッドを実行した効果がわかるでしょう。(\menu{World \go \ind{restore display}~(r)}を実行して、元に戻してください。)

\ind{Java}や\ind{C++}に慣れている人には、クラスメソッドは静的メソッドに似ているように見えるかもしれません。
しかし、\stの一貫性から、いくぶんの違いがあります。
Javaの静的メソッドは本当に単に静的に名前解決される手続きに過ぎないのですが、\stのクラスメソッドは動的に探索されるメソッドです。つまり、\stではクラスメソッドを継承したりオーバーライドしたりsuperに送信したりできますが、Javaの静的メソッドではできません。

%---------------------------------------------------------
\subsection{クラスインスタンス変数}
通常のインスタンス変数では、あるクラスの全てのインスタンスは同じ変数名のセットを持っていて、サブクラスのインスタンスはそれらの名前を継承します。しかし、各インスタンスはそれぞれのインスタンス自身にプライベートな値のセットを持ちます。
\subind{クラス}{インスタンス変数}{}も全く同様です。各クラスはそれぞれ自身にプライベートなクラスインスタンス変数を持っています。
サブクラスはそれらのクラスインスタンス変数を継承しますが、\emph{サブクラスはそれ自身にプライベートなコピーのクラスインスタンス変数を持っています。}
オブジェクトがインスタンス変数を共有しないのと同様に、クラスとサブクラスもクラスインスタンス変数を共有しません。

あるクラスについて生成したインスタンスの数を管理するために\ct{count}というクラスインスタンス変数を使うことはできますが、
そのサブクラスはどれも自分自身の\ct{count}変数を持っているので、サブクラスのインスタンスの数は別々に数えられることになります。

\paragraph{例題: クラスインスタンス変数はサブクラスと共有されない。}
2つのクラス\ct{Dog}(犬)と\ct{Hyena}(ハイエナ)を定義します。\ct{Hyena}は\ct{Dog}からクラスインスタンス変数\ct{count}を継承します。

\begin{classdef}[dog]{犬とハイエナ}
Object subclass: #Dog
	instanceVariableNames: ''
	classVariableNames: ''
	poolDictionaries: ''
	category: 'PBE-CIV'

Dog class
	instanceVariableNames: 'count'

Dog subclass: #Hyena
	instanceVariableNames: ''
	classVariableNames: ''
	poolDictionaries: ''
	category: 'PBE-CIV'
\end{classdef}

さあ、\ct{Dog}の\ct{count}を\ct{0}に初期化するクラスメソッドと、新しいインスタンスが生成されたら1を足すクラスメソッドを定義しましょう:

\begin{method}[dogcount]{新しい犬の数を数える}
Dog class>>>initialize
	super initialize.
	count := 0.

Dog class>>>new
	count := count +1.
	^ super new

Dog class>>>count
	^ count
\end{method}

これで、新しい\ct{Dog}(犬)が生成されるたびに、頭数に1が足されていきます。\ct{Hyena}(ハイエナ)も同様ですが、別々に数えられていきます:
\begin{code}{}
Dog initialize.
Hyena initialize.
Dog count     --> 0
Hyena count --> 0
Dog new.
Dog count     --> 1
Dog new.
Dog count     --> 2
Hyena new.
Hyena count --> 1
\end{code}
% ON: In order to make this a test, I need the previous code to be part of the setup. Bleh.

インスタンス変数がインスタンスごとにプライベートなのと同様に、クラスインスタンス変数はクラスごとにプライベートだということにも注意してください。
クラスとそのインスタンスは別々のオブジェクトなことから、
以下のことが言えます。
\important{クラスはそのインスタンスのインスタンス変数にアクセスすることはできない。}
\important{クラスのインスタンスは、そのクラスのクラスインスタンス変数にアクセスすることはできない。}
このことから、インスタンスを初期化するメソッドは常に\subind{ブラウザ}{インスタンス側}で定義しなければなりません --- \subind{ブラウザ}{クラス側}からはインスタンス変数にアクセスできないので、初期化できません!
クラスができることは、新しく生成されたインスタンスに、\ind{初期化}メッセージを送信することと、可能ならばアクセサを使うことです。

同じようにインスタンスからクラスインスタンス変数にアクセスするには、クラスにアクセサメッセージを送ることによって間接的にアクセスするしかありません。

\ind{Java}にはクラスインスタンス変数に相当するものはありません。
Javaと\ind{C++}の静的変数は、\stではどちらかというと\secref{classVars}で説明するクラス変数に似ています:全てのサブクラスやインスタンスは同一の静的変数を共有します。

\paragraph{例題: シングルトンを定義する。}
\ind{シングルトンパターン}~\cite{Alpe98a}はクラスインスタンス変数とクラスメソッドを使う典型例です。
\ct{WebServer}クラスを実装してみましょう。シングルトンパターンを使って、\ct{WebServer}クラスには1つしかインスタンスがないことを保証します。

ブラウザの\button{instance}ボタンをクリックして、\clsind{WebServer}クラスを以下のように定義します。(\clsref{singleton})

\begin{classdef}[singleton]{シングルトンクラス}
Object subclass: #WebServer
	instanceVariableNames: 'sessions' 	
	classVariableNames: '' 	
	poolDictionaries: '' 	

	category: 'Web'
\end{classdef}

そして、\button{class}ボタンをクリックして、\subind{ブラウザ}{クラス側}でインスタンス変数\ct{uniqueInstance}を追加します。

\begin{classdef}[webserver]{シングルトンクラスのクラス側}
WebServer class 	
	instanceVariableNames: 'uniqueInstance'
\end{classdef}

この結果、\ct{WebServer}クラスにはもう1つのインスタンス変数が追加されます。
それまでは、\ct{superclass}や\ct{methodDict}といった変数を継承していました。

さて、これで\subind{クラス}{メソッド}\ct{luniqueInstance}を定義することができます。\mthref{uniqueInstance}に示します。
このメソッドはまず、\ct{uniqueInstance}が初期化されているかどうかをチェックします。
もしまだならば、インスタンスを生成して、クラスインスタンス変数\ct{uniqueInstance}に代入します。
最後に、\ct{uniqueInstance}の値が返されます。
\ct{uniqueInstance}はクラスインスタンス変数なので、このメソッドは直接アクセスすることができます。
    
\begin{method}[uniqueInstance]{uniqueInstance (クラス側)}
WebServer class>>>uniqueInstance
     uniqueInstance ifNil: [uniqueInstance := self new].
     ^ uniqueInstance
\end{method}

\ct{Webserver uniqueInstance}が最初に実行される時には、\ct{WebServer}クラスのインスタンスが生成され、変数\ct{uniqueInstance}に代入されます。
次回の実行では、新しいインスタンスを生成することなしに、前回生成されたインスタンスが返されます。

\mthref{uniqueInstance}の条件分岐の内側にあるインスタンス生成のコードが\ct{self new}という形であって、\ct{WebServer new}ではないことに留意してください。
何が違うのでしょうか?
\ct{uniqueInstance}メソッドは\ct{WebServer class}に定義されているので、同じことだと思うかもしれません。
\ct{WebServer}のサブクラスが作られるまでは本当に同じなのですが、\ct{ReliableWebServer}が\lct{WebServer}のサブクラスで、\ct{uniqueInstance}メソッドを継承していたりしたらどうでしょうか。
\ct{ReliableWebServer uniqueInstance}の結果は\lct{ReliableWebServer}であることを期待するでしょう。
\selfを使うことで、そうなることが保証されます。\selfはメッセージを受けたクラスなので。

\ct{WebServer}と{ReliableWebServer}はそれぞれ自分のクラスインスタンス変数\ct{uniqueInstance}を持っていることにも留意してください。それぞれの変数は、もちろん、別々の値を持ちます。

%=========================================================
\section{全てのクラスはスーパークラスを持っている}

%\ruleref{inheritance}

\stでは、それぞれのクラスは単一の\emphind{スーパークラス}から振舞いと構造の記述を継承します。
つまり\stは単一\ind{継承}を持っています。

\needlines{2}
\begin{code}{@TEST}
SmallInteger superclass --> Integer
Integer superclass          --> Number
Number superclass        --> Magnitude
Magnitude superclass    --> Object
Object superclass           --> ProtoObject
ProtoObject superclass  --> nil
\end{code}

伝統的には、\stの継承階層のルートは\clsind{Object}クラスです。(なぜなら、全てのものはオブジェクトだからです。)
\pharoでは、実際にはルートは\clsind{ProtoObject}と呼ばれるクラスです。しかし、通常はこのクラスに注意する必要はありません。
\ct{ProtoObject}は全てのオブジェクトが持って\emph{いなければならない}最低限のメッセージをカプセル化しています。
しかしながら、ほとんどのクラスは\ct{Object}から継承していて、\ct{Object}はほとんど全てのオブジェクトが理解し対応するべき沢山のメッセージを追加して定義しています。
特別な理由がないかぎり、アプリケーションクラスを生成する時には、通常\ct{Object}やそのサブクラスのどれかのサブクラスとするべきです。

\dothis{新しいクラスを生成するには、既存のクラスに
\ct{subclass: instanceVariableNames: ...}
というメッセージを送ります。
他にもいくつかクラスを生成するメソッドがあります。
\prot{Kernel-Classes \go Class \go subclass creation}プロトコルを見て、どのようなメソッドがあるか見てみてください。}
\scatindex{Kernel-Classes}
\protindex{creation}

%There is no special syntax for creating abstract classes in \st.
%An abstract class is an ordinary class in which the implementation of some methods is deferred to a subclass.
%This is repeated in the next section

\pharoには多重継承はありませんが、互いに関係のないクラスで同じ振る舞いを共有する\emphind{trait}{}と呼ばれるメカニズムをサポートしています。
トレイトはメソッドの集合で、継承関係にない複数のクラスの間で再利用されます。
トレイトを使うと、同じをコードを繰り返し定義することなく、別々のクラスの間で共有させることができます。

%---------------------------------------------------------
\subsection{抽象メソッドと抽象クラス}

\subind{クラス}{抽象}クラスは、インスタンスを生成するのではなく、サブクラスを生成するために存在するクラスです。
抽象クラスは通常は不完全で、必要なメソッドの全てが実装されてはいません。
``ない''メソッド---他のメソッドがそのメソッドがあることを想定しているのに定義されていないメソッド---を\subind{メソッド}{抽象}メソッドと呼びます。
\seeindex{抽象クラス}{クラス, 抽象}
\seeindex{抽象メソッド}{メソッド, 抽象}

\stには、抽象メソッドや抽象クラスであることを指定する特別な構文はありません。
コード規約として、抽象メソッドの本文は\mbox{\ct{self subclassResponsibility}}と書きます。
これはいわゆる``マーカーメソッド''といって、サブクラスにそのメソッドの具体的な定義を定義する責任があることを示しています。
\ct{self subclassResponsibility}メソッドは必ずオーバーライドされなければならず、したがって、絶対に実行してはならないというものです。
もしオーバーライドし忘れたら、そのメソッドが実行され、例外が発生してしまいます。
\cmindex{Object}{subclassResponsibility}

抽象メソッドを持つクラスが抽象クラスです。
実際には抽象クラスのインスタンスを生成しないようにする仕組みはありません。抽象メソッドが実行されるまでは、普通に動作します。

\subsubsection{例題: \ct{Magnitude}クラス}
\clsind{Magnitude}は互いに比較するクラスを定義するのに役立つ抽象クラスです。
\ct{Magnitude}のサブクラスは\ct{<}および\ct{=}、\ct{hash}の3つのメソッドを定義しなければなりません。
\ct{Magnitude}では、それらのメッセージを使って\ct{>}および\ct{>=}、\ct{<=}、\ct{max:}、\ct{min:}、\ct{between:and:}等のメソッドを定義しています。
\mthind{Magnitude}{<}は抽象メソッドで、\mthref{MagnitudeLessThan}に示すように定義されています。

\begin{method}[MagnitudeLessThan]{\ct{Magnitude>>><}}
Magnitude>>>< aMagnitude 
	"Answer whether the receiver is less than the argument."
	^self subclassResponsibility
\end{method}

\noindent
対照的に、\mthind{Magnitude}{>=}は具象メソッドで、\ct{<}を使って定義されています。

\begin{method}[Magnitude>=]{\ct{Magnitude>>>>=}}
>= aMagnitude 
	"Answer whether the receiver is greater than or equal to the argument."
	^(self < aMagnitude) not
\end{method}
他の比較メソッドも同様です。

\clsind{Character}は\ct{Magnitude}のサブクラスで、\ct{<}メソッドの\mthind{Object}{subclassResponsibility}をオーバーライドして独自のメソッドを定義しています(\mthref{CharacterLessThan}参照)。\ct{Character}は\ct{=}メソッドおよび\ct{hash}メソッドも定義していて、\ct{>=}および\ct{<=}、\ct{~=}等のメソッドを\ct{Magnitude}から継承しています。

\begin{method}[CharacterLessThan]{\ct{Character>>><}}
Character>>>< aCharacter 
	"Answer true if the receiver's value < aCharacter's value."
	^self asciiValue < aCharacter asciiValue
\end{method}

%---------------------------------------------------------
\subsection{トレイト}
\emphind{トレイト}は、継承なしにクラスの振る舞いに取り込むことができるメソッドの集合です。
これによって、単一継承の容易さを保ちながら、別々のクラスに有用なメソッドを共通化することができます。

新しいトレイトを定義する方法は、単にサブクラス定義のテンプレートの代わりに、\clsind{Trait}クラスへのメッセージに置き換えるだけです。

\needspace{5\baselineskip}
\begin{classdef}[tauthor]{新しいトレイトを定義する}
Trait named: #TAuthor
	uses: { }
	category: 'PBE-LightsOut'
\end{classdef}

\noindent
ここでは、\scat{PBE-LightsOut}カテゴリに\ct{TAuthor}トレイトを定義しています。
このトレイトは既存のトレイトは\emph{使って}いません。
一般には、\ct{uses:}キーワードの引数の一部に、他のトレイトの\emph{トレイト合成式}を使うことができます。
ここでは単に空の配列を与えています。

トレイトはメソッドは持つことができますが、インスタンス変数は持ちません。
以下のようにすると、色々なクラスに、階層のどこに位置するかに関係なく\ct{author}メソッドを追加できるようになります。

\begin{method}[author]{authorメソッド}
TAuthor>>>authorj
    "著者の頭文字を返す"
	^ 'on'    "oscar nierstraszの頭文字"
\end{method}

\noindent
さあ、これで既にスーパークラスを持っているクラスに対してこのトレイトを使うことができます。例えば、\charef{firstApp}で定義した\ct{LOGame}クラスです。
単に、\ct{LOGame}クラスを生成するテンプレートを書き変えて、\ct{uses:}キーワードの引数で\ct{TAuthor}を使うように指定するだけです。

\begin{classdef}[sbegamewithtrait]{トレイトを使う}
BorderedMorph subclass: #LOGame
	uses: TAuthor
	instanceVariableNames: -'cells'ss
	classVariableNames: ''e
	poolDictionaries: ''
	category: 'PBE-LightsOut'
\end{classdef}

これで、\ct{LOGame}のインスタンスを生成したら、そのインスタンスは期待通り\ct{author}メッセージに答えてくれるでしょう。

\begin{code}{}
LOGame new author --> 'on'
\end{code}

トレイト合成式では複数のトレイトを\ct{+}演算子でつなげることができます。
もし不整合(例えば、複数のトレイトが同じ名前のメソッドを定義したら)があれば、\ct{-}演算子でそれらのメソッドを明示的に除外したり、あるいは、今定義しているクラスやトレイトで再定義することで、不整合を解消することができます。
\ct{@}演算子でメソッドに\emph{エイリアス}として新しい名前を与えることも可能です。

トレイトはシステムの中核部分で使われています。
\mbox{\clsind{Behavior}}クラスが良い例です。

\needlines{8}
\begin{classdef}[behaviorwithtraits]{\ct{Behavior}の定義で使われているトレイト}
Object subclass: #Behavior
	uses: TPureBehavior @ {#basicAddTraitSelector:withMethod:->#addTraitSelector:withMethod:}
	instanceVariableNames: 'superclass methodDict format'
	classVariableNames: 'ObsoleteSubclasses'
	poolDictionaries: ''
	category: 'Kernel-Classes'
\end{classdef}
\noindent
ここでは、\ct{TPureBehavior}で定義されている\ct{addTraitSelector:withMethod:}メソッドが\ct{basicAddTraitSelector:withMethod:}にエイリアスされています。
現在、ブラウザにトレイトへのサポートが追加されているところです。

%=========================================================
\section{あらゆるコトは、メッセージ送信で起こる}

%\ruleref{message}

この規則は、\stプログラミングの本質を捉えています。

手続き的プログラミングでは、手続きが呼ばれた時にどのコードが使われるかは、呼び出し側が決めます。
呼び出し側が、名前で、\emph{静的に}、手続きや関数を選んで実行します。

オブジェクト指向プログラミングでは、``メソッドを呼び出し''\emph{ません}。``メッセージを\subind{メッセージ}{送信}''ます。
この用語の違いは重要です。
それぞれのオブジェクトにはそれぞれの責任があります。
オブジェクトを手続きに渡すことで、そのオブジェクトが何をすべきか\emph{指示する}ようなことはしません。
そうではなく、オブジェクトにメッセージを送ることで、オブジェクトに何かをするように\emph{お願い}するのです。
ここで、メッセージはコードでは\emph{ありません}。まさに名前と引数のリストだけです。
メッセージを受けたレシーバは、そこで、頼まれたことを実行するための\emph{メソッド}を選択して、どう対応するのかを決めます。
同じメッセージに対してオブジェクトによって別々のメソッドを持っていることがあるので、メソッドはメッセージを受け取った時に\emph{動的に}選択されなければなりません。
\begin{code}{@TEST}
3 + 4         --> 7          "整数「3」にメッセージ「+」と引数「4」を送信する"
(1@2) + 4 --> 5@6    "点「(1@2)」にメッセージ「+」と引数「4」を送信する"
\end{code}
\noindent
したがって、\emph{同じメッセージ}を異なるオブジェクトに送信することができ、それぞれのオブジェクトはそのメッセージに対応するために\emph{オブジェクト毎のメソッド}を持つことができます。
\ct{+ 4}というメッセージにどう対応するのかを\ct{SmallInteger} \ct{3}や\ct{Point} \ct{1@2}に指示しないのです。
それぞれが\ct{+}メソッドを持っていて、そのメソッドに従って\ct{+ 4}に対応するのです。

\stのメッセージ送信のモデルの帰結として言えることがあります。オブジェクトはとても小さなメソッドを持ち他のオブジェクトに仕事を移譲し、巨大で手続き的なメソッドで過大な責任をかかえることを避ける傾向を持つということです。
ジョセフ・ペルリン(Joseph Pelrine)はこの原則を次のように簡潔に表現しています。
\ab{Citation?}
\on{sorry, just personal communication and my own lecture notes!}
\important{誰かに押しつけられる仕事を自分でやらないこと。}
\index{Pelrine, Joseph}

多くのオブジェクト指向言語はオブジェクトに対して静的動的両方の操作方法を提供していますが、\stでは動的なメッセージ送信だけです。
静的な操作の代わりに、例えば、クラスがオブジェクトであることで、単にクラスにメッセージを送ります。

\stでは、\emph{ほぼ}全てのコトはメッセージ送信で起こります。
いずれかの時点で、メッセージ送信ではない処理が必ず発生します。
\begin{itemize}
  \item \emph{変数宣言}はメッセージ送信ではありません。
		実際、変数\subind{変数}{宣言}{}は実行可能ですらありません。
		変数宣言はオブジェクトへの参照を格納する領域を確保する意味しかありません。
  \item \emph{代入}はメッセージ送信ではありません。
		変数への\ind{代入}はその定義スコープ内で名前を束縛します。
  \item \emph{リターン}はメッセージ送信ではありません。
		\ind{リターン}は計算結果をメッセージの送り手に返します。
  \item \emph{プリミティブ}はメッセージ送信ではありません。
		プリミティブは\ind{仮想マシン}に実装されています。
		\index{プリミティブ}
\end{itemize}
これら僅かな例外を除いて、他の色々なコトはメッセージ送信で起こるのです。
特に、\stには``パブリックなフィールド''はないので、他のオブジェクトの\ind{インスタンス変数}を更新するには、メッセージを送ってそのオブジェクト自身のフィールドを更新するようにお願いするしかありません。
もちろん、全てのインスタンス変数についてセッターメソッドやゲッターメソッドを定義するのは、オブジェクト指向として良いスタイルとは言えません。
ジョセフ・ペルリンは次のようにも言っています。
\important{あなたが持っているデータを他人に弄らせるな。}

%=========================================================
\section{メソッド探索は継承の連鎖を辿る}

%\ruleref{lookup}

オブジェクトがメッセージを受け取る時、何が起こるのでしょう?

とても単純な過程をたどります:
メッセージを受けたレシーバのクラスはそのメッセージを処理するのに用いるメソッドを探索します。
もしそのクラスがメソッドを持っていなければ、その\ind{スーパークラス}に頼みます。それを続けて、\ind{継承}の連鎖を上がっていきます。
メソッドが見つかれば、渡された引数はそのメソッドの引数に束縛されて、\ind{仮想マシン}がそれを実行します。
\index{メソッド!探索}

メソッド探索の本質はこんなに簡単なのですが、
いくつか注意しなければならないことがあります。

\begin{itemize}
  \item \emph{メソッドが明示的に値を返していない場合は、どうなるのでしょうか?}
  \item \emph{あるクラスがスーパークラスのメソッドを再実装している場合は、どうなるのでしょうか?}
  \item \emph{\pvind{self}送信と\pvind{super}送信は何が違うのでしょうか?}
  \item \emph{メソッドが見つからない場合には、どうなるのでしょうか?}
\end{itemize}

ここでは、メソッド探索の概念的な規則を示します。仮想マシンを実装する人はありとあらゆるトリックや最適化を駆使してメソッド探索を高速化するのですが、
それは彼等に任せておきましょう。ここで示す規則から外れているとは判らないように、やっているはずです。

まず、探索の基本戦略から見ていきましょう。その上で、上記の一連の疑問文について考察しましょう。

%---------------------------------------------------------
\subsection{メソッド探索}
\ct{EllipseMorph}のインスタンスを生成してみます。
\begin{code}{@TEST | anEllipse |}
anEllipse := EllipseMorph new.
\end{code}
\noindent
ここで、このオブジェクトに\ct{defaultColor}メッセージを送ったら、\ct{Color yellow}という結果が得られます。
\begin{code}{@TEST | anEllipse | anEllipse := EllipseMorph new.}
anEllipse defaultColor --> Color yellow
\end{code}
\noindent
\ct{EllipseMorph}クラスは\ct{defaultColor}を実装しているので、適切なメソッドが直ちに見つかります。

\begin{method}[defaultColor]{レシーバのクラス自体に実装されたメソッド}
EllipseMorph>>>defaultColor
	"answer the default color/fill style for the receiver"
	^ Color yellow
\end{method}
\cmindex{EllipseMorph}{defaultColor}

一方、\ct{anEllipse}に\ct{openInWorld}メッセージを送ると、メソッドは直ちには見つかりません。\ct{anEllipse}クラスは\ct{openInWorld}を実装していないからです。
したがって、続いてスーパークラスである\lct{BorderedMorph}を探索していって、\ct{Morph}クラスで\ct{openInWorld}メソッドが見つかるまで続きます。

\begin{method}[openInWorld]{継承されたメソッド}
Morph>>>openInWorld
	"Add this morph to the world."

	self openInWorld: self currentWorld
\end{method}
\cmindex{Morph}{openInWorld}

\begin{figure}[htb]
\begin{center}
	{\includegraphics[width=0.8\textwidth]{openInWorldLookup}}
\caption{継承階層に従っていくメソッド探索\figlabel{openInWorldLookup}}
\end{center}
\end{figure}

%---------------------------------------------------------
\subsection{selfを返す}

\ct{EllipseMorph>>>defaultColor} (\mthref{defaultColor})は\ct{Color yellow}を明示的に返している一方で、\ct{Morph>>>openInWorld} (\mthref{openInWorld})は返り値を示すものがないことに注目してください。

実際にはメソッドは\emph{常に}値を答えます---もちろん、値というのはオブジェクトです。
答える値はメソッド中の\ct{^}で決まることもありますが、\ct{^}を実行することなくメソッドの終わりまで来てしまった場合、それでもなおメソッドは値を答えます。その場合、メソッドは、そのメッセージを受け取ったオブジェクトを答えます。
そのことを、``\selfを答える''と言います。Smalltalkでは疑似変数\selfはメッセージを受けたレシーバを指します。\ind{Java}の\ct{this}みたいなものです。
\index{変数!疑似}
\index{return}
\seeindex{caret}{return}

つまり、\mthref{openInWorld}は\mthref{openInWorldReturnSelf}と同じことです。

\needlines{5}
\begin{method}[openInWorldReturnSelf]{明示的にselfを返す}
Morph>>>openInWorld
	"Add this morph to the world."
	
    self openInWorld: self currentWorld
	^ self		"不要なので、通常はわざわざ書かないこと!"
\end{method}

なぜ上記のような場合には\ct{^ self}と明示的に書かないほうがいいのかと言うと、
何か明示的に返す場合には、メッセージを送ったセンダーにとって何か意味があることだということをコードに書いていると言えます。
\selfを明示的に返すのであれば、センダーにその返り値を使ってほしい、と言う意味になるのです。
上記のコードではそういう意図はないので、\selfを返すということを明示しないほうがよいでしょう。

これは\stでは日常的なイディオムで、ケント・ベックは``興味を惹くような返り値'' \cite{Beck97a} と呼んでいます。
\index{Beck, Kent}

\important{センダーにその値を使ってほしい時にだけ、返り値を示すこと。}

%---------------------------------------------------------
\subsection{オーバーライドと拡張}

\figref{openInWorldLookup}で示された\ct{EllipseMorph}クラスの階層では、\ct{Morph}クラスと\mbox{\ct{EllipseMorph}}クラスのどちらも\ct{defaultColor}を実装しています。
実際、\ct{Morph new openInWorld}で新しいモーフを開くと、青いモーフが出ます。楕円の場合にはデフォルトで黄色です。
\index{メソッド!オーバーライド}
\index{メソッド!拡張}
\seeindex{オーバーライド}{メソッド, オーバーライド}
\seeindex{拡張}{メソッド, 拡張}

これを、\ct{EllipseMorph}が\ct{Morph}から継承した\ct{defaultColor}メソッドを\emph{オーバーライド}する、と言います。
継承したメソッドは、\ct{anEllipse}の視点からは見えなくなります。

継承したメソッドをオーバーライドするのではなく、機能を追加して\emph{拡張}したいこともあります。つまり、サブクラスで定義する新しい機能\emph{に加えて}オーバーライドされたメソッドを実行したい、ということです。
\stでは、\super送信すれば可能です。単一継承をサポートする多くのオブジェクト指向言語によく見られる方法です。

\ct{initialize}メソッドは、この仕組みの一番重要な応用例です。
あるクラスの新しいインスタンスを初期化する時には、スーパークラスから継承したインスタンス変数も必ず初期化しなければなりません。
しかし、どうやって初期化するのかは、継承の連鎖の中でスーパークラスそれぞれの\ct{initialize}メソッドに既に記述されています。
サブクラスは継承したインスタンス変数の初期化には本来関わりがありません。

以上のことから、良いプラクティスでは、\ct{initialize}メソッドを実装する時には常にまず\ct{super initialize}を送信してから他の\ind{初期化}を続けます。
\index{super!initialize}

\needlines{6}
\begin{method}[morphinit]{Super initialize}
BorderedMorph>>>initialize
	"initialize the state of the receiver"
	super initialize.
	self borderInitialize
\end{method}

\important{ \ct{initialize}メソッドは、必ず\ct{super initialize}で始めること。}

%---------------------------------------------------------
\subsection{self送信とsuper送信}

\super\subind{super}{送信}{}は、継承されオーバーライドされてしまう振る舞いを組み入れるのに用います。
しかしながら、継承されたものであろうとそうでなかろうと、メソッドを組み入れるには通常、\self\subind{self}{送信}{}を用います。

\self送信と\super送信の違いは何でしょうか? 
\selfと同様に、\superはメッセージを受けたレシーバを表わします。
違うのは、\ind{メソッド探索}だけです。
\super送信があると、レシーバのクラスから探索を始めるのではなく、そのメソッドを定義しているクラスのスーパークラスから探索を始めます。

\superはスーパークラス\emph{ではない}ことに注意してください!
スーパークラスのことだと思うのは、よく見られる間違いです。
また、レシーバのスーパークラスから探索を始めるというのも、間違いです。
どういうことか、以下に例を示します。

\ct{constructorString}メッセージは、どんなモーフにも送信することができます。
\begin{code}{@TEST | anEllipse | anEllipse := EllipseMorph new.}
anEllipse constructorString --> '(EllipseMorph newBounds: (0@0 corner: 50@40) color: Color yellow)'
\end{code}
返り値は、評価するとそのモーフを再生成できる文字列です。

\selfと\superをどう組み合わせて、この文字列が得られるのでしょうか?
まず最初に、\ct{anEllipse constructorString}が実行されると、\ct{Morph}クラスの\ct{constructorString}が見つかります。
\figref{constructorStringLookup}を見てください。

\begin{figure}[htb]
\begin{center}
\ifluluelse
	{\includegraphics[width=\textwidth]{constructorStringLookup}}
	{\includegraphics[width=0.8\textwidth]{constructorStringLookup}}
\caption{\self送信と\super送信\figlabel{constructorStringLookup}}
\end{center}
\end{figure}

\needlines{2}
\begin{method}[constructorString]{\self送信}
Morph>>>constructorString
	^ String streamContents: [:s | self printConstructorOn: s indent: 0].
\end{method}
\cmind{Morph}{constructorString}メソッドが\lct{printConstructorOn:indent:}を\self送信します。
このメッセージもまた探索され、探索は\lct{EllipseMorph}クラスから始まって、\ct{Morph}で見つかります。
次にこのメソッドは\lct{printConstructorOn:indent:nodeDict:}を\self送信します。それがまた\ct{fullPrintOn:}を\self送信します。
さらに、\ct{fullPrintOn:}は\ct{EllipseMorph}クラスから探索を始め、\mthind{BorderedMorph}{fullPrintOn:}が\ct{BorderedMorph}で見つかります。(もう一度、\figref{constructorStringLookup}を参照してください。)
注意していただきたいのは、\self送信ではレシーバのクラス、この場合には\ct{anEllipse}クラスから、メソッド探索を始めます。

\important{\self送信は、\emph{動的な}メソッド探索を、レシーバのクラスから開始します。}

\needlines{4}
\begin{method}[fullPrintOn]{\super送信と\self送信を組み合わせる}
BorderedMorph>>>fullPrintOn: aStream
	aStream nextPutAll: '('.
	!\textbf{super fullPrintOn: aStream.}!
	aStream nextPutAll: ') setBorderWidth: '; print: borderWidth;
		nextPutAll: ' borderColor: ' , (self colorString: borderColor)
\end{method}
こうすると\ct{BorderedMorph>>>fullPrintOn:}は、\super送信をして、スーパークラスから継承した\ct{fullPrintOn:}を拡張しています。
\super送信なので、メソッド探索は\super送信が発生したクラスのスーパークラスである\ct{Morph}クラスから開始します。
すると、\ct{Morph>>>fullPrintOn:}が即座に見つかるので、それを評価します。

\super送信の探索はレシーバのスーパークラスから開始しているのではない、ということに注意してください。
もしレシーバのスーパークラスから探索を開始すると、\ct{BorderedMorph}から探索を始めるということになり、無限ループになってしまいます!

\important{\super送信は、\emph{静的な}メソッド探索を、\super送信を行なっているメソッドを定義しているクラスのスーパークラスから開始します。}

\super送信と\figref{constructorStringLookup}から、\superは静的束縛だということが判ります。\super送信をしているソースコードがどのクラスにあるかだけで全て決まります。
対照的に、\self送信は動的です。今実行しているメッセージのレシーバを表わしています。つまり、\selfに送られた\emph{全ての}メッセージは、そのレシーバのクラスから探索を開始します。

%---------------------------------------------------------
\subsection{理解されないメッセージ}

探しているメソッドが見つからなかったら、どうなるのでしょうか?
\index{メッセージ!理解されない}

楕円に\ct{foo}メッセージを送ってみましょう。
まず最初に、通常のメソッド探索が開始され、\ct{foo}メソッドを探して\clsind{Object}(正確には\clsind{ProtoObject})まで継承の連鎖を一通り探索します。
メソッドが見つからない場合、\ind{仮想マシン}がそのオブジェクトに\ct{self doesNotUnderstand: #foo}メッセージを送信させます。(\figref{fooNotFound}を参照)

\begin{figure}[htb]
\begin{center}
\ifluluelse
	{\includegraphics[width=\textwidth]{fooNotFound}}
	{\includegraphics[width=0.8\textwidth]{fooNotFound}}
\caption{Message \lct{foo} is not understood\figlabel{fooNotFound}}
\end{center}
\end{figure}

さあ、これは全く普通の動的なメッセージ送信なので、探索はまた\ct{EllipseMorph}から始まります。ただし、今度は\ct{doesNotUnderstand:}メソッドを探します。
結局のところ、\ct{Object}が\ct{doesNotUnderstand:}を実装しています。
このメソッドは、\ct{MessageNotUnderstood}オブジェクトを生成します。\ct{MessageNotUnderstood}オブジェクトは、現在実行しているコンテキストでデバッガを立ち上げることができます。

なぜこんな明らかなエラーを扱うのにこんなに複雑なことをするのかというと、
こうすると開発者がエラーを捕まえて別のアクションをするようにするのが簡単になるのです。
\mthind{Object}{doesNotUnderstand:}は\ct{Object}のどのサブクラスでもオーバーライドして、エラーを別のやり方で処理することができます。

実際、あるオブジェクトから別のオブジェクトにメッセージを自動的に移譲する実装が簡単になります。
\ct{Delegator}オブジェクトは実装していない全てのメッセージを単純に別のオブジェクトに移譲して、そのメッセージを処理させたりそのオブジェクトにエラーを発生させることができます。

%=========================================================
\section{共有変数}

\stの側面のなかで、5つの規則ではあまり簡単にカバーされていないものを見ていきましょう。\subind{変数}{共有}変数のことです。

\stには3種類の共有変数があります:(1) \empty{グローバル}な共有変数 (2) インスタンスとクラスの間で共有される変数(\emph{クラス変数}) (3) クラスのグループの間で共有される変数(\emph{プール変数}) です。
これらの共有変数の名前は大文字で始まります。これは、その変数が複数のオブジェクトの間で共有されている警告です。
\index{変数!グローバル}
\index{クラス!変数}
\index{変数!プール}

%---------------------------------------------------------
\subsection{グローバル変数}
\pharoでは、全てのグローバル変数は\glbind{Smalltalk}という名前空間に格納されます。\glbind{Smalltalk}は\clsind{SystemDictionary}クラスのインスタンスとして実装されています。
グローバル変数はどこからでもアクセス可能です。
各クラスはグローバル変数で名付けられています。加えて、特別なオブジェクトや共通して有用なオブジェクトもグローバル変数で名前付けられています。

変数\glbind{Transcript}(トランスクリプト)は\clsind{TranscriptStream}のインスタンスを指します。\clsind{TranscriptStream}はスクロール付きウィンドウに書き込むストリームです。
以下のコードは、\ct{Transcript}に情報を表示して改行します。

\begin{code}{}
Transcript show: 'Pharoは楽しくてパワフル!' ; cr
\end{code}

\noindent
\menu{do it}する前に、\menu{World \go Tools \ldots \go Transcript}でトランスクリプトを開いてください。

\hint{トランスクリプトへの書き込みは遅いです。特に、トランスクリプトウィンドウが開いている時は。
なので、速度が低下した時は、トランスクリプトに書き込んでいるのなら窓を畳んでみることも1つの手です。}

\subsubsection{他の便利なグローバル変数}

\begin{itemize}
\item
\ct{Smalltalk}は\ct{SystemDictionary}のインスタンスで、\ct{Smalltalk}自身を含む全てのグローバル変数を定義します。
この辞書のキーは\stコードでグローバル変数名として使われているシンボルです。
従って、例えば、
\begin{code}{@TEST}
Smalltalk at: #Boolean --> Boolean
\end{code}
\ct{Smalltalk}はそれ自身がグローバル変数なので、
\begin{code}{}
Smalltalk at: #Smalltalk-->a SystemDictionary(lots of globals)}
\end{code} 
さらに、
\begin{code}{@TEST}
(Smalltalk at: #Smalltalk) == Smalltalk --> true
\end{code}

\item \clsind{Sensor}は\clsind{EventSensor}のインスタンスで、\pharoへの入力を代表します。
例えば、\lct{Sensor keyboard}はキーボードからの次の入力を答えます。
 and \ct{Sensor leftShiftDown} answers \ct{true} if the left shift key is being held down,
そして、\ct{Sensor leftShiftDown}は、左シフトキーが押されている時に\ct{true}と答えて、
さらに、\ct{Sensor mousePoint}は現在のマウスの位置を示す\ct{Point}を答えます。

\item \glbind{World} is an instance of \clsind{PasteUpMorph} that represents the screen.
\item \glbind{World}は\clsind{PasteUpMorph}のインスタンスで、スクリーンを代表します。
\ct{World bounds}はスクリーン全体の範囲を示す矩形を答えます。全てのモーフは\ct{World}のサブモーフです。
\index{Morphic}

\item 
\glbind{ActiveHand}は\clsind{HandMorph}の現在のインスタンスで、カーソルのグラフィックを代表します。
\ct{ActiveHand}のサブモーフはマウスでドラッグしているものを持ちます。
\ab{I have never used this, and had to browse the image to see what it is!  What do you use it for?}

\item
\glbind{Undeclared}も辞書です。全ての未宣言の変数を持っています。
未宣言の変数を参照するメソッドを書くと、通常はブラウザがそれを宣言するよう促します。例えば、グローバル変数やそのクラスのインスタンス変数として。
しかし、後で変数宣言を削除した場合には、コードは未宣言の変数を参照することになります。
\ct{Undeclared}をインスペクトすることで、奇妙な振舞いの原因がわかることもあるでしょう。

\item
\glbind{SystemOrganization}は\clsind{SystemOrganizer}のインスタンスです。クラスの構成をパッケージに記録します。より正確に言えば、\emph{クラス名}をカテゴリーに分類します。
\end{itemize}

\begin{code}{@TEST}
SystemOrganization categoryOfElement: #Magnitude --> #'Kernel-Numbers'
\end{code}

現在のプラクティスでは、グローバル変数の使用を厳格に制限します。通常はクラスインスタンス変数やクラス変数を使ってクラスメソッドでアクセスすることが望ましいです。
もし仮に\pharoを1から再実装するようなことがあれば、グローバル変数はクラスを除いてほとんどがシングルトンに置き換えられるでしょう。

通常、グローバル変数を定義するには、大文字で始まる未宣言の識別子への代入を\menu{do it}するだけでいいです。
すると、パーサがグローバル変数を定義するよう提案します。
プログラムでグローバル変数を定義するには、\ct{Smalltalk at: #AGlobalName put: nil}を実行するだけでいいです。
削除するには、\ct{Smalltalk removeKey: #AGlobalName}です。
\glbindex{Smalltalk}

%---------------------------------------------------------
\subsection{Class variables}
\seclabel{classVars}

Sometimes we need to share some data amongst all the instances of a 
class and the class itself.
This is possible using \emph{class variables}. 
The term class \subind{class}{variable} indicates that the lifetime of the variable is the same as that of the class. However, what the
term does not convey is that these variables are shared amongst all the instances of a class as well as the class itself,
as shown in \figref{privateSharedVar}.
Indeed, a better name would have been \emph{shared variables} since this expresses more clearly their role, and also warns of the danger of using them, particularly if they are modified.

\begin{figure}[htb]
\begin{center}
\ifluluelse
	{\includegraphics[width=\textwidth]{privateSharedVarColor}}
	{\includegraphics[width=0.7\textwidth]{privateSharedVarColor}}
\caption{Instance and class methods accessing different 
variables.\figlabel{privateSharedVar}}
\end{center}
\end{figure}

In \figref{privateSharedVar} we see that \ct{rgb} and \ct{cachedDepth} are instance variables of \ct{Color}, hence only accessible to instances of \clsind{Color}.
We also see that \lct{superclass}, \lct{subclass}, \lct{methodDict} and so on are class instance variables, \ie instance variables only accessible to the \lct{Color} class.

But we can also see something new: \ct{ColorNames} and \ct{CachedColormaps} are \emph{class variables} defined for \ct{Color}.
The capitalization of these variables gives us a hint that they are shared.
In fact, not only may all instances of \ct{Color} access these shared variables, but also the \ct{Color} class itself, \emph{and any of its subclasses}.
Both instance methods and class methods can access these shared variables.

%\begin{figure}
%\begin{center}\includegraphics[width=6cm]{dateToday}\caption{A date is an object that  represents only anumber of days; all the information about month names, day names, etc.\ is shared among all the instances \figlabel{dateToday}}\end{center}.
%\end{figure}

A class \subind{class}{variable} is declared in the class definition template.
For example, the class \ct{Color} defines a large number of class variables to speed up color creation; its definition is shown below (\clsref{Color}).
\needlines{5}
\begin{classdef}[Color]{Color and its class variables}
Object subclass: #Color 	
        instanceVariableNames: 'rgb cachedDepth cachedBitPattern'
        classVariableNames: 'Black Blue BlueShift Brown CachedColormaps ColorChart ColorNames ComponentMask ComponentMax Cyan DarkGray Gray GrayToIndexMap Green GreenShift HalfComponentMask HighLightBitmaps IndexedColors LightBlue LightBrown LightCyan LightGray LightGreen LightMagenta LightOrange LightRed LightYellow Magenta MaskingMap Orange PaleBlue PaleBuff PaleGreen PaleMagenta PaleOrange PalePeach PaleRed PaleTan PaleYellow PureBlue PureCyan PureGreen PureMagenta PureRed PureYellow RandomStream Red RedShift TranslucentPatterns Transparent VeryDarkGray VeryLightGray VeryPaleRed VeryVeryDarkGray VeryVeryLightGray White Yellow'
        poolDictionaries: '' 	
        category: 'Graphics-Primitives'
\end{classdef}

The class variable \cvind{ColorNames} is an array containing the name of frequently-used colors. This array is shared by all the instances of \ct{Color} and its subclass \clsind{TranslucentColor}. It is accessible from all the instance and class methods. % (see \figref{ClassVarAccess2}).

\ct{ColorNames} is initialized once in \cmind{Color class}{initializeNames}, but it is accessed from instances of \ct{Color}.
The method \cmind{Color}{name} uses the variable to find the name of a color.
Since most colors do not have names, it was thought inappropriate to add an instance variable \ct{name} to every color.

\subsubsection{Class initialization}

The presence of class variables raises the question: how do we initialize them?  
One solution is lazy \subind{class}{initialization}.
This can be done by introducing an accessor method which, when executed, initializes the variable if it has not yet been initialized. This implies that we must use the accessor all the time and never use the class variable directly.  This furthermore imposes the cost of the accessor send and the initialization test.
It also arguably defeats the point of using a class variable, since in fact it is no longer shared.

\begin{method}[colorclasscolornames]{Color class>>colorNames}
Color class>>>colorNames	
	ColorNames ifNil: [self initializeNames].
	^ ColorNames
\end{method}	
\cmindex{Color class}{colorNames}

Another solution is to override the class method \ct{initialize}.

\needlines{3}
\begin{method}[colorclassinit]{Color class>>initialize}
Color class>>>initialize	
	!\ldots!
	self initializeNames
\end{method}	
\cmindex{Color class}{initialize}

\noindent
If you adopt this solution, you need to remember to invoke the \ct{initialize} method after you define it, \eg by evaluating \ct{Color initialize}.
Although \subind{browser}{class side} \ct{initialize} methods are executed automatically when code is loaded into memory, they are \emph{not} executed automatically when they are first typed into the browser and compiled, or when they are edited and re-compiled.

%---------------------------------------------------------
\subsection{Pool variables}
Pool variables are variables that are shared between several classes that may not be related by inheritance.
Pool variables were originally stored in pool dictionaries; now they should be defined as class variables of dedicated classes (subclasses of SharedPool). Our advice is to avoid them; you will need them only in rare and specific circumstances.
Our goal here is therefore to explain \subind{variable}{pool} variables just enough so that you can understand them when you are reading code. 

A class that accesses a pool variable must mention the pool in its class definition. 
For example, the class \clsind{Text} 
indicates that it is using the pool dictionary \ct{TextConstants}, which contains all the text constants such as \glbind{CR} and \glbind{LF}. 
This dictionary has a key \ct{#CR} that is bound to the value \ct{Character cr}, \ie the carriage return character.
\cmindex{Character class}{cr}

\begin{classdef}[textpooldict]{Pool dictionaries in the \ct{Text} class}
ArrayedCollection subclass: #Text
        instanceVariableNames: 'string runs' 	
        classVariableNames: '' 	
        !\textbf{poolDictionaries: 'TextConstants'}!
        category: 'Collections-Text'
\end{classdef}
   
This allows methods of the class \ct{Text} to access the keys of the dictionary in the method body \emph{directly}, \ie by using variable syntax rather than an explicit dictionary lookup.
For example, we can write the following method. 
  
\begin{method}[texttestcr]{Text>>testCR}
Text>>>testCR 	
      ^ CR == Character cr
\end{method}

Once again,  we recommend that you avoid the use of pool variables and pool dictionaries. 

%=========================================================
\section{Chapter summary}

The object model of \pharo is both simple and uniform.
Everything is an object, and pretty much everything happens by sending messages.

\begin{itemize}
  \item Everything is an object.
  Primitive entities like integers are objects, but also classes are first-class objects.

  \item Every object is an instance of a class.
  Classes define the structure of their instances via \emph{private} instance variables and the behaviour of their instances via \emph{public} methods. Each class is the unique instance of its metaclass.  Class variables are private variables shared by the class and all the instances of the class.
  Classes cannot directly access instance variables of their instances, and instances cannot access instance variables of their class.
  Accessors must be defined if this is needed.

  \item Every class has a superclass.
  The root of the single inheritance hierarchy is \lct{ProtoObject}.
  Classes you define, however, should normally inherit from \ct{Object} or its subclasses.
  There is no syntax for defining abstract classes.
  An abstract class is simply a class with an abstract method\,---\,one 
  whose implementation consists of the expression \ct{self subclassResponsibility}.
  Although \pharo supports only single inheritance, it is easy to share implementations of methods by packaging them as \emph{traits}.

  \item Everything happens by sending messages.
	We do not ``call methods'', we ``send messages''.
	The receiver then chooses its own method for responding to the message.

  \item Method lookup follows the inheritance chain;
  \self sends are dynamic and start the method lookup again in the class of the receiver, whereas
  \super sends are static, and start in the superclass of class in which the \super send is written.
  
  \item There are three kinds of shared variables.
  		Global variables are accessible everywhere in the system.
		Class variables are shared between a class, its subclasses and its instances.
		Pool variables are shared between a selected set of classes.
		You should avoid shared variables as much as possible.

\end{itemize}

%=========================================================
\ifx\wholebook\relax\else
   \bibliographystyle{jurabib}
   \nobibliography{scg}
   \end{document}
\fi
%=========================================================

%=================================================================
%%% Local Variables:
%%% coding: utf-8
%%% mode: latex
%%% TeX-master: t
%%% TeX-PDF-mode: t
%%% ispell-local-dictionary: "english"
%%% End:

%---------------------------------------------------------
