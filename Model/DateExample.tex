% $Author$
% $Date$
% $Revision$

% HISTORY:
% 2007-06-21 - split off data example

%=================================================================
\ifx\wholebook\relax\else
% --------------------------------------------
% Lulu:
	\documentclass[a4paper,10pt,twoside]{book}
	\usepackage[
		papersize={6.13in,9.21in},
		hmargin={.75in,.75in},
		vmargin={.75in,1in},
		ignoreheadfoot
	]{geometry}
	\input{../common.tex}
	\pagestyle{headings}
	\setboolean{lulu}{true}
% --------------------------------------------
% A4:
%	\documentclass[a4paper,11pt,twoside]{book}
%	\input{../common.tex}
%	\usepackage{a4wide}
% --------------------------------------------
    \graphicspath{{figures/} {../figures/}}
	\begin{document}
	\renewcommand{\nnbb}[2]{} % Disable editorial comments
	\sloppy
\fi
%=================================================================


\chapter{ケーススタディー}

\subsubsection{ケーススタディー: \ct{Date}クラス}
Dateは日付を表わすオブジェクトです。
\ct{Date today}は本日を表わすオブジェクトを返します。
このオブジェクトに対して、表示するよう依頼すると、例えば'27 November 2002'が得られます。%%%/// update this?
\figref{dateToday}では、式\ct{Date today inspect}で日付オブジェクトをインスペクトしています。
日付オブジェクトには日数しか持っていません。
月の名前や、曜日の名前、各月の日数などを保持するインスタンス変数はありません。
事実、そのような情報は日付クラスの全インスタンスで共有されています。つまり、\clsref{cls:dateclass}に示されているように、\ct{Date}クラスのクラス変数\ct{DaysInMonth FirstDayOfMonth MonthNames  SecondsInDay WeekDayNames} %%%/// should there be commas between these names?
で表わされています。

\begin{classdef}{}\clslabel{dateclass}
Magnitude subclass: \#Date
   instanceVariableNames: 'julianDayNumber '
   \ct{classVariableNames: 'DaysInMonth FirstDayOfMonth MonthNames 
     SecondsInDay WeekDayNames '}
   poolDictionaries: ''
   category: 'Kernel-Magnitudes'
\end{classdef}


\ct{Date}クラス(とそのサブクラス)のすべてのインスタンスメソッドは\ct{Date}クラスで定義されたクラス変数に直接アクセスします。
下に示す\ct{mth:monthName}メソッドはクラス変数\ct{MonthNames}にアクセスします。

\begin{method}{}\mthlabel{monthName}
Date>>>monthName
   "レシーバの月の名前を答える。"

   ^MonthNames at: self monthIndex
\end{method}

同じように、すべてのクラスメソッドはクラス変数にアクセスできます。
クラスメソッド\mthref{nameOfDay}と\ct{nameOfDay:}は、クラス変数\ct{WeekDayNames}にアクセスします。

\begin{method}{}\mthlabel{nameOfDay}
Date class>>>nameOfDay: dayIndex 
   "1から7のdayIndexで指される曜日を表わすシンボルを答える。"

   ^WeekDayNames at: dayIndex
\end{method}


%%%/// Should this go after or in the paragraph below called "pool dictionaries"?
プール辞書はまったく静的な概念です。 %%%/// Unclear. "A pool dictionary is a static concept"?  
%%%/// "and it"? 
プール辞書はメソッドがそれを使う前に定義されていなければなりません。%%%/// can use it?
プール辞書を使わないことを強く推奨します。%%%/// not to use pool dictionaries?

%=================================================================
\ifx\wholebook\relax\else
\end{document}\fi
%=================================================================
