% $Author$
% $Date$
% $Revision$

% HISTORY:
% 2006-10-24 - Stef started
% 2006-11-16 - Stef completes first draft
% 2007-04-09 - Andrew review and edit
% 2007-08-23 - Oscar edit
% 2007-09-05 - Andrew edit

%=================================================================
\ifx\wholebook\relax\else
% --------------------------------------------
% Lulu:
	\documentclass[a4paper,10pt,twoside]{book}
	\usepackage[
		papersize={6.13in,9.21in},
		hmargin={.75in,.75in},
		vmargin={.75in,1in},
		ignoreheadfoot
	]{geometry}
	\input{../common.tex}
	\pagestyle{headings}
	\setboolean{lulu}{true}
% --------------------------------------------
% A4:
%	\documentclass[a4paper,11pt,twoside]{book}
%	\input{../common.tex}
%	\usepackage{a4wide}
% --------------------------------------------
    \graphicspath{{figures/} {../figures/}}
	\begin{document}
	\renewcommand{\nnbb}[2]{} % Disable editorial comments
	\sloppy
\fi
%=================================================================
\chapter{クラスとメタクラス}
\chalabel{metaclasses}

\on{The section on responsibilities of Class, Behavior and Metaclass are weak, and need to be fleshed out with convincing examples. Marcus, can you help?}

\charef{model}で見てきたように、\st では、すべてがオブジェクトで、それらはいずれもクラスのインスタンスです。クラスも例外ではなく、オブジェクトであり、なんらかのクラスのインスタンスです。
このオブジェクトモデルは、無駄なく、シンプルで、エレガントに統一されており、オブジェクト指向プログラミングの本質を捉えています。
しかしながら、慣れていないうちは、この一貫性のために混乱してしまうかもしれません。この章では、それが複雑ではなく、魔法でも特別なものでもない、ただシンプルなルールが一貫して適用されている、ということを示します。
これらのルールにより、なぜそうなっているのかが分かるでしょう。

%=================================================================
\section{クラスとメタクラスのルール}

Smalltalkのオブジェクトモデルは、一貫した、数少ないコンセプトの上に成り立っています。Smalltalkの設計者は、必要以上の複雑さにつながるものは削ぎ落としました。
\charef{model}で見た、オブジェクトモデルのルールを思い出しましょう。

\begin{enumerate}[label={\textbf{Rule \arabic{*}}.}, ref={Rule \arabic{*}}, leftmargin=*, widest=10]
% NB: rule labels must not be multiply defined across chapters!
\item{} % \rulelabel{everything}
	すべてのものはオブジェクトである

\item{} % \rulelabel{instance}
	すべてのオブジェクトは、クラスのインスタンスである

\item{} % \rulelabel{inheritance}
	すべてのクラスは、スーパークラスを持つ

\item{} % \rulelabel{message}
	すべての出来事は、メッセージを送ることで生じる

\item{} % \rulelabel{lookup}
	メソッドは、継承チェーンに沿って探す

\end{enumerate}

この章の冒頭で述べたように、\ref{rule:everything}より\emph{クラスもオブジェクト}です。そして、\ref{rule:instance}より、クラスもなんらかのクラスのインスタンスです。
この、クラスのクラスを、\emph{メタクラス}といいます。
\seclabel{metaclassIntro}
\indmain{メタクラス}は、クラスを定義した際、自動的に作られます。
普段はメタクラスを気にする必要はないでしょう。
しかしながら、ブラウザで``\subind{browser}{クラス側}''を見てるとき、実は異なるクラスを見ているのです。
クラスとメタクラスは、それぞれが別々のクラスであり、クラスがメタクラスのインスタンスである、という関係です。

クラスとメタクラスを正確に説明するには、\charef{model}のルールに付け足す必要があります。

\begin{enumerate}[label={\textbf{Rule \arabic{*}}.}, ref={Rule \arabic{*}}, leftmargin=*, widest=10]
\setcounter{enumi}{5}
\item{} \rulelabel{metaclass}
	すべてのクラスは、メタクラスのインスタンスである

\item{} \rulelabel{parallelhierarchy}
	メタクラスの\subind{metaclass}{階層}は、クラスの階層と並列に存在する

\item{} \rulelabel{behavior}
	すべてのメタクラスは、\clsind{Class}と\clsind{Behavior}を継承している

\item{} \rulelabel{metaclassclass}
	すべてのメタクラスは、\clsind{Metaclass}のインスタンスである

\item{} \rulelabel{metaclassmetaclass}
	\ct{Metaclass}のメタクラスは、\ct{Metaclass}のインスタンスである

\end{enumerate}

\noindent
これら10個のルールが、\st のオブジェクトモデルのすべてです。

最初に、\charef{model}で紹介した5つのルールを、もう一度小さな例で説明します。
そして、同じ例を用いて、新しいルールを深く見ていきます。

%=================================================================
\section{Smalltalkオブジェクトモデルの復習}

すべてがオブジェクトなので、「青色」も\st ではオブジェクトです。
\begin{code}{@TEST}
Color blue --> Color blue
\end{code}

\noindent
すべてのオブジェクトはクラスのインスタンスです。「青色」のクラスは\clsind{Color}です。
\begin{code}{@TEST}
Color blue class --> Color
\end{code}

\noindent
おもしろいことに、\emph{alpha}(透明度)を設定すると、異なるクラス(\clsind{TranslucentColor})のインスタンスになります。
\begin{code}{@TEST}
(Color blue alpha: 0.4) class --> TranslucentColor
\end{code}

\noindent
Morphを作って、それを半透明にすることができます。
\begin{code}{}
EllipseMorph new color: (Color blue alpha: 0.4); openInWorld
\end{code}
\noindent
実行すると、\figref{translucentEllipse}のようになります。

\begin{center}
\begin{figure}[!ht]
\ifluluelse
	{\centerline {\includegraphics[scale=0.7]{TranslucentEllipse}}}
	{\centerline {\includegraphics[width=8cm]{TranslucentEllipse}}}
\caption{半透明の楕円Morph\figlabel{translucentEllipse}}
\end{figure}
\end{center}

\ref{rule:inheritance}より、すべてのクラスはスーパークラスを持ちます。
\clsind{TranslucentColor}のスーパークラスは\clsind{Color}です。そして、\ct{Color}のスーパークラスは\clsind{Object}です。
\begin{code}{@TEST}
TranslucentColor superclass --> Color
Color superclass                   --> Object
\end{code}

\ref{rule:message}「すべての出来事は、\ind{メッセージを送ること}{}で生じる」より、\mthind{Color class}{blue}は\ct{Color}へのメッセージであると推測できます。\mthind{Object}{class}と\mthind{Color}{alpha:}は「青色」へのメッセージ、\mthind{Morph}{openInWorld}は楕円Morphオブジェクトへのメッセージ、そして、\mthind{Behavior}{superclass}は\ct{TranslucentColor}と\ct{Color}へのメッセージです。
どのケースもメッセージの受け手はオブジェクトです。すべてがオブジェクトであることから、いくつかのケースでクラスが受け手となっています。

\ref{rule:lookup}「メソッドは、継承チェーンに沿って探す」より、\ct{Color blue alpha: 0.4}で得られたオブジェクトに\ct{class}メッセージを送った場合、メッセージは、対応するメソッドが見つかる\ct{Object}で取り扱われます。\figref{classmessage}をご覧ください。

\begin{center}
\begin{figure}[!ht]
\ifluluelse
	{\centerline{\includegraphics[width=\textwidth]{TranslucentClassMessage}}}
	{\centerline{\includegraphics[width=0.8\textwidth]{TranslucentClassMessage}}}
\caption{透過色オブジェクトへのメッセージ送信\figlabel{classmessage}}
\end{figure}
\end{center}

この図は、\emphind{is-a}の継承関係を示したものです。
\ct{Color blue alpha: 0.4}で得られたオブジェクトは、\ct{TranslucentColor}のインスタンスです。また、それは\ct{Color}オブジェクトであるとも、\ct{Object}オブジェクトであるとも言え、それぞれのクラスで応ずることの出来るメッセージが定義されています。
さらに補足すると、\mthind{Object}{isKindOf:}メッセージをオブジェクトに送ることで、引数で与えられたクラスと\emph{is a}の関係にあるかどうかを調べることができます。
\begin{code}{@TEST | translucentBlue |}
translucentBlue := Color blue alpha: 0.4.
translucentBlue isKindOf: TranslucentColor --> true
translucentBlue isKindOf: Color                    --> true
translucentBlue isKindOf: Object                  --> true
\end{code}

%=================================================================
\section{すべてのクラスは、メタクラスのインスタンス}

% \ruleref{metaclass}

\secref{metaclassIntro}で述べたように、すべてのクラスは、\ind{メタクラス}{}のインスタンスです。

\paragraph{メタクラスは暗黙的な存在です。}
メタクラスは、クラスを定義すると、自動的に作成されます。\emph{暗黙的}なものなので、プログラマが気にかける必要はありません。クラスを定義する度に、対応する\subind{metaclass}{暗黙的}なメタクラスが一つだけ作成されます。
% At a more advanced level, this implies that sharing between metaclasses is difficult except by subclassing. 

通常のクラスはグローバル変数ですが、メタクラスはそうではなく、匿名の存在です。しかし、クラスを通して、いつでもメタクラスのインスタンスを参照することができます。\clsind{Color}は\clsind{Color class}のインスタンス。\ct{Object}は、\clsind{Object class}のインスタンスです。
\begin{code}{@TEST}
Color class   --> Color class
Object class --> Object class
\end{code}

\noindent
\figref{translucentmetaclasses}に、クラスが、その(\subind{metaclass}{匿名}の)メタクラスのインスタンスであることを示します。

\begin{center}
\begin{figure}[!ht]
\ifluluelse
	{\centerline {\includegraphics[width=\textwidth]{TranslucentMetaclasses}}}
	{\centerline {\includegraphics[width=0.8\textwidth]{TranslucentMetaclasses}}}
\caption{TranslucentColorクラスのメタクラスと、スーパークラス\figlabel{translucentmetaclasses}}
\end{figure}
\end{center}

クラスがオブジェクトであることで、メッセージを送って問い合わせることが簡単にできます。以下をご覧ください。

\begin{code}{@TEST}
Color subclasses                           --> {TranslucentColor}
TranslucentColor subclasses         --> #()
TranslucentColor allSuperclasses  --> an OrderedCollection(Color Object ProtoObject)
TranslucentColor instVarNames     --> #('alpha')
TranslucentColor allInstVarNames --> #('rgb' 'cachedDepth' 'cachedBitPattern' 'alpha')
TranslucentColor selectors             -->  an IdentitySet(#pixelWord32 #asNontranslucentColor #privateAlpha #pixelValueForDepth: #isOpaque #isTranslucentColor #storeOn: #pixelWordForDepth: #scaledPixelValue32 #alpha #bitPatternForDepth: #hash #isTransparent #isTranslucent #balancedPatternForDepth: #setRgb:alpha: #alpha: #storeArrayValuesOn:)
\end{code}
\cmindex{Class}{subclasses}
\cmindex{Behavior}{allSuperclasses}
\cmindex{Behavior}{instVarNames}
\cmindex{Behavior}{allInstVarNames}
\cmindex{Behavior}{selectors}

%=================================================================
\section{メタクラス階層は、クラス階層に並列に存在する}
% \ruleref{parallelhierarchy}

\ref{rule:parallelhierarchy}より、メタクラスのスーパークラスには、任意のクラスを指定できません。スーパークラスのメタクラスとなります。

\begin{code}{@TEST}
TranslucentColor class superclass --> Color class
TranslucentColor superclass class --> Color class
\end{code}

\noindent
この意味は、メタクラス\subindmain{metaclass}{階層}が、クラス階層に並列に存在するということです。\figref{parallelHierarchies}に、\clsind{TranslucentColor}の階層がどのようになっているかを示します。

\begin{center}
\begin{figure}[!ht]
\ifluluelse
	{\centerline {\includegraphics[width=\textwidth]{TranslucentMetaclassHierarchy}}}
	{\centerline {\includegraphics[width=0.8\textwidth]{TranslucentMetaclassHierarchy}}}
\caption{メタクラス階層と、並列に存在するクラス階層\figlabel{parallelHierarchies}}
\end{figure}
\end{center}

\begin{code}{@TEST}
TranslucentColor class                                     --> TranslucentColor class
TranslucentColor class superclass                   --> Color class
TranslucentColor class superclass superclass --> Object class
\end{code}

\paragraph{クラスとオブジェクトの一貫性。}オブジェクトやクラスにメッセージを送る場合と違いがないことは、興味深いことです。どちらのケースも、対応するメソッドはメッセージの受け手のクラスから始まり、継承チェーンを辿っていきます。

従って、クラスに送ったメッセージは、メタクラス階層チェーンを辿らなければなりません。例えば、\mthind{Color class}{blue}メソッドは\ct{Color}の\subind{browser}{クラス側}に定義されています。\ct{TranslucentColor}に\ct{blue}メッセージを送ると、instance側に定義されたメッセージと同じようにメソッドを探します。\ct{TranslucentColor class}からスタートし、メタクラス階層を、\ct{Color class}まで辿ります。\figref{metaclasslookup}をご覧ください。

\begin{code}{@TEST}
TranslucentColor blue --> Color blue
\end{code}
\noindent
\ct{TranslucentColor}のインスタンスではなく、普通の\ct{Color blue}オブジェクトが得られます。マジックではありません。

\begin{center}
\begin{figure}[!ht]
\ifluluelse
	{\centerline {\includegraphics[width=\textwidth]{TranslucentColorBlue}}}
	{\centerline {\includegraphics[width=0.8\textwidth]{TranslucentColorBlue}}}
\caption{通常のオブジェトと同様のメッセージ検索\figlabel{metaclasslookup}}
\end{figure}
\end{center}

Smalltalkのメソッド\subind{method}{検索}が、一つの決まった方法で行われることが分かります。クラスはオブジェクトで、他のオブジェクトと同じように振る舞います。
クラスは、新しいインスタンスを作る能力を持ちます。これは、クラスだけがたまたま\ct{new}メッセージに応えられ、\ct{new}メソッドだけが新しいインスタンスを作る方法だからです。
通常、クラス以外のオブジェクトは、このメッセージに応えられません。しかし、もし十分な理由があって、それに応えられるようにしたければ、\ct{new}メソッドをメタクラス以外に追加しても良いでしょう。

クラスはオブジェクトなので、インスペクトすることができます。
\index{inspector}

\dothis{\ct{Color blue}と\ct{Color}をインスペクトしてみましょう。}

\noindent
片方は\ct{Color}のインスタンス、もう片方は\ct{Color}自身をインスペクトしていることに注意してください。インスペクタのタイトルバーに、どちらも\emph{class}の名前が表示されているので、少し混乱するかもしれません。

\ct{Color}のインスペクタには、\figref{inspectingColor}のように、スーパークラス(superclass)、インスタンス変数(instanceVariables)、メソッド\subind{method}{辞書}(methodDict)などが見えています。

\begin{center}
\begin{figure}[!ht]
\ifluluelse
	{\centerline{\includegraphics[width=\textwidth]{InspectingColor}}}
	{\centerline{\includegraphics[width=10cm]{InspectingColor}}}
\caption{クラスもオブジェクト\figlabel{inspectingColor}}
\end{figure}
\end{center}

%=================================================================
\section{すべてのメタクラスは、\lct{Class}と\lct{Behavior}を継承する}
% \ruleref{behavior}

すべての\ind{メタクラス}はクラスなので、\clsind{Class}を継承しています。\clsind{Class}のスーパークラスは\clsind{ClassDescription}、さらにそのスーパークラスは\clsind{Behavior}です。\st ではすべてがオブジェクトなので、つまるところ、これらのクラスも\ct{Object}を継承しています。\figref{inheritbehavior}に、最終的な姿を示します。

\begin{center}
\begin{figure}
\ifluluelse
	{\centerline{\includegraphics[width=\textwidth]{TranslucentBehavior}}}
	{\centerline{\includegraphics[width=0.8\textwidth]{TranslucentBehavior}}}
\caption{メタクラスは、\lct{Class}と\lct{Behavior}を継承する\figlabel{inheritbehavior}}
\end{figure}
\end{center}

\paragraph{\lct{new}はどこに定義されているのでしょうか?}
\ct{new}がどこに定義されていて、それをどのように見つけるのか、ということが、メタクラスが\ct{Class}、\ct{Behavior}を継承していることの重要性を理解する手助けになります。\ct{new}メッセージをクラスに送ると、\figref{sendingnew}のように、メタクラスチェインを辿って、スーパークラスである、\ct{Class}、\ct{ClassDescription}、\ct{Behavior}の順にメソッド探索を行います。

\emph{「どこに\ct{new}が定義されているのか」}は、大切な問いです。\mthind{Behavior}{new}は、最初に\ct{Behavior}で定義されています。そして必要に応じて、メタクラスを含む\ct{Behavior}のサブクラスで再定義することができます。\ct{new}メッセージをクラスに送ると、通常はまず、そのクラスのメタクラスから検索し、見つからなければ、\ct{Behavior}か、再定義しているクラスまで、順にスーパークラスを辿って検索します。

\ct{TranslucentColor new}を評価した結果は、\clsind{TranslucentColor}のインスタンスで、\ct{Behavior}のインスタンスでは\emph{ありません}。例え、\ct{Behavior}まで辿って\ct{new}が見つかった場合であっても、\self のインスタンスとなります。これは、\ct{new}が他のクラスで実装されている場合でも同じです。

\begin{code}{@TEST}
TranslucentColor new class --> TranslucentColor    "Behaviorのインスタンスではない!"
\end{code}

\begin{center}
\begin{figure}
\ifluluelse
	{\centerline{\includegraphics[width=\textwidth]{TranslucentSendingNew}}}
	{\centerline{\includegraphics[width=0.8\textwidth]{TranslucentSendingNew}}}
\caption{\lct{new}は一般的なメッセージで、メタクラスチェインを辿って検索されます\figlabel{sendingnew}}
\end{figure}
\end{center}

よくある間違いは、メッセージを受けるクラスのスーパークラスを辿って\ct{new}を探すことです。\ct{new:}も同じで、これは引数で与えられたサイズに応じたオブジェクトを生成するための標準的なメッセージです。たとえば、\lct{Array new: 4}は、4つの要素を保つ配列を生成します。\ct{Array}とそのスーパークラスからは、このメソッドの定義を見つけることは出来ないでしょう。代わりに、\ct{Array class}とそのスーパークラスから探索を始めるべきです。

\on{The text below needs more details and convincing examples ...}

\paragraph{\lct{Behavior}、\lct{ClassDescription}および\lct{Class}の責務について}
\clsind{Behavior}は、インスタンスを持つための必要最小限の機能を備えます。すなわち、スーパークラスへのリンク、メソッド辞書、インスタンスの振る舞い(\ie 数字の表示)などです。
\on{not sure I understand the last point}
\ct{Behavior}は\ct{Object}を継承しているので、すべてのサブクラスは、オブジェクトとして振る舞うことができます。
\ct{Behavior}も、コンパイラへの基本的なインターフェースです。メソッド辞書の生成、メソッドのコンパイル、インスタンスの生成(\ie \mthind{Behavior}{new}, \mthind{Behavior}{basicNew}, \mthind{Behavior}{new:}, \mthind{Behavior}{basicNew:})、クラス階層の取り扱い(\ie \mthindex{Behavior}{superclass:} \lct{superclass:}, \mthind{Behavior}{addSubclass:})、メソッドへのアクセス(\ie \mthind{Behavior}{selectors}, \lmthind{Behavior}{allSelectors}, \mthind{Behavior}{compiledMethodAt:})、インスタンス、および属性へのアクセス(\ie \mthind{Behavior}{allInstances}, \mthind{Behavior}{instVarNames}\ldots)、クラス階層へのアクセス(\ie \mthind{Behavior}{superclass}, \mthind{Behavior}{subclasses})、クラス階層関連の問い合わせ(\ie \mthind{Behavior}{hasMethods}, \mthind{Behavior}{includesSelector}, \mthind{Behavior}{canUnderstand:}, \mthind{Behavior}{inheritsFrom:}, \mthind{Behavior}{isVariable})、などを提供します。

\clsind{ClassDescription}は抽象クラスです。clsind{Class}と\clsind{Metaclass}のスーパークラスで、それらに必要な機能を提供しています。
\clsind{ClassDescription}は、\ct{Behavior}で提供された基礎に、多くの機能を付け足します。インスタンス変数、プロトコルによるメソッドの分類、サブクラスでオーバーライドすべき抽象メソッドの定義、チェンジセットや変更ログの取り扱い、ファイルアウト機能の大部分などです。

\clsind{Class}は、一般的なクラスの振る舞いを提供します。
すなわち、クラス名、メソッドのコンパイル、メソッドの保管、インスタンス変数に関わる機能です。
また、クラス変数名、共有プール変数に関わる機能(\mthind{Class}{addClassVarName:}, \mthind{Class}{addSharedPool:}, \mthind{Class}{initialize})も持ちます。
\ct{Class}がインスタンス変数を生成する機能を持つので、つまるところ、すべてのメタクラスは\ct{Class}を継承することになります。

%=================================================================
\section{すべてのメタクラスは、\lct{Metaclass}のインスタンス}
% \ruleref{metaclassclass}

メタクラスもオブジェクトで、\figref{metaclassclass}に示すように、\clsind{Metaclass}のインスタンスです。\ct{Metaclass}のインスタンスは、匿名のメタクラスです。そのそれぞれは、ただ一つだけインスタンスを持ち、つまるところ、それはクラスです。

\begin{center}
\begin{figure}
\ifluluelse
	{\centerline{\includegraphics[width=\textwidth]{TranslucentMetaclassClass}}}
	{\centerline{\includegraphics[width=0.8\textwidth]{TranslucentMetaclassClass}}}
\caption{すべてのメタクラスは、\lct{Metaclass}のインスタンス\figlabel{metaclassclass}}
\end{figure}
\end{center}

\ct{Metaclass}は、一般的なメタクラスの振る舞いを提供します。
すなわち、インスタンス生成のためのメソッド(\lct{sub\-class\-Of:})、メタクラスの唯一のインスタンス、クラス変数の初期化、メタクラスのインスタンス、メソッドのコンパイル、クラスの情報(例:継承関係、インスタンス変数、\etc)などです。
% (\ct{name:inEnvironment:subclassOf:})
% Actually, this is in ClassBuilder, it seems!
% (different semantics can be introduced)
\ab{This is too cryptic for me.}

%=================================================================
\section{\lct{Metaclass}のメタクラスは、\lct{Metaclass}のインスタンス}
% \ruleref{metaclassmetaclass}

最後の問題です。\clsind{Metaclass class}のクラスは何でしょうか?

答えは簡単です。それはメタクラスで、\ct{Metaclass}のインスタンスです。システム内にある、他すべてのメタクラスと同じです。\figref{metaclassclassclass}をご覧ください。

\begin{center}
\begin{figure}
\ifluluelse
	{\centerline{\includegraphics[width=\textwidth]{TranslucentMetaclassClassClass}}}
	{\centerline{\includegraphics[width=0.8\textwidth]{TranslucentMetaclassClassClass}}}
\caption{すべてのメタクラスは、\lct{Metaclass}のインスタンスです。\lct{Metaclass}のメタクラスも同様です\figlabel{metaclassclassclass}}
\end{figure}
\end{center}

この図は、すべてのメタクラスは、\ct{Metaclass}のインスタンスで、その関係は、\ct{Metaclass}自身も同じだということを示しています。
\ref{fig:metaclassclass}と\ref{fig:metaclassclassclass}を比較すると、どのようにしてメタクラス階層がクラス階層と完全に対称となっており、最後には\ct{Object class}に辿り着くことが分かるでしょう。

次の例では、クラス階層への問い合わせる方法を紹介します。\figref{metaclassclassclass}が正確であることのデモンストレーションです。
(本当のことを言えば、簡単のため、ちょっとしたウソが混じっています。\ct{Object class superclass -->} \clsind{ProtoObject class}、で、\ct{Class}ではありません。\pharo では、\ct{Class}に辿りつくために、もう一段上がらなければなりません。)

\begin{example}{クラス階層}{@TEST}
TranslucentColor superclass --> Color
Color superclass                   --> Object
\end{example}

\needlines{4}
\begin{example}{クラス階層に並列するメタクラス階層}{@TEST}
TranslucentColor class superclass   --> Color class
Color class superclass                     --> Object class
Object class superclass superclass --> Class    "NB: skip ProtoObject class"
Class superclass                              --> ClassDescription
ClassDescription superclass            --> Behavior
Behavior superclass                         --> Object
\end{example}

\begin{example}{Metaclassのインスタンス}{@TEST}
TranslucentColor class class --> Metaclass
Color class class                   --> Metaclass
Object class class                 --> Metaclass
Behavior class class              --> Metaclass
\end{example}
\begin{example}{Metaclass classはMetaclassのインスタンス}{@TEST}
Metaclass class class --> Metaclass
Metaclass superclass --> ClassDescription
\end{example}

%=================================================================
\section{章のまとめ}
今、あなたは、クラスがどのように組織されており、一貫性のあるオブジェクトモデルがどのような影響を及ぼしているのかということを、以前よりも理解できていることでしょう。もし忘れたり、混乱したりしたときには、いつでもメッセージ・パッシングが鍵であることを思い出してください。メッセージの受け手となるオブジェクトのクラスから、メソッドを探せばよいのです。
これは、\emph{どのオブジェクトでも}同じです。
もしオブジェクトのクラスからメソッドがみつからなければ、そのスーパークラスを探せばよいのです。

\begin{itemize}
\item すべてのクラスはメタクラスのインスタンスです。
	メタクラスは暗黙的なものです。メタクラスはクラス定義をした際、自動的に作成されます。クラス1つにつき、メタクラスは1つです。

\item メタクラス階層は、クラス階層と並列に存在します。
	クラスのメソッド探索は、普通のオブジェクトのメソッド探索と同じく、そのメタクラスの継承チェーンに沿って行います。

\item すべてのメタクラスは、\ct{Class}と\ct{Behavior}を継承します。
	すべてのクラスは\ct{Class}のインスタンスです。メタクラスもクラスなので、\ct{Class}を継承します。
	\ct{Behavior}は、すべてのインスタンスに共通の振る舞いを提供します。

\item すべてのメタクラスは、\ct{Metaclass}のインスタンスです。
	\ct{ClassDescription}は、\ct{Class}、\ct{Metaclass}に共通するすべてを提供します。

\item \ct{Metaclass}のメタクラスは、\ct{Metaclass}のインスタンスです。
	クラス-インスタンスの関係(\emph{instance-of})は、閉じたループを形成します。\ct{Metaclass class class --> Metaclass}です。
\end{itemize}
%=================================================================
\ifx\wholebook\relax\else\end{document}\fi
%=================================================================

%-----------------------------------------------------------------
