% $Author$
% $Date$
% $Revision$

% HISTORY:
% 2006-10-24 - Stef started
% 2006-11-16 - Stef completes first draft
% 2007-04-09 - Andrew review and edit
% 2007-08-23 - Oscar edit
% 2007-09-05 - Andrew edit

%=================================================================
\ifx\wholebook\relax\else
% --------------------------------------------
% Lulu:
	\documentclass[a4paper,10pt,twoside]{book}
	\usepackage[
		papersize={6.13in,9.21in},
		hmargin={.75in,.75in},
		vmargin={.75in,1in},
		ignoreheadfoot
	]{geometry}
	\input{../common.tex}
	\pagestyle{headings}
	\setboolean{lulu}{true}
% --------------------------------------------
% A4:
%	\documentclass[a4paper,11pt,twoside]{book}
%	\input{../common.tex}
%	\usepackage{a4wide}
% --------------------------------------------
    \graphicspath{{figures/} {../figures/}}
	\begin{document}
	\renewcommand{\nnbb}[2]{} % Disable editorial comments
	\sloppy
\fi
%=================================================================
\chapter{クラスとメタクラス}
\chalabel{metaclasses}

\on{The section on responsibilities of Class, Behavior and Metaclass are weak, and need to be fleshed out with convincing examples. Marcus, can you help?}

\charef{model}で見てきたように、\st ではすべてがオブジェクトで、それらはいずれもあるクラスのインスタンスです。クラスもまた例外なくオブジェクトであり、あるクラスのインスタンスです。
このオブジェクトモデルは、簡潔で、シンプルかつエレガントに統一されており、オブジェクト指向プログラミングの本質を捉えています。
しかしながら、慣れていないうちは、この一貫性のために混乱してしまうかもしれません。この章では、それが複雑ではなく、魔法でも特別なものでもない、ただシンプルな規則が一貫して適用されている、ということを示します。
これらの規則を追えば、システムがなぜこのように成り立っているのかが分かるでしょう。

%=================================================================
\section{クラスとメタクラスのルール}

Smalltalkのオブジェクトモデルは、一貫した数少ないコンセプトの上に成り立っています。
Smalltalkの設計者は、オッカムの剃刀と呼ばれる規則を使って、モデルが必要以上に複雑になる要因を削ぎ落としました。
\charef{model}で見た、オブジェクトモデルのルールを思い出しましょう。

\begin{enumerate}[label={\textbf{Rule \arabic{*}}.}, ref={Rule \arabic{*}}, leftmargin=*, widest=10]
% NB: rule labels must not be multiply defined across chapters!
\item{} % \rulelabel{everything}
	すべてのものはオブジェクトである。

\item{} % \rulelabel{instance}
	すべてのオブジェクトは、あるクラスのインスタンスである。

\item{} % \rulelabel{inheritance}
	すべてのクラスは、スーパークラスを持つ。

\item{} % \rulelabel{message}
	すべての出来事は、メッセージを送ることで生じる。

\item{} % \rulelabel{lookup}
	メソッド検索は継承チェーンに沿って行われる。

\end{enumerate}

この章の冒頭で述べたように、\ref{rule:everything}の規則より\emph{クラスもオブジェクト}が成り立ちます。そして、\ref{rule:instance}より、クラスもなんらかのクラスのインスタンスです。
このクラスのクラスを\emph{メタクラス}といいます。
\seclabel{metaclassIntro}
あるクラスの\indmain{メタクラス}は、そのクラスを定義した際に自動的に作られます。
普段はメタクラスのことを気にする必要はないでしょう。
しかしながら、ブラウザで``\subind{browser}{クラス側}''を選んだときには、実は異なるクラスを見ているということを思い返してください。
クラスとそのメタクラスは、前者が後者のインスタンスであるということを除けば、それぞれ別々のクラスであるということです。

クラスとメタクラスを正確に説明するには、\charef{model}のルールに付け足す必要があります。

\begin{enumerate}[label={\textbf{Rule \arabic{*}}.}, ref={Rule \arabic{*}}, leftmargin=*, widest=10]
\setcounter{enumi}{5}
\item{} \rulelabel{metaclass}
	すべてのクラスは、メタクラスのインスタンスである。

\item{} \rulelabel{parallelhierarchy}
	メタクラスの\subind{metaclass}{階層}は、クラスの階層と並列に存在する。

\item{} \rulelabel{behavior}
	すべてのメタクラスは\clsind{Class}と\clsind{Behavior}を継承している。

\item{} \rulelabel{metaclassclass}
	すべてのメタクラスは\clsind{Metaclass}のインスタンスである。

\item{} \rulelabel{metaclassmetaclass}
	\ct{Metaclass}のメタクラスは、\ct{Metaclass}のインスタンスである。

\end{enumerate}

\noindent
これら10個のルールが、\st のオブジェクトモデルのすべてです。

以下では、まず\charef{model}で紹介した5つのルールを小さな例を使いながら説明します。
そして、同じ例を用いて、新しいルールを深く見ていきます。

%=================================================================
\section{Smalltalkオブジェクトモデルの復習}

すべてがオブジェクトなので、「青色オブジェクト」も\st ではオブジェクトです。
\begin{code}{@TEST}
Color blue --> Color blue
\end{code}

\noindent
すべてのオブジェクトはあるクラスのインスタンスです。青色オブジェクトのクラスは\clsind{Color}です。
\begin{code}{@TEST}
Color blue class --> Color
\end{code}

\noindent
興味深いことに、\emph{alpha}(透明度)を指定すると、異なるクラス(\clsind{TranslucentColor})のインスタンスが得られます。
\begin{code}{@TEST}
(Color blue alpha: 0.4) class --> TranslucentColor
\end{code}

\noindent
Morphを作り、その色としてこの半透明オブジェクトを指定することができます。
\begin{code}{}
EllipseMorph new color: (Color blue alpha: 0.4); openInWorld
\end{code}
\noindent
これを実行すると、\figref{translucentEllipse}のようになります。

\begin{center}
\begin{figure}[!ht]
\ifluluelse
	{\centerline {\includegraphics[scale=0.7]{TranslucentEllipse}}}
	{\centerline {\includegraphics[width=8cm]{TranslucentEllipse}}}
\caption{半透明の楕円Morph\figlabel{translucentEllipse}}
\end{figure}
\end{center}

\ref{rule:inheritance}より、すべてのクラスはスーパークラスを持ちます。
\clsind{TranslucentColor}のスーパークラスは\clsind{Color}です。そして、\ct{Color}のスーパークラスは\clsind{Object}です。
\begin{code}{@TEST}
TranslucentColor superclass --> Color
Color superclass                   --> Object
\end{code}

\ref{rule:message}「すべての出来事は、\ind{メッセージを送ること}{}で生じる」より、\mthind{Color class}{blue}は\ct{Color}へのメッセージであると推測できます。\mthind{Object}{class}と\mthind{Color}{alpha:}は青色オブジェクトへのメッセージ、\mthind{Morph}{openInWorld}は楕円Morphへのメッセージ、そして、\mthind{Behavior}{superclass}は\ct{TranslucentColor}と\ct{Color}へのメッセージです。
どのケースでも、メッセージのレシーバーはオブジェクトです。すべてがオブジェクトであることから、いくつかのケースではレシーバーがクラスとなっています。

\ref{rule:lookup}「メソッド検索は継承チェーンに沿って行われる」より、\ct{Color blue alpha: 0.4}で得られたオブジェクトに\ct{class}メッセージを送った場合、メッセージは、対応するメソッドが見つかる\ct{Object}で取り扱われます。\figref{classmessage}をご覧ください。

\begin{center}
\begin{figure}[!ht]
\ifluluelse
	{\centerline{\includegraphics[width=\textwidth]{TranslucentClassMessage}}}
	{\centerline{\includegraphics[width=0.8\textwidth]{TranslucentClassMessage}}}
\caption{半透明色オブジェクトへのメッセージ送信\figlabel{classmessage}}
\end{figure}
\end{center}

この図は\emphind{is-a}の継承関係を示したものです。
\ct{Color blue alpha: 0.4}で得られたオブジェクトは、\ct{TranslucentColor}のインスタンスですが、同時に\ct{Color}オブジェクトであるとも、\ct{Object}オブジェクトであるとも言えます。これらのクラスで定義されたすべてのメッセージに応答することができるからです。
さらに補足すると、\mthind{Object}{isKindOf:}メッセージをオブジェクトに送ることで、レシーバーが引数で与えられたクラスと\emph{is a}の関係にあるかどうかを調べることができます。
\begin{code}{@TEST | translucentBlue |}
translucentBlue := Color blue alpha: 0.4.
translucentBlue isKindOf: TranslucentColor --> true
translucentBlue isKindOf: Color                    --> true
translucentBlue isKindOf: Object                  --> true
\end{code}

%=================================================================
\section{すべてのクラスは、メタクラスのインスタンス}

% \ruleref{metaclass}

\secref{metaclassIntro}で述べたように、インスタンスとしてクラスを持つようなクラスは\ind{メタクラス}{}と呼ばれます。

\paragraph{メタクラスは暗黙的な存在です。}
メタクラスはクラスを定義すると自動的に作成されます。これはプログラマが気にする必要はないことなので、\emph{暗黙的}であると言われます。作成されるクラスごとに、それに対応する\subind{metaclass}{暗黙的}なメタクラスが一つだけ作成されます。
% At a more advanced level, this implies that sharing between metaclasses is difficult except by subclassing. 

通常のクラスはグローバル変数によって参照することができますが、メタクラスは匿名の存在です。しかし、クラスを通して、いつでもメタクラスのインスタンスを参照することができます。例えば、\clsind{Color}は\clsind{Color class}のインスタンスであり、\ct{Object}は、\clsind{Object class}のインスタンスです。
\begin{code}{@TEST}
Color class   --> Color class
Object class --> Object class
\end{code}

\noindent
\figref{translucentmetaclasses}は、クラスがその(\subind{metaclass}{匿名}の)メタクラスのインスタンスであることを示しています。

\begin{center}
\begin{figure}[!ht]
\ifluluelse
	{\centerline {\includegraphics[width=\textwidth]{TranslucentMetaclasses}}}
	{\centerline {\includegraphics[width=0.8\textwidth]{TranslucentMetaclasses}}}
\caption{TranslucentColorのメタクラスと、そのスーパークラス\figlabel{translucentmetaclasses}}
\end{figure}
\end{center}

クラスもまたオブジェクトであることより、それらにメッセージを送って情報を得ることも簡単にできます。以下をご覧ください。

\begin{code}{@TEST}
Color subclasses                           --> {TranslucentColor}
TranslucentColor subclasses         --> #()
TranslucentColor allSuperclasses  --> an OrderedCollection(Color Object ProtoObject)
TranslucentColor instVarNames     --> #('alpha')
TranslucentColor allInstVarNames --> #('rgb' 'cachedDepth' 'cachedBitPattern' 'alpha')
TranslucentColor selectors             -->  an IdentitySet(#pixelWord32 #asNontranslucentColor #privateAlpha #pixelValueForDepth: #isOpaque #isTranslucentColor #storeOn: #pixelWordForDepth: #scaledPixelValue32 #alpha #bitPatternForDepth: #hash #isTransparent #isTranslucent #balancedPatternForDepth: #setRgb:alpha: #alpha: #storeArrayValuesOn:)
\end{code}
\cmindex{Class}{subclasses}
\cmindex{Behavior}{allSuperclasses}
\cmindex{Behavior}{instVarNames}
\cmindex{Behavior}{allInstVarNames}
\cmindex{Behavior}{selectors}

%=================================================================
\section{メタクラス階層は、クラス階層に並列に存在する}
% \ruleref{parallelhierarchy}

\ref{rule:parallelhierarchy}より、メタクラスのスーパークラスには任意のクラスがなれるわけではありません。メタクラスのスーパクラスとして許されているのは、メタクラスの単一インスタンスであるクラスのスーパークラスのメタクラスとなります。

\begin{code}{@TEST}
TranslucentColor class superclass --> Color class
TranslucentColor superclass class --> Color class
\end{code}

\noindent
メタクラス\subindmain{metaclass}{階層}はクラス階層と並列に存在するというのは、この事実のことです。\figref{parallelHierarchies}に、\clsind{TranslucentColor}の階層がどのようになっているかを示します。

\begin{center}
\begin{figure}[!ht]
\ifluluelse
	{\centerline {\includegraphics[width=\textwidth]{TranslucentMetaclassHierarchy}}}
	{\centerline {\includegraphics[width=0.8\textwidth]{TranslucentMetaclassHierarchy}}}
\caption{メタクラス階層と、並列に存在するクラス階層\figlabel{parallelHierarchies}}
\end{figure}
\end{center}

\begin{code}{@TEST}
TranslucentColor class                                     --> TranslucentColor class
TranslucentColor class superclass                   --> Color class
TranslucentColor class superclass superclass --> Object class
\end{code}

\paragraph{クラスとオブジェクトの一貫性}
ここで一息入れて考え直してみると、オブジェクトでもクラスでもメッセージを送る場合に区別がないということは、興味深いことです。どちらのケースも、メソッド検索はメッセージのレシーバークラスから始まり、継承チェーンを辿っていきます。

従って、クラスに送ったメッセージの検索は、メタクラスの階層チェーンを辿らなければなりません。例えば、\mthind{Color class}{blue}メソッドは\ct{Color}の\subind{browser}{クラス側}に定義されています。\ct{TranslucentColor}に\ct{blue}メッセージを送信した場合にも、他のメッセージとまったく同様にメソッドを検索します。すなわち、検索は\ct{TranslucentColor class}でまず行われ、対応するメソッドが見つかるまでメタクラス階層を\ct{Color class}まで辿っていきます。\figref{metaclasslookup}をご覧ください。

\begin{code}{@TEST}
TranslucentColor blue --> Color blue
\end{code}
\noindent
\ct{TranslucentColor}のインスタンスではなく、普通の\ct{Color blue}オブジェクトが得られることに注意してください\,---\,不思議なところはどこにもありません!

\begin{center}
\begin{figure}[!ht]
\ifluluelse
	{\centerline {\includegraphics[width=\textwidth]{TranslucentColorBlue}}}
	{\centerline {\includegraphics[width=0.8\textwidth]{TranslucentColorBlue}}}
\caption{通常のオブジェトと同様のメッセージ検索\figlabel{metaclasslookup}}
\end{figure}
\end{center}

すなわち、Smalltalkのメソッド\subind{method}{検索}は、唯一の統一された方法で行われていると言えます。クラスもまた単なるオブジェクトであり、他のオブジェクトと同じように振る舞います。
クラスは、新しいインスタンスを作る能力を持ちますが、これは、クラスがたまたま\ct{new}という新しいインスタンスを作ることのできるメソッドを持ち、それに対応する\ct{new}メッセージに応答できるからです。
通常はクラス以外のオブジェクトはこのメッセージに応答できませんが、もし十分な理由があれば、\ct{new}メソッドをメタクラス以外に追加することを止めるものはなにもありません。

クラスもオブジェクトなので、同様にインスペクトすることができます。
\index{inspector}

\dothis{\ct{Color blue}と\ct{Color}をインスペクトしてみましょう。}

\noindent
前者は\ct{Color}のインスタンス、後者は\ct{Color}自身をインスペクトしていることに注意してください。インスペクタのタイトルバーにはどちらの場合も\emph{class}の名前が表示されるので、少し混乱するかもしれません。

\figref{inspectingColor}にあるように、\ct{Color}のインスペクタによって、スーパークラス(superclass)、インスタンス変数(instanceVariables)、メソッド\subind{method}{辞書}(methodDict)などを見ることできます。

\begin{center}
\begin{figure}[!ht]
\ifluluelse
	{\centerline{\includegraphics[width=\textwidth]{InspectingColor}}}
	{\centerline{\includegraphics[width=10cm]{InspectingColor}}}
\caption{クラスもオブジェクト\figlabel{inspectingColor}}
\end{figure}
\end{center}

%=================================================================
\section{すべてのメタクラスは、\lct{Class}と\lct{Behavior}を継承する}
% \ruleref{behavior}

すべての\ind{メタクラス}はクラスなので(\emphind{is-a})、\clsind{Class}を継承しています。\clsind{Class}のスーパークラスは\clsind{ClassDescription}、さらにそのスーパークラスは\clsind{Behavior}です。\st ではすべてがオブジェクトなので(\emph{is-a})、最終的には、これらのクラスも\ct{Object}を継承していることになります。\figref{inheritbehavior}に、最終的な図を示します。

\begin{center}
\begin{figure}
\ifluluelse
	{\centerline{\includegraphics[width=\textwidth]{TranslucentBehavior}}}
	{\centerline{\includegraphics[width=0.8\textwidth]{TranslucentBehavior}}}
\caption{メタクラスは、\lct{Class}と\lct{Behavior}を継承する\figlabel{inheritbehavior}}
\end{figure}
\end{center}

\paragraph{\lct{new}はどこに定義されているのでしょうか?}
\ct{new}がどこに定義されていて、どのように検索されるのかについて考えてみることが、メタクラスが\ct{Class}と\ct{Behavior}を継承しているということの重要性を理解する手助けになります。\ct{new}メッセージがクラスに送られたときは、\figref{sendingnew}のように、メタクラス継承チェインを辿って、スーパークラスである、\ct{Class}、\ct{ClassDescription}、そして\ct{Behavior}の順にメソッド検索が行われます。

\emph{「どこに\ct{new}が定義されているのか」}という問いは決定的に重要です。\mthind{Behavior}{new}は、最上層のものとしてまず\ct{Behavior}で定義されています。そして、必要に応じてプログラマが定義したクラスのメタクラスを含むすべての\ct{Behavior}のサブクラスで再定義することができます。\ct{new}メッセージをクラスに送った場合、まずは通常通りそのクラスのメタクラスから検索が開始され、そこで見つからなければ、\ct{new}を再定義しているクラスまたは最終的には\ct{Behavior}まで、順にスーパークラスを辿って検索されます。

ここでは、\ct{TranslucentColor new}を評価した結果は\clsind{TranslucentColor}のインスタンスであり、メソッドが\ct{Behavior}まで辿った時にようやく見つかった場合であっても\ct{Behavior}のインスタンスになるわけでは\emph{ない}ことに注意してください。\ct{new}は常に\self のインスタンスを返します。これは\ct{new}が他のクラスで実装されている場合でも同じです。

\begin{code}{@TEST}
TranslucentColor new class --> TranslucentColor    "Behaviorのインスタンスではない!"
\end{code}

\begin{center}
\begin{figure}
\ifluluelse
	{\centerline{\includegraphics[width=\textwidth]{TranslucentSendingNew}}}
	{\centerline{\includegraphics[width=0.8\textwidth]{TranslucentSendingNew}}}
\caption{\lct{new}は一般的なメッセージで、メタクラスチェインを辿って検索されます\figlabel{sendingnew}}
\end{figure}
\end{center}

よくある間違いは、\ct{new}メッセージのレシーバーとなっているクラスのスーパークラスをブラウズして定義を探そうとすることです。引数で与えられた大きさのオブジェクトを生成するための標準的なメッセージ\ct{new:}も同様です。たとえば、\lct{Array new: 4}は、4つの要素を保つ配列を生成しますが、\ct{Array}とそのスーパークラスからは、このメソッドの定義を見つけることは出来ないでしょう。代わりに、実際のメソッド検索が行われる\ct{Array class}とそのスーパークラス階層に目を向けてみるべきです。

\on{The text below needs more details and convincing examples ...}

\paragraph{\lct{Behavior}、\lct{ClassDescription}および\lct{Class}の責務について}
\clsind{Behavior}は、インスタンスを持つための必要最小限の状態を保持しています。すなわち、スーパークラスへのリンク、メソッド辞書、およびクラスのフォーマット(ポインターオブジェクト・非ポインターオブジェクトの区別、コンパクトクラス・非コンパクトクラスの区別、およびインスタンス変数の個数をコード化したもの)などです。
\on{not sure I understand the last point}
\ct{Behavior}は\ct{Object}を継承しているので、それ自身、およびそのすべてのサブクラスは、一般のオブジェクトとしても振る舞うことができます。

\ct{Behavior}は、コンパイラへの基本的なインターフェースも提供します。メソッド辞書の生成、メソッドのコンパイル、インスタンスの生成(\ie \mthind{Behavior}{new}, \mthind{Behavior}{basicNew}, \mthind{Behavior}{new:}, \mthind{Behavior}{basicNew:})、クラス階層の操作(\ie \mthindex{Behavior}{superclass:} \lct{superclass:}, \mthind{Behavior}{addSubclass:})、メソッドへのアクセス(\ie \mthind{Behavior}{selectors}, \lmthind{Behavior}{allSelectors}, \mthind{Behavior}{compiledMethodAt:})、インスタンス、およびインスタンス変数へのアクセス(\ie \mthind{Behavior}{allInstances}, \mthind{Behavior}{instVarNames}\ldots)、クラス階層へのアクセス(\ie \mthind{Behavior}{superclass}, \mthind{Behavior}{subclasses})、クラス階層関連の問い合わせ(\ie \mthind{Behavior}{hasMethods}, \mthind{Behavior}{includesSelector}, \mthind{Behavior}{canUnderstand:}, \mthind{Behavior}{inheritsFrom:}, \mthind{Behavior}{isVariable})などを提供します。

\clsind{ClassDescription}は抽象クラスであり、\clsind{Class}と\clsind{Metaclass}という直接のサブクラスが必要とする機能を提供しています。
\clsind{ClassDescription}は\ct{Behavior}で提供されている最小限の基礎に以下のような多くの機能を追加しています。それらは、
インスタンス変数の名前の管理、
プロトコルによるメソッドの分類、
チェンジセットや変更ログの取り扱い、
そしてファイルアウト機能の大部分などです。

\clsind{Class}は、すべてのクラスに共通する一般的な振る舞いを提供します。
それらは、クラス名、メソッドのコンパイル、メソッドの保管、インスタンス変数に関わる機能です。
また、クラス変数名、共有プール変数に関わる機能(\mthind{Class}{addClassVarName:}, \mthind{Class}{addSharedPool:}, \mthind{Class}{initialize})も持ちます。
メタクラスは通常のクラスという形でインスタンスを持ち、そのインスタンスにクラスとしてのサービスを提供しなくてはならないので、すべてのメタクラスは\ct{Class}を継承することになります。

%=================================================================
\section{すべてのメタクラスは\lct{Metaclass}のインスタンス}
% \ruleref{metaclassclass}

メタクラスもまたオブジェクトです。\figref{metaclassclass}が示すように、それらは\clsind{Metaclass}のインスタンスです。\ct{Metaclass}のインスタンスは全部が匿名のメタクラスです。そのようなメタクラスのそれぞれは、ただ一つだけインスタンスを持ちます。つまるところ、それがクラスというわけです。

\begin{center}
\begin{figure}
\ifluluelse
	{\centerline{\includegraphics[width=\textwidth]{TranslucentMetaclassClass}}}
	{\centerline{\includegraphics[width=0.8\textwidth]{TranslucentMetaclassClass}}}
\caption{すべてのメタクラスは\lct{Metaclass}のインスタンス\figlabel{metaclassclass}}
\end{figure}
\end{center}

\ct{Metaclass}は、一般的なメタクラスの振る舞いを提供します。
例えば、メタクラスの唯一のインスタンスを生成のためのメソッド(\lct{sub\-class\-Of:})、メタクラスの唯一のインスタンス、クラス変数の初期化、メタクラスのインスタンス、メソッドのコンパイル、クラスの情報(例:継承関係、インスタンス変数、\etc)などです。
% (\ct{name:inEnvironment:subclassOf:})
% Actually, this is in ClassBuilder, it seems!
% (different semantics can be introduced)
\ab{This is too cryptic for me.}

%=================================================================
\section{\lct{Metaclass}のメタクラスは\lct{Metaclass}のインスタンス}
% \ruleref{metaclassmetaclass}

残る質問は1つだけです。\clsind{Metaclass class}のクラスは何でしょうか?

答えは簡単です。それはメタクラスで、\ct{Metaclass}のインスタンスであり、システム内の他のすべてのメタクラスと同じです。\figref{metaclassclassclass}をご覧ください。

\begin{center}
\begin{figure}
\ifluluelse
	{\centerline{\includegraphics[width=\textwidth]{TranslucentMetaclassClassClass}}}
	{\centerline{\includegraphics[width=0.8\textwidth]{TranslucentMetaclassClassClass}}}
\caption{すべてのメタクラスは、\lct{Metaclass}のインスタンスです。\lct{Metaclass}のメタクラスも同様です\figlabel{metaclassclassclass}}
\end{figure}
\end{center}

この図は、すべてのメタクラスは\ct{Metaclass}のインスタンスであり、\ct{Metaclass}のメタクラスもまた同様であるということを示しています。
\ref{fig:metaclassclass}と\ref{fig:metaclassclassclass}を比較すると、メタクラスの階層は\ct{Object class}に辿り着くところまでクラスの階層を完全に鏡に映したようになっていることが分かるでしょう。

次の例で、\figref{metaclassclassclass}が正確であることを示すために。クラス階層に関するどのような問い合わせができるのかを示します。
(本当のことを言えば、簡略化のためのちょっとしたウソが混じっています。\ct{Object class superclass -->} \clsind{ProtoObject class}、で\ct{Class}ではありません。\pharo では、\ct{Class}に辿りつくために、もう一段上がらなければなりません。)

\begin{example}{クラス階層}{@TEST}
TranslucentColor superclass --> Color
Color superclass                   --> Object
\end{example}

\needlines{4}
\begin{example}{クラス階層に並列するメタクラス階層}{@TEST}
TranslucentColor class superclass   --> Color class
Color class superclass                     --> Object class
Object class superclass superclass --> Class    "NB: skip ProtoObject class"
Class superclass                              --> ClassDescription
ClassDescription superclass            --> Behavior
Behavior superclass                         --> Object
\end{example}

\begin{example}{Metaclassのインスタンス}{@TEST}
TranslucentColor class class --> Metaclass
Color class class                   --> Metaclass
Object class class                 --> Metaclass
Behavior class class              --> Metaclass
\end{example}
\begin{example}{Metaclass classはMetaclassのインスタンス}{@TEST}
Metaclass class class --> Metaclass
Metaclass superclass --> ClassDescription
\end{example}

%=================================================================
\section{章のまとめ}
ここまでで、あなたはクラスがどのように組織されており、一貫性のあるオブジェクトモデルがどのような効果を及ぼしているのかということを、以前よりも理解できていることでしょう。もし忘れたり混乱してしまったりしたときには、いつでもメッセージ・パッシングが鍵であることを思い出してください。メッセージのレシーバーのクラスから、メソッドを探せばよいのです。
これは\emph{どのオブジェクトでも}同じです。
もしレシーバーのクラスからメソッドがみつからなければそのスーパークラスを探せばよいのです。

\begin{itemize}
\item すべてのクラスはメタクラスのインスタンスです。
	メタクラスは暗黙的に定義されます。メタクラスはクラスを定義をしたときに、そのクラスを唯一のインスタンスとするように自動的に作成されます。

\item メタクラスの階層はクラスの階層をなぞるようになっています。
	クラスへのメッセージのメソッド探索は、普通のオブジェクトのメソッド探索と同様にそのメタクラスの継承チェーンに沿って行います。

\item すべてのメタクラスは\ct{Class}と\ct{Behavior}を継承します。
	すべてのクラスは\ct{Class}のインスタンスです(\emph{is-a})。メタクラスもクラスなので\ct{Class}を継承しています。
	\ct{Behavior}はインスタンスを持つようなすべての存在に共通の振る舞いを提供します。

\item すべてのメタクラスは\ct{Metaclass}のインスタンスです。
	\ct{ClassDescription}は\ct{Class}および\ct{Metaclass}に共通することがらすべてを提供します。

\item \ct{Metaclass}のメタクラスは\ct{Metaclass}のインスタンスです。
	クラス-インスタンスの関係(\emph{instance-of})は閉じたループを形成します。すなわち、\ct{Metaclass class class --> Metaclass}ということです。
\end{itemize}
%=================================================================
\ifx\wholebook\relax\else\end{document}\fi
%=================================================================

%-----------------------------------------------------------------
