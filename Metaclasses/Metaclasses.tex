% $Author$
% $Date$
% $Revision$

% HISTORY:
% 2006-10-24 - Stef started
% 2006-11-16 - Stef completes first draft
% 2007-04-09 - Andrew review and edit
% 2007-08-23 - Oscar edit
% 2007-09-05 - Andrew edit

%=================================================================
\ifx\wholebook\relax\else
% --------------------------------------------
% Lulu:
	\documentclass[a4paper,10pt,twoside]{book}
	\usepackage[
		papersize={6.13in,9.21in},
		hmargin={.75in,.75in},
		vmargin={.75in,1in},
		ignoreheadfoot
	]{geometry}
	\input{../common.tex}
	\pagestyle{headings}
	\setboolean{lulu}{true}
% --------------------------------------------
% A4:
%	\documentclass[a4paper,11pt,twoside]{book}
%	\input{../common.tex}
%	\usepackage{a4wide}
% --------------------------------------------
    \graphicspath{{figures/} {../figures/}}
	\begin{document}
	\renewcommand{\nnbb}[2]{} % Disable editorial comments
	\sloppy
\fi
%=================================================================
\chapter{クラスとメタクラス}
\chalabel{metaclasses}

\on{The section on responsibilities of Class, Behavior and Metaclass are weak, and need to be fleshed out with convincing examples. Marcus, can you help?}

\charef{model}で見てきたように、\st ではすべてがオブジェクトであり、いずれも何らかのクラスのインスタンスです。
クラスも例外ではありません。つまり、クラスはオブジェクトであり、クラスオブジェクトは他のクラスのインスタンスとなっています。
このオブジェクトモデルは、簡潔で、シンプルかつエレガントに統一されており、オブジェクト指向プログラミングの本質を捉えています。
しかし、不慣れなうちはこの一貫性のために混乱してしまうかもしれません。この章では、それが複雑ではなく、``魔法''でも特別なものでもない、ただシンプルな規則が一貫して適用されているということを示します。
これらの規則を追えば、システムがなぜこのように成り立っているのかがわかるでしょう。

%=================================================================
\section{クラスとメタクラスのルール}

Smalltalkのオブジェクトモデルは、一貫した数少ないコンセプトの上に成り立っています。
Smalltalkの設計者は、オッカムの剃刀と呼ばれる規則を使って、モデルが必要以上に複雑になる要因を削ぎ落としました。

\charef{model}で見た、オブジェクトモデルのルールを思い出しましょう。

\begin{enumerate}[label={\textbf{Rule \arabic{*}}.}, ref={Rule \arabic{*}}, leftmargin=*, widest=10]
% NB: rule labels must not be multiply defined across chapters!
\item{} % \rulelabel{everything}
	すべてのものはオブジェクトである。

\item{} % \rulelabel{instance}
	すべてのオブジェクトは、あるクラスのインスタンスである。

\item{} % \rulelabel{inheritance}
	すべてのクラスは、スーパークラスを持つ。

\item{} % \rulelabel{message}
	すべての出来事は、メッセージを送ることで生じる。

\item{} % \rulelabel{lookup}
	メソッド探索は継承チェーンに沿って行われる。

\end{enumerate}

この章の冒頭で述べたように、\ref{rule:everything}の規則より\emph{クラスもオブジェクトである}が成り立ちます。そして、\ref{rule:instance}より、クラスもなんらかのクラスのインスタンスであることがわかります。
このクラスのクラスを\emph{メタクラス}といいます。
\seclabel{metaclassIntro}
あるクラスの\indmain{メタクラス}は、そのクラスを定義した際に自動的に作られます。
普段はメタクラスのことを気にする必要はないでしょう。
しかし、ブラウザで``\subind{browser}{クラス側}''を選んだときには、実は異なるクラスを見ているということを思い返してください。
クラスとそのメタクラスは、前者が後者のインスタンスであるということを除けば、それぞれ別々のクラスであるということです。

クラスとメタクラスを正確に説明するには、\charef{model}のルールに付け足す必要があります。

\begin{enumerate}[label={\textbf{Rule \arabic{*}}.}, ref={Rule \arabic{*}}, leftmargin=*, widest=10]
\setcounter{enumi}{5}
\item{} \rulelabel{metaclass}
	すべてのクラスはメタクラスのインスタンスである。

\item{} \rulelabel{parallelhierarchy}
	メタクラスの\subind{metaclass}{階層}はクラスの階層と並列に存在する。

\item{} \rulelabel{behavior}
	すべてのメタクラスは\clsind{Class}と\clsind{Behavior}を継承している。

\item{} \rulelabel{metaclassclass}
	すべてのメタクラスは\clsind{Metaclass}のインスタンスである。

\item{} \rulelabel{metaclassmetaclass}
	\ct{Metaclass}のメタクラスは、\ct{Metaclass}のインスタンスである。

\end{enumerate}

\noindent
これら10個のルールが、\st のオブジェクトモデルのすべてです。

以下では、まず\charef{model}で紹介した5つのルールを小さな例を使いながら説明します。
そして、同じ例を用いて、新しいルールを深く見ていきます。

%=================================================================
\section{Smalltalkオブジェクトモデルの復習}

すべてがオブジェクトなので、「青色オブジェクト」も\st ではオブジェクトです。
\begin{code}{@TEST}
Color blue --> Color blue
\end{code}

\noindent
すべてのオブジェクトはあるクラスのインスタンスです。青色オブジェクトのクラスは\clsind{Color}です。
\begin{code}{@TEST}
Color blue class --> Color
\end{code}

\noindent
興味深いことに、\emph{alpha}(透明度)を指定すると、異なるクラス(\clsind{TranslucentColor})のインスタンスが得られます。
\begin{code}{@TEST}
(Color blue alpha: 0.4) class --> TranslucentColor
\end{code}

\noindent
Morphを作り、その色としてこの半透明オブジェクトを指定することができます。
\begin{code}{}
EllipseMorph new color: (Color blue alpha: 0.4); openInWorld
\end{code}
\noindent
これを実行すると、\figref{translucentEllipse}のようになります。

\begin{center}
\begin{figure}[!ht]
\ifluluelse
	{\centerline {\includegraphics[scale=0.7]{TranslucentEllipse}}}
	{\centerline {\includegraphics[width=8cm]{TranslucentEllipse}}}
\caption{半透明の楕円モーフ。\figlabel{translucentEllipse}}
\end{figure}
\end{center}

\ref{rule:inheritance}より、どのクラスもスーパークラスを持ちます。
\clsind{TranslucentColor}のスーパークラスは\clsind{Color}です。そして、\ct{Color}のスーパークラスは\clsind{Object}です。
\begin{code}{@TEST}
TranslucentColor superclass --> Color
Color superclass                   --> Object
\end{code}

\ref{rule:message}「すべての出来事は、\ind{メッセージを送ること}{}で生じる」より、以下のことが導きだせます。\mthind{Color class}{blue}は\ct{Color}へのメッセージ、\mthind{Object}{class}と\mthind{Color}{alpha:}は青色オブジェクトへのメッセージ、\mthind{Morph}{openInWorld}は楕円Morphへのメッセージ、そして、\mthind{Behavior}{superclass}は\ct{TranslucentColor}と\ct{Color}へのメッセージです。
すべてがオブジェクトなので、どのケースでもメッセージのレシーバはオブジェクトです。ただし、いくつかのケースではレシーバがクラスとなっています。

\ref{rule:lookup}「メソッド探索は継承チェーンに沿って行われる」より、\ct{Color blue alpha: 0.4}で得られたオブジェクトに\ct{class}メッセージを送ると、\figref{classmessage}で示すように、メッセージは対応するメソッドが見つかる\ct{Object}で取り扱われます。

\begin{center}
\begin{figure}[!ht]
\ifluluelse
	{\centerline{\includegraphics[width=\textwidth]{TranslucentClassMessage}}}
	{\centerline{\includegraphics[width=0.8\textwidth]{TranslucentClassMessage}}}
\caption{半透明色オブジェクトへのメッセージ送信。\figlabel{classmessage}}
\end{figure}
\end{center}

この図は\emphind{is-a}の継承関係を示したものです。
図のように(\ct{Color blue alpha: 0.4}で得られた)translucentBlueオブジェクトは、\ct{TranslucentColor}のインスタンス\emph{である}わけですが、同時に\ct{Color}オブジェクト\emph{である}とも、\ct{Object}オブジェクト\emph{である}とも言えるのは、これらのクラスで定義されたすべてのメッセージに応答することができるからです。
さらに補足すると、\mthind{Object}{isKindOf:}メッセージをオブジェクトに送ることで、レシーバが引数で与えられたクラスと\emph{である(is a)}の関係にあるかどうかを調べることができます。
\begin{code}{@TEST | translucentBlue |}
translucentBlue := Color blue alpha: 0.4.
translucentBlue isKindOf: TranslucentColor --> true
translucentBlue isKindOf: Color                    --> true
translucentBlue isKindOf: Object                  --> true
\end{code}

%=================================================================
\section{すべてのクラスはメタクラスのインスタンスである}

% \ruleref{metaclass}

\secref{metaclassIntro}で述べたように、そのインスタンスがクラスであるようなクラスは\ind{メタクラス}{}と呼ばれます。

\paragraph{メタクラスは暗黙的な存在である}
メタクラスはクラスを定義すると自動的に作成されます。これはプログラマが気にする必要はないので、\emph{暗黙的}であると言えます。作成されるクラスごとに、そのクラスを唯一のインスタンスとして持つ\subind{metaclass}{暗黙的}なメタクラスが作成されます。
% At a more advanced level, this implies that sharing between metaclasses is difficult except by subclassing. 

通常のクラスはグローバル変数によって参照できますが、メタクラスは匿名の存在です。
しかし、インスタンスであるクラスを通して、いつでもメタクラスを参照できます。
例えば、\clsind{Color}のクラスは\clsind{Color class}のインスタンスであり、\ct{Object}のクラスは、\clsind{Object class}のインスタンスです。
\begin{code}{@TEST}
Color class   --> Color class
Object class --> Object class
\end{code}

\noindent
\figref{translucentmetaclasses}は、クラスがその(\subind{metaclass}{匿名}の)メタクラスのインスタンスであることを示しています。

\begin{center}
\begin{figure}[!ht]
\ifluluelse
	{\centerline {\includegraphics[width=\textwidth]{TranslucentMetaclasses}}}
	{\centerline {\includegraphics[width=0.8\textwidth]{TranslucentMetaclasses}}}
\caption{TranslucentColorのメタクラスと、そのスーパークラス。\figlabel{translucentmetaclasses}}
\end{figure}
\end{center}

クラスもまたオブジェクトであることから、メッセージを送って情報を得ることも簡単にできます。以下を見てみましょう。

\begin{code}{@TEST}
Color subclasses                           --> {TranslucentColor}
TranslucentColor subclasses         --> #()
TranslucentColor allSuperclasses  --> an OrderedCollection(Color Object ProtoObject)
TranslucentColor instVarNames     --> #('alpha')
TranslucentColor allInstVarNames --> #('rgb' 'cachedDepth' 'cachedBitPattern' 'alpha')
TranslucentColor selectors             -->  an IdentitySet(#pixelWord32 #asNontranslucentColor #privateAlpha #pixelValueForDepth: #isOpaque #isTranslucentColor #storeOn: #pixelWordForDepth: #scaledPixelValue32 #alpha #bitPatternForDepth: #hash #isTransparent #isTranslucent #balancedPatternForDepth: #setRgb:alpha: #alpha: #storeArrayValuesOn:)
\end{code}
\cmindex{Class}{subclasses}
\cmindex{Behavior}{allSuperclasses}
\cmindex{Behavior}{instVarNames}
\cmindex{Behavior}{allInstVarNames}
\cmindex{Behavior}{selectors}

%=================================================================
\section{メタクラスの階層はクラスの階層と並列に存在する}
% \ruleref{parallelhierarchy}

\ref{rule:parallelhierarchy}は、メタクラスのスーパークラスに任意のクラスがなれるわけではないことを示しています。メタクラスのスーパクラスとして許されるのは、メタクラスの唯一のインスタンスであるクラスのスーパークラスのメタクラスです。

\begin{code}{@TEST}
TranslucentColor class superclass --> Color class
TranslucentColor superclass class --> Color class
\end{code}

\noindent
これが、「メタクラス\subindmain{metaclass}{階層}はクラス階層と並列に存在する」ことの意味です。 \figref{parallelHierarchies}は、\clsind{TranslucentColor}の階層において、どのようになっているかを示しています。

\begin{center}
\begin{figure}[!ht]
\ifluluelse
	{\centerline {\includegraphics[width=\textwidth]{TranslucentMetaclassHierarchy}}}
	{\centerline {\includegraphics[width=0.8\textwidth]{TranslucentMetaclassHierarchy}}}
\caption{メタクラス階層と並列に存在するクラス階層。\figlabel{parallelHierarchies}}
\end{figure}
\end{center}

\begin{code}{@TEST}
TranslucentColor class                                     --> TranslucentColor class
TranslucentColor class superclass                   --> Color class
TranslucentColor class superclass superclass --> Object class
\end{code}

\paragraph{クラスとオブジェクトの一貫性}
ここで一息入れて考え直してみると、オブジェクトであろうとクラスであろうとメッセージを送るのに違いがないことは興味深いことです。
どちらのケースも、メソッド探索はメッセージのレシーバクラスから始まり、継承チェーンをたどっていきます。

従って、クラスに送ったメッセージの探索は、メタクラスの階層チェーンをたどらなければなりません。
例えば、\mthind{Color class}{blue}メソッドは\ct{Color}の\subind{browser}{クラス側}に定義されています。
\ct{TranslucentColor}に\ct{blue}メッセージを送信した場合は、他のメッセージとまったく同様にメソッドを探索します。
すなわち、探索は\ct{TranslucentColor class}でまず行われ、対応するメソッドが見つかるまでメタクラス階層を\ct{Color class}までたどっていきます。(\figref{metaclasslookup}を参照)

\begin{code}{@TEST}
TranslucentColor blue --> Color blue
\end{code}
\noindent
\ct{TranslucentColor}のインスタンスではなく、普通の\ct{Color blue}オブジェクトが得られることに注意してください\,---\,不思議なところはどこにもありません!

\begin{center}
\begin{figure}[!ht]
\ifluluelse
	{\centerline {\includegraphics[width=\textwidth]{TranslucentColorBlue}}}
	{\centerline {\includegraphics[width=0.8\textwidth]{TranslucentColorBlue}}}
\caption{クラスに対するメッセージ探索は、通常のオブジェトの場合と同じ。\figlabel{metaclasslookup}}
\end{figure}
\end{center}

すなわち、Smalltalkでは統一された方法でメソッド\subind{method}{探索}が行われていると言えます。クラスもまた単なるオブジェクトであり、他のオブジェクトと同じように振る舞います。
クラスは、新しいインスタンスを作る能力を持ちますが、これは、クラスがたまたま\ct{new}というメッセージに反応することができて、\ct{new}に対するメソッドが新しいインスタンスの作り方を知っているからにすぎません。
通常はクラス以外のオブジェクトはこのメッセージに応答できませんが、そうする理由があるなら、\ct{new}メソッドをメタクラス以外に追加することを止めるものはなにもありません。

クラスもオブジェクトなので、インスペクトすることもできます。
\index{inspector}

\dothis{\ct{Color blue}と\ct{Color}をインスペクトしてみましょう。}

\noindent
前者は\ct{Color}のインスタンスを、後者は\ct{Color}自身をインスペクトしていることに注意してください。
どちらの場合もインスペクタのタイトルバーにはオブジェクトの\emph{クラス}が表示されるので、少し混乱するかもしれません。

\figref{inspectingColor}にあるように、\ct{Color}のインスペクタによって、スーパークラス(superclass)、インスタンス変数(instanceVariables)、メソッド\subind{method}{辞書}(methodDict)などを見ることできます。

\begin{center}
\begin{figure}[!ht]
\ifluluelse
	{\centerline{\includegraphics[width=\textwidth]{InspectingColor}}}
	{\centerline{\includegraphics[width=10cm]{InspectingColor}}}
\caption{クラスもオブジェクトである。\figlabel{inspectingColor}}
\end{figure}
\end{center}

%=================================================================
\section{すべてのメタクラスは\lct{Class}と\lct{Behavior}を継承している}
% \ruleref{behavior}

どの\ind{メタクラス}もクラスである(\emphind{is-a})ため、\clsind{Class}を継承しています。
\clsind{Class}のスーパークラスは\clsind{ClassDescription}であり、さらにそのスーパークラスは\clsind{Behavior}です。
\st ではすべてがオブジェクトである(\emph{is-a})ため、結局これらのクラスも\ct{Object}を継承していることになります。
\figref{inheritbehavior}に、完全な図を示します。

\begin{center}
\begin{figure}
\ifluluelse
	{\centerline{\includegraphics[width=\textwidth]{TranslucentBehavior}}}
	{\centerline{\includegraphics[width=0.8\textwidth]{TranslucentBehavior}}}
\caption{メタクラスは、\lct{Class}と\lct{Behavior}を継承している。\figlabel{inheritbehavior}}
\end{figure}
\end{center}

\paragraph{\lct{new}はどこで定義されているのか?}
\ct{new}がどこで定義されており、どのように探索されるのかについて考えてみることが、メタクラスが\ct{Class}と\ct{Behavior}を継承しているということの重要性を理解する手助けになります。\ct{new}メッセージがクラスに送られるとき、\figref{sendingnew}のように、メタクラスの継承チェーンをたどって、スーパークラスである\ct{Class}、\ct{ClassDescription}、そして\ct{Behavior}の順にメソッド探索が行われます。

\emph{「どこに\ct{new}が定義されているのか」}という問いは決定的に重要です。\mthind{Behavior}{new}は、まず\ct{Behavior}で定義され、必要に応じてそのサブクラスで再定義されます。これには、プログラマが定義したクラスのメタクラスも含まれます。\ct{new}メッセージをクラスに送った場合、まずは通常通りそのクラスのメタクラスから探索が開始され、そこで見つからなければ、\ct{new}を再定義しているクラスまたは最終的には\ct{Behavior}まで、順にスーパークラスをたどって探索されます。

\ct{TranslucentColor new}を評価した結果は\clsind{TranslucentColor}のインスタンスであり、たとえメソッドが\ct{Behavior}で見つかったとしても\ct{Behavior}のインスタンスになるわけでは\emph{ない}ことに注意してください。\ct{new}は常にそのメッセージを受け取ったクラスである \self のインスタンスを返します。これは\ct{new}が他のクラスで実装されている場合でも同じです。

\begin{code}{@TEST}
TranslucentColor new class --> TranslucentColor    "Behaviorのインスタンスではない!"
\end{code}

\begin{center}
\begin{figure}
\ifluluelse
	{\centerline{\includegraphics[width=\textwidth]{TranslucentSendingNew}}}
	{\centerline{\includegraphics[width=0.8\textwidth]{TranslucentSendingNew}}}
\caption{\lct{new}は普通のメッセージで、メタクラスチェーンをたどって探索される。\figlabel{sendingnew}}
\end{figure}
\end{center}

よくある間違いは、レシーバとなっているクラスのスーパークラスで\ct{new}を探すことです。同じことは、指定された大きさのオブジェクトを生成するための標準的なメッセージである、\ct{new:}についても言えます。
例えば、\lct{Array new: 4}は、4つの要素を持つ配列を生成します。
\ct{Array}やそのスーパークラスで、このメソッドの定義を見つけることはできないでしょう。
代わりに、実際のメソッド探索が行われる\ct{Array class}とそのスーパークラスの階層に目を向ける必要があります。

\on{The text below needs more details and convincing examples ...}

\paragraph{\lct{Behavior}、\lct{ClassDescription}および\lct{Class}の責務について}
\clsind{Behavior}は、インスタンスを持つオブジェクトのための必要最小限の状態を保持しています。すなわち、スーパークラスへのリンク、メソッド辞書、およびクラスのフォーマットです。クラスのフォーマットは、ポインターか否かの区別、コンパクトクラスか否かの区別、およびインスタンスの基本的な大きさをコード化したものです。
\ct{Behavior}は\ct{Object}を継承しているので、それ自身と全てのサブクラスは、普通のオブジェクトとして振る舞うことができます。

\ct{Behavior}は、コンパイラへの基本的なインターフェースも提供します。
メソッド辞書の生成、
メソッドのコンパイル、
インスタンスの生成(\ie \mthind{Behavior}{new}, \mthind{Behavior}{basicNew}, \mthind{Behavior}{new:}, \mthind{Behavior}{basicNew:})、
クラス階層の操作(\ie \mthindex{Behavior}{superclass:} \lct{superclass:}, \mthind{Behavior}{addSubclass:})、
メソッドへのアクセス(\ie \mthind{Behavior}{selectors}, \lmthind{Behavior}{allSelectors}, \mthind{Behavior}{compiledMethodAt:})、
インスタンス、およびインスタンス変数へのアクセス(\ie \mthind{Behavior}{allInstances}, \mthind{Behavior}{instVarNames}\ldots)、
クラス階層へのアクセス(\ie \mthind{Behavior}{superclass}, \mthind{Behavior}{subclasses})、
および問い合わせ(\ie \mthind{Behavior}{hasMethods}, \mthind{Behavior}{includesSelector}, \mthind{Behavior}{canUnderstand:}, \mthind{Behavior}{inheritsFrom:}, \mthind{Behavior}{isVariable})のメソッドを提供します。


\clsind{ClassDescription}は抽象クラスであり、\clsind{Class}と\clsind{Metaclass}という直接のサブクラスが必要とする機能を提供しています。
\clsind{ClassDescription}は\ct{Behavior}が提供する基礎に、以下のような多くの機能を加えます。それらは、
インスタンス変数の名前の管理、
プロトコルによるメソッドの分類、
チェンジセットの管理と変更の記録、
そして変更のファイルアウトに必要なほとんどの機能などです。

\clsind{Class}は、すべてのクラスに共通する振る舞いを提供し、
クラス名、メソッドのコンパイル、メソッドの保存、インスタンス変数を提供します。
また、クラス変数名、共有プール変数(\mthind{Class}{addClassVarName:}, \mthind{Class}{addSharedPool:}, \mthind{Class}{initialize})の実際のインターフェーイスも提供します。
メタクラスは、唯一のインスタンス(\ie, メタクラスでないクラス)に対するクラスであり、クラスとしてインスタンスにサービスを提供するため、すべてのメタクラスは最終的に\ct{Class}を継承しています。

%=================================================================
\section{すべてのメタクラスは\lct{Metaclass}のインスタンスである}
% \ruleref{metaclassclass}

メタクラスもまたオブジェクトです。\figref{metaclassclass}が示すように、メタクラスは\clsind{Metaclass}のインスタンスです。\ct{Metaclass}のインスタンスは匿名のメタクラスで、それぞれ唯一のインスタンスを持ちます。
つまり、それがクラスというわけです。

\begin{center}
\begin{figure}
\ifluluelse
	{\centerline{\includegraphics[width=\textwidth]{TranslucentMetaclassClass}}}
	{\centerline{\includegraphics[width=0.8\textwidth]{TranslucentMetaclassClass}}}
\caption{すべてのメタクラスは\lct{Metaclass}のインスタンスである。\figlabel{metaclassclass}}
\end{figure}
\end{center}

\ct{Metaclass}は、メタクラスに共通する振る舞いを提供し、
メタクラスの唯一のインスタンスに対して初期化されたインスタンスを生成する
インスタンス生成(\lct{sub\-class\-Of:})、
クラス変数の初期化、
メタクラスのインスタンス、
% (\ct{name:inEnvironment:subclassOf:})
% Actually, this is in ClassBuilder, it seems!
メソッドのコンパイル、% (different semantics can be introduced)
およびクラスの情報(継承関係、インスタンス変数、\etc)のメソッドを提供します。
\ab{This is too cryptic for me.}

%=================================================================
\section{\lct{Metaclass}のメタクラスは\lct{Metaclass}のインスタンスである}
% \ruleref{metaclassmetaclass}

残る質問は1つだけです。\clsind{Metaclass class}のクラスは何でしょうか?

答えは簡単でメタクラスです。もちろんシステム内の他のすべてのメタクラスと同様に、\ct{Metaclass}のインスタンスであるべきです。(\figref{metaclassclassclass}を参照)

\begin{center}
\begin{figure}
\ifluluelse
	{\centerline{\includegraphics[width=\textwidth]{TranslucentMetaclassClassClass}}}
	{\centerline{\includegraphics[width=0.8\textwidth]{TranslucentMetaclassClassClass}}}
\caption{すべてのメタクラスは\lct{Metaclass}のインスタンスである。\lct{Metaclass}のメタクラスも同様。\figlabel{metaclassclassclass}}
\end{figure}
\end{center}

この図は、すべてのメタクラスが\ct{Metaclass}のインスタンスであり、\ct{Metaclass}のメタクラスもまた同様であるということを示しています。
\ref{fig:metaclassclass}と\ref{fig:metaclassclassclass}を比べると、\ct{Object class}に至るまでメタクラスの\subind{metaclass}{階層}がクラス階層の完全な鏡像になっていることがわかるでしょう。

以下の例は、クラス階層の問い合わせによって、\figref{metaclassclassclass}が正確であることを示しています。
(現実にはちょっとしたウソが混じっており、\ct{Object class superclass}の結果は\clsind{ProtoObject class}であって\ct{Class}ではありません。\pharo では、\ct{Class}に至るため、もう一つ上のスーパークラスに行く必要があります。)

\begin{example}{クラス階層}{@TEST}
TranslucentColor superclass --> Color
Color superclass                   --> Object
\end{example}

\needlines{4}
\begin{example}{クラス階層に並列するメタクラス階層}{@TEST}
TranslucentColor class superclass   --> Color class
Color class superclass                     --> Object class
Object class superclass superclass --> Class    "NB: skip ProtoObject class"
Class superclass                              --> ClassDescription
ClassDescription superclass            --> Behavior
Behavior superclass                         --> Object
\end{example}

\begin{example}{Metaclassのインスタンス}{@TEST}
TranslucentColor class class --> Metaclass
Color class class                   --> Metaclass
Object class class                 --> Metaclass
Behavior class class              --> Metaclass
\end{example}
\begin{example}{Metaclass classはMetaclassのインスタンスである}{@TEST}
Metaclass class class --> Metaclass
Metaclass superclass --> ClassDescription
\end{example}

%=================================================================
\section{この章のまとめ}
クラスがどのように組織されており、一貫性のあるオブジェクトモデルがどのような効果を及ぼしているのかについて、よく理解できたのではないでしょうか。忘れたり混乱したなら、いつでもメッセージ・パッシングが鍵であることを思い出してください。メッセージのレシーバのクラスからメソッドを探せばよいのです。
このことは\emph{どんな}レシーバについても同様です。
レシーバのクラスでメソッドがみつからなければ、そのスーパークラスを探せばよいのです。

\begin{itemize}
\item すべてのクラスはメタクラスのインスタンスである。
	メタクラスは暗黙的である。メタクラスはクラスを定義をするとき、そのクラスを唯一のインスタンスとするように自動的に作成されます。

\item メタクラスの階層はクラスの階層と並列に存在する。
	クラスのメソッド探索は、普通のオブジェクトのメソッド探索と並行しており、メタクラスの継承チェーンに沿って行われます。

\item すべてのメタクラスは\lct{Class}と\lct{Behavior}を継承している。
	すべてのクラスは\ct{Class}のインスタンスです(\emph{is-a})。メタクラスはクラスでもあるため\ct{Class}を継承しています。
	\ct{Behavior}はインスタンスを持つすべての存在に共通した振る舞いを提供します。

\item すべてのメタクラスは\ct{Metaclass}のインスタンスである。
	\ct{ClassDescription}は\ct{Class}および\ct{Metaclass}に共通することがらすべてを提供します。

\item \ct{Metaclass}のメタクラスは\ct{Metaclass}のインスタンスである。
	クラス-インスタンスの関係(\emph{instance-of})は閉じたループを形成します。すなわち、\ct{Metaclass class class --> Metaclass}ということです。
\end{itemize}
%=================================================================
\ifx\wholebook\relax\else\end{document}\fi
%=================================================================

%-----------------------------------------------------------------
