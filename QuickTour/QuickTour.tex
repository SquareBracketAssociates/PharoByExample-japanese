% $Author$
% $Date$
% $Revision$

% HISTORY:
% 2006-12-01 - Andrew edited (split from FirstApp?)
% 2006-12-03 - Andrew first draft
% 2006-12-06 - Stef edit
% 2007-06-11 - Oscar edit
% 2007-07-03 - Stef review
% 2007-08-22 - Andrew corrections
% 2007-09-11 - Marcus review
% 2007-09-11 - Orla review
% 2009-07-04 - Oscar migrated to Pharo

%=================================================================
\ifx\wholebook\relax\else
% --------------------------------------------
% Lulu:
	\documentclass[a4paper,10pt,twoside]{book}
	\usepackage[
		papersize={6.13in,9.21in},
		hmargin={.75in,.75in},
		vmargin={.75in,1in},
		ignoreheadfoot
	]{geometry}
	\input{../common.tex}
	\pagestyle{headings}
	\setboolean{lulu}{true}
% --------------------------------------------
% A4:
%	\documentclass[a4paper,11pt,twoside]{book}
%	\input{../common.tex}
%	\usepackage{a4wide}
% --------------------------------------------
    \graphicspath{{figures/} {../figures/}}
	\begin{document}
	% \renewcommand{\nnbb}[2]{} % Disable editorial comments
	\sloppy
\fi
%=================================================================
\newcommand{\clover}{%
	\raisebox{-0.8ex}[0pt][0pt]{%
		\includegraphics[width=1em]{cloverleafKey}}}
%=================================================================
\chapter{\pharo 早巡り}
\chalabel{quick}

この章では、\pharo 環境に親しんでもらうために、その概要を解説します。
実際に \pharo を試してみる機会がたくさんあるので、この章は、パソコンを手元に置いて読むと良いでしょう。

\pharo を使って実際に試してみてもらいたい箇所には、このアイコン: \dothisicon{} で印をつけておきます。
この印がついた箇所では、特に、\pharo を起動したり、システムとやりとりする異った方法を学んだり、基本的なツールについて知ることになるでしょう。
また、メソッドの作成方法や、オブジェクトを作成し、それにメッセージを送る方法を学ぶでしょう。

%=================================================================
\section{入門}

\pharo は \pharoweb から自由に\ind{ダウンロード}することができます。
ダウンロードしなければならないのは、3 種のパーツです。これらは実際には 4 個のファイルからなります(\figref{download})。

\begin{figure}[htb]
\centerline {\includegraphics[width=\textwidth]{annotatedDownload-flat}}
\caption{\pharo がサポートしている、あるプラットフォーム用のダウンロードファイル。\figlabel{download}}
\end{figure}

\begin{enumerate}

  \item \emphind{バーチャルマシン} (VM)。バーチャルマシンは、システムで唯一、各 OS と CPU ごとに異なるパーツです。すべての主要なプラットフォーム用に、コンパイル済みのバーチャルマシンが用意されています。\figref{download} にあるのは Windows 用の \textit{Pharo.exe} という名前のバーチャルマシンです。

  \item \emphind{ソースファイル}。ソースファイルには、\pharo のすべてのパーツのソースコードが入っています。これらのパーツは、あまり頻繁には変更されません。\figref{download} では、\emph{SqueakV39.sources} がソースファイルです。\footnote{\pharo は \squeak 3.9 をベースに作らています。そして、現在のバーチャルマシンは、\squeak と同じものを使っています。}

  \item \ind{仮想イメージ}。\emph{システム仮想イメージ}は、実行中の \pharo システムのスナップショットです。このスナップショットは二つのファイルからなります: \emphind{イメージファイル}(\emph{.}\ind{image})。イメージファイルには、システムのすべてのオブジェクト(クラスやメソッドもオブジェクトです)の状態が入っています。それから、\emphind{チェンジファイル}(\emph{.}\ind{changes})。チェンジファイルには、システムのソースコードの変更のログがすべて入っています。
\figref{download} では、これらのファイルは、\emph{pharo.image}、\emph{pharo.changes} です。
\end{enumerate}

\dothis{\pharo をダウンロードし、インストールしましょう。}
\pharo by Example のウェブページから仮想イメージをダウンロードすることをお勧めします。\footnote{\pbe}
\index{ダウンロード}
\seclabel{sbeImage}

「入門」で扱う題材のほとんどが、\pharo のどのバージョンでも動作するので、もし \pharo を既にインストールしているのなら、それを使っても構いません。
しかし、外見や振る舞いのちょっとした違いがあっても、驚かないで下さい。
% On the other hand, if you are about to download \pharo for the first time, you may as well grab the \emph{\pharo by Example} image.

\pharo で何か作業している時、イメージファイルとチェンジファイルは常に内容が更新されるため、確実に書き込み可能にしておいてください。
そして、これら二つのファイルは、常に同じ場所に置いて下さい。
これらを、絶対にテキストエディタで直接変更しないでください。\pharo はこれらを、オブジェクトを格納したり、ソースコードの変更のログを書き出すのに使います。
ダウンロードしたイメージファイルとチェンジファイルのバックアップコピーを取っておくのは良い考えです。こうしておけばいつでも、まっさらな仮想イメージから \pharo を開始し、(それが保存してあれば)あなたのコードを読み込むことができます。

\emphind{ソースファイル}とバーチャルマシンは、書き込み不可にすることができます\,---\,これらを異なるユーザで共有することができます。
バーチャルマシン・ソースファイル・イメージファイル・チェンジファイルは、同じ場所に置いても構いませんし、バーチャルマシン・ソースファイルだけ別の共有ディレクトリに置いておくこともできます。
OS や作業スタイルに合わせて、一番良い方法を選んでください。

%\sd{it would be really nice to have a setup and startup section on PC, Mac and Linux}
%\ab{I agree entirely; the reason that this is not here is because I could do only the Mac section.  Damien can probably do Windoze.  Perhaps we can ask on the list for a Linux volunteer?}

%-----------------------------------------------------------------
\begin{figure}[htb]
% \centerline {\includegraphics[width=0.6\textwidth]{download}}
\centerline {\includegraphics[width=\textwidth]{startup}}
\caption{\pbe からダウンロードしたままの、まっさらな仮想イメージ。\figlabel{startup}}
\end{figure}

\index{\pharo を起動する}
\paragraph{起動する。} \pharo の起動は、普通のアプリケーションと同様です: バーチャルマシンのアイコンに\emphind{イメージファイル}をドラッグ\&ドロップしたり、\emph{イメージファイル}をダブルクリックしたり、コマンドラインで、バーチャルマシンの名前の後に\emph{イメージファイル}へのパスを入力することになるでしょう。使っている OS に合わせて下さい(異なるバージョンのバーチャルマシンがあると、OS は、正しいバーチャルマシンを自動で選んでくれないかも知れません; そのような場合は、ドラッグ\&ドロップまたはコマンドラインから起動した方が安全でしょう)。

\pharo を起動すると、1 個の大きなウィンドウが現れます。このウィンドウの中には、ワークスペースウィンドウが、いくつか開いているかも知れません(\figref{startup})。しかし、このあとどう続けたらいいか、明かではありません!
メニューバーが出ることもありますが、\pharo ではもっぱら、コンテキスト依存のポップアップメニューを活用します。
% You will notice that there is no menu bar, or at least not a useful one.  
% Instead,  \pharo makes heavy use of context-dependent pop-up menus.

\dothis{\pharo を始めましょう。ウィンドウの左上にある赤いクローズボタンを \click すると、ワークスペースを閉じることができます。}

ウィンドウを最小化するにはオレンジ色のボタン、最大化するには緑色のボタンを \click します。

%-----------------------------------------------------------------
\paragraph{最初のやりとり。}

\figref{threeButtons:click} の \ind{ワールドメニュー}から入門するのがいいでしょう。

\dothis{マウスカーソルがメインウィンドウの背景にある状態で、マウスのボタンをクリックし、ワールドメニューを表示しましょう。そして、\menu{Workspace} を選択して、新しいワークスペースを開きましょう。}

%\begin{figure}[tbh]
%	\centering
%	\subfigure[The world menu]{\figlabel{threeButtons:click}% click
%		\includegraphics[width=0.25\linewidth]{worldMenu}}\hfill
%	\subfigure[The contextual menu]{\figlabel{threeButtons:actclick}% action click
%		\includegraphics[width=0.35\linewidth]{yellowButtonMenuOnWorkspace}}\hfill
%	\subfigure[The morphic halo]{\figlabel{threeButtons:metaclick}% meta click
%		\includegraphics[width=0.35\linewidth]{morphicHaloOnWorkspace}}% these braces needed (else no whitespace at end of line)
%	\caption{The world menu (brought up by \click{ing}), a contextual menu (\actclick{ing}), and a morphic \subind{Morphic}{halo} (\metaclick{ing}).\figlabel{threeButtons}}
%\end{figure}

\begin{figure}[tbh]
	\centering
	\subfigure[ワールドメニュー]{\figlabel{threeButtons:click}% click
		\includegraphics[width=0.40\linewidth]{worldMenu}}\hfill
	\subfigure[コンテキストメニュー]{\figlabel{threeButtons:actclick}% action click
		\includegraphics[width=0.55\linewidth]{yellowButtonMenuOnWorkspace}}\hfill
	\subfigure[ハロー]{\figlabel{threeButtons:metaclick}% meta click
		\includegraphics[width=0.60\linewidth]{morphicHaloOnWorkspace}}% these braces needed (else no whitespace at end of line)
	\caption{ワールドメニュー(\click すれば出ます)、コンテキストメニュー(\actclick)、\subind{Morphic}{ハロー}(\metaclick)。\figlabel{threeButtons}}
\end{figure}
\seeindex{ハロー}{Morphic}

%% ON: I had to shrink this and move it up to avoid
%% it running over the end of the page.
%\begin{wrapfigure}[19]{r}{0.25\linewidth}
%% The parameters are the number of narrow lines to the right of the figure [19],
%% the placement {r} for right, and the width of the figure. Capital R will allow some float.
%% Inside the wrapfig environment, linewidth is special --- the width of the figure.
%\includegraphics[width=0.95\linewidth]{colouredMouse}
%\caption{The author's mouse. \click{ing} the scroll wheel activates the blue button.}
%\figlabel{colouredMouse}
%\end{wrapfigure}

\st はもともと、\ind{3 ボタンマウス}付きのコンピュータのために設計されています。マウスのボタンが足りないときは、修飾キーを押しながらマウスを \click してください。2 ボタンマウスでも、十分に \pharo を使うことができますが、もし 1 ボタンマウスしか持っていなら、クリックできるスクロールホイールが付いた 2 ボタンマウスを買うことを、真剣に考えましょう: これで \pharo を、よりいっそう楽しく使えるようになります。

色々なコンピュータやマウス、キーボード、個人設定があるため、\pharo では「マウスの左ボタンをクリック」という表現は避けています。
もともと \st は、異なるマウスボタンを、色で表わしていました。\footnote{マウスボタンの色は、\emph{赤}、\emph{黄}そして\emph{青}です。この本の著者は、どの色がどのボタンを指していたのか、さっぱり思い出せません。}
\index{赤ボタン}
\index{黄ボタン}
\index{青ボタン}
多くのユーザが、同じ目的のために様々な修飾キー(\emph{control}、\emph{ALT}、\emph{meta} \etc)を使うので、この本では、代わりに次の用語を使います:
\begin{description}
\item [\click:] 一番良く使われるマウスボタンで、普通は、修飾キーを使わずに 1 ボタンマウスを \click するのと同じです; 背景を \click して「ワールド」メニュー(\figref{threeButtons:click})を表示してみましょう。
\item [\actclick:] 次に良く使われるボタンです; このボタンでコンテキストメニューを表示します。コンテキストメニューが表示するアクションの一覧は、マウスが指している場所によって異なります; \figref{threeButtons:actclick}を見てください。マルチボタンのマウスが無い場合、普通は、\emph{control} 修飾キーを使って \actclick するように設定します。
\item [\metaclick:] 最後が \metaclick です。画面上の任意のオブジェクトを \metaclick して、その「\subind{Morphic}{ハロー}」を表示させることができます。ハローは、画面上のオブジェクトを、回転させたりリサイズしたりするためのハンドルの集りです; \figref{threeButtons:metaclick} を見てください。\footnote{\pharo はデフォルトで、ハンドルを無効にしていますが、これらはプリファレンス・ブラウザを使って、有効にすることができます。プリファレンス・ブラウザについては、すぐに述べます。}
マウスをかざしておけば、バルーンヘルプが出て、ハンドルの機能について教えてくれます。
\pharo で \metaclick する方法は、オペレーティングシステムに依存します:
{\sc shift} \emph{ctrl} または {\sc shift} \emph{option} を押しながらクリックすることになるでしょう。
% \ab{This makes it sound like either {\sc shift} \emph{ctrl} or {\sc shift} \emph{alt} will work.  On my (Mac OS) system, only the latter works.  Perhaps we want to say: In \pharo, how you meta-click depends on your operating system. On Linux \ldots}
% Typically you will use a third modifier key, such as \emph{command} or \emph{meta} to \metaclick.
\end{description}

\dothis{\ct{Time now} とワークスペースに入力しましょう。
そして、ワークスペースで \actclick して、
\menu{print it} を選択しましょう。}

%Now we will activate \metaclick{ing}.

%\dothis{Open the preference browser (\menu{System {\ldots\go} Preferences {\ldots\go} Preference Browser\ldots}) and find the \menu{halosEnabled} option using the search box.}

%\begin{figure}[htb]
%\centerline{\includegraphics[width=\textwidth]{PreferenceBrowser}}
%\caption{The Preference Browser.\figlabel{prefBrowser}}
%\end{figure}

%\dothis{Now you should be able to \metaclick on the workspace. (See \figref{threeButtons:metaclick}.)
%Grab the blue \raisebox{-0.4ex}{\includegraphics[width=1em]{morphicRotate}} handle near the bottom left corner and drag it to rotate the workspace.}

右利きの人なら、\click は左ボタンに、\actclick は右ボタンに、クリックできるスクロールホイールがあるマウスなら、\metaclick をそれに設定することをお勧めします。
% If you don't have a clickable scroll wheel, then you can get the Morphic halo by holding down the \ct{alt} or \ct{option} key while \click{ing}. 
% \ab{This doesn't work any more.  This sentence either repeats or contradicts the meta-click item above; neither is a good idea.}
1 ボタンのマウスで Macintosh を使っているのなら、\clover{} を押しながらマウスを \click することで、\actclick か \metaclick をシミュレートすることもできます。それでも、これから \pharo を頻繁に使おうと言うのなら、最低でも二つのボタンの付いたマウスに投資することを、お勧めします。

オペレーティングシステムやマウスドライバの設定を変えることで、マウスの動作を望み通りにすることができます。
%\ab{How can I get meta-click without a three-finger salute?  Is this a secret?}
\pharo には、マウスやキーボードのメタキーの設定を変更できる機能もあります。
プリファレンス・ブラウザ(\menu{System {\ldots\go} Preferences {\ldots\go} Preference Browser\ldots})では、\menu{keyboard} カテゴリの \menu{swapControlAndAltKeys} オプションを使うことで、\actclick と \metaclick の機能を交換することができます。
プリファレンスブラウザには他にも、様々なキーボードショートカットを設定するオプションがあります。

\begin{figure}[htb]
\centerline{\includegraphics[width=\textwidth]{PreferenceBrowser}}
\caption{プリファレンス・ブラウザ。\figlabel{prefBrowser}}
\end{figure}


%=================================================================
\section{ワールドメニュー}
\index{ワールドメニュー}

\dothis{\pharo の背景で、もう一度 \click してみましょう。}
再び \menu{World} メニューが表示されるはずです。
ほとんどの \pharo のメニューには、モードがありません; モードの無いメニューは、その右上の画鋲アイコンを \click して、いつまでも画面上に残しておくことができます。やってみましょう。
% Also, notice that menus appear when you click the mouse, but do not disappear when you release it; they stay visible until you make a selection, or until you click outside of the menu. You can even move the menu around by grabbing its title bar.

ワールドメニューを使えば、\pharo の色々なツールに簡単にアクセスすることができます。

\dothis{\menu{World} メニューと、\menu{{}Tools \ldots} メニューを良く見てみましょう(\figref{threeButtons:click})。}

そこには、ブラウザやワークスペースと言った \pharo の主要なツールが、いくつかリストされるでしょう。
次の章では、これらのツールのほとんどを、見ることになります。

%=================================================================
\section{メッセージを送る}

\dothis{ワークスペースを開けましょう。そして以下のテキストを入力しましょう:}

\begin{code}{}
BouncingAtomsMorph new openInWorld
\end{code}

\dothis{\actclick しましょう。メニューが現われるはずです。\menu{do it (d)} を選択しましょう(\figref{doit})。}

\begin{figure}[htb]
\centerline {\includegraphics[width=0.8\textwidth]{Doit}}
\caption{式を「do it」する。\figlabel{doit}}
\end{figure}

\pharo の画面左上に、たくさんの原子が中で弾んでいるウインドウが、表示されたはずです。

あなたはたった今、最初の \st の式を評価しました!
あなたは \bam クラスへ \ct{new} メッセージを送り、\bam のインスタンスが生成され、続く \ct{openInWorld} メッセージが、このインスタンスに送られました。
\ct{new} メッセージを受け取った時にすることを決めたのは、\bam クラスです。つまり \bam クラスは、\ct{new} メッセージを扱う\emph{メソッド}を探索して、適切に反応しました。
同様に、\bam インスタンスも、\ct{openInWorld} に反応するメソッドを探索して、適切なアクションを取りました。

スモールトーカーたちとしばらく話せば、すぐに、彼らが「手続きをコールする」または「メソッドを呼び出す」などの表現を使わず、代わりに「メッセージを送る」と言うことに気付くでしょう。
これは、オブジェクトは自分自身のアクションに責任を持つ、という思想を反映しています。
あなたは決して、オブジェクトに、何々をしろと\emph{命令する}ことはありません\,---\,代わりにメッセージを送って、何かしてほしい、と礼儀正しく\emph{頼み}ます。
あなたではなくオブジェクトが、メッセージに反応するための適切なメソッドを選びます。

%=================================================================
\section{\pharo のセッションを、保存・終了・再開する}

\dothis{弾む原子のウィンドウを \click し、好きな所へウィンドウをドラッグしましょう。デモウィンドウは、もはや「意のまま」です。\click して、適当な場所に置きましょう。}

\begin{figure}[htb]
\begin{minipage}[b]{0.48\textwidth}
\centerline {\includegraphics[width=0.7\textwidth]{atoms}}
\caption{\bam。\figlabel{atoms}}
\end{minipage}
\hfill
\begin{minipage}[b]{0.48\textwidth}
\centerline {\includegraphics[width=0.7\textwidth]{saveAs}}
\caption{\menu{save as \ldots} ダイアログ。\figlabel{saveas}}
\end{minipage}
\end{figure}

\dothis{\menu{World\go{}Save as \ldots} を選択して「myPharo」と入力し、\button{OK} ボタンを \click しましょう。
そして、\menu{World\go{}Save and quit} を選択しましょう。}

もともとのイメージファイルとチェンジファイルがあった場所に、「myPharo.\ind{image}」と「myPharo.\ind{changes}」というファイルができているはずです。これらのファイルには、\menu{Save and quit} する直前の \pharo 仮想イメージの、動作中の状態が入っています。
この二つのファイルは何処にでも移動させて構いませんが、(OS によっては)バーチャルマシンと\emph{ソースファイル}も、その場所へ、移動・コピー・リンクする必要があるかもしれません。

\dothis{今作った「myPharo.image」ファイルを使って \pharo を起動しましょう。}

先ほど \pharo を終了した時とそっくりそのままの状態に戻ったことにお気づきでしょう。\bam も同じところにあり、原子も弾み続けているでしょう。

\pharo を起動すると、\pharo \ind{バーチャルマシン}は、指定されたイメージファイルを読み込みます。このファイルには、たくさんのオブジェクトのスナップショットが入っています。これらのオブジェクトには、既に書かれた大量のコードや、たくさんのプログラミングツール(これらはすべて、オブジェクトです)が含まれます。\pharo を使っていると、これらのオブジェクトへメッセージ送ったり、新しいオブジェクトを作ったりすることになります。また、いくつかのオブジェクトは命を終え、オブジェクトに割り当てられていたメモリは回収(\ie ガベージコレクション)されます。

\pharo を終了するとき、普通は、すべてのオブジェクトのスナップショットを保存することになるでしょう。ここで通常、古いイメージファイルは上書きされることになりますが、先ほど行ったように、新しく名前を付けてイメージファイルを保存することもできます。

\emph{イメージファイル}の他に、\emph{チェンジファイル}もあります。
このファイルは、標準のツールを使って行ったソースコードの変更を、すべて記録しています。
通常、このファイルは、まったく意識する必要はありません。
しかし後で見るように、\emph{チェンジファイル}は、エラーから回復したり、保存しそこなった変更を再現するのに重宝します。
これについては、後ほど!

あなたが使ってきた仮想イメージは、1970年代後半に作られた、オリジナルの \st-80 の仮想イメージの子孫です。
この仮想イメージの中には、何10年と生き続けているオブジェクトもあります!

ソフトウェアプロジェクトを保存・管理するのに、仮想イメージは基本的なメカニズムであると考えるかも知れませんが、それは違います。
すぐ後で見るように、コードを管理し、ソフトウェアをチームで共有するには、もっと良いツールがあります。
仮想イメージは非常に便利ですが、Monticello などのツールを使えば、バージョン管理やコードの共有がもっとうまくできるので、仮想イメージに執着せず、これを無造作に作ったり捨てたりすることに慣れた方が良いでしょう。

\dothis{マウス(と必要ならば修飾キー)を使って \bam を \metaclick しましょう。\footnote{うまくいかない場合は、プリファレンス・ブラウザの \ct{halosEnabled} オプションをチェックしてみてください。}}
色とりどりの円が表示されたはずです。これらをまとめて、\bam の\subind{Morphic}{ハロー}と呼びます。
一つ一つの円は、\emph{ハンドル}と呼びます。
十字のピンクのハンドルをクリックしましょう; \bam は消えるはずです。

%=================================================================
\section{ワークスペースとトランスクリプト}
\seclabel{トランスクリプト}

\dothis{すべてのウィンドウを閉じましょう。\ind{トランスクリプト}と\ind{ワークスペース}を開きましょう。(トランスクリプトは、\menu{World{\go}Tools ...} サブメニューから開けます。)}

\dothis{トランスクリプトとワークスペースの位置やサイズを変えて、ワークスペースをトランスクリプトにオーバーラップさせましょう。}
ウィンドウをリサイズするには、ウィンドウの角をドラッグするか、\metaclick してハローを出し、黄色の(右下の)ハンドルをドラッグします。

常に、1 個のウィンドウだけがアクティブです; そのウィンドウは前面にあって、枠がハイライトされています。
% The mouse cursor must be in the window in which you wish to type.

トランスクリプトは、システムメッセージのログを取るのにしばしば使われるオブジェクトです。
トランスクリプトは「システムコンソール」の一種です。
%Note that the transcript is terribly slow, so if you keep it open and write to it certain operations can become 10 times slower.
%In addition the transcript is not thread-safe so you may experience strange problems if multiple objects write concurrently to the transcript.
% ON: I think the transcript has been made thread-safe now, right?

ワークスペースは、試してみたい \st コードの断片を入力するのに役立ちます。
ワークスペースにはまた、任意のテキストをメモしておくことができます。TODO リストや、あなたの仮想イメージを使う人への手引きなどを、書き残しておくことができます。
ワークスペースは、しばしば、保存された仮想イメージについてのドキュメントとして使われます。先にダウンロードした標準の仮想イメージが一例です(\figref{startup})。

\dothis{ワークスペースに次のテキストを入力しましょう。}
\begin{code}{}
Transcript show: 'hello world'; cr.
\end{code}

ワークスペースの中で、今入力したテキストの色々な箇所をダブル\click してみましょう。
単語の上、文字列の終り、式全体の終り、という具合に \click してみて、単語全体、文字列全体、テキスト全体が選択される様子を確認しましょう。

\dothis{入力したテキスト全体を選択して、\actclick しましょう。
\menu{do it (d)} を選択しましょう。}
トランスクリプトウィンドウに「hello world」と表示されたのを確認しましょう
(\figref{helloworld})。
もう一度やってみましょう。
(メニュー項目 \menu{do it (d)} の中の \menu{(d)} は、\emph{do it} へのキーボードショートカットが \short{d} であることを示しています。これについて詳しい話は、次の節で!)

\begin{figure}[htb]
\centerline {\includegraphics[width=\textwidth]{HelloWorld}}
\caption{オーバーラップしたウィンドウ。ワークスペースがアクティブ。\figlabel{helloworld}}
\end{figure}

%=================================================================
\section{キーボードショートカット}

式を評価したいとき、いつも \actclick する必要はありません。代わりに、\ind{キーボードショートカット}を使うこともできます。メニューの括弧書きの部分が該当します。プラットフォームに応じて、いずれかの修飾キー(control、alt、command、あるいは meta)を押すことになるでしょう
(このようなキーボードショートカットを、一般的に、\short{\emph{キー}}と表記することにします)。

\dothis{もう一度ワークスペースの式を評価してみましょう。ただし、キーボードショートカットを使って: \short{d}。}
\index{キーボードショートカット!do it}

\menu{do it} の他に、\menu{print it}、\menu{inspect it}、\menu{explore it} にも気付いたでしょう。これらについて手短に説明します。

\dothis{\ct{3 + 4} とワークスペースの中に入力しましょう。そして、キーボードショートカットを使って、\menu{do it} しましょう。}

何も起きなかったことに驚かないでください! ここでは \ct{3} という数に \ct{+} というメッセージを、\ct{4} という引数付きで送ったことになります。普通に \ct{7} が計算され、返されたのですが、ワークスペースはこの答えをどうすべきか知らないので、答えを捨ててしまいます。結果を見たければ、代わりに \menu{print it} を使う必要があります。\menu{print it} は、式をコンパイルし、実行し、その結果に \ct{printString} を送って得た文字列を表示します。

\dothis{\ct{3+4} を選択し、\menu{print it} しましょう(\short{p})。}
今度は期待通りの結果になります(\figref{printit})。
\index{キーボードショートカット!print it}

\begin{figure}[htb]
% \centerline {\includegraphics[width=0.4\textwidth]{PrintIt}}
\centerline {\includegraphics[width=0.8\textwidth]{PrintIt}}
\caption{「do it」ではなく「print it」。\figlabel{printit}}
\end{figure}

\needlines{3}
\begin{code}{@TEST}
3 + 4 --> 7
\end{code}
\noindent
この本の約束事として、特定の \pharo の式を \menu{print it} したらどうなるかを示すのに、\ct{-->} の表記を用います。

\dothis{選択されている「\ct{7}」を削除し(\pharo は既に「\ct{7}」を選択しているはずです。だから delete キーを押すだけです)、\ct{3+4} をもう一度選択してから、今度は \menu{inspect it} (\short{i})しましょう。}
\noindent
タイトルに \ct{SmallInteger: 7} と書かれた\emphind{インスペクタ}ウィンドウが表示されるはずです。
インスペクタは極めて便利なツールで、これを用いれば、システムのどんなオブジェクトもブラウズできますし、どんなオブジェクトともやりとりすることができます。
タイトルの意味は、\ct{7} は \clsind{SmallInteger} クラスのインスタンスであるということです。
左のペインを使って、オブジェクトのインスタンス変数をブラウズすることができます。インスタンス変数の値は、右のペインに表示されます。
下のペインには、オブジェクトへメッセージを送るための式を書くことができます。

\begin{figure}[htb]
\centerline {\includegraphics[width=\textwidth]{InspectIt}}
\caption{オブジェクトをインスペクトする。\figlabel{inspectit}}
\end{figure}

\dothis{\ct{7} のインスペクタの下のペインに \ct{self squared} と入力し、\menu{print it} しましょう。}

\needlines{2}
\dothis{インスペクタを閉じましょう。\ct{Object} をワークスペースに入力して、今度は、\menu{explore it} (\short{I}, 大文字 i)しましょう。}
\index{キーボードショートカット!explore it}
\index{エクスプローラ}

今度は、\clsind{Object} のタイトルが付いたウィンドウが現れるはずです。
このウィンドウの中には、\mbox{$\triangleright$ \ct{root: Object}} というテキストがあります。
三角をクリックして、中身を開いてください(\figref{exploreit})。

\begin{figure}[htb]
\centerline {\includegraphics[width=0.7\textwidth]{ExploreIt}}
\caption{\ct{Object} をエクスプロアする。\figlabel{exploreit}}
\end{figure}

エクスプローラはインスペクタと似ていますが、複雑なオブジェクトのツリービューを提供する点で異ります。
この例では、見ているオブジェクトは、\ct{Object} クラスです。
ここでは、このオブジェクトに格納されている要素を直に見ることができますし、さらに要素の内部構造をたどって行くことも容易です。

%=================================================================
\section{クラスブラウザ}

クラス\ind{ブラウザ}\footnote{紛らわしいことに「システムブラウザ」や「コードブラウザ」など呼ばれることがあります。\pharo では \ind{OmniBrowser} というブラウザの実装が使われています。OmniBrowser は「OB」または「パッケージブラウザ」としても知られています。
この本では単に「ブラウザ」という用語を用いますが、曖昧さが問題になるケースでは「クラスブラウザ」を用います。}は、プログラミングにおいて重要なツールの一つです。
後で見るように、\pharo にはいくつかの興味深いブラウザがありますが、どの仮想イメージを使う場合でも、クラスブラウザは、最も基本的なブラウザです。
\seeindex{クラスブラウザ}{ブラウザ}

\dothis{\menu{World \go Class browser} を選択してブラウザを開きましょう。\footnote{もし、browser が \figref{classBrowser} に似ていなければ、デフォルトのブラウザを変更する必要があるでしょう。\faqref{packagebrowser} を見てください。}}

\begin{figure}[htb]
\ifluluelse
	{\centerline {\includegraphics[width=\textwidth]{ClassBrowser1}}}
	{\centerline {\includegraphics[width=0.7\textwidth]{ClassBrowser1}}}
\caption{Object クラスの \ct{printString} メソッドを表示しているブラウザ。
\figlabel{classBrowser}}
\end{figure}

\figref{classBrowser} にあるのがブラウザです。
タイトルバーは、\clsind{Object} クラスをブラウズしていることを示しています。
%\footnote{If the browser you have seems to differ from the one described in this book, you may be using an image with a different default browser. See \faqref{omnibrowser}.}

ブラウザを最初に開くと、一番左のペイン以外はすべて空欄です。
この最初のペインは、すべての \emph{パッケージ}をリストしています。各パッケージには、関連するクラスのグループが入っています。
\index{パッケージ}

\dothis{\scatind{Kernel} パッケージをクリックしましょう。}
選択されたパッケージに含まれるクラスが、2 番目のペインにリストされます。

\dothis{\clsind{Object} クラスを選択しましょう。}
今度は、残りの二つのペインにテキストが表示されます。
3 番目のペインには、選択されたクラスの\emph{プロトコル}が表示されます。
プロトコルは、関連するメソッドを、扱いやすいようにグループ分けします。
\ind{プロトコル}を何も選択していなければ、すべてのメソッドが 4 番目のペインに表示されます。

\dothis{\protind{printing} プロトコルを選択しましょう。}
printing プロトコルを見つけるには、下にスクロールしなければならないかも知れません。
プリントに属するメソッドだけが、4 番目のペインに表示されます。

\dothis{\mthind{Object}{printString} メソッドを選択しましょう。}
今度は、下のペインに、\ct{printString} メソッドのソースコードが表示されます。このソースコードは、すべてのオブジェクトで共有されます(このメソッドをオーバーライドするものを除く)。

%=================================================================
\section{クラスを見つける}

\pharo でクラスを見つけるには、いくつかの方法があります。一つ目は、今見たように、ブラウザを使ってパッケージからたどっていく方法です。この場合パッケージ名は、最初からわかっているか推測することになります。
\index{ブラウザ}
\seeindex{ブラウザ!クラスを見つける}{クラス, 見つける}
\index{クラス!見つける}

二つ目は、クラスに\ct{browse} メッセージを送り、自身のブラウザを開いてもらう方法です。\clsind{Boolean} クラスをブラウズしたかったとしましょう。

\dothis{ワークスペースに \ct{Boolean browse} と入力し、\menu{do it} しましょう。}
Boolean クラスのブラウザが開きます(\figref{browseBoolean})。
\ind{キーボードショートカット} \short{b} (browse)もあります。このキーボードショートカットは、クラス名が現われる場所なら、どのツールの中でも使えます;
\index{キーボードショートカット!browse it}
つまり、クラス名を選択して \short{b} をタイプします。

\dothis{キーボードショートカットを使って、\ct{Boolean} クラスをブラウズしましょう。}

\begin{figure}[hbt]
\centerline {\includegraphics[width=\textwidth]{Kernel-objects-boolean}}
\caption{Boolean クラスの定義を表示しているブラウザ。
\figlabel{browseBoolean}}
\end{figure}

\ct{Boolean} クラスが選択されていて、プロトコルもメソッドも選択されていないときは、メソッドのソースコードの代わりに、\emph{クラス定義}が表示されることに注意してください(\figref{browseBoolean})。
このクラス定義自体、通常の \st のメッセージです。\st では、クラスは、親クラスにサブクラスの作成を依頼することで定義されます。
ここでは \ct{Object} クラスが、\ct{Boolean} という名前のサブクラスを作成するよう依頼されているのが分かります。\ct{Boolean} クラスは、インスタンス変数、クラス変数、「プール辞書」が全部空で、\scatind{Kernel-Objects} カテゴリ内に収められます。
% The lower pane shows the \emph{class comment} --- a piece of plain text describing the class.
クラスペインの下の \button{?} を \click すると、専用のペインの中に、クラス\subind{クラス}{コメント}を見ることができます(\figref{classComment})。

\begin{figure}[hbt]
\centerline {\includegraphics[width=\textwidth]{classComment}}
\caption{\ct{Boolean} クラスのクラスコメント。
\figlabel{classComment}}
\end{figure}

クラスを見つけるには、名前で検索するのが、しばしば一番の早道です。例えば、日付と時間を表現するクラスを探しているとしましょう。

\dothis{ブラウザのパッケージペインにマウスを置き、\short{f} とタイプするか、\actclick して \menu{find class \ldots (f)} を選択しましょう。ダイアログボックスに「time」と入力しましょう。} 
\noindent
「time」を含んだクラス名のリストが表示されます(\figref{findit})。\ct{Time} を選択しましょう。ブラウザが \ct{Time} クラスを表示します。\ct{Time} のクラスコメントには、他のお勧めのクラスが書いてあります。これらをブラウズしたければ、その名前を選んで \short{b} をタイプします(実際は、どのテキストペインでもこの方法は使えます)。
\index{キーボードショートカット!find ...}
\index{キーボードショートカット!browse it}

\begin{figure}[hbt]
\centerline{
	\includegraphics[width=0.45\textwidth]{FindIt}
	\hspace{1cm}
	\includegraphics[width=0.45\textwidth]{TimeClasses}
}
\caption{名前でクラスを検索する。
\figlabel{findit}}
\end{figure}

検索ダイアログで、完全なクラス名を入力した場合(そして単語の頭を正しく大文字化した場合)、ブラウザは、候補リストを表示することなく、そのクラスに直行することに注意してください。

%=================================================================
\section{メソッドを見つける}
\seclabel{quick:methodFinder}

メソッド名または少なくともメソッド名の一部が、クラス名よりも簡単に推測できることもあるでしょう。例えばあなたは、現在の時間が知りたかったとして、それには「now」というメソッドや、「now」を名前に含んだメソッドが使える、と期待するかも知れません。しかしどこにあるのでしょう?
そのような時は、\emphind{メソッド・ファインダ}が助けてくれます。
\seeindex{ブラウザ!メソッドを見つける}{メソッド, 見つける}
\index{メソッド!見つける}

\dothis{\menu{World \go Tools ... \go Method finder} を選択しましょう。
左上のペインに「now」を入力し、\menu{accept} しましょう(または、\textsc{return} キーを押しましょう)。}
メソッド・ファインダは、「now」を含むすべてのメソッド名のリストを表示します。
\ct{now} までスクロールするには、カーソルをリストに移動して「\ct{n}」をタイプします; このトリックは、すべてのスクロールするウィンドウで使えます。「now」を選択しましょう。すると右側のペインには、この名前のメソッドを定義しているクラスの一覧が表示されます(\figref{MethodFinder})。クラス名を選択すると、そのクラスのブラウザが開きます。

\begin{figure}[hbt]
\centerline {\includegraphics[width=0.7\textwidth]{methodFinder-now}}
\caption{メソッド・ファインダ。\ct{now} の定義を含んだすべてのクラスをリストしている。
\figlabel{MethodFinder}}
\end{figure}

時には、メソッドが存在することは分かっていても、それが何という名前なのか、見当が付かないこともあるでしょう。
メソッド・ファインダは、このような場合にも助けてくれます! 例えば、文字列を大文字化するメソッドを探したかったとしましょう。\ct{'eureka'} を \ct{'EUREKA'} という具合にです。

\dothis{\ct{'eureka' . 'EUREKA'} とメソッド・ファインダに入力し、
  \textsc{return} キーを押しましょう
  (\figref{methodFinder-example1})。}
\noindent
メソッド・ファインダは、お望みのメソッドを勧めてくれるでしょう。\footnote{もしこのとき、ウィンドウがポップアップして、推奨されないメソッドについて警告されても、驚かないでください --- メソッド・ファインダは、単に、ありそうな候補すべてを表示しようとするだけです。その中に、推奨されないメソッドが含まれることもあります。そのような場合は、慌てず騒がず、~\button{Proceed} を \click してください。}

メソッド・ファインダの右のペインの行頭のアスタリスクは、そのメソッドが、望んだ結果を得るのに実際に使われたことを示します。
だから、\ct{String asUppercase} の先頭のアスタリスクを見れば、\clsind{String} で定義された \mthind{String}{asUppercase} が、実際の \ct{'eureka' . 'EUREKA'} の変換に使われたことがわかります。アスタリスクが付いていないメソッドは、単に名前が同じだと言うだけで挙げられたものです。\cmind{Character}{asUppercase} は、この場合実行されていません。\ct{'eureka'} は、\clsind{Character} オブジェクトではないからです。

\begin{figure}[hbt]
\centerline {\includegraphics[width=\textwidth]{MethodFinder-example1}}
\caption{例を使ってメソッドを見つける。
\figlabel{methodFinder-example1}}
\end{figure}

引数のあるメソッドについても、メソッド・ファインダを使うことができます; 例えば、二つの整数の最大公約数を見つけるメソッドを探すときは、\ct{25. 35. 5} を例として入力することができます。また、複数の例を使って、検索の範囲を絞りこむことができます; 下のペインのヘルプテキストに、やり方が書いてあります。

%=================================================================
\section{メソッドを新しく定義する}

\ind{テスト駆動開発}\cite{Beck03a} (TDD)が登場してから、コードを書く方法は一変しました。
TDD の背景にある思想は、テストを、コードそれ自身より先に書く、というものです。コードに期待する振舞いは、テストが定義します。
TDDでは、テストを書いて初めて、そのテストを満足させるコードを書きます。
\seeindex{振舞い駆動開発}{テスト駆動開発}
% \orla{describe the technique where we write a test hat ... subsequently we write ...}

「何かを大声で強調して言う」メソッドを書く、という課題があったとしましょう。これは正確にはどういう意味でしょう? そのようなメソッドの名前として相応わしいのは何でしょう? 将来そのメソッドを保守することになるかも知れないプログラマに、このメソッドが何をすべきなのか、確実に伝えるにはどうすればいいでしょう? 次の例を示すことによって、これらの疑問のすべてに答えることができます:

\begin{quote}
文字列「Don't panic」にメッセージ \ct{shout} を送ると、結果は「DON'T PANIC!」でなければならない。
\end{quote}

\noindent
この例を、システムが理解できるように、テストメソッドに変換します:
\index{testing}
\index{SUnit}

\needlines{3}
\begin{method}[testShout]{shout メソッドのテスト}
testShout
	self assert: ('Don''t panic' shout = 'DON''T PANICBANG')
\end{method} % BANG is the escape for !

どうしたら、\pharo で新しくメソッドが作れるでしょうか? 最初に、そのメソッドがどのクラスに属するか決めなければいけません。
この場合、今からテストしようとしている \ct{shout} メソッドが \clsind{String} クラスのものなので、そのテストは、習慣として、\clsind{StringTest} クラス内に作成します。

\begin{figure}[hbt]
\centerline {\includegraphics[width=\textwidth]{StringTest-newMethodTemplate}}
\caption{\ct{StringTest} の新規メソッドのテンプレート。
\figlabel{newMethodTemplate}}
\end{figure}

\dothis{\ct{StringTest} のブラウザを開きましょう。\ct{testShout} のための適切なプロトコルとして、\menu{tests - converting} を選択しましょう(\figref{newMethodTemplate})。
下のペインで選択されているのは、メソッドのテンプレートです。これを見れば、\st のメソッドの大まかな姿がわかります。
これを消して、\mthref{testShout} のコードを入力しましょう。}
ブラウザにテキストを入力したときに、下のペインが赤く縁取りされることに注意してください。これにより、このペインへの変更が、まだ保存されていないことがわかります。
それでは、下のペインで \actclick して \menu{accept (s)} を選択するか \short{s} をタイプして、メソッドをコンパイル・保存してください。
\index{キーボードショートカット}
\index{キーボードショートカット!accept}
\seeindex{accept it}{キーボードショートカット, accept}

あなたがこの仮想イメージでコードをアクセプトするのが初めてであれば、多分、あなたの名前を入力するように促されるはずです。仮想イメージは、沢山の人が書いたコードからできているので、メソッドを作った・変更したのは誰か、記録しておくことは重要です。単純に、ファーストネーム、ラストネームを、空白を入れずにあるいはドットでつないで、入力します。

%\begin{figure}[hbt]
%\centerline {\includegraphics[width=0.35\textwidth]{initials}}
%\caption{Entering your initials.
%\figlabel{initials}}
%\end{figure}

\ct{shout} というメソッドはまだ無いので、ブラウザは、これが本当にあなたが意図した名前であるか確認するよう求めます\,---\,そして、他の可能性のある名前についても提案してきます(\figref{testShoutConfirm})。
単にタイプミスした時にはこの機能は非常に助かりますが、今回は \ct{shout} で\emph{間違いない}ので、というのは、これが今から作ろうとするメソッドなので、\figref{testShoutConfirm} のように、一番上の項目を選択しなければなりません。


%\begin{figure}[htb]
%\begin{minipage}[b]{0.48\textwidth}
%\centerline {\includegraphics[width=0.9\textwidth]{name}}
%\caption{Entering your name.\figlabel{name}}
%\end{minipage}
%\hfill
%\begin{minipage}[b]{0.48\textwidth}
%\centerline {\includegraphics[width=\textwidth]{testShoutConfirm}}
%\caption{Accepting the \ct{StringTest} method \ct{testShout}.\figlabel{testShoutConfirm}}
%\end{minipage}
%\end{figure}

\begin{figure}[htb]
\centerline {\includegraphics[width=0.6\textwidth]{name}}
\caption{あなたの名前を入力する。\figlabel{name}}
\end{figure}

\begin{figure}[htb]
\centerline {\includegraphics[width=\textwidth]{testShoutConfirm}}
\caption{\ct{StringTest} の \ct{testShout} メソッドをアクセプトする。\figlabel{testShoutConfirm}}
\end{figure}


%\begin{figure}[hbt]
%\ifluluelse
%	{\centerline {\includegraphics[width=\textwidth]{testShoutConfirm}}}
%	{\centerline {\includegraphics[width=0.7\textwidth]{testShoutConfirm}}}
%\caption{Accepting the \ct{StringTest} method \ct{testShout}.
%\figlabel{testShoutConfirm}}
%\end{figure}

\dothis{今作ったテストを実行しましょう: \menu{World} から \ind{SUnit} \emphind{TestRunner} を開けましょう。}

一番左に二つ並んだペインは、ブラウザの上部のペインに少し似ています。左側のペインには、カテゴリのリストが表示されますが、テストクラスを含んだカテゴリに限られます。

\dothis{\scat{CollectionsTests-Text} を選択しましょう。右側のペインには、そのカテゴリのすべてのテストクラスが表示されます。その中に、\ct{StringTest} もあります。テストクラスの名前は既に選択されているので、\menu{Run Selected} を \click してこれらを実行しましょう。}

\begin{figure}[hbt]
\centerline {\includegraphics[width=\textwidth]{testRunnerOnStringTest}}
\caption{StringTest を実行する。
\figlabel{testRunnerTestShout}}
\end{figure}

テストを実行した結果が、\figref{testRunnerTestShout} のように表示され、テストの実行時にエラーがあったことが分かります。エラーを起こしたテストのリストは、右下のペインに表示されます; ご覧の通り、\ct{StringTest>>>#testShout} が犯人です
(\st では \ct{StringTest} クラスの \mthind{StringTest}{testShout} メソッドを、\ct{StringTest>>>#testShout} と表記して識別する習慣があります)。
\ct{StringTest>>>#testShout} を \click すると、再度そのテストだけが実行され、「\ct{MessageNotUnderstood: ByteString>>>shout}」というウィンドウが開きます。
\seeindex{\ct{>>}}{Behavior, \ct{>>}}
\cmindex{Behavior}{>>}

エラーメッセージとともに開いたこのウィンドウは、\st のデバッガです(\figref{predebugger})。
% \ab{Well, it's actually the \emph{pre-}debugger.  Does this matter?}\damien{I don't think it's important at this point.}
\ind{デバッガ}とその使い方については、\charef{env} で解説します。

\begin{figure}[hbt]
\centerline {\includegraphics[width=\textwidth]{Predebugger}}
\caption{(プリ)デバッガ}
\figlabel{predebugger}
\end{figure}

このエラーは、もちろん期待した通りのものです: 文字列が \ct{shout} するためのメソッドをまだ書いていないので、テストを実行するとエラーが発生します。
それでもなお、テストが失敗することを確認するのは良い習慣です。それによって、テスト装置が正しく設置され、新しいテストが実際に実行されたことが確認できるからです。
いったんエラーを確認したら、テストの実行をやめて(\button{Abandon})、デバッガウィンドウを閉じることができます。
ちなみに、\st ではしばしば、まだ無いメソッドを、\button{Create} を使ってその場で書いてしまうことができます。デバッガの中で、新しく作成されたメソッドを編集し、テストを続行(\button{Proceed})することができます。

では、テストを成功させるメソッドを作成しましょう!

\dothis{ブラウザで \clsind{String} クラスの \menu{converting} プロトコルを選択し、\mthref{shout} のテキストでメソッド作成テンプレートを上書きし、\menu{accept} しましょう
(注: \mbox{\ct{^}} のところには、\caret を入力します)。}
\begin{method}[shout]{shout メソッド}
shout
	^ self asUppercase, 'BANG'
\end{method}

カンマはこの場合、文字列連結演算子になります。つまりこのメソッドの本体は、\ct{shout} を受け取った任意の \ct{String} オブジェクトを大文字化して、末尾に感嘆符を付け足します。
\pharo では、$\uparrow$ 以下の式が、メソッドの戻り値になります。ここでは戻り値は、大文字化され、感嘆符が連結された、\emph{新しい}文字列です。
\seeindex{comma}{Collection, カンマ演算子}
\index{Collection!カンマ演算子}

このメソッドは、思いどおり動くでしょうか? もう一度テストを実行して、どうなるか見てみましょう。

\dothis{テストランナーで \menu{Run Selected} を再びクリックしましょう。今度は緑のバーが出て、すべてのテストが、失敗もエラーもなく実行されたことを示すテキストが、表示されます。}
緑のバーが出せた時は、仕事の成果を保存して一休みするのが良い考えです。
そうしましょう!
%\footnotetext{Actually, you might not get a green bar since some images contains tests for bugs that still need to be fixed.
%Don't worry about this.
%\pharo is constantly evolving.
%}

\begin{figure}[hbt]
\ifluluelse
	{\centerline{\includegraphics[width=\textwidth]{String-Shout}}}
	{\centerline{\includegraphics[width=0.7\textwidth]{String-Shout}}}
\caption{\ct{String} クラスで定義された \ct{shout} メソッド。
\figlabel{String-shout}}
\end{figure}

%=================================================================
\section{まとめ}
この章では、\pharo の環境を紹介し、ブラウザ、メソッド・ファインダ、テストランナーと言った主要なツールの使い方を学びました。\pharo の構文についても、全部ではありませんが、少しだけ学びました。
\begin{itemize}
  \item 実行中の \pharo システムは、\emph{バーチャルマシン}、\emph{ソースファイル}、\emph{イメージファイル}と\emph{チェンジファイル}で構成されます。この中で、最後の二つだけが変化します; これらは、実行中のシステムのスナップショットを記録します。
  \item \pharo 仮想イメージをロードすると、以前とまったく同じ状態\,---\,実行中のオブジェクトもそのままに\,---\,仮想イメージを保存したときの状態が再現されます。
  \item \pharo は、3 ボタンマウスを使って \click、\actclick、\metaclick するように設計されています。3 ボタンマウスが無くても、キーボードの修飾キーを使えば、同じことができます。
  \item \pharo の背景部分を \click すれば、\emph{ワールドメニュー}が表示できます。ワールドメニューからは、様々なツールが起動できます。
  \item \emph{ワークスペース}は、コードの断片を書いて評価するためのツールです。ワークスペースにはまた、任意のテキストを書いておくことができます。
  \item コードを評価するのに、ワークスペースや他の任意のツールのテキストの上で、\ind{キーボードショートカット}を使うことができます。最も重要なのは、\menu{do it} (\short{d})、\menu{print it} (\short{p})、\menu{inspect it} (\short{i})、\menu{explore it} (\short{I})、そして \menu{browse it} (\short{b})です。
%  \item \sqmap is a tool for loading useful packages from the Internet.
  \item \emph{ブラウザ}は、\pharo のコードをブラウズしたり新しくコードを開発したりするための、最も重要なツールです。
  \item \emph{テストランナー}は、ユニットテストを実行するためのツールです。テストランナーは、テスト駆動開発もサポートします。
\end{itemize}

%=================================================================
\ifx\wholebook\relax\else 
   \bibliographystyle{jurabib}
   \nobibliography{scg}
   \end{document}
\fi
%=================================================================

%%% Local Variables:
%%% coding: utf-8
%%% mode: latex
%%% TeX-master: t
%%% TeX-PDF-mode: t
%%% ispell-local-dictionary: "english"
%%% End:
