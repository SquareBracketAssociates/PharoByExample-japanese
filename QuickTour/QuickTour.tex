% $Author$
% $Date$
% $Revision$

% HISTORY:
% 2006-12-01 - Andrew edited (split from FirstApp?)
% 2006-12-03 - Andrew first draft
% 2006-12-06 - Stef edit
% 2007-06-11 - Oscar edit
% 2007-07-03 - Stef review
% 2007-08-22 - Andrew corrections
% 2007-09-11 - Marcus review
% 2007-09-11 - Orla review
% 2009-07-04 - Oscar migrated to Pharo

%=================================================================
\ifx\wholebook\relax\else
% --------------------------------------------
% Lulu:
	\documentclass[a4paper,10pt,twoside]{book}
	\usepackage[
		papersize={6.13in,9.21in},
		hmargin={.75in,.75in},
		vmargin={.75in,1in},
		ignoreheadfoot
	]{geometry}
	\input{../common.tex}
	\pagestyle{headings}
	\setboolean{lulu}{true}
% --------------------------------------------
% A4:
%	\documentclass[a4paper,11pt,twoside]{book}
%	\input{../common.tex}
%	\usepackage{a4wide}
% --------------------------------------------
    \graphicspath{{figures/} {../figures/}}
	\begin{document}
	% \renewcommand{\nnbb}[2]{} % Disable editorial comments
	\sloppy
\fi
%=================================================================
\newcommand{\clover}{%
	\raisebox{-0.8ex}[0pt][0pt]{%
		\includegraphics[width=1em]{cloverleafKey}}}
%=================================================================
\chapter{\pharo の使い方}
\chalabel{quick}

この章では、\pharo の快適な使い方について解説します。
読書するばかりで無く、パソコンを用意して実際にPharoを使っていくことができれば、一層理解が深まると思います。

\pharo を使って実際に試してみてもらいたい箇所には、このアイコン: \dothisicon{} でマークしておきます。
\pharo の起動や、\pharo と対話する色々な方法を学んだり、基本的なツールを見付けたり、
メソッドの作成方法や、オブジェクトを作成し、それにメッセージを送る方法を学ぶことになるでしょう。

%=================================================================
\section{使ってみよう}

\pharo は \pharoweb から自由に \ind{download} することができます。
3種のパーツから構成されていますが、実際には4個のファイルをダウンロードすることになります(\figref{download})。

\begin{figure}[htb]
\centerline {\includegraphics[width=\textwidth]{annotatedDownload-flat}}
\caption{サポートしているOS毎にダウンロードする \pharo のファイル。\figlabel{download}}
\end{figure}

\begin{enumerate}

  \item \emphind{仮想マシン} (VM)。VMは、システムで唯一、各OSとCPU毎に異なるパーツです。主要なプラットフォーム用の、コンパイル済みの仮想マシンが手に入ります。\figref{download}には、Windows用のVMである \textit{Pharo.exe} があります。

  \item \emphind{sources} ファイル。sources ファイルは、\pharo のすべてのパーツのソースコードを含みます。このファイルは、ほぼ変更されることはありません。\figref{download} では \emph{SqueakV39.sources}が sources ファイルです。\footnote{\pharo は \squeak 3.9 をベースに作らています。現在のVMは、\squeak と同じものを使っています。}

  \item 現在の \emph{システム\ind{イメージ}}は、実行中の \pharo システムのスナップショットです。このスナップショットは二つのファイルからなります: \emph{.}\emphind{image} ファイル。 .image ファイルは、システムのすべてのオブジェクトの状態を含みます(クラスやメソッドも含みます。これらもオブジェクトです)と、 \emph{.}\emphind{changes} ファイル。 .changes ファイルは、システムのソースコードの変更のログをすべて含みます。
\figref{download}では、これらのファイルは、 \emph{pharo.image} 、 \emph{pharo.changes} となっています。
\end{enumerate}

\dothis{\pharo をダウンロードし、インストールしてみましょう。}
\pharo by Example のウェブページからイメージをダウンロードすることをお勧めします。\footnote{\pbe}
\index{download}
\seclabel{sbeImage}

ここで紹介する内容は、殆どが \pharo のどんなバージョンでも動作します。もし、\pharo を既にインストールしているのなら、それを使っても問題有りません。
しかし、外見や動作のちょっとした違いがあっても驚かないで下さいね。
% On the other hand, if you are about to download \pharo for the first time, you may as well grab the \emph{\pharo by Example} image.

\pharo で何か作業している時、 .image ファイルと .changes ファイルは常に内容が更新されているため、確実に書き込み可能ファイルにしてください。
そして、これらは二つのファイルは、常に同じ場所に置いて下さい。
また、これらを、絶対にテキストエディターで直接変更しないでください。\pharo はこれらを、オブジェクトを格納したり、ソースコードの変更のログを書き出すのに使います。
ダウンロードしたファイルのバックアップコピーを取っておくのは良い考えです。こうしておけばいつでも、フレッシュなイメージから \pharo を開始し、(それがセーブしてあれば)あなたのプログラムを読み込むことができます。

\emphind{sources} ファイルとVMは書き込み不可にすることが可能です\,---\,これらを複数のユーザーで共有することができます。
仮想マシン・sources ファイル・image ファイル・changes ファイルは、同じ場所に置いても構いませんし、仮想マシン・sources ファイルだけ別の共有ディレクトリに置いても構いません。
OSや作業スタイルに合わせて使ってください。

%\sd{it would be really nice to have a setup and startup section on PC, Mac and Linux}
%\ab{I agree entirely; the reason that this is not here is because I could do only the Mac section.  Damien can probably do Windoze.  Perhaps we can ask on the list for a Linux volunteer?}

%-----------------------------------------------------------------
\begin{figure}[htb]
% \centerline {\includegraphics[width=0.6\textwidth]{download}}
\centerline {\includegraphics[width=\textwidth]{startup}}
\caption{\pbe からダウンロードした状態の イメージ\figlabel{startup}}
\end{figure}

\index{launching \pharo}
\paragraph{起動してみましょう。} \pharo の起動は、普通のアプリケーションと同様です。仮想マシンのアイコンに \emph{.}\emphind{image} ファイルをドラッグ・アンド・ドロップしたり、 \emph{.image} ファイルをダブルクリックしたり、コマンドラインで、仮想マシンの名前の後に \emph{.image} へのパスを入力することになるでしょう。使っているOSに合わせて下さい(もし、異なるバージョンのVMを持っておく必要があるなら、どのVMが動作しているかはっきりさせるためにも、ドラッグ・アンド・ドロップまたはコマンドラインからの起動を勧めます)。

\pharo を起動すると、大きめのウィンドウが現れます。このウィンドウの中ではさらに複数のワークスペースウィンドウを開くことができます(\figref{startup}を見てください)。しかし、このあとどう続けたらいいか、明かではありません!
メニューバーも有りますが、\pharo では主にコンテキスト(文脈)に依存したポップアップメニューを使うことになります。
% You will notice that there is no menu bar, or at least not a useful one.  
% Instead,  \pharo makes heavy use of context-dependent pop-up menus.

\dothis{\pharo を始めましょう。ウィンドウの左上にある赤いクローズボタンを \click すると、ワークスペースを閉じることができます。}

ウィンドウを最小化するにはオレンジ色のボタン、最大化するには緑色のボタンを \click してください。

%-----------------------------------------------------------------
\paragraph{最初の対話。}

\figref{threeButtons:click} のように、\ind{world メニュー} を表示しましょう。

\dothis{マウスカーソルがメインウィンドウの背景にある状態で、マウスのボタンをクリックし、world メニューを表示しましょう。そして、\menu{Workspace} を選択して、新しいワークスペースを出しましょう。}

%\begin{figure}[tbh]
%	\centering
%	\subfigure[The world menu]{\figlabel{threeButtons:click}% click
%		\includegraphics[width=0.25\linewidth]{worldMenu}}\hfill
%	\subfigure[The contextual menu]{\figlabel{threeButtons:actclick}% action click
%		\includegraphics[width=0.35\linewidth]{yellowButtonMenuOnWorkspace}}\hfill
%	\subfigure[The morphic halo]{\figlabel{threeButtons:metaclick}% meta click
%		\includegraphics[width=0.35\linewidth]{morphicHaloOnWorkspace}}% these braces needed (else no whitespace at end of line)
%	\caption{The world menu (brought up by \click{ing}), a contextual menu (\actclick{ing}), and a morphic \subind{Morphic}{halo} (\metaclick{ing}).\figlabel{threeButtons}}
%\end{figure}

\begin{figure}[tbh]
	\centering
	\subfigure[world メニュー]{\figlabel{threeButtons:click}% click
		\includegraphics[width=0.40\linewidth]{worldMenu}}\hfill
	\subfigure[コンテキストメニュー]{\figlabel{threeButtons:actclick}% action click
		\includegraphics[width=0.55\linewidth]{yellowButtonMenuOnWorkspace}}\hfill
	\subfigure[morphic ハロー]{\figlabel{threeButtons:metaclick}% meta click
		\includegraphics[width=0.60\linewidth]{morphicHaloOnWorkspace}}% these braces needed (else no whitespace at end of line)
	\caption{world メニュー(\click)、コンテキストメニュー(\actclick)、morphic \subind{Morphic}{ハロー} (\metaclick).\figlabel{threeButtons}}
\end{figure}
\seeindex{morphic ハロー}{Morphic}

%% ON: I had to shrink this and move it up to avoid
%% it running over the end of the page.
%\begin{wrapfigure}[19]{r}{0.25\linewidth}
%% The parameters are the number of narrow lines to the right of the figure [19],
%% the placement {r} for right, and the width of the figure. Capital R will allow some float.
%% Inside the wrapfig environment, linewidth is special --- the width of the figure.
%\includegraphics[width=0.95\linewidth]{colouredMouse}
%\caption{The author's mouse. \click{ing} the scroll wheel activates the blue button.}
%\figlabel{colouredMouse}
%\end{wrapfigure}

\st は、元々の設計時点から、3ボタンマウスを想定しています。使っているマウスのボタンが足りないときは、修飾キーを押しながらマウスボタンを\click してください。2ボタンマウスでも、十分に \pharo を使うことができますが、もし1ボタンマウスしか持っていなら、クリックできるスクロールホイールが付いた2ボタンマウスを買うことを、真剣に考えましょう: これだけで、\pharo を、よりいっそう快適に使うことができます。

色々なコンピューターやマウス、キーボード、個人環境が有るため、\pharo では``左マウスボタンのクリック''という表現は避けています。
元来、\st は異なるマウスボタンを色で表現していました。\footnote{マウスボタンの色は、\emph{赤}、\emph{黄}そして\emph{青}です。この本の著者は、どの色がどのボタンを指していたのか、さっぱり思い出せません。}
\index{赤ボタン}
\index{黄ボタン}
\index{青ボタン}
マウスボタンが違ったり、修飾キー(\emph{control}キー、\emph{ALT}キー、\emph{meta}キー、\etc)を押しながらマウスを使うような状況を統一するため、次の用語を使います:
\begin{description}
\item [\click:] 一番使われるマウスボタンで、普通は、修飾キー等を使わずに1ボタンマウスを \click するのと同じです; 背景部分を \click して ``World'' メニュー(\figref{threeButtons:click})を表示してみましょう。
\item [\actclick:] 次によく使われるボタンです; これはコンテキストメニューを表示するボタンです。コンテキストメニューは、マウスポインタが指している場所により違うメニュー表示を行うことがあります; \figref{threeButtons:actclick}を見てください。マルチボタンのマウスを使わない場合、普通は、\emph{control} 修飾キーを使って \actclick をすることになります。
\item [\metaclick:] 最後に、画面に表示されたどんなオブジェクトに対しても、それを \metaclick して ``morphic \subind{Morphic}{ハロー}'' を有効にすることができます。ハローは、画面上のオブジェクトを、回転させたりリサイズしたりするためのハンドルの集りです; \figref{threeButtons:metaclick} を見てください。\footnote{\pharo は、いくつかの Morphic ハンドルの操作を、標準で制限しています。以降の ``Preferences Browser の使い方'' には、その制限を外す方法も書いてあります。}
バルーンヘルプ機能もありますので、マウスの使い方に困ることはありません。
\pharo で \metaclick する方法は、使っているオペレーティングシステムに依存ます。
{\sc shift} \emph{ctrl} または {\sc shift} \emph{option} を押しながらクリックすることになります。
% \ab{This makes it sound like either {\sc shift} \emph{ctrl} or {\sc shift} \emph{alt} will work.  On my (Mac OS) system, only the latter works.  Perhaps we want to say: In \pharo, how you meta-click depends on your operating system. On Linux \ldots}
% Typically you will use a third modifier key, such as \emph{command} or \emph{meta} to \metaclick.
\end{description}

\dothis{\ct{Time now} とワークスペースに入力しましょう。
そして、ワークスペースで \actclick して、
\menu{print it} を選択しましょう。}

%Now we will activate \metaclick{ing}.

%\dothis{Open the preference browser (\menu{System {\ldots\go} Preferences {\ldots\go} Preference Browser\ldots}) and find the \menu{halosEnabled} option using the search box.}

%\begin{figure}[htb]
%\centerline{\includegraphics[width=\textwidth]{PreferenceBrowser}}
%\caption{The Preference Browser.\figlabel{prefBrowser}}
%\end{figure}

%\dothis{Now you should be able to \metaclick on the workspace. (See \figref{threeButtons:metaclick}.)
%Grab the blue \raisebox{-0.4ex}{\includegraphics[width=1em]{morphicRotate}} handle near the bottom left corner and drag it to rotate the workspace.}

右利きの人なら、\click は左ボタンに、\actclick は右ボタンに、クリックできるスクロールホイールがあるマウスなら、\metaclick をそれに設定することをお勧めします。
% If you don't have a clickable scroll wheel, then you can get the Morphic halo by holding down the \ct{alt} or \ct{option} key while \click{ing}. 
% \ab{This doesn't work any more.  This sentence either repeats or contradicts the meta-click item above; neither is a good idea.}
あなたが1ボタンのマウスで Macintosh を使っているのなら、\clover{} を押しながらマウスを \click することで、\actclick か \metaclick をシミュレートすることもできます。それでも、これから \pharo を頻繁に使おうと言うのなら、最低でも二つのボタンの付いたマウスに投資することを、お勧めします。

オペレーティングシステムのマウスドライバの設定を変えることで、あなた好みの動作を行うこともできます。
\ab{How can I get meta-click without a three-finger salute?  Is this a secret?}
\pharo は、キーボードのメタキーや、マウスの設定を変更できる機能も持っています。
Preference Browser (\menu{System {\ldots\go} Preferences {\ldots\go} Preference Browser\ldots})では、\menu{keyboard} カテゴリーの \menu{swapControlAndAltKeys} オプションを使うことで、\actclick と \metaclick の機能を交換することができます。
コマンドキーの設定を変更するオプションも、用意されています。

\begin{figure}[htb]
\centerline{\includegraphics[width=\textwidth]{PreferenceBrowser}}
\caption{Preference Browser。\figlabel{prefBrowser}}
\end{figure}


%=================================================================
\section{Worldメニュー}
\index{worldメニュー}

\dothis{\pharo の背景で、もう一度 \click をしてみましょう。}
再び \menu{World} メニューが表示されるはずです。
\pharo のメニューは殆どがモーダルメニューではありませんので、右上の画鋲アイコンを \click することで、画面上に残しておくことができます。やってみましょう。
% Also, notice that menus appear when you click the mouse, but do not disappear when you release it; they stay visible until you make a selection, or until you click outside of the menu. You can even move the menu around by grabbing its title bar.

world メニューを使えば、\pharo の色々なツールに簡単にアクセスすることができます。

\dothis{\menu{World} と \menu{{}Tools \ldots} を良く見てみましょう。(\figref{threeButtons:click})}

そこには、ブラウザーやワークスペースと言った \pharo のコアツールが、いくつかリストされるでしょう。
次の章では、これらリストされたツールのほとんどを、見ることになります。

%=================================================================
\section{メッセージを送る}

\dothis{ワークスペースを開けましょう。そして以下のテキストを入力しましょう:}

\begin{code}{}
BouncingAtomsMorph new openInWorld
\end{code}

\dothis{\actclick でメニューを表示し、\menu{do it (d)} を選択しましょう(\figref{doit}を見てください)。}

\begin{figure}[htb]
\centerline {\includegraphics[width=0.8\textwidth]{Doit}}
\caption{式を``do it''している様子\figlabel{doit}}
\end{figure}

左上に、たくさんの原子が中で弾んでいるウインドウが、表示されたはずです。

あなたはたった今、最初の \st プログラムを評価したことになります!
\bam クラスへ \ct{new} メッセージを送り、生成された \bam のインスタンスに \ct{openInWorld} メッセージを送ったことになります。
\bam クラスは \ct{new} メッセージを理解しました。つまり、\ct{new} メッセージを扱う \emph{メソッド} を検索して、適切に動作しました。
同様に、\bam インスタンスも、\ct{openInWorld} に応答するメソッドを検索して、適切に動作しました。

Smalltalkerたちとしばらく話せば、あなたはすぐに、彼らが、``手続きをコールする''または、``メソッドを呼び出す''などの表現を使わず、代わりに、``メッセージを送る''と言うことに気付くでしょう。
これは、「オブジェクトは自分の動作に責任を持つ」というアイディアを反映しています。
あなたは決して 、オブジェクトに、何々をしろと\emph{命令する}ことはありません\,---\,代わに、メッセージを送って、何かしてほしいと、礼儀正しく\emph{頼み}ます。
あなたではなくオブジェクトが、メッセージを理解し、適切なメソッドを選択して実行します。

%=================================================================
\section{保存、終了と再起動について}

\dothis{先ほど表示させた弾む原子のウィンドウを \click クリックし、好きな所へウィンドウをドラッグしましょう。もう一度 \click して、ウィンドウを適当な場所に置きましょう。}

\begin{figure}[htb]
\begin{minipage}[b]{0.48\textwidth}
\centerline {\includegraphics[width=0.7\textwidth]{atoms}}
\caption{\bam のウィンドウ\figlabel{atoms}}
\end{minipage}
\hfill
\begin{minipage}[b]{0.48\textwidth}
\centerline {\includegraphics[width=0.7\textwidth]{saveAs}}
\caption{\menu{save as \ldots} ダイアログ\figlabel{saveas}}
\end{minipage}
\end{figure}

\dothis{\menu{World\go{}Save as \ldots} を選択して、``myPharo'' と入力し、\button{OK} ボタンを押しましょう。
そして、\menu{World\go{}Save and quit} を選択しましょう。}

もともとの image ファイルと changes ファイルがあった場所に、``myPharo.\ind{image}''と``myPharo.\ind{changes}''ができているはずです。これらのファイルには、\menu{Save and quit}したときの \pharo の動作状態が入っています。
この二つのファイルは何処にでも移動させて構いませんが、(オペレーティングシステムによっては)仮想マシンと \emph{sources} ファイルも一緒にしておく必要があるかもしれません。

\dothis{今作った``myPharo.image''ファイルを使って \pharo を起動しましょう。}

先ほど \pharo を終了した時とそっくりそのままの状態に戻ったことに気付くはずです。\bam も同じところにあり、原子は弾み続けているでしょう。

\pharo を起動すると、\pharo \ind{仮想マシン}は、指定された image ファイルを読み込みます。このファイルには、たくさんのオブジェクトのスナップショットが入っています。これらのオブジェクトには、既に書かれた大量のコードや、多くのプログラミングツール(これらはすべて、オブジェクトです)が含まれます。\pharo を使っていると、オブジェクトへメッセージ送ったり、新しいオブジェクトを作ったりすることになります。また、いくつかのオブジェクトは命を終え、オブジェクトに割り当てられていたメモリーは回収(\ie ガベージコレクション)されます。

\pharo を終了するとき、通常はいわゆる上書き保存になります。あるいは、先ほど行ったみたいに、別名での保存もできます。

\emph{.image} ファイルとは別に、\emph{.changes} ファイルというものがあります。
このファイルは、標準のツールを使ってソースコードに対して行ったすべての変更を、記録しています。
通常、このファイルはあまり意識する必要はありません。
しかし、\emph{.changes} ファイルは、エラーを回復したり、保存しそこなった変更を再現するのに重宝します。
これについては、後ほど!

あなたが使ってきたイメージは、1970年代後半に作られた、オリジナルの \st-80 のイメージの子孫です。
このイメージの中には、何10年と生き続けているオブジェクトもあります。

ソフトウェアプロジェクトを保存・管理する方法として、イメージは重要なメカニズムであると考えるかも知れませんが、それは違います。
すぐ後で見るように、ソースコードを管理し、ソフトウェアをチームで共有するには、Monticello などの、もっと良いツールがあります。
イメージは非常に便利ですが、こうしたツールがあるので、イメージを、無造作に作ったり捨てたりすることに慣れましょう。

\dothis{マウス(必要ならば修飾キーを使って) \bam を \metaclick してみましょう。\footnote{うまくいかない場合は、Preferences Browser の \ct{halosEnabled} オプションをチェックしてみてください。}}
\bam の morphic \subind{Morphic}{ハロー} と呼ばれる、色とりどりの円が表示されたはずです。
それらの円は、\emph{ハンドル}と呼ばれます。
十字のピンクのハンドルをクリックすると、\bam は消えるはずです。

%=================================================================
\section{ワークスペースと Transcript}
\seclabel{transcript}

\dothis{すべてのウィンドウを閉じましょう。\ind{transcript} と \ind{ワークスペース}を新しく開きましょう。(transcript は、\menu{World{\go}Tools ...} のサブメニューにあります。)}

\dothis{transcript とワークスペースの位置やサイズを変えて、ワークスペースをtranscriptにオーバーラップさせましょう。}
ウィンドウをリサイズするには、ウィンドウの角をドラッグするか、\metaclick して morphic ハローを出し、右下の黄色のハンドルをドラッグします。

常に、1個のウィンドウだけがアクティブです; そのウィンドウは前面にあって、枠が強調されています。
% The mouse cursor must be in the window in which you wish to type.

transcript はシステムメッセージのログを確認するときに良く使われるオブジェクトです。
いわば、``システムコンソール''と言えるでしょう。
%Note that the transcript is terribly slow, so if you keep it open and write to it certain operations can become 10 times slower.
%In addition the transcript is not thread-safe so you may experience strange problems if multiple objects write concurrently to the transcript.
% ON: I think the transcript has been made thread-safe now, right?

ワークスペースは、実験的に試してみたい \st コードの断片を入力するのに役立ちます。
ワークスペースには、単に覚書のテキストを書いておくこともできます。TODOリストや、あなたのイメージを使う人への手引きなどに使えるでしょう。
ワークスペースは、しばしば、保存されたイメージについてのドキュメントとして使われます。先にダウンロードした標準のイメージが一例です(\figref{startup}を見てください)。

\dothis{ワークスペースに次のテキストを入力してみましょう。}
\begin{code}{}
Transcript show: 'hello world'; cr.
\end{code}

ワークスペースの中で、今入力したテキストの色々な箇所を、ダブル \click してみましょう。
単語の上、文字列の終り(\ct{'})、テキスト全体の終り(\ct{.})、という具合に \click してみて、単語全体、文字列全体、テキスト全体が選択される様子を確認しましょう。

\dothis{入力したテキスト全体を選択して、\actclick で
\menu{do it (d)} を選択します。}
transcript ウィンドウに``hello world''と表示されたのを確認しましょう
(\figref{helloworld})。
もう一度やってみましょう。
(\menu{do it (d)}の中の\menu{(d)} は、 \emph{do it}へのキーボードショートカットが \short{d}であることを示しています。これについて詳しい話は、次の節で!)

\begin{figure}[htb]
\centerline {\includegraphics[width=\textwidth]{HelloWorld}}
\caption{オーバーラップしたウィンドウ。ワークスペースがアクティブ。\figlabel{helloworld}}
\end{figure}

%=================================================================
\section{キーボードショートカット}

式を評価したいときは、\actclick ではなく \ind{キーボードショートカット}を使うこともできます。メニューの括弧書きの部分が該当します。ショートカットキーを使う時には、プラットフォームに応じて、control、alt、command、meta 等の修飾キーを同時に使用することになるでしょう
(このようなキーボードショートカットを、一般的に、\short{\emph{キー}}と表記することにします)。

\dothis{キーボードショートカットを使って、再びワークスペースの式を評価してみましょう: \short{d}。}
\index{キーボードショートカット!do it}

\menu{do it} だけでなく、\menu{print it}、 \menu{inspect it}、 \menu{explore it} にも気付いたでしょう。これらについて手短に説明します。

\dothis{\ct{3 + 4} とワークスペースの中に入力し、キーボードショートカットを使って \menu{do it} を実行してみましょう。}

何も起きなかったことに驚かないでください。ここでは数字の \ct{3} に数字の \ct{4} を引数とした \ct{+} メッセージを送ったことになります。普通に \ct{7} が計算され、返されたのですが、ワークスペースはこの答えをどうするべきか知らないので、答えを捨ててしまいます。結果を表示させたければ、\menu{print it} を使うことになります。\menu{print it} は、式をコンパイルし、実行し、その結果に \ct{printString} を送って得た文字列を表示します。

\dothis{\ct{3+4} を選択し、\menu{print it} を実行しましょう(\short{p})。}
今度は予想通りの結果です(\figref{printit})。
\index{キーボードショートカット!print it}

\begin{figure}[htb]
% \centerline {\includegraphics[width=0.4\textwidth]{PrintIt}}
\centerline {\includegraphics[width=0.8\textwidth]{PrintIt}}
\caption{``do it'' では無く ``Print it''。\figlabel{printit}}
\end{figure}

\needlines{3}
\begin{code}{@TEST}
3 + 4 --> 7
\end{code}
\noindent
この本の習慣として、特定の \pharo の式を\menu{print it}したらどうなるかを示すのに、\ct{-->}の表記を用います。

\dothis{選択されている ``\ct{7}'' を削除し(\pharo は既に``\ct{7}''を選択しているはずです。だから、delete キーを押すだけです)、\ct{3+4}をもう一度選択してから、今度は \menu{inspect it} (\short{i})しましょう。}
\noindent
整数クラスの一つである \ct{SmallInteger: 7} (\figref{inspectit}) の \emphind{inspector}(インスペクタ)ウィンドウが表示されるはずです。
inspector はどんなオブジェクトでもあなたと対話することができる優れたツールです。
タイトルの意味は、\ct{7} は \clsind{SmallInteger} クラスのインスタンスであるという意味です。
左のパネルはオブジェクトのインスタンス変数を示し、その値が右のパネルに表示されます。
下のパネルは、オブジェクトへメッセージを送るための式を記述することができます。

\begin{figure}[htb]
\centerline {\includegraphics[width=\textwidth]{InspectIt}}
\caption{オブジェクトのインスペクト中(検査中の意味)。 \figlabel{inspectit}}
\end{figure}

\dothis{下のパネルに \ct{self squared} と入力し、\menu{print it} を実行した結果。}

\needlines{2}
\dothis{inspector を閉じ、\ct{Object} をワークスペースに入力して、\menu{explore it} (\short{I}, アルファベット大文字 i)を実行してみましょう。}
\index{キーボードショートカット!explore it}
\index{explorer}

今度は、\clsind{Object} のタイトルが付いた、
\mbox{$\triangleright$ \ct{root: Object}} と表示されているウィンドウが開きます。
三角のアイコンを使って、中身を表示してください (\figref{exploreit})。

\begin{figure}[htb]
\centerline {\includegraphics[width=0.7\textwidth]{ExploreIt}}
\caption{\ct{Object} の explorer(調査)。\figlabel{exploreit}}
\end{figure}

explorer は inspector と似ていますが、ツリー構造により複雑なオブジェクトの中身を表示します。
今私たちは、\ct{Object} クラスの中身を explorer で見ていることになります。
このウィンドウではすべての情報を参照し、操作することが可能です。

%=================================================================
\section{Class Browser}

class \ind{browser}(クラスブラウザー)\footnote{紛らわしいことに、``system browser'' や ``code browser'' など呼ばれることがあります。\pharo は \ind{OmniBrowser} (``OB'' や ``Package browser'' と呼ばれてもいます)を実装しています。
文中ではシンプルに ``browser''(ブラウザー) を用いますが、曖昧さが問題になるケースでは、``class browser''(クラスブラウザー) と表記します。} はプログラミングにおいて、重要なツールです。
\pharo にはいくつかのブラウザーが有りますが、``class browser'' は最も基本的なブラウザーです。
\seeindex{class browser}{browser}

\dothis{\menu{World で \go Class browser を選択してください}。\footnote{もし、browser が \figref{classBrowser} に似ていないものであれば、デフォルトの browser を変更しましょう。\faqref{packagebrowser} を参照してください。}}

\begin{figure}[htb]
\ifluluelse
	{\centerline {\includegraphics[width=\textwidth]{ClassBrowser1}}}
	{\centerline {\includegraphics[width=0.7\textwidth]{ClassBrowser1}}}
\caption{browser で Object クラスの \ct{printString} メソッドを表示。
\figlabel{classBrowser}}
\end{figure}

\figref{classBrowser} を見てください。
タイトルに書かれているとおり、\clsind{Object} クラスをブラウズしています。
%\footnote{If the browser you have seems to differ from the one described in this book, you may be using an image with a different default browser. See \faqref{omnibrowser}.}

最初に browser を開くと、左のパネル以外はすべて空欄です。
左のパネルは、クラスを含んだすべての \emph{packages}(パッケージ) をリスト表示しています。
\index{category}

\dothis{\scatind{Kernel} パッケージを選択しましょう。}
選択されたパッケージに含まれるクラスが二番目のパネルにリスト表示されます。

\dothis{\clsind{Object} クラスを選択しましょう。}
%stub: 訳していないのか?
三番目のパネルは選択されたクラスの\emph{protocols}(プロトコル)が表示されます。
%stub: 訳していないのか?
\ind{protocol} を選択していなければ、すべてのメソッドが四番目のパネルに表示されます。

\dothis{\protind{printing} プロトコルを選択しましょう。}
%stub: 訳していないのか?
printing のプロトコルに属するメソッドが四番目のパネルに表示されます。

\dothis{\mthind{Object}{printString} メソッドを選択しましょう。}
下のパネルに\ct{printString} メソッドのソースコードが表示されます。このソースコードは、すべてのクラスで共有される、オーバーライドされていないソースコードになります。

%=================================================================
\section{クラスの検索}

\pharo でクラスを見つけるには、いくつかの方法が有ります。一つ目は、browser のナビゲーションを使ってカテゴリからたどっていく方法です。
\index{browser}
\seeindex{browser!クラスを見つけるには}{クラス, 検索}
\index{クラス!finding}

二つ目として、クラスに \ct{browse} メッセージを送り、browser を開く方法があります。\clsind{Boolean} クラスをブラウズしたいと考えてみましょう。

\dothis{ワークスペースに \ct{Boolean browse} と記述し、\menu{do it} してみましょう。}
Boolean クラスをポイントした状態の browser が開きましたか(\figref{browseBoolean})?
クラスをブラウズする場合は、クラス名を選択して、\ind{keyboard shortcut} \short{b} (browse) を使う方法もあります。
\index{キーボードショートカット!browse it}
%stub: 訳していないのか?

\dothis{キーボードショートカットを使って \ct{Boolean} クラスをブラウズしてみましょう。}

\begin{figure}[hbt]
\centerline {\includegraphics[width=\textwidth]{Kernel-objects-boolean}}
\caption{Boolean クラスを表示している状態の browser。
\figlabel{browseBoolean}}
\end{figure}

\ct{Boolean} クラスが選択されていますが、メソッドを選択していないときに\emph{クラス定義}が表示されていることに注意してください(\figref{browseBoolean})。
このクラス定義は、親クラスにサブクラスを生成するメッセージそのものです。
\ct{Object} クラスに対して、インスタンス変数、クラス変数、``pool dictionaries''(プール辞書)の定義を空っぽとし、\scatind{Kernel-Objects} カテゴリで \ct{Boolean} クラスを定義しています。
% The lower pane shows the \emph{class comment} --- a piece of plain text describing the class.
\button{?} をクリックすると、\subind{class}{comment}(クラスコメント)を見ることができます(\figref{classComment})。

\begin{figure}[hbt]
\centerline {\includegraphics[width=\textwidth]{classComment}}
\caption{\ct{Boolean} クラスのクラスコメント。
\figlabel{classComment}}
\end{figure}

クラスを見つける良い方法は、名前を使って探し出すことです。例えば、日付と時間に関するクラスを探していると仮定してみましょう。

\dothis{マウスをパッケージを表示しているペインに持って行き、\short{f} または \actclick{ing} で \menu{find class \ldots (f)} を選択し、表示されたダイアログに ``time'' と入力してみましょう。} 
\noindent
``time'' を含んだクラス名のリストが表示されます(see \figref{findit})。\ct{Time} を選択してみましょう。\ct{Time} クラスをポイントした状態で browser が表示されます。\short{b} を実行すると、その時にポイントしていたクラスを選択した状態で別の browser が開きます(訳注:\pharo のバージョンにより動作が微妙に異なりますが、感覚的に直ぐにわかります)。
\index{キーボードショートカット!find ...}
\index{キーボードショートカット!browse it}

\begin{figure}[hbt]
\centerline{
	\includegraphics[width=0.45\textwidth]{FindIt}
	\hspace{1cm}
	\includegraphics[width=0.45\textwidth]{TimeClasses}
}
\caption{名前からクラスを検索。
\figlabel{findit}}
\end{figure}

検索のためにクラス名を入力する時(先頭文字は大文字です)、リストに載っていなくてもブラウザー上ではそのクラスに移動してしまうことに注意してください。

%=================================================================
\section{メソッドの検索}
\seclabel{quick:methodFinder}

メソッド名はクラス名よりも簡単に推測することができます。例えば、今現在の時間に関することを考えた場合、``now''(現在) という言葉が思い当たるでしょう。
\emphind{method finder} はそのようなケースで非常に重宝されるツールです。
\seeindex{browser!メソッドの検索}{メソッド, 検索}
\index{メソッド!検索}

\dothis{\menu{World \go Tools ... \go Method finder} を選択し、
左上のペインに ``now'' を入力し、\textsc{return} キーを押してみましょう。}
method finder は、``now'' を名称に含むリストを表示します。
リスト表示のペインで ``\ct{n}'' をキーボードから入力すると、インクリメンタルサーチのような動作を行います。ここで、``now'' を選択すると、右側のペインにはメソッドを定義しているクラス名のリストが表示されます(\figref{MethodFinder} )。クラス名を選択すると、そのクラスをポイントした状態でブラウザーが立ち上がります。

\begin{figure}[hbt]
\centerline {\includegraphics[width=0.7\textwidth]{methodFinder-now}}
\caption{method finder でメソッド \ct{now} を定義しているクラスのリスト表示。
\figlabel{MethodFinder}}
\end{figure}

method finder はまだ他の使い方もあります。例えば、\ct{'eureka'} を \ct{'EUREKA'} とアルファベット大文字に変換するケースを考えてみましょう。
%@stub: 2行 -> 1行

\dothis{\ct{'eureka' . 'EUREKA'} と method finder に入力し、
  \textsc{return} キーを押してください
  (\figref{methodFinder-example1})。}
\noindent
method finder はあなたの思った通りを動作をしましたか?\footnote{警告のメッセージダイアログが出てもびっくりしないでください。method finder は単にありそうな候補すべてを表示するだけなので、推奨されないメソッドも含まれます。慌てず騒がず、\button{Proceed} をクリックしてください。}

右のペインでアスタリスクが付いているクラスは、実際にアルファベット大文字へ変換したときに使われたことを示します。このケースでは、\ct{String asUppercase} が実際に使われたということです。
アスタリスクが付いていないクラスは、単に同じ名前のメソッドが定義されているというだけでリストアップされています。\ct{'eureka'} は \clsind{Character} のオブジェクトではないので、\cmind{Character}{asUppercase} は使われなかったということです。

\begin{figure}[hbt]
\centerline {\includegraphics[width=\textwidth]{MethodFinder-example1}}
\caption{メソッド検索例。
\figlabel{methodFinder-example1}}
\end{figure}

他にも使い方は色々有ります。二つの整数の最大公約数を見つけるメソッドを見つけたいときは、\ct{25. 35. 5} と入力してみてください。下のペインではいくつかのサンプルを見ることができます。

%=================================================================
\section{メソッドの新規定義}

\ind{Test Driven Development}\cite{Beck03a} (TDD) は革新的なコードの記述方式です。
TDD の考え方はテスト先導です。プログラムコードを書く前にテストコードを作成します。
後は、そのテストコードを満足させるプログラムコードを書くという方式です。
\seeindex{Behavior Driven Development}{Test Driven Development}
% \orla{describe the technique where we write a test hat ... subsequently we write ...}

将来そのメソッドを修正するプログラマに、何を意図したメソッドであるか、そのメソッドの名前は何を付けるべきか注意する必要が有ります。例えば、次のような例を考えてみましょう。

\begin{quote}
文字列 ``Don't panic'' にメッセージ \ct{shout} を送った場合、``DON'T PANIC!'' という文字列を結果として得る。
\end{quote}

\noindent
テスト用のメソッドを作成してみます。
\index{testing}
\index{SUnit}

\needlines{3}
\begin{method}[testShout]{shout メソッドのテストメソッド}
testShout
	self assert: ('Don''t panic' shout = 'DON''T PANICBANG')
\end{method} % BANG is the escape for !

\pharo で新しくメソッドを作成するには、そのメソッドがどのクラスに属するかを決めなければいけません。
今回のケースでは、今から作成しようとしている \ct{shout} メソッドが \clsind{String} クラスに属することになりますので、テストメソッドは \clsind{StringTest} に作成することにしましょう。

\begin{figure}[hbt]
\centerline {\includegraphics[width=\textwidth]{StringTest-newMethodTemplate}}
\caption{\ct{StringTest} に作成した新規メソッドの雛形。
\figlabel{newMethodTemplate}}
\end{figure}

\dothis{browser で \ct{StringTest} クラスを選択してください。適切なプロトコルとして、\menu{tests - converting} を選択しましょう(\figref{newMethodTemplate})}。
browser の下のペインで表示されているのはテンプレートコードです。
すべて消した後に、\mthref{testShout} を入力してください。
browser にプログラムコードを入力したときに、ペインの一部が赤くなっているはずです。これは、入力したプログラムコードがまだ保存されていないというマーカーです。
\actclick のメニューから \menu{accept (s)} を選択するか、\short{s} でメソッドのプログラムコードを保存してください。
\index{キーボードショートカット}
\index{キーボードショートカット!accept}
\seeindex{accept it}{キーボードショートカット, accept}

使用しているイメージにはじめて書き込んだのであれば、あなたの名前を入力するように促されるはずです。たくさんの寄贈プログラムコードの一つとしてあなたの名前を入力しておきましょう。難しいことは考えずに、ファーストネーム、ラストネーム(間に空白或いはドットで繋ぎましょう)でいいでしょう。

%\begin{figure}[hbt]
%\centerline {\includegraphics[width=0.35\textwidth]{initials}}
%\caption{Entering your initials.
%\figlabel{initials}}
%\end{figure}

まだ \ct{shout} メソッドを作成していませんので、似たような名前がリストされた確認ダイアログが表示されます(\figref{name})。
タイプミスなどの時にはこの機能は非常に助かりますが、今回は本当に作成したい状況ですので、\figref{testShoutConfirm} のように一番上の項目を選択します。


%\begin{figure}[htb]
%\begin{minipage}[b]{0.48\textwidth}
%\centerline {\includegraphics[width=0.9\textwidth]{name}}
%\caption{Entering your name.\figlabel{name}}
%\end{minipage}
%\hfill
%\begin{minipage}[b]{0.48\textwidth}
%\centerline {\includegraphics[width=\textwidth]{testShoutConfirm}}
%\caption{Accepting the \ct{StringTest} method \ct{testShout}.\figlabel{testShoutConfirm}}
%\end{minipage}
%\end{figure}

\begin{figure}[htb]
\centerline {\includegraphics[width=0.6\textwidth]{name}}
\caption{あなたの名前を入力しましょう。\figlabel{name}}
\end{figure}

\begin{figure}[htb]
\centerline {\includegraphics[width=\textwidth]{testShoutConfirm}}
\caption{\ct{StringTest} クラスの \ct{testShout} メソッド。\figlabel{testShoutConfirm}}
\end{figure}


%\begin{figure}[hbt]
%\ifluluelse
%	{\centerline {\includegraphics[width=\textwidth]{testShoutConfirm}}}
%	{\centerline {\includegraphics[width=0.7\textwidth]{testShoutConfirm}}}
%\caption{Accepting the \ct{StringTest} method \ct{testShout}.
%\figlabel{testShoutConfirm}}
%\end{figure}

\dothis{\menu{World} から \ind{SUnit} \emphind{TestRunner} を選択し、新しいテストを実行しましょう。}

一番左のペインはカテゴリ、その右のペインはカテゴリに含まれるテスト用のクラスです。

\dothis{\scat{CollectionsTests-Text}を選択してください。表示された \ct{StringTest} を選択し、\menu{Run Selected} ボタンで実行しましょう。}

\begin{figure}[hbt]
\centerline {\includegraphics[width=\textwidth]{testRunnerOnStringTest}}
\caption{StringTest クラスでのテスト実行。
\figlabel{testRunnerTestShout}}
\end{figure}

テストを実行した結果が、\figref{testRunnerTestShout} のように表示されます。エラーがある箇所は右下のペインに表示されます(\ct{StringTest>>>#testShout})。
%@stub: 訳していないのか?
\ct{StringTest>>>#testShout} をクリックすると、再度そのテストだけが実行され、``\ct{MessageNotUnderstood: ByteString>>>shout}'' が表示されます。
\seeindex{\ct{>>}}{振る舞い, \ct{>>}}
\cmindex{振る舞い}{>>}

このウィンドウは、\st のデバッガ(debugger) です(\figref{predebugger})。
% \ab{Well, it's actually the \emph{pre-}debugger.  Does this matter?}\damien{I don't think it's important at this point.}
\ind{debugger} については、\charef{env} で解説します。

\begin{figure}[hbt]
\centerline {\includegraphics[width=\textwidth]{Predebugger}}
\caption{debugger.}
\figlabel{predebugger}
\end{figure}

このエラーは、意図したとおりの動作になります。まだ \ct{shout} を実装していませんので、当然のことになります。
しかしながら、当然のエラーが発生したということは、今までの実践内容がすべて正しかったという裏返しの確認でもあります。
取り敢えず、\button{Abandon} ボタンを押して、ウィンドウを閉じましょう。
因みに、\button{Create} はデバッガの中で新しくメソッドを作成することを促されますし、\button{Proceed} ボタンはそのまま続行することになります。

では、テストが成功するためのメソッドを作成しましょう!

\dothis{browser で \clsind{String} クラスの \menu{converting} プロトコルを選択し、メソッドとして \mthref{shout} を定義し、保存しましょう
(Note: \mbox{\ct{^}} は、\caret を入力します)。 }
\begin{method}[shout]{shout メソッド}
shout
	^ self asUppercase, 'BANG'
\end{method}

カンマは連結演算子です。文字列がアルファベット大文字に変換された後、最後に感嘆符を付けます。
$\uparrow$ は \pharo では返される答えになります。$\uparrow$ 以下の文字列(感嘆符も付きます)が返される答えです。
\seeindex{comma}{Collection, カンマ演算子}
\index{Collection!カンマ演算子}

できましたか?では、もう一度テストを実行しましょう。

\dothis{\menu{Run Selected} を再実行します。今度は緑色でエラー無し状態のウィンドウを見るはずです。}
どうでしたか?うまくいきました?
%@stub: 3行 -> 2行
%\footnotetext{Actually, you might not get a green bar since some images contains tests for bugs that still need to be fixed.
%Don't worry about this.
%\pharo is constantly evolving.
%}

\begin{figure}[hbt]
\ifluluelse
	{\centerline{\includegraphics[width=\textwidth]{String-Shout}}}
	{\centerline{\includegraphics[width=0.7\textwidth]{String-Shout}}}
\caption{\ct{String} クラスに定義された \ct{shout} メソッド。
\figlabel{String-shout}}
\end{figure}

%=================================================================
\section{本章のまとめ}
本章では、\pharo の構文をすこしばかりと、様々なツールの簡単な使い方を学びました。

\begin{itemize}
  \item \pharo は、\emph{仮想マシン}、 \emph{sources} ファイル、 \emph{image} ファイルと \emph{changes} ファイルで構成されており、最後の二つだけが、実行中のシステムのスナップショットとして変化していきます。
  \item \pharo は、あなたが前回終了したときと全く同じ状態で再起動します。
  \item \pharo は、3ボタンマウスを使うことで最高の使い勝手を提供します。もちろん、マウスがなくても、キーボードの修飾キーを使うことで、\click 、 \actclick と \metaclick が使えます。
  \item \pharo の背景部分をクリックすることで、\emph{World menu} が表示できます。
  \item \emph{ワークスペース}では、プログラムコードを試すこともできますし、単なるメモ帳としても使えます。
  \item プログラムコードを評価するのに、\menu{do it} (\short{d})、 \menu{print it} (\short{p})、 \menu{inspect it} (\short{i})、 \menu{explore it} (\short{I}) と \menu{browse it} (\short{b}) が使えます。
%  \item \sqmap is a tool for loading useful packages from the Internet.
  \item \emph{browser} は \pharo のプログラムコードを見たり開発したりするための重要なツールです。
  \item \emph{test runner} はユニットテストのための環境を提供します。
\end{itemize}

%=================================================================
\ifx\wholebook\relax\else 
   \bibliographystyle{jurabib}
   \nobibliography{scg}
   \end{document}
\fi
%=================================================================

%%% Local Variables:
%%% coding: utf-8
%%% mode: latex
%%% TeX-master: t
%%% TeX-PDF-mode: t
%%% ispell-local-dictionary: "english"
%%% End:
