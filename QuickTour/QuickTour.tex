% $Author$
% $Date$
% $Revision$

% HISTORY:
% 2006-12-01 - Andrew edited (split from FirstApp?)
% 2006-12-03 - Andrew first draft
% 2006-12-06 - Stef edit
% 2007-06-11 - Oscar edit
% 2007-07-03 - Stef review
% 2007-08-22 - Andrew corrections
% 2007-09-11 - Marcus review
% 2007-09-11 - Orla review
% 2009-07-04 - Oscar migrated to Pharo

%=================================================================
\ifx\wholebook\relax\else
% --------------------------------------------
% Lulu:
	\documentclass[a4paper,10pt,twoside]{book}
	\usepackage[
		papersize={6.13in,9.21in},
		hmargin={.75in,.75in},
		vmargin={.75in,1in},
		ignoreheadfoot
	]{geometry}
	\input{../common.tex}
	\pagestyle{headings}
	\setboolean{lulu}{true}
% --------------------------------------------
% A4:
%	\documentclass[a4paper,11pt,twoside]{book}
%	\input{../common.tex}
%	\usepackage{a4wide}
% --------------------------------------------
    \graphicspath{{figures/} {../figures/}}
	\begin{document}
	% \renewcommand{\nnbb}[2]{} % Disable editorial comments
	\sloppy
\fi
%=================================================================
\newcommand{\clover}{%
	\raisebox{-0.8ex}[0pt][0pt]{%
		\includegraphics[width=1em]{cloverleafKey}}}
%=================================================================
\chapter{\pharo の使い方}
\chalabel{quick}

この章では、\pharo の快適な使い方について解説します。
読書するばかりで無く、パソコンを用意して実際にPharoを使っていくことが出来れば、一層理解が深まると思います。
\pharo を使って実際に試してみてもらいたい箇所には、このアイコン: \dothisicon{} でマークしておきます。
\pharo の起動や、\pharo と対話する色々な方法を知ったり、基本的なツールの使い方を理解し、メソッドの作成やクラスの作成、メッセージの使い方を覚える事になるでしょう。

%=================================================================
\section{使ってみよう}

\pharo は \pharoweb から \ind{download} することが出来ます。
3種のパーツから構成されていますが、実際には4個のファイルをダウンロードする事になります(\figref{download})。

\begin{figure}[htb]
\centerline {\includegraphics[width=\textwidth]{annotatedDownload-flat}}
\caption{サポートしているOS毎にダウンロードする \pharo のファイル\figlabel{download}}
\end{figure}

\begin{enumerate}

  \item \emphind{バーチャルマシン} (VM)(\figref{download} の \textit{Pharo.exe})は、各OS毎に異なるファイルです。

  \item \emphind{sources}ファイル(\figref{download} の \emph{SqueakV39.sources})は、\pharo の全てのソースコードを含んでいます。
このファイルはほぼ変更することはありません。\footnote{\pharo は\squeak 3.9 をベースに作らています。現在のVMは、\squeak と同じものを使っています。}

  \item 現在の \emph{system \ind{image}} は \emph{.}\emphind{image} ファイル(クラスやメソッド、オブジェクトなどをすべて含みます)と \emph{.}\emphind{changes} ファイル(システムイメージに行った変更のログをすべて含みます)の2つで構成された、\pharo システムのスナップショットになります(\figref{download} の \emph{pharo.image} と \emph{pharo.changes})。
\end{enumerate}

\dothis{\pharo をダウンロードし、インストールしてみましょう。}
\pharo by Example のウェブページからimageファイルをダウンロードしてみましょう。\footnote{\pbe}
\index{download}
\seclabel{sbeImage}

ここで紹介する内容は、殆どが \pharo のどんなバージョンでも動作します。
もし、\pharo を既にインストールしているのなら、それを使っても問題有りません。
しかし、外見や動作のちょっとした違いがあっても驚かないで下さいね。

\pharo でなにか作業している時、imageファイルとchangesファイルは常に内容が更新されているため、確実に書き込み可能ファイルにし、同じ場所に置いて下さい。
また、絶対に他のアプリケーションで直接変更しないでください。
ダウンロードしたファイルは、バックアップのコピーを取っておきましょう。

\emphind{sources} ファイルとVMは書き込み不可にすることが可能で、複数のユーザーで共有することが出来ます。
同じ場所に置いたものを共有で使っても構わないですし、個人ごとにファイルをコピーしてもいいでしょう。
OSや作業スタイルに合わせて使ってください。

%-----------------------------------------------------------------
\begin{figure}[htb]
\centerline {\includegraphics[width=\textwidth]{startup}}
\caption{\pbe からダウンロードした状態の imageファイル\figlabel{startup}}
\end{figure}

\index{launching \pharo}
\paragraph{起動してみましょう。} \pharo の起動は、普通のアプリケーションと同様です。
VMファイルに\emph{.}\emphind{image} ファイルをドラッグ・アンド・ドロップしたり、\emph{.image} ファイルをダブルクリックする、コマンドラインで VM ファイルの後にパスを付けて \emph{.image} ファイルをタイプしてもいいでしょう。
使っているOSに合わせて下さい(もし、異なるバージョンのVMを持っておく必要があるなら、どのVMが動作しているかはっきりさせるためにも、ドラッグ・アンド・ドロップまたはコマンドラインからの起動を勧めます)。

\pharo を起動すると、大きめのウィンドウが現れます(\figref{startup}を見てください)。
このウィンドウの中では更に複数の workspace ウィンドウを開くことが出来ます。
メニューバーも有りますが、\pharo では主にコンテキスト(文脈)に依存したポップアップメニューを使うことになります。

\dothis{\pharo を始めましょう。ウィンドウの左上にあるクローズボタンを\click すると、workspace ウィンドウを閉じることができます。}

ウィンドウを最小化するにはオレンジ色のボタン、最大化するには緑色のボタンを\click してください。

%-----------------------------------------------------------------
\paragraph{\pharo を使い始めましょう。}

\figref{threeButtons:click} のように、\ind{world menu} を表示しましょう。

\dothis{マウスカーソルがメインウィンドウのバックグラウンドにある状態で、マウスのボタンをクリックし、表示された world メニューから \menu{Workspace} を選択すると、新しい workspace ウィンドウが出てきます。}

\begin{figure}[tbh]
	\centering
	\subfigure[world メニュー]{\figlabel{threeButtons:click}% click
		\includegraphics[width=0.40\linewidth]{worldMenu}}\hfill
	\subfigure[コンテキストメニュー]{\figlabel{threeButtons:actclick}% action click
		\includegraphics[width=0.55\linewidth]{yellowButtonMenuOnWorkspace}}\hfill
	\subfigure[morphic halo]{\figlabel{threeButtons:metaclick}% meta click
		\includegraphics[width=0.60\linewidth]{morphicHaloOnWorkspace}}% these braces needed (else no whitespace at end of line)
	\caption{world メニュー(\click)、コンテキストメニュー(\actclick)、morphic \subind{Morphic}{halo} (\metaclick).\figlabel{threeButtons}}
\end{figure}
\seeindex{morphic halo}{Morphic}

\st は、元々の設計時点から3ボタンマウスを想定しています。
使っているマウスのボタンが足りないときは、キーボードの\emph{control}キーや\emph{ALT}キー、\emph{meta}キーなどが修飾キーとして使われます(キーを押しながらマウスボタンを\click してください)。
それでも、一番良い方法は3つボタンマウスを用意することです。
これだけで、\pharo をビックリするほど快適に使うことができます。

色々なコンピュータやマウス、キーボード、個人環境が有るため、\pharo では``左マウスボタンのクリック''という表現は避けています。
元来、\st は異なるマウスボタンを色で表現していました。\footnote{マウスボタンの色は、\emph{赤}、\emph{黄}そして\emph{青}です。}
\index{red button}
\index{yellow button}
\index{blue button}
マウスボタンが違ったり、修飾キー(\emph{control}キー、\emph{ALT}キー、\emph{meta}キー、\etc)を押しながらマウスを使うような状況を統一するため、次の用語を使います:

\begin{description}
\item [\click:] 
修飾キー等を使うことがない、一番使われる(\click される)マウスボタンです。背景部分を\click して ``World'' メニュー(\figref{threeButtons:click})を表示してみましょう。
\item [\actclick:] 
次によく使われるボタンはコンテキストメニューを表示するボタンです。コンテキストメニューは、マウスポインタが指している場所により違うメニュー表示を行うことがあります(\figref{threeButtons:actclick})。マルチボタンのマウスを使わない場合、\emph{control} 等の修飾キーを使った \actclick を行い、表示することになります。
\item [\metaclick:] 
  ``morphic \subind{Morphic}{halo}''を動かすには、どんな物の上でも \metaclick を実行することが出来ます。物を回転させたりリサイズを行ったり、\figref{threeButtons:metaclick} にあるような様々なことが可能です。\footnote{\pharo ではいくつかの機能は意図的に操作を制限しています。以降の Preferences Browser の使い方では、その制限を外す設定も書かれています。}
バルーンヘルプ機能もありますので、マウスの使い方に困ることはありません。
\pharo は、あなたの \metaclick の方法を、使っているオペレーティングシステムに依存しています。{\sc shift} \emph{ctrl} または {\sc shift} \emph{option} を押しながらクリックする事になります。

% \ab{This makes it sound like either {\sc shift} \emph{ctrl} or {\sc shift} \emph{alt} will work.  On my (Mac OS) system, only the latter works.  Perhaps we want to say: In \pharo, how you meta-click depends on your operating system. On Linux \ldots}
% Typically you will use a third modifier key, such as \emph{command} or \emph{meta} to \metaclick.
\end{description}

\dothis{\ct{Time now} を workspace 上で記述してみてください。
\actclick を表示して、
\menu{print it} を選択してみましょう。}

%Now we will activate \metaclick{ing}.

%\dothis{Open the preference browser (\menu{System {\ldots\go} Preferences {\ldots\go} Preference Browser\ldots}) and find the \menu{halosEnabled} option using the search box.}

%\begin{figure}[htb]
%\centerline{\includegraphics[width=\textwidth]{PreferenceBrowser}}
%\caption{The Preference Browser.\figlabel{prefBrowser}}
%\end{figure}

%\dothis{Now you should be able to \metaclick on the workspace. (See \figref{threeButtons:metaclick}.)
%Grab the blue \raisebox{-0.4ex}{\includegraphics[width=1em]{morphicRotate}} handle near the bottom left corner and drag it to rotate the workspace.}

右利きの人なら、\click は左ボタンに、\actclick は右ボタンに、ボタンとして機能するスクロールホイールがあるマウスなら、\metaclick をそれに設定することをお勧めします。
% If you don't have a clickable scroll wheel, then you can get the Morphic halo by holding down the \ct{alt} or \ct{option} key while \click{ing}. 
% \ab{This doesn't work any more.  This sentence either repeats or contradicts the meta-click item above; neither is a good idea.}
あなたがワンボタンのマウスで Macintosh を使っているのなら、\clover{} を押しながらマウスを \click することで、シミュレートすることも出来ます。それでも、\pharo を使うためには、やはり2ボタンのマウスを使ったほうが良いでしょう。
オペレーティングシステムのマウスドライバの設定を変えることで、あなた好みの動作を行うことも出来ます。
\ab{どうやったら3本指の洗礼無く、メタクリックを使うことが出来ますか?秘密じゃないですよね?}
\pharo は、あなたが使っているキーボードのメタキーやマウスの設定を変更できる機能も持っています。
Preference Browser (\menu{System {\ldots\go} Preferences {\ldots\go} Preference Browser\ldots})では、\menu{keyboard} カテゴリの \menu{swapControlAndAltKeys} オプションを使うことで、\actclick と \metaclick の機能を変更することが出来ます。
コマンドキーの設定を変更するオプションも用意されています。

\begin{figure}[htb]
\centerline{\includegraphics[width=\textwidth]{PreferenceBrowser}}
\caption{The Preference Browser.\figlabel{prefBrowser}}
\end{figure}


%=================================================================
\section{World menu}
\index{world menu}

dothis{\pharo のバックグラウンドで、もう一度 \click をしてみましょう。}
\menu{World} menu が表示されましたか?
\pharo のメニューは殆どがモーダルメニューではありませんので、右上の画鋲アイコンを \click することで、画面上に残しておくことが出来ます。
% Also, notice that menus appear when you click the mouse, but do not disappear when you release it; they stay visible until you make a selection, or until you click outside of the menu. You can even move the menu around by grabbing its title bar.

world メニューは、\pharo の色々なツールを使うときに使います。

\dothis{\menu{World} と \menu{{}Tools \ldots} を見てみましょう。(\figref{threeButtons:click})}

次の章では、workspace や browser 等の \pharo のツールについて見ていきましょう。

%=================================================================
\section{メッセージ送信}

\dothis{workspaceを使って、以下のテキストを入力してみてください:}

\begin{code}{}
BouncingAtomsMorph new openInWorld
\end{code}

\dothis{\actclick でメニューを表示し、\menu{do it (d)} を選択してください。(\figref{doit}.)}

\begin{figure}[htb]
\centerline {\includegraphics[width=0.8\textwidth]{Doit}}
\caption{プログラムを実行します\figlabel{doit}}
\end{figure}

左上にアトム(小さな玉)が弾むウインドウが表示されたはずです。

あなたは最初の \st プログラムを実行したことになります。
\bam クラスへ \ct{new} メッセージを送り、生成された \bam のインスタンスに続く \ct{openInWorld} メッセージを送った事になります。
\bam クラスは \ct{new} メッセージを理解し、適切に動作を行いました。
同様に、\bam インスタンスも \ct{openInWorld} を適切に処理しています。

Smalltalker(スモールトーカー:\st を使ってプログラミングする人たちをこう呼びます)は、``メソッドを呼び出す''などの言い方をせず、``メッセージを送る''という表現を使います。あなたもすぐにそうなるでしょう。
どうしたらいいのかは object が知っているので、あなたは object にメッセージを送るだけです。
object はメッセージを理解し、適切なメソッドを選択して実行します。


%=================================================================
\section{終了と再起動、保存について}

\dothis{先ほど表示させたアトムのウィンドウを \click クリックすると、ウィンドウを移動させることが出来ます。もう一度 \click すると、ウィンドウを置くことが出来るはずです。}

\begin{figure}[htb]
\begin{minipage}[b]{0.48\textwidth}
\centerline {\includegraphics[width=0.7\textwidth]{atoms}}
\caption{\bam のウィンドウ\figlabel{atoms}}
\end{minipage}
\hfill
\begin{minipage}[b]{0.48\textwidth}
\centerline {\includegraphics[width=0.7\textwidth]{saveAs}}
\caption{\menu{save as \ldots} ダイアログ\figlabel{saveas}}
\end{minipage}
\end{figure}

\dothis{\menu{World\go{}Save as \ldots} を選択して、``myPharo'' と入力し、\button{OK} ボタンを押してください。
その後、\menu{World\go{}Save and quit} を選択しましょう。}

もともとの image ファイルと changes ファイルがあった場所に、``myPharo.\ind{image}'' と ``myPharo.\ind{changes}'' が出来ているはずです。
この2つのファイルは何処にでも移動させて構いませんが、バーチャルマシンと \emph{sources} も一緒にしておく必要があります。

\dothis{では、``myPharo.image'' を使って \pharo を起動しましょう。}

先ほど、\pharo を終了したときのままに成っているはずです。\bam も同じところにあり、アトムは弾み続けているでしょう?

\pharo は起動時に \ind{バーチャルマシン}でイメージファイルを読み込みます。このファイルは、プログラムのコードやツール、オブジェクトそのものを数多く含んでいます。\pharo を使っていると、この object へメッセージを送ることで新しく作り出したり、消したり、使用していたメモリを回収(ガーベッジコレクション)したりすることになります。

\pharo を終了するとき、通常はいわゆる上書き保存になります。勿論、先ほど行ったみたいに、更に別名での保存もできます。

\emph{.changes} ファイルは、あなたが行った追加変更を全て記録しています。このファイルはエラーが起きたときなどに重宝しますが、通常はあまり意識する必要はありません。

このやり方は 1970年代の \st-80 から変わりません。

ソースコード管理という意味では、Monticello などの優れたツールが有ります。従って、このイメージというメカニズムによって作成されたファイルを消したり作ったり、とにかく無駄にすることを普通だと思う様になりましょう。

\dothis{マウス(必要ならば修飾キーを使って) \bam を \metaclick してみましょう。\footnote{うまくいかない場合は、Preferences Browser の \ct{halosEnabled} オプションをチェックしてみてください。}}
\bam の morphic \subind{Morphic}{halo} と呼ばれる、色とりどりの円が表示されたはずです。
それらの円は、\emph{handle} と呼ばれます。
十字ピンク色の円をクリックすると、\bam は消えるはずです。

%=================================================================
\section{Workspaces と Transcripts}
\seclabel{transcript}

\dothis{全てのウィンドウを閉じて、\ind{transcript} と \ind{workspace} を新しく開いてください。(transcript は、\menu{World{\go}Tools ...} のサブメニューにあります。)}

\dothis{transcript と workspace の位置やサイズを変えてみましょう。}
\metaclick で morphic halo の右下の黄色ハンドルを使うことでリサイズが出来ます。

常に、枠が強調されているウィンドウだけがアクティブになります。
% The mouse cursor must be in the window in which you wish to type.

transcript はシステムメッセージのログを確認するときに良く使われます。
いわば、``システムコンソール''と言えるでしょう。
%Note that the transcript is terribly slow, so if you keep it open and write to it certain operations can become 10 times slower.
%In addition the transcript is not thread-safe so you may experience strange problems if multiple objects write concurrently to the transcript.
% ON: I think the transcript has been made thread-safe now, right?

workspaces は実験的に試してみたい \st コードを使うときに役立ちます。
勿論、メモ帳としても使えます。
イメージに保存されるので、付箋がわりに使うのも良いアイデアです(\figref{startup})。

\dothis{workspace に次のコードを入力してみてください。}
\begin{code}{}
Transcript show: 'hello world'; cr.
\end{code}

workspace のウィンドウの中で、単語の上や単語の間、コードの最後のピリオドの後ろなどを \click して、選択されるエリアの確認をやってみましょう。

\dothis{コード全体を選択して、\actclick で \menu{do it (d)} を選択します。}
transcript ウィンドウに``hello world''と表示されたのを確認してみましょう(\figref{helloworld})。
(\menu{(d)} は \menu{do it (d)} のキーボードショートカットです。例えば、 \emph{do it} を \short{d} のように使うことが出来ます。)

\begin{figure}[htb]
\centerline {\includegraphics[width=\textwidth]{HelloWorld}}
\caption{workspace がアクティブとなった状態の transcript と workspace。\figlabel{helloworld}}
\end{figure}

%=================================================================
\section{キーボードショートカット}

コードを実行するときは、\actclick ではなく \ind{キーボードショートカット}を使うことも出来ます。キーボードショートカットを使えば、操作が簡単になります。メニューの括弧書きの部分が該当します。ショートカットキーを使う時には、\short{\emph{key}} のように control / alt / command / meta 等の修飾キーを同時に使用することになるでしょう。


\dothis{キーボードショートカットを使って、式を評価してみましょう(例:\short{d})。}
\index{キーボードショートカット!do it}

\menu{do it} だけでなく、\menu{print it}、 \menu{inspect it}、 \menu{explore it} も使えますが、それらについては後述します。

\dothis{\ct{3 + 4} と workspace の中に入力し、キーボードショートカットを使って \menu{do it} を実行してみましょう。}

何も起きなかったことに驚かないでください。ここでは数字の \ct{3} に数字の \ct{4} を引数とした \ct{+} メッセージを送ったことになります。プログラムが実行され、計算はされたのですが、表示することを知らないのです。結果を表示させたければ、\menu{print it} を使うことになります。実行されて計算され、最後に \ct{printString} を送ることで文字列として表示させるまでを行います。

\dothis{\ct{3+4} を選択肢、\menu{print it} を実行しましょう(\short{p})。}
今度は予想通りの結果です(\figref{printit})。
\index{キーボードショートカット!print it}

\begin{figure}[htb]
% \centerline {\includegraphics[width=0.4\textwidth]{PrintIt}}
\centerline {\includegraphics[width=0.8\textwidth]{PrintIt}}
\caption{``do it'' では無く ``Print it''。\figlabel{printit}}
\end{figure}

\needlines{3}
\begin{code}{@TEST}
3 + 4 --> 7
\end{code}
\noindent
これからは、\menu{print it} を使って結果の表示が行われたときには、\ct{-->} の表記をします。

\dothis{選択されている ``\ct{7}'' を削除し(delete キーを押します)、\ct{3+4} をもう一度選択してから \menu{inspect it} (\short{i}) を実行しましょう。}
\noindent
整数クラスの一つである \ct{SmallInteger: 7} (\figref{inspectit}) の \emphind{inspector}(インスペクタ)ウィンドウが表示されるはずです。
inspector はどんなオブジェクトでもあなたと対話することが出来る優れたツールです。
タイトルの意味は、\ct{7} は \clsind{SmallInteger} クラスのインスタンスであるという意味です。
左のパネルはオブジェクトのインスタンス変数を示し、その値が右のパネルに表示されます。下のパネルは、オブジェクトへメッセージを送るための式を記述することが出来ます。

\begin{figure}[htb]
\centerline {\includegraphics[width=\textwidth]{InspectIt}}
\caption{オブジェクトのインスペクト中(検査中の意味)。 \figlabel{inspectit}}
\end{figure}

\dothis{下のパネルに \ct{self squared} と入力し、\menu{print it} を実行した結果。}

\needlines{2}
\dothis{inspector を閉じ、\ct{Object} を workspace に入力して、\menu{explore it} (\short{I}, アルファベット大文字 i)を実行してみましょう。}
\index{キーボードショートカット!explore it}
\index{explorer}

今度は、\clsind{Object} のタイトルが付いた、\mbox{$\triangleright$ \ct{root: Object}} と表示されているウィンドウが開きます。
三角のアイコンを使って、中身を表示してください (\figref{exploreit})。

\begin{figure}[htb]
\centerline {\includegraphics[width=0.7\textwidth]{ExploreIt}}
\caption{\ct{Object} の explorer(調査)。\figlabel{exploreit}}
\end{figure}

explorer は inspector と似ていますが、ツリー構造により複雑なオブジェクトの中身を表示します。
今私たちは、\ct{Object} クラスの中身を explorer で見ていることになります。
このウィンドウでは全ての情報を参照し、操作することが可能です。

%=================================================================
\section{Class Browser}

class \ind{browser}(クラスブラウザ)\footnote{
紛らわしいことに、``system browser'' や ``code browser'' など呼ばれることがあります。\pharo は \ind{OmniBrowser} (``OB'' や ``Package browser'' と呼ばれてもいます)を実装しています。
文中ではシンプルに ``browser''(ブラウザ) を用いますが、曖昧さが問題になるケースでは、``class browser''(クラスブラウザ) と表記します。} はプログラミングにおいて、重要なツールです。
\pharo にはいくつかのブラウザが有りますが、``class browser'' は最も基本的なブラウザです。

\seeindex{class browser}{browser}

\dothis{\menu{World で \go Class browser を選択してください}。\footnote{もし、browser が \figref{classBrowser} に似ていないものであれば、デフォルトの browser を変更しましょう。\faqref{packagebrowser} を参照してください。}}

\begin{figure}[htb]
\ifluluelse
	{\centerline {\includegraphics[width=\textwidth]{ClassBrowser1}}}
	{\centerline {\includegraphics[width=0.7\textwidth]{ClassBrowser1}}}
\caption{browser で Object クラスの \ct{printString} メソッドを表示。
\figlabel{classBrowser}}
\end{figure}

\figref{classBrowser} を見てください。
タイトルに書かれているとおり、\clsind{Object} クラスをブラウズしています。
%\footnote{If the browser you have seems to differ from the one described in this book, you may be using an image with a different default browser. See \faqref{omnibrowser}.}

最初に browser を開くと、左のパネル以外は全て空欄です。左のパネルは、クラスを含んだ全ての \emph{packages}(パッケージ) をリスト表示しています。
\index{category}

\dothis{\scatind{Kernel} パッケージを選択しましょう。}
選択されたパッケージに含まれるクラスが二番目のパネルにリスト表示されます。

\dothis{\clsind{Object} クラスを選択しましょう。}
三番目のパネルは選択されたクラスの\emph{protocols}(プロトコル)が表示されます。
\ind{protocol} を選択していなければ、全てのメソッドが四番目のパネルに表示されます。

\dothis{\protind{printing} プロトコルを選択しましょう。}
printing のプロトコルに属するメソッドが四番目のパネルに表示されます。

\dothis{\mthind{Object}{printString} メソッドを選択しましょう。}
下のパネルに\ct{printString} メソッドのソースコードが表示されます。
このソースコードは、全てのクラスで共有される、オーバーライドされていないソースコードになります。

%=================================================================
\section{クラスの検索}

\pharo でクラスを見つけるには、いくつかの方法が有ります。一つ目は、browser のナビゲーションを使ってカテゴリからたどっていく方法です。
\index{browser}
\seeindex{browser!クラスを見つけるには}{クラス, 検索}
\index{クラス!finding}

二つ目として、クラスに \ct{browse} メッセージを送り、browser を開く方法があります。\clsind{Boolean} クラスをブラウズしたいと考えてみましょう。

\dothis{workspace に \ct{Boolean browse} と記述し、\menu{do it} してみましょう。}
Boolean クラスをポイントした状態の browser が開きましたか(\figref{browseBoolean})?
クラスをブラウズする場合は、クラス名を選択して、\ind{keyboard shortcut} \short{b} (browse) を使う方法もあります。
\index{キーボードショートカット!browse it}

\dothis{キーボードショートカットを使って \ct{Boolean} クラスをブラウズしてみましょう。}

\begin{figure}[hbt]
\centerline {\includegraphics[width=\textwidth]{Kernel-objects-boolean}}
\caption{Boolean クラスを表示している状態の browser。
\figlabel{browseBoolean}}
\end{figure}

\ct{Boolean} クラスが選択されていますが、メソッドを選択していないときに\emph{クラス定義}が表示されていることに注意してください(\figref{browseBoolean})。
このクラス定義は、親クラスにサブクラスを生成するメッセージそのものです。
\ct{Object} クラスに対して、インスタンス変数、クラス変数、``pool dictionaries''(プール辞書)の定義を空っぽとし、\scatind{Kernel-Objects} カテゴリで \ct{Boolean} クラスを定義しています。
% The lower pane shows the \emph{class comment} --- a piece of plain text describing the class.
\button{?} をクリックすると、\subind{class}{comment}(クラスコメント)を見ることが出来ます(\figref{classComment})。

\begin{figure}[hbt]
\centerline {\includegraphics[width=\textwidth]{classComment}}
\caption{\ct{Boolean} クラスのクラスコメント。
\figlabel{classComment}}
\end{figure}

クラスを見つける良い方法は、名前を使って探し出すことです。例えば、日付と時間に関するクラスを探していると仮定してみましょう。

\dothis{マウスをパッケージを表示しているペインに持って行き、\short{f} または \actclick{ing} で \menu{find class \ldots (f)} を選択し、表示されたダイアログに ``time'' と入力してみましょう。} 
\noindent
``time'' を含んだクラス名のリストが表示されます(see \figref{findit})。\ct{Time} を選択してみましょう。\ct{Time} クラスをポイントした状態で browser が表示されます。\short{b} を実行すると、その時にポイントしていたクラスを選択した状態で別の browser が開きます(訳注:\pharo のバージョンにより動作が微妙に異なりますが、感覚的に直ぐにわかります)。
\index{キーボードショートカット!find ...}
\index{キーボードショートカット!browse it}

\begin{figure}[hbt]
\centerline{
	\includegraphics[width=0.45\textwidth]{FindIt}
	\hspace{1cm}
	\includegraphics[width=0.45\textwidth]{TimeClasses}
}
\caption{名前からクラスを検索。
\figlabel{findit}}
\end{figure}

Note that if you type the complete (and correctly capitalized) name of a class in the find dialog, the browser will go directly to that class without showing you the list of options.

%=================================================================
\section{メソッドの検索}
\seclabel{quick:methodFinder}

メソッド名はクラス名よりも簡単に推測することができます。例えば、今現在の時間に関することを考えた場合、``now''(現在) という言葉が思い当たるでしょう。 \emphind{method finder} はそのようなケースで非常に重宝されるツールです。
\seeindex{browser!メソッドの検索}{メソッド, 検索}
\index{メソッド!検索}

\dothis{\menu{World \go Tools ... \go Method finder} を選択し、左上のペインに ``now'' を入力し、\textsc{return} キーを押してみましょう。}

method finder は、``now'' を名称に含むリストを表示します。リスト表示のペインで ``\ct{n}'' をキーボードから入力すると、インクリメンタルサーチのような動作を行います。ここで、``now'' を選択すると、右側のペインにはメソッドを定義しているクラス名のリストが表示されます(\figref{MethodFinder} )。クラス名を選択すると、そのクラスをポイントした状態でブラウザが立ち上がります。

\begin{figure}[hbt]
\centerline {\includegraphics[width=0.7\textwidth]{methodFinder-now}}
\caption{method finder でメソッド \ct{now} を定義しているクラスのリスト表示。
\figlabel{MethodFinder}}
\end{figure}

method finder はまだ他の使い方もあります。例えば、\ct{'eureka'} を \ct{'EUREKA'} とアルファベット大文字に変換するケースを考えてみましょう。

\dothis{\ct{'eureka' . 'EUREKA'} と method finder に入力し、\textsc{return} キーを押してください(\figref{methodFinder-example1})。}
\noindent
method finder はあなたの思った通りを動作をしましたか?\footnote{警告のメッセージダイアログが出てもびっくりしないでください。method finder は単にありそうな候補全てを表示するだけなので、推奨されないメソッドも含まれます。慌てず騒がず、\button{Proceed} をクリックしてください。}

右のペインでアスタリスクが付いているクラスは、実際にアルファベット大文字へ変換したときに使われたことを示します。このケースでは、\ct{String asUppercase} が実際に使われたということです。
アスタリスクが付いていないクラスは、単に同じ名前のメソッドが定義されているというだけでリストアプされています。\ct{'eureka'} は \clsind{Character} のオブジェクトではないので、\cmind{Character}{asUppercase} は使われなかったということです。

\begin{figure}[hbt]
\centerline {\includegraphics[width=\textwidth]{MethodFinder-example1}}
\caption{メソッド検索例。
\figlabel{methodFinder-example1}}
\end{figure}

他にも使い方は色々有ります。二つの整数の最大公約数を見つけるメソッドを見つけたいときは、\ct{25. 35. 5} と入力してみてください。下のペインではいくつかのサンプルを見ることが出来ます。

%=================================================================
\section{メソッドの新規定義}

\ind{Test Driven Development}\cite{Beck03a} (TDD) は革新的なコードの記述方式です。
TDD の考え方はテスト先導です。プログラムコードを書く前にテストコードを作成します。後は、そのテストコードを満足させるプログラムコードを書くという方式です。
\seeindex{Behavior Driven Development}{Test Driven Development}
% \orla{describe the technique where we write a test hat ... subsequently we write ...}

将来そのメソッドを修正するプログラマに、何を意図したメソッドであるか、そのメソッドの名前は何を付けるべきか注意する必要が有ります。例えば、次のような例を考えてみましょう。

\begin{quote}
文字列 ``Don't panic'' にメッセージ \ct{shout} を送った場合、``DON'T PANIC!'' という文字列を結果として得る。
\end{quote}

\noindent
テスト用のメソッドを作成してみます。
\index{testing}
\index{SUnit}

\needlines{3}
\begin{method}[testShout]{shout メソッドのテストメソッド}
testShout
	self assert: ('Don''t panic' shout = 'DON''T PANICBANG')
\end{method} % BANG is the escape for !

\pharo で新しくメソッドを作成するには、そのメソッドがどのクラスに属するかを決めなければいけません。今回のケースでは、今から作成しようとしている \ct{shout} メソッドが \clsind{String} クラスに属することになりますので、テストメソッドは \clsind{StringTest} に作成することにしましょう。

\begin{figure}[hbt]
\centerline {\includegraphics[width=\textwidth]{StringTest-newMethodTemplate}}
\caption{\ct{StringTest} に作成した新規メソッドの雛形。
\figlabel{newMethodTemplate}}
\end{figure}

\dothis{browser で \ct{StringTest} クラスを選択してください。適切なプロトコルとして、\menu{tests - converting} を選択しましょう(\figref{newMethodTemplate})}。
browser の下のペインで表示されているのはテンプレートコードです。全て消した後に、\mthref{testShout} を入力してください。
browser にプログラムコードを入力したときに、ペインの一部が赤くなっているはずです。これは、入力したプログラムコードがまだ保存されていないというマーカーです。
\actclick のメニューから \menu{accept (s)} を選択するか、\short{s} でメソッドのプログラムコードを保存してください。
\index{キーボードショートカット}
\index{キーボードショートカット!accept}
\seeindex{accept it}{キーボードショートカット, accept}

使用している image にはじめて書き込んだのであれば、あなたの名前を入力するように促されるはずです。たくさんの寄贈プログラムコードの一つとしてあなたの名前を入力しておきましょう。難しいことは考えずに、ファーストネーム、ラストネーム(間に空白或いはドットで繋ぎましょう)でいいでしょう。

%\begin{figure}[hbt]
%\centerline {\includegraphics[width=0.35\textwidth]{initials}}
%\caption{Entering your initials.
%\figlabel{initials}}
%\end{figure}

まだ \ct{shout} メソッドを作成していませんので、似たような名前がリストされた確認ダイアログが表示されます(\figref{name})。
% 訳注:P21-22
% 原文は(\figref{testShoutConfirm})を使っていますが、恐らく間違いだと思います。
% (\figref{testShoutConfirm}) → (\figref{name}) に変更してあります。

タイプミスなどの時にはこの機能は非常に助かりますが、今回は本当に作成したい状況ですので、\figref{testShoutConfirm} のように一番上の項目を選択します。

%\begin{figure}[htb]
%\begin{minipage}[b]{0.48\textwidth}
%\centerline {\includegraphics[width=0.9\textwidth]{name}}
%\caption{Entering your name.\figlabel{name}}
%\end{minipage}
%\hfill
%\begin{minipage}[b]{0.48\textwidth}
%\centerline {\includegraphics[width=\textwidth]{testShoutConfirm}}
%\caption{Accepting the \ct{StringTest} method \ct{testShout}.\figlabel{testShoutConfirm}}
%\end{minipage}
%\end{figure}

\begin{figure}[htb]
\centerline {\includegraphics[width=0.6\textwidth]{name}}
\caption{あなたの名前を入力しましょう。\figlabel{name}}
\end{figure}

\begin{figure}[htb]
\centerline {\includegraphics[width=\textwidth]{testShoutConfirm}}
\caption{\ct{StringTest} クラスの \ct{testShout} メソッド。\figlabel{testShoutConfirm}}
\end{figure}


%\begin{figure}[hbt]
%\ifluluelse
%	{\centerline {\includegraphics[width=\textwidth]{testShoutConfirm}}}
%	{\centerline {\includegraphics[width=0.7\textwidth]{testShoutConfirm}}}
%\caption{Accepting the \ct{StringTest} method \ct{testShout}.
%\figlabel{testShoutConfirm}}
%\end{figure}

\dothis{\menu{World} から \ind{SUnit} \emphind{TestRunner} を選択し、新しいテストを実行しましょう。}
一番左のペインはカテゴリ、その右のペインはカテゴリに含まれるテスト用のクラスです。
\scat{CollectionsTests-Text}を選択してください。表示された \ct{StringTest} を選択し、\menu{Run Selected} ボタンで実行しましょう。

\begin{figure}[hbt]
\centerline {\includegraphics[width=\textwidth]{testRunnerOnStringTest}}
\caption{StringTest クラスでのテスト実行。
\figlabel{testRunnerTestShout}}
\end{figure}

テストを実行した結果が、\figref{testRunnerTestShout} のように表示されます。エラーがある箇所は右下のペインに表示されます(\ct{StringTest>>>#testShout})。
\ct{StringTest>>>#testShout} をクリックすると、再度そのテストだけが実行され、``\ct{MessageNotUnderstood: ByteString>>>shout}'' が表示されます。
\seeindex{\ct{>>}}{振る舞い, \ct{>>}}
\cmindex{振る舞い}{>>}
このウィンドウは、\st のデバッガ(debugger) です(\figref{predebugger})。
% \ab{Well, it's actually the \emph{pre-}debugger.  Does this matter?}\damien{I don't think it's important at this point.}
\ind{debugger} については、\charef{env} で解説します。

\begin{figure}[hbt]
\centerline {\includegraphics[width=\textwidth]{Predebugger}}
\caption{debugger.}
\figlabel{predebugger}
\end{figure}

このエラーは、意図したとおりの動作になります。まだ \ct{shout} を実装していませんので、当然の事になります。しかしながら、当然のエラーが発生したということは、今までの実践内容が全て正しかったという裏返しの確認でもあります。
取り敢えず、\button{Abandon} ボタンを押して、ウィンドウを閉じましょう。
因みに、\button{Create} はデバッガの中で新しくメソッドを作成することを促されますし、\button{Proceed} ボタンはそのまま続行することになります。

では、テストが成功するためのメソッドを作成しましょう!

\dothis{browser で \clsind{String} クラスの \menu{converting} プロトコルを選択し、メソッドとして \mthref{shout} を定義し、保存しましょう(Note: \mbox{\ct{^}} は、\caret を入力します)。 }
\begin{method}[shout]{shout メソッド}
shout
	^ self asUppercase, 'BANG'
\end{method}

カンマは連結演算子です。文字列がアルファベット大文字に変換された後、最後に感嘆符を付けます。
$\uparrow$ は \pharo では返される答えになります。$\uparrow$ 以下の文字列(感嘆符も付きます)が返される答えです。
\seeindex{comma}{Collection, カンマ演算子}
\index{Collection!カンマ演算子}

出来ましたか?では、もう一度テストを実行しましょう。

\dothis{\menu{Run Selected} を再実行します。今度は緑色でエラー無し状態のウィンドウを見るはずです。}
どうでしたか?うまくいきました?
% 訳注:
% \footnotemarkが定義されている(文章中では10)が、対応する文章がない。そのため、カット。

%\footnotetext{Actually, you might not get a green bar since some images contains tests for bugs that still need to be fixed.
%Don't worry about this.
%\pharo is constantly evolving.
%}

\begin{figure}[hbt]
\ifluluelse
	{\centerline{\includegraphics[width=\textwidth]{String-Shout}}}
	{\centerline{\includegraphics[width=0.7\textwidth]{String-Shout}}}
\caption{\ct{String} クラスに定義された \ct{shout} メソッド。
\figlabel{String-shout}}
\end{figure}

%=================================================================
\section{本章のまとめ}
本章では、\pharo の構文をすこしばかりと、様々なツールの簡単な使い方を学びました。

\begin{itemize}
  \item \pharo は、\emph{virtual machine}、\emph{sources} ファイル、\emph{image} と \emph{changes} ファイルで構成されており、\emph{image} と \emph{changes} ファイルはスナップショットとして変化していきます。
  \item \pharo は、あなたが前回終了したときと全く同じ状態で再起動します。
  \item \pharo は、3ボタンマウスを使うことで最高の使い勝手を提供します。勿論、マウスがなくても、キーボードの修飾キーを使うことで、\click 、 \actclick と \metaclick が使えます。
  \item \pharo の背景部分をクリックすることで、\emph{World menu} が表示できます。
  \item \emph{workspace} では、プログラムコードを試すことも出来ますし、単なるメモ帳としても使えます。
  \item プログラムコードを評価するのに、\menu{do it} (\short{d})、 \menu{print it} (\short{p})、 \menu{inspect it} (\short{i})、 \menu{explore it} (\short{I}) と \menu{browse it} (\short{b}) が使えます。
%  \item \sqmap is a tool for loading useful packages from the Internet.
  \item \emph{browser} は \pharo のプログラムコードを見たり開発したりするための重要なツールです。
  \item \emph{test runner} はユニットテストのための環境を提供します。
\end{itemize}

%=================================================================
\ifx\wholebook\relax\else 
   \bibliographystyle{jurabib}
   \nobibliography{scg}
   \end{document}
\fi
%=================================================================

%%% Local Variables:
%%% coding: utf-8
%%% mode: latex
%%% TeX-master: t
%%% TeX-PDF-mode: t
%%% ispell-local-dictionary: "english"
%%% End:
