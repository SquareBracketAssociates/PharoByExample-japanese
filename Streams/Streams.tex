% $Author$
% $Date$
% $Revision$

% HISTORY:
% 2007-03-30 - Cassou splits of Streams from Collection chapter
% 2007-07-05 - Cassou partial draft complete?
% 2007-08-02 - Stef pass
% 2007-08-16 - Cassou continues
% 2007-08-21 - Oscar edit
% 2007-08-21 - Cassou review
% 2009-07-07 - Oscar migrate to Pharo; fixed broken tests

%=================================================================
\ifx\wholebook\relax\else
% --------------------------------------------
% Lulu:
	\documentclass[a4paper,10pt,twoside]{book}
	\usepackage[
		papersize={6.13in,9.21in},
		hmargin={.75in,.75in},
		vmargin={.75in,1in},
		ignoreheadfoot
	]{geometry}
	\input{../common.tex}
	\pagestyle{headings}
	\setboolean{lulu}{true}
% --------------------------------------------
% A4:
%	\documentclass[a4paper,11pt,twoside]{book}
%	\input{../common.tex}
%	\usepackage{a4wide}
% --------------------------------------------
    \graphicspath{{figures/} {../figures/}}
	\begin{document}
	\renewcommand{\nnbb}[2]{} % Disable editorial comments
	\sloppy
\fi
%=================================================================
%\chapter{Streams}\chalabel{streams}
\chapter{ストリーム}\chalabel{ストリーム}

%\ew{Streams are presented as a way to navigate collection. From my point of view, stream are important not to navigate collection, but to produce/consume data:
%(a)	memory constraint. Data can not hold into memory and must be processed in a stream fashion, e.g: encryption
%(b)	blocking IO. A stream is a nice abstraction to deal with, and the stream manages internally data availability, buffering, etc. to simplify the consumption/production of data
%Only few streams have random access capability.}

\clsindexmain{ストリーム}

ストリームはコレクション,ファイル,ネットワークストリームのような
連続する要素に対する反復処理に用いられます。
ストリームは読み取り可能か,書き込み可能か,もしくはその両方です。
読み取りまたは書き込みは常にストリームの現在位置に対して相対的です。
ストリームはコレクションに簡単に変換可能ですし,その逆も同様です。

%\lr{"Streams can easily be converted into collections." I wouldn't say it like this, because it is not true for all streams (infinite streams). According to Kent Beck we should only talk about conversion when the same protocol is supported. Collections and Streams do not support the same protocol. (p. 249)}


%=============================================================
\section{二続きの要素}

ストリーミングを理解する良いたとえは以下のようなものです。
ストリームは二続きの要素として表されます。
過去の要素と未来の要素です。
ストリームはこの二つの連続の間に位置します。
Smalltalkのストリームに対する全ての操作はこれによるので,このモデルを理解しておくのは大切です。
このような理由で,ほとんどの\clsind{Stream}クラスは\clsind{PositionableStream}のサブクラスになっています。
図\figref{_abcde}は5個の文字を含むストリームを表しています。このストリームはその初期位置にあります。\ie 過去には一つの要素もありません。\mthind{PositionableStream}{reset}メッセージによりこの位置に戻すことができます。



\begin{figure}[ht]
\centerline{\includegraphics[scale=0.5]{_abcdeStef}}
\caption{最初のストリームの位置.}
\figlabel{_abcde}
\vspace{.2in}
\end{figure}

要素の読み取りは概念的に,未来の要素列の先頭を取り除き,それを過去の要素列の最後に付け加えることを意味します。
\ct{next}メッセージを使って一つの要素を読むと,ストリームの状態は図\figref{a_bcde}に示されているようになります。

\begin{figure}[ht]
\centerline{\includegraphics[scale=0.5]{a_bcdeStef}}
\caption{\ct{next}メソッドを実行したあとの同じストリーム: 文字 \ct{a} は 「過去」 で \ct{b}, \ct{c}, \ct{d} \ct{e} は 「未来」です.}
\figlabel{a_bcde}
\vspace{.2in}
\end{figure}

要素の書き込みは未来の先頭要素を新しいものに置き換えて過去に移動させることを意味します。図\figref{ax_cde}は\mthind{Stream}{nextPut:} \ct{anElement}メッセージを使って同じ要素を\ct{x}で書き換えた状態を示しています。


\begin{figure}[ht]
\centerline{\includegraphics[scale=0.5]{ax_cdeStef}}
\caption{\ct{x}を書き込んだ同じストリーム.}
\figlabel{ax_cde}
\vspace{.2in}
\end{figure}

%=============================================================
\section{ストリーム対コレクション}

コレクションのプロトコルは要素の格納,除去と列挙の手段を提供しますが,それらの合成は許されていません。
例を挙げると,\clsind{OrderedCollection}の要素を\ct{do:}メソッドによって処理する場合,
\ct{do:}のブロック内で要素の追加と削除はできません。
あるいは,コレクションのプロトコルには二つのコレクションを同時に反復処理し,片方は前進させて
もう片方をそうしないように選ぶような方法を提供していません。
このような処理にはトラバースインデックスまたはコレクション自身の外部に保存された
位置のリファレンスが必要になります。
これこそ\clsind{ReadStream},  \clsind{WriteStream} そして \clsind{ReadWriteStream}の役割です。



これら三つのクラスはコレクションではなく, \emph{ストリームに}定義されています。
例を挙げると,次のコードはインターバルのストリームを作り,それから二つの要素を読み取ります。


\needlines{5}
\begin{code}{@TEST |r|}
r := ReadStream on: (1 to: 1000).
r next.   --> 1
r next.   --> 2
r atEnd.--> false
\end{code}

\ct{WriteStream}はコレクションにデータを書き込むことができます。


\begin{code}{@TEST |w|}
w := WriteStream on: (String new: 5).
w nextPut: $a.
w nextPut: $b.
w contents. -->  'ab'
\end{code}

読み取りと書き込みの両方のプロトコルを提供する\ct{ReadWriteStream}を作ることもできます。

\ct{WriteStream}と\ct{ReadWriteStream}の主な問題は,\pharoでは配列と文字列しか
サポートしていないことです。
これは,Nileと名付けられた新しいライブラリの開発で変わりつつありますが,
今のところそれ以外のコレクションに対する操作はエラーになります。

\needlines{3}
\begin{code}{}
w := WriteStream on: (OrderedCollection new: 20).
w nextPut: 12. -->  エラー
\end{code}

ストリームはコレクションに対してのみ有効ではなく,ファイルやソケットにも使えます。
次の例は\ct{test.txt}というファイルを作成し,改行で分離された2つの文字列を書き込み,ファイルを閉じます。

\needlines{3}
\begin{code}{}
StandardFileStream
  fileNamed: 'test.txt'
  do: [:str | str
                nextPutAll: '123';
                cr;
                nextPutAll: 'abcd'].
\end{code}
\cmindex{FileStream class}{fileNamed:do:}

次の章ではプロトコルについてさらに解説します。

%=============================================================
\section{ストリームによるコレクションの操作}

ストリームはコレクションの要素を扱うのにとても便利で,コレクションの要素の読み取りと書き込みに使われます。
これからコレクションに対するストリームの特徴を探っていきます。


%-----------------------------------------------------------------
\subsection{コレクションを読み取る}

このセクションではコレクションの読み取りの特徴について見ていきます。
コレクションの読み取りにストリームを使うことは,本質的にはコレクションへのポインタを提供することになります。
ポインタは読み取りによって前進し,また希望する場所へ自由に移動させることができます。
コレクションの要素を読み込むのには\clsindmain{ReadStream}クラスを使います。

コレクションから一つ以上の要素を読み取るには\mthind{ReadStream}{next}と\mthind{ReadStream}{next:}メソッドを使います。

\begin{code}{@TEST |stream|}
stream := ReadStream on: #(1 (a b c) false).
stream next. -->   1
stream next. -->   #(#a #b #c)
stream next. -->   false
\end{code}
\cmindex{PositionableStream class}{on:}

\begin{code}{@TEST |stream|}
stream := ReadStream on: 'abcdef'.
stream next: 0. -->   ''
stream next: 1. -->   'a'
stream next: 3. -->   'bcd'
stream next: 2. -->   'ef'
\end{code}

\mthind{PositionableStream}{peek}メッセージはストリームの次の要素が
何であるかを,ポインタを進ませないで知りたいときに使われます。

\begin{code}{@TEST |stream negative number|}
stream := ReadStream on: '-143'.
negative := (stream peek = $-).    "読み取ることなく最初の要素を調べる"
negative. --> true
negative ifTrue: [stream next].       "マイナスの文字を無視する"
number := stream upToEnd.
number. --> '143'
\end{code}
\noindent

このコードは,ストリーム中の数の符合をブール変数\ct{negative}に設定し,絶対値を\ct{number}に設定します。
\mthind{ReadStream}{upToEnd}メソッドは現在の位置から最後までのストリームの全てを返し,
ストリームのポインタを最後に設定します。このコードは,引数と同じ時は先に進み,
異なるときは進まない\mthind{PositionableStream}{peekFor:}メソッドを使うことで単純化できます。


\begin{code}{@TEST |stream negative number|}
stream := '-143' readStream.
(stream peekFor: $-) --> true
stream upToEnd         --> '143'
\end{code}
\noindent

\ct{peekFor:}メソッドはまた,引数と要素が等しいかを示す真偽値を返します。

上のコードの例からストリームを作る新しい方法に気付いたと思います。
一連のコレクションに対して単に\mthind{SequenceableCollection}{readStream}メッセージを送ることで,
そのコレクションからストリームを読み取ることができます。

\paragraph{位置決め} 
ストリームのポインタを位置決めするメソッドがあります。
もしインデックスがあるなら\mthind{PositionableStream}{position:}メソッドで直接そこへ移動できます。
\mthind{PositionableStream}{position}メソッドで現在位置を求めることができます。
ストリームは要素と要素の間を示しているのであって,要素そのものを指しているわけではないことを覚えておいて下さい。
インデックスはストリームの先頭を0として対応します。


図\figref{ab_cde}に描かれたストリームの状態を次のコードで得ることができます。


\begin{code}{@TEST |stream|}
stream := 'abcde' readStream.
stream position: 2.
stream peek --> $c
\end{code}

\begin{figure}[h!t]
\centerline{\includegraphics[scale=0.5]{ab_cdeStef}}
\caption{2番目の位置にあるストリーム}
\figlabel{ab_cde}
\vspace{.2in}
\end{figure}


ストリームの先頭か最後に移動する場合はそれぞれ\mthind{PositionableStream}{reset}メソッド,
\mthind{PositionableStream}{setToEnd}メソッドを使います。
\mthind{PositionableStream}{skip:}と\mthind{PositionableStream}{skipTo:}メソッドは現在位置を基準として前に進む場合に使います。
\ct{skip:}メソッドは引数に数を取ることができ,その数の要素を飛ばします。
一方,\ct{skipTo:}メソッドは与えられた引数と同じ要素が見つかるまでストリームの要素を飛ばしていきます。
そのときの位置は,見つかった要素の後ろになることに注意して下さい。

\begin{code}{@TEST |stream|}
stream := 'abcdef' readStream.
stream next.        --> $a    "ストリームはaの真後ろに位置している"
stream skip: 3.                           "dの後ろに位置している"
stream position.  -->   4
stream skip: -2.                          "bの後ろに位置している"
stream position.  --> 2
stream reset.
stream position.  --> 0
stream skipTo: $e.                      "ストリームはeの後ろに移った"
stream next.        --> $f
stream contents. --> 'abcdef'
\end{code}

ご覧のように\ct{e}の文字は飛ばされています。

\mthind{PositionableStream}{contents}メソッドは常にストリーム全体のコピーを返します。

\paragraph{テスト}現在のストリームの状態をテストするためのメソッドがあります。
\mthind{PositionableStream}{atEnd}メソッドはこれ以上読める要素がない場合のみtrueを返します。
\mthind{PositionableStream}{isEmpty}メソッドはコレクションに全く要素がない場合のみtrueを返します。

二つのソートされたコレクションを引数としてこれらのコレクションを別のソートされたコレクションと結合するアルゴリズムを
\ct{atEnd}メソッドを使って実装する例を示します。

\needlines{4}
\begin{code}{@TEST |stream1 stream2 result|}
stream1 := #(1 4 9 11 12 13) readStream.
stream2 := #(1 2 3 4 5 10 13 14 15) readStream.

"変数resultにはソートされたコレクションが入る."
result := OrderedCollection new.
[stream1 atEnd not & stream2 atEnd not]
  whileTrue: [stream1 peek < stream2 peek
    "両方のストリームのもっとも小さな要素が取り除かれてresultに加えられる."
    ifTrue: [result add: stream1 next]
    ifFalse: [result add: stream2 next]].

%"One of the two streams might not be at its end. Copy whatever remains."
"二つのストリームのどちらかは終端にきていないだろう.残っている全てをコピーする."
result
  addAll: stream1 upToEnd;
  addAll: stream2 upToEnd.

result. -->   an OrderedCollection(1 1 2 3 4 4 5 9 10 11 12 13 13 14 15)
\end{code}

%-----------------------------------------------------------------
\subsection{コレクションへの書き込み}

\ct{ReadStream}を使ってコレクションの要素をどのように読み取るかについては既にみてきました。
次に,\clsindmain{WriteStream}{}を使ってコレクションを作成する方法について学びましょう。


コレクションの様々な場所にたくさんのデータを加えるとき,\ct{WriteStream}は便利です。
次の例のように,静的部分と動的部分からなる文字列を生成するときにもしばしば用いられます。


\begin{code}{NB: can't be tested due to the changing number of classes}
stream := String new writeStream.
stream
  nextPutAll: 'This Smalltalk image contains: ';
  print: Smalltalk allClasses size;
  nextPutAll: ' classes.';
  cr;
  nextPutAll: 'This is really a lot.'.

stream contents. --> 'This Smalltalk image contains: 2322 classes.
This is really a lot.'
\end{code}

この方法は,例えば\ct{printOn:}メソッドの異なる実装にも使えます。
ストリームの内容だけにしか興味がない場合,これはもっと単純で効果的なストリームの作成法です。

\begin{code}{@TEST |string|}
string := String streamContents:
		[:stream |
			stream
                 print: #(1 2 3);
                 space;
                 nextPutAll: 'size';
                 space;
                 nextPut: $=;
                 space;
                 print: 3.	].
string. -->   '#(1 2 3) size = 3'
\end{code}

\mthind{SequenceableCollection class}{streamContents:} \seclabel{streamContents}メソッドは
コレクションと,そのコレクションを流すためのストリームを作ります。
それから引数として与えられたブロックを実行します。
ブロックが終了すると\ct{streamContents:}メソッドはコレクションの中身を返します。


%The following \ct{WriteStream} methods are especially useful in this context:
次のような状況に合わせた\ct{WriteStream}メソッドがあります。

\begin{description}
\item[\lct{nextPut:}] 引数をストリームに加えるとき。
\item[\lct{nextPutAll:}] 引数として与えたコレクションの要素をそれぞれストリームに追加するとき。
\item[\lct{print:}] ストリームにテキストによる表示手段を追加する。
	\cmindex{Stream}{print:}
\end{description}

\mthind{WriteStream}{space},\mthind{WriteStream}{tab},\mthind{WriteStream}{cr} 
などの異なる種類の文字を表示するのに便利なメソッドもあります。
\mthind{WriteStream}{ensureASpace}は,ストリームの最後の文字がスペースでない場合には追加し,
最後がスペースをあることを保証します。

\paragraph{連結について}

\ct{WriteStream}の\mthind{WriteStream}{nextPut:}メソッド,\mthind{WriteStream}{nextPutAll:}メソッドは多くの場合,
文字を連結する一番良い方法になります。
カンマ演算子(\ct{,})による結合はあまり効率的ではありません。

\index{Collection!comma operator}

\begin{code}{}
[| temp |
  temp := String new.
  (1 to: 100000)
    do: [:i | temp := temp, i asString, ' ']] timeToRun --> 115176 "(milliseconds)"

[| temp |
  temp := WriteStream on: String new.
  (1 to: 100000)
    do: [:i | temp nextPutAll: i asString; space].
  temp contents] timeToRun --> 1262 "(milliseconds)"
\end{code}

カンマ演算子は,レシーバと引数の文字列を新たに作成するのでどちらも
コピーしなければなりません。これがストリームの方がカンマ演算子より効果的であることの理由です。
もし,同じレシーバに対してカンマ演算子で繰り返し連結をしてみると,
指数関数的にコピーしなければならない文字数が増えてくるために,
どんどん時間がかかるようになってきます。
またこの方法はメモリ上に回収しなければならないゴミをたくさん残します。
カンマ演算子の代わりにストリームを使うのはよく知られている最適化方法です。
実際は\mthind{SequenceableCollection class}{streamContents:}メソッド(\pageref{sec:streamContents}参照)
がこれを手伝ってくれます。


\begin{code}{}
String streamContents: [ :tempStream |
  (1 to: 100000)
       do: [:i | tempStream nextPutAll: i asString; space]] 
\end{code}

%-----------------------------------------------------------------
\subsection{読み取りと書き込みを同時に行う}

ストリームを使うことでコレクションに対して同時に読み取りと書き込みを行うことができます。ウェブブラウザの
前進・後退ボタンの動作を管理する\ct{History}クラスを作りたいという状況を考えてみましょう。
履歴は図\ref{fig:emptyStream}から\ref{fig:page4Stef}のような動作をします。


\begin{figure}[!ht]
\centerline{\includegraphics[scale=0.5]{emptyStef}}
\caption{履歴は空です。ブラウザには何も表示されていません。}
\figlabel{emptyStream}
\vspace{.2in}
\end{figure}

\begin{figure}[!ht]
\centerline{\includegraphics[scale=0.5]{page1Stef}}
\caption{ユーザはページ1を開きます。}
\figlabel{page1}
\vspace{.2in}
\end{figure}

\begin{figure}[!ht]
\centerline{\includegraphics[scale=0.5]{page2Stef}}
%\caption{The user clicks on a link to page 2.}
\caption{ユーザはページ2へのリンクをクリックします。}
\figlabel{page2}
\vspace{.2in}
\end{figure}

\begin{figure}[!ht]
\centerline{\includegraphics[scale=0.5]{page3Stef}}
\caption{ユーザはページ3へのリンクをクリックします。}
\figlabel{page3}
\vspace{.2in}
\end{figure}

\begin{figure}[!ht]
\centerline{\includegraphics[scale=0.5]{page2_Stef}}
\caption{ユーザは戻るボタンをクリックします。今,再びページ2を見ています。}
\figlabel{page2_}
\vspace{.2in}
\end{figure}

\begin{figure}[!ht]
\centerline{\includegraphics[scale=0.5]{page1_Stef}}
\caption{ユーザは更に戻るボタンをクリックします。ページ1が表示されています。}
\figlabel{page1_}
\vspace{.2in}
\end{figure}

\begin{figure}[!ht]
\centerline{\includegraphics[scale=0.5]{page4Stef}}
\caption{1ページからユーザは4ページへのリンクをクリックしました。ページ2,ページ3の履歴はなくなります。}
\figlabel{page4}
\vspace{.2in}
\end{figure}

この振る舞いは\clsind{ReadWriteStream}を使うことで実装できます。

\needlines{6}
\begin{code}{}
Object subclass: #History
  instanceVariableNames: 'stream'
  classVariableNames: ''
  poolDictionaries: ''
  category: 'PBE-Streams'

History>>initialize
    super initialize.
    stream := ReadWriteStream on: Array new.
\end{code}

ストリームを持つクラスを定義しました。このストリームは\ct{initialize}メソッドで作られます。
ここまで特にも難しいことはありません。

次に,前進と後進のメソッドが必要になります。

\begin{code}{}
History>>goBackward
  self canGoBackward ifFalse: [self error: 'Already on the first element'].
  stream skip: -2.
  ^ stream next.

History>>goForward
  self canGoForward ifFalse: [self error: 'Already on the last element'].
  ^ stream next
\end{code}

ここまでのコードはとても単純明快です。
次に,ユーザがリンクをクリックしたときに動作する\ct{goTo:}メソッドに取りかかりましょう。
たとえば次のようなコードが思いつくかも知れません。

\begin{code}{}
History>>goTo: aPage
    stream nextPut: aPage.
\end{code}

しかし,これでは不十分です。ユーザがリンクをクリックしたとき,履歴上では進めるページがないはずだからです。
つまり,前進ボタンは無効になってなければいけません。
これを実現するもっともシンプルな方法は,履歴の指している要素の後ろに\ct{nil}を書き込むことです。


\begin{code}{}
History>>goTo: anObject
  stream nextPut: anObject.
  stream nextPut: nil.
  stream back.
\end{code}

これで,あとは\ct{canGoBackward}メソッドと\ct{canGoForward}メソッドを実装するだけです。

ストリームは常に二つの要素の間に位置しています。
戻るには今の位置から2ページ分(1ページ目が現在のページ,2ページ目が戻ったときに表示したいページ)
戻る必要があります。

\begin{code}{}
History>>canGoBackward
  ^ stream position > 1

History>>canGoForward
  ^ stream atEnd not and: [stream peek notNil]
\end{code}

ストリームの中身を確認できるメソッドを追加します。

\begin{code}{}
History>>contents
  ^ stream contents
\end{code}

これで履歴が狙い通りに動きます。

\begin{code}{}
History new
	goTo: #page1;
	goTo: #page2;
	goTo: #page3;
	goBackward;
	goBackward;
	goTo: #page4;
	contents --> #(#page1 #page4 nil nil)
\end{code}

%=============================================================
\section{ファイルアクセスにストリームを使う}

コレクションの要素に対してどのようにストリームを使うかをこれまで見てきました。
ハードディスクのファイルに対しても同じようにストリームを使うことができます。
一度作成すれば,ストリームはファイルに対してのストリームはコレクションに対してのストリームとほとんど変わりません。
同じプロトコルで読み取り,書き込み,位置の移動ができます。
大きな違いはストリームの作成にあります。
これからいくつか異なるファイルストリームの作成方法を見ていきましょう。


%-----------------------------------------------------------------
\subsection{ファイルストリームを作成する}

\seclabel{creat-file-stre}

ファイルストリームを作るには\clsindmain{FileStream}クラスが提供する
次のインスタンス作成メソッドのどれかを使います。

\begin{description}

\item[\lct{fileNamed:}] 与えられた名前のファイルを読み取り,書き込みのために開きます。
  もしファイルが既に存在していたら,その内容は変更したり置き換えられたりしますが,ファイルを閉じるときに切り捨てられません。
  もし,ファイル名にディレクトリ部分が明示されていない場合,デフォルトディレクトリにファイルは作成されます。
  \cmindex{FileStream class}{fileNamed:}
  
\item[\lct{newFileNamed:}] 与えられた名前の新しいファイルを作り,そのファイルに書き込むためのストリームを返します。
  もしファイルが既に存在していた場合はユーザに対してどうするか尋ねます。
  \cmindex{FileStream class}{newFileNamed:}
  
\item[\lct{forceNewFileNamed:}] 与えられた名前のファイルを作成し,そのファイルに書き込むためのストリームを返します。
  もしファイルが既に存在していた場合,新しいファイルを作成するかの確認をせずに古いファイルを消去します。
  \cmindex{FileStream class}{forceNewFileNamed:}

\item[\lct{oldFileNamed:}] 読み込みと書き込みのために与えられた名前の既にあるファイルを開きます。
  もしファイルが既に存在していたら,その内容は変更したり置き換えられたりしますが,ファイルを閉じるときに切り捨てられません。
  もし,ファイル名にディレクトリ部分が明示されていない場合,デフォルトディレクトリにファイルは作成されます。
  \cmindex{FileStream class}{oldFileNamed:}

\item[\lct{readOnlyFileNamed:}] ファイルを読み込みのために開きます。
  \cmindex{FileStream class}{readOnlyFileNamed:}

\end{description}

ファイルをストリームで開くたびに,必ず閉じなければならないことを覚えておいて下さい。
これには\mthind{FileStream}{close}メソッドを使います。


\begin{code}{@TEST |stream|}
stream := FileStream forceNewFileNamed: 'test.txt'.
stream
    nextPutAll: 'This text is written in a file named ';
    print: stream localName.
stream close.

stream := FileStream readOnlyFileNamed: 'test.txt'.
stream contents. --> 'This text is written in a file named ''test.txt'''
stream close.
\end{code}


\mthind{FileStream}{localName}メソッドはファイルの一番最後の内容を返します。
\mthind{StandardFileStream}{fullName}メソッドでファイルのフルパスを得ることができます。


%You will soon notice that manually closing the file stream is
%painful and error-prone. That's why \ct{FileStream} offers a message called \mthind{FileStream class}{forceNewFileNamed:do:} to
%automatically close a new stream after evaluating a block that
%sets its contents.
そのうち,ファイルストリームを手動で閉じることは面倒な上にエラーの元になることに気付くでしょう。
内容についてのブロックを評価したあと自動的にストリームを
閉じる\mthind{FileStream class}{forceNewFileNamed:do:}:メソッドを\ct{FileStream}クラスが提供しているのはそのためです。


\begin{code}{@TEST |string|}
FileStream
    forceNewFileNamed: 'test.txt'
    do: [:stream |
        stream
            nextPutAll: 'This text is written in a file named ';
            print: stream localName].
string := FileStream
            readOnlyFileNamed: 'test.txt'
            do: [:stream | stream contents].
string --> 'This text is written in a file named ''test.txt'''
\end{code}

ブロックを引数にとるストリーム作成のメソッドは,まずファイルに対してのストリームを作成し,
次に引数のブロックを実行し,最後にストリームを閉じます。
これらのメソッドはブロックの返すもの,つまり最後のブロック表現の値を返します。
前の例でこれは,ファイルの中身を得てそれを変数\ct{string}に入れるところで使われています。

%-----------------------------------------------------------------
\subsection{バイナリストリーム}

\seclabel{binary-streams}

デフォルトでストリームは文字の読み取りと書き込み可能なテキストベースで作成されます。
もしバイナリでストリームを作りたい場合,\mthind{FileStream}{binary}メッセージを送る必要があります。


%When your stream is in binary mode, you can only write numbers from 0
%to 255 (1 Byte). If you want to use \ct{nextPutAll:} to write more
%than one number at a time, you have to pass a \clsind{ByteArray} as
%argument.
もし,ストリームがバイナリモードの場合,0から255(1バイト)の数を書き込むことしかできません。
1つ以上の文字を同時に書き込むのに使いたい場合\ct{nextPutAll:}メソッドを使い,\clsind{ByteArray}を引数として渡す必要があります。

\begin{code}{@TEST}
FileStream
  forceNewFileNamed: 'test.bin'
  do: [:stream |
          stream
            binary;
            nextPutAll: #(145 250 139 98) asByteArray].

FileStream
  readOnlyFileNamed: 'test.bin'
  do: [:stream |
          stream binary.
          stream size.         --> 4
          stream next.         --> 145
          stream upToEnd. --> #[250 139 98]
      ].
\end{code}

次の例は「test.pgm」(portable graymap file format)を作成する例です。
作成したファイルはお気に入りのドロープログラムで開くことができます。


% The following does not assert anything, but @TEST is used to ensure
% that no error is thrown.
\begin{code}{@TEST}
FileStream
  forceNewFileNamed: 'test.pgm' 
  do: [:stream |
	stream
		nextPutAll: 'P5'; cr;
		nextPutAll: '4 4'; cr;
		nextPutAll: '255'; cr;
		binary;
		nextPutAll: #(255 0 255 0) asByteArray;
		nextPutAll: #(0 255 0 255) asByteArray;
		nextPutAll: #(255 0 255 0) asByteArray;
		nextPutAll: #(0 255 0 255) asByteArray
	]
\end{code}

これは図\figref{checkerboard4x4}のような4x4のチェッカーボード(市松模様)を作ります。

\begin{figure}[!ht]
\centerline{\includegraphics[width=0.25\textwidth]{checkerboard4x4}}
\caption{バイナリストリームで描くことができる4x4のチェッカーボード(市松模様)}
\figlabel{checkerboard4x4}
\vspace{.2in}
\end{figure}

%=============================================================
\section{章のまとめ}

ストリームは一連の要素を連続的に読み取る,あるいは書き込むための,
コレクションよりも良い方法を提供します。
また,ストリームとコレクションを相互に変換する簡単な方法があります。
\begin{itemize}
  \item ストリームは読み取り可能,または書き込み可能,あるいは読み取り・書き込みが同時に可能です。
  \item コレクションをストリームに変換する場合,コレクション「で」ストリームを定義します。\eg \ct{ReadStream on: (1 to: 1000)}または,コレクションに対して\ct{readStream}メッセージを送る,など。
  \item ストリームをコレクションに変換する場合,\ct{contents}メッセージを送ります。
  \item 大きなコレクションを連結する場合はカンマ演算子を使うよりも,ストリームを作成してから\ct{nextPutAll:}メソッドでコレクションをストリームに結合し,\ct{contents}メソッドを送って結果を取り出す方法が有効です。
  \item ファイルストリームは初期設定では文字型です。\ct{binary}メッセージを送ることで明示的にバイナリにすることができます。
\end{itemize}

%=================================================================
\ifx\wholebook\relax\else\end{document}\fi
%=================================================================

%-----------------------------------------------------------------

%%% Local Variables: 
%%% coding: utf-8
%%% mode: latex
%%% TeX-master: t
%%% TeX-PDF-mode: t
%%% ispell-local-dictionary: "english"
%%% End:
