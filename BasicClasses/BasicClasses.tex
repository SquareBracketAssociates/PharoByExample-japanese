% $Author$
% $Date$
% $Revision$

% HISTORY:
% 2006-10-31 - Oscar started
% 2007-08-19 - Stef revised
% 2007-11-09 - Andrew corrections
% 2008-03-28 - Cassou corrections
% 2009-07-07 - Oscar fixed broken tests

%=================================================================
\ifx\wholebook\relax\else
% --------------------------------------------
% Lulu:
  \documentclass[a4paper,10pt,twoside]{book}
  \usepackage[
    papersize={6.13in,9.21in},
    hmargin={.75in,.75in},
    vmargin={.75in,1in},
    ignoreheadfoot
  ]{geometry}
  \input{../common.tex}
  \pagestyle{headings}
  \setboolean{lulu}{true}
% --------------------------------------------
% A4:
% \documentclass[a4paper,11pt,twoside]{book}
% \input{../common.tex}
% \usepackage{a4wide}
% --------------------------------------------
    \graphicspath{{figures/} {../figures/}}
  \begin{document}
  \renewcommand{\nnbb}[2]{} % Disable editorial comments
  \sloppy
\fi
%=================================================================
\chapter{基本的なクラス群}
\chalabel{basic}

Smalltalkの魔法の大部分は言語そのものではなくクラスライブラリにあります。Smalltalkで効率的にプログラミングするには、クラスライブラリがどのように言語や環境を支えているのか学ばなければなりません。クラスライブラリは全てがSmalltalkで記述されていて、ある機能があるクラスに定義されていなかったとしても、パッケージは新しい機能をクラスに追加できるため、簡単に拡張することができます。


私たちの目的は\pharo のクラスライブラリ全体に渡って細かい部分を周りくどく説明することではなく、プログラムを効率的に利用する、または書き換えるのに必要な核となるクラスやメソッドを示すことです。本章ではほぼ全てのアプリケーションで必要になると思われる次の基本的なクラスを取り上げます: \ct{Object}・\ct{Number}とそのサブクラス・\ct{Character}・\ct{String}・\ct{Symbol}・\ct{Boolean}

\md{Here are some comments:\\
- copying: Good question... the copying in \pharo  is much too complicated... there is for one the "old" smalltalk way of
  overrifing postCopy, and then the "automatic" deepCopy... which is quite complex and (I think) was no good idea...
 (see class comment in  DeepCopier)\\
- Debugging: Yes, needs its own chapter. We should talk about haltIf, haltOnce...\\
- assert: Object>>>assert: can take both a block and a boolean, because boleen implements \#value.
  (I will fix SUnit to allow both, too).\\
- Characters and Strings: we should talk about Unicode stuff... but I don't know too much myself.}

%=================================================================
\section{Object}
あらゆる意味で、\clsindmain{Object}クラスは継承階層のルートです。本当のところは\pharo では、オブジェクトとして振舞うことができる最小の実体を定義するために使用される、\clsind{ProtoObject}クラスが階層の真のルートですが、とりあえずそれについては無視していいでしょう。
% (後ほどリフレクションの章で詳しく説明します。)

\ct{Object}クラスは\scatind{Kernel-Objects}カテゴリの中にあります。そこには(拡張も含めると)400ものメソッドがあるということにきっと驚くでしょう。これはつまり、新しいクラスを定義すると、あなたがそれを知っていようといまいと、全てに自動的に400個のメソッドが提供されるということです。ただ、削除されるべき不要なメソッドもいくつかあり、\pharo の新しいバージョンではそれら不要なメソッドのいくつかはおそらく削除されることになることを覚えておきましょう。

\sd{変更される可能性のあるものを引用することは好みませんし、簡単に一覧を見つけられるので、今のところはそのままにしておきます。}
\ct{Object}クラスのクラスコメントには次のように書かれています:

\needlines{4}
\begin{quote}
\textit{\ct{Object}クラスはクラス階層内のほぼ全てのクラスのルートクラスです。例外は\ct{ProtoObject}クラス(\ct{Object}のスーパークラス)とその直接のサブクラスです。
\ct{Object}クラスでは通常のオブジェクトの、アクセス・コピー・比較・エラー処理・メッセージ送信・\ind{リフレクション}などのデフォルトの振舞いが提供されます。また、全てのオブジェクトが応答する必要があるユーティリティメッセージもここで定義されています。
\ct{Object}はインスタンス変数を持っておらず、追加すべきでもありません。これは(例えば\ct{SmallInteger}や\ct{UndefinedObject}などの)\ct{Object}から継承された特別な実装を持つオブジェクトのクラスがいくつかあることや、VMがある標準的なクラス群の構造とレイアウトを知っていてそれに依存しているためです。
}
\end{quote}

\ct{Object}のインスタンス側のメソッドカテゴリをブラウズすると、提供されているいくつかの主だった振舞いを理解できるようになるでしょう。

%-----------------------------------------------------------------
\subsection{プリント}
Smalltalkのすべてのオブジェクトは自分自身を表示用の形式で返すことができます。ワークスペースで好きな式を選び、\menu{print it}メニューを選択してください: 式が実行され、その結果を表示するためのオブジェクトが要求されます。実際には、\ct{printString}メッセージが、返り値のオブジェクトに送られています。\mthind{Object}{printString}は\ind{テンプレートメソッド}で、その核となる部分はレシーバーに送られる\mthind{Object}{printOn:}メッセージにあります。なお、\ct{printOn:}メッセージはいわゆるフックメソッドでオーバーライド可能です。

おそらく\ct{Object>>>printOn:}は最も頻繁にオーバーライドされるメソッドのひとつになるでしょう。このメソッドは引数に\clsind{Stream}を受け取り、そこにオブジェクトの\clsind{String}表現を書き込みます。デフォルトの実装は単純に``\ct{a}''または``\ct{an}''の後にクラス名を書きこむだけです。\ct{Object>>>printString}はそれが実際に書きこまれた\ct{String}を返します。

例えば\clsind{Browser}クラスは\ct{printOn:}メソッドを再定義していないので、printStringメソッドをインスタンスに送ると\ct{Object}で定義されたメソッドが実行されます。
\begin{code}{@TEST}
Browser new printString --> 'a Browser'
\end{code}

\ct{Color}では\mthind{Color}{printOn:}を特殊化した例が見られます。実行するとクラスの名前に続いて色を生成するために使用したクラスメソッドの名前が表示されます。

\needlines{7}
\begin{method}[zork]{printOn:の再定義。}
Color>>>printOn: aStream
  | name |
  (name := self name) ifNotNil: 
    [ ^ aStream
      nextPutAll: 'Color ';
      nextPutAll: name ].
  self storeOn: aStream
\end{method}\ignoredollar$

\begin{code}{@TEST}
Color red printString --> 'Color red'
\end{code}

\ct{printOn:}メッセージは\mthind{Object}{storeOn:}メッセージとは異なることに注意してください。\ct{storeOn:}メッセージはレシーバーを再生成するのに利用できる式を、引数として受け取ったストリームに流します。そしてストリームが\ct{readFrom:}メッセージによって読み込まれるときにその式が評価されます。一方、\ct{printOn:}は単にレシーバーの文字列表現を返すだけです。もちろん、その文字列表現がレシーバーの自己評価型の表現ということもありえますが。

\paragraph{表現と自己評価型表現という単語について}
関数型プログラミングでは、式を実行すると値を返します。一方Smalltalkでは、メッセージ(式)はオブジェクト(値)を返します。オブジェクトの中には自分自身を表す値を持つ便利なプロパティを持つものがあります。例えば、\ct{true}オブジェクトの値はそれ自身、\ie\ct{true}オブジェクトです。このようなオブジェクトは\emphind{自己評価型オブジェクト}と呼ばれます。ワークスペースでオブジェクトをprintすると、オブジェクトの値の\emph{printed}バージョンを見ることができます。次のものが自己評価型表現の例です。

\begin{code}{@TEST}
true         --> true
3@4       --> 3@4
$a           --> $a
#(1 2 3)   --> #(1 2 3)
Color red --> Color red
\end{code}

配列のようなオブジェクトはそこに含まれるオブジェクトによって自己評価型であったり、そうでは無かったりすることに注意して下さい。例えば、真偽値の配列は自己評価型ですが、personオブジェクトの配列はそうではありません。
次の例は、\subind{Array}{dynamic}配列が要素も自己評価型である場合だけ自己評価型になることを示しています:
\begin{code}{@TEST}
{10@10. 100@100}           --> {10@10. 100@100}
{Browser new . 100@100} --> an Array(a Browser 100@100)
\end{code}

\subind{Array}{リテラル}配列はリテラルだけを保持できることに注意して下さい。そのため、次の配列は二つの座標を保持するのではなく、6つのリテラルを持つことになります。
\begin{code}{@TEST}
#(10@10 100@100) --> #(10 #@ 10 100 #@ 100)
\end{code}

\ct{printOn:}メソッドは多くの場合、自己評価型として振舞うように特殊化された実装を持ちます。例えば\cmind{Point}{printOn:}や\cmind{Interval}{printOn:}の実装は自己評価型です。

\begin{method}[Self-evaluating points]{\ct{Point}の自己評価型\footnote{訳注:\pharo 1.3ではコードが変更されています}}
Point>>>printOn: aStream 
    "レシーバーは中置記法でaStream上に書き込まれます"
    x printOn: aStream.
    aStream nextPut: $@.
    y printOn: aStream
\end{method}\ignoredollar$

\begin{method}[Self-evaluating intervals]{\ct{Interval}の自己評価型}
Interval>>>printOn: aStream
    aStream nextPut: $(;
        print: start;
        nextPutAll: ' to: ';
        print: stop.
    step ~= 1 ifTrue: [aStream nextPutAll: ' by: '; print: step].
    aStream nextPut: $)
\end{method}

\begin{code}{@TEST}
1 to: 10 --> (1 to: 10)    "インターバルは自己評価可能です"
\end{code}

%-----------------------------------------------------------------
\subsection{同一性と同値性}

Smalltalkでは、\ct{=}メッセージは\emphsubindmain{Object}{同値性}を確かめます(\ie 二つのオブジェクトが同じ値を持つかどうか)。一方\ct{==}は\emphsubindmain{Object}{同一性}を確かめます(\ie 二つの表現が同じオブジェクトを示すかどうか)。
\seeindex{\ct{=}}{Object, 同値性}
\seeindex{\ct{==}}{Object, 同一性}
\seeindex{equality}{Object, 同値性}
\seeindex{identity}{Object, 同一性}

オブジェクト同値性チェックのデフォルトの実装ではオブジェクトの同一性をテストします:
\begin{method}{オブジェクトの同値性}
Object>>>= anObject
    "レシーバーと引数が同じオブジェクトを表すかどうかを返します。
    もし = をサブクラスによって再定義するなら、hashメッセージも合わせて再定義してください。"
    ^ self == anObject
\end{method}
\cmindex{Object}{=}

これは頻繁にオーバーライドされることのあるメソッドです。\ct{Complex}数の場合を見てみましょう\footnote{訳注:\ct{Complex}クラスは\pharo 1.3に存在しません}:

\begin{code}{@TEST}
(1 + 2 i) = (1 + 2 i)   --> true     "同じ値"
(1 + 2 i) == (1 + 2 i) --> false    "だけど、違うオブジェクト"
\end{code}

このように動作するのは\ct{Complex}が\ct{=}を次のようにオーバーライドしているからです:
\cmindex{Complex}{=}
\needlines{5}
\begin{method}{複素数の同値性}
Complex>>>= anObject
    anObject isComplex
        ifTrue: [^ (real = anObject real) & (imaginary = anObject imaginary)]
        ifFalse: [^ anObject adaptToComplex: self andSend: #=]
\end{method}

\ct{Object>>>~=}のデフォルトの実装は単純に\ct{Object>>>=}を反転しているだけですが、通常は変更する必要はありません。
%\cmindex{Object}{\~=}
\index{Object!~=@\ct{~=}} % needs special treatment due to ~

\begin{code}{@TEST}
(1 + 2 i) ~= (1 + 4 i) --> true
\end{code}

もし\ct{=}をオーバーライドした場合は、\mthind{Object}{hash}をオーバーライドすることを検討すべきです。クラスのインスタンスが\clsind{Dictionary}のキーとして使用された場合に、同値であると見なされるインスタンスはハッシュ値が同じになるようにしておく必要があります:
\cmindex{Complex}{hash}
\begin{method}{複素数のHashは必ず再実装が必要です}
Complex>>>hash
    "= を実装したのでHashも再実装します"
    ^ real hash bitXor: imaginary hash.
\end{method}

\ct{=}と\ct{hash}は同時にオーバーライドするべきですが、\ct{==}は\emph{決して}オーバーライドしてはいけません。(オブジェクトが同一であることの意味は全てのクラスで同じでなければいけません。)\ct{==}は\clsind{ProtoObject}のプリミティブメソッドです。

\pharo は他のSmalltalkと比べていくつか奇妙な振舞いがあることに注意してください: 例えばシンボルと文字列は同値と見なされることがあります。(これは仕様ではなくバグではないかと思います。)

\begin{code}{@TEST}
#'lulu' = 'lulu' --> true
'lulu' = #'lulu' --> true
\end{code}


%-----------------------------------------------------------------
\subsection{クラスメンバーシップ}
幾つかのメソッドを使ってオブジェクトのクラスに問い合わせることができます。

\paragraph{\mthind{Object}{class}.} \ct{class}メッセージを使用してオブジェクトに自身がどのクラスに属するかを尋ねることができます。
\begin{code}{@TEST}
1 class --> SmallInteger
\end{code}

反対に、オブジェクトがあるクラスのインスタンスであるかどうかを尋ねることもできます:
\cmindex{Object}{isMemberOf:}
\begin{code}{@TEST}
1 isMemberOf: SmallInteger --> true    "まさにこのクラスに属していなければいけません"
1 isMemberOf: Integer          --> false
1 isMemberOf: Number        --> false
1 isMemberOf: Object           --> false
\end{code}

SmalltalkはSmalltalk自身によって記述されているので、スーパークラスとクラスメッセージの正しい組み合わせを利用してその構造を自在にたどることができます(\charef{metaclasses}を参照)。

\paragraph{\ct{isKindOf:}}
\cmind{Object}{isKindOf:}はレシーバーのクラスが引数のクラスと等しいか、またはそのサブクラスであるかどうかを返します。

\begin{code}{@TEST}
1 isKindOf: SmallInteger --> true
1 isKindOf: Integer          --> true
1 isKindOf: Number         --> true
1 isKindOf: Object           --> true
1 isKindOf: String            --> false

1/3 isKindOf: Number      --> true
1/3 isKindOf: Integer        --> false
\end{code}

\clsind{Fraction}である\ct{1/3}は\clsind{Number}の一種です。これは\ct{Number}クラスが\ct{Fraction}クラスのスーパークラスだからですが、しかし\ct{1/3}は\ct{Integer}ではありません。

\paragraph{\ct{respondsTo:}}
\cmind{Object}{respondsTo:}はレシーバーが引数として与えられたメッセージセレクタを理解できるかどうかを返します。

\begin{code}{@TEST}
1 respondsTo: #, --> false
\end{code}

一般的には、オブジェクトにクラスを尋ねたり、メッセージを理解できるかどうかを尋ねたりするのはいいやり方ではありません。
オブジェクトのクラスに基づいて判断する代わりに、オブジェクトにただメッセージを送信し、どう振る舞うかはオブジェクトの(クラスに基づいた)決定に任せるべきです。

%-----------------------------------------------------------------
\subsection{コピー}

オブジェクトのコピーにはいくつかちょっとした問題があります。インスタンス変数は参照によってアクセスされるので、あるオブジェクトの\emphsubind{Object}{浅いコピー}は元のオブジェクトとインスタンス変数の参照を共有することになります。
\seeindex{コピー}{Object, \ct{copy}}
\seeindex{浅いコピー}{Object, \ct{shallowCopy}}
\seeindex{深いコピー}{Object, \ct{deepCopy}}

\begin{code}{@TEST | a1 a2 |}
a1 := { { 'harry' } }.
a1 --> #(#('harry'))
a2 := a1 shallowCopy.
a2 --> #(#('harry'))
(a1 at: 1) at: 1 put: 'sally'.
a1 --> #(#('sally'))
a2 --> #(#('sally'))    "内側の配列が共有されています!"
\end{code}

\cmind{Object}{shallowCopy}はオブジェクトの浅いコピーを作成するためのプリミティブなメソッドです。\ct{a2}は\ct{a1}の単なる浅いコピーなので、二つの配列は、中に含まれるネストされた\ct{Array}の参照を共有しています。

\ct{Object>>>shallowCopy}は\cmind{Object}{copy}の``公開インターフェース''で、インスタンスが特殊であればオーバーライドする必要があります。例えば、\clsind{Boolean}クラス、\clsind{Character}、\clsind{SmalInteger}、\clsind{UndefinedObject}などがその例です。

\cmind{Object}{copyTwoLevel}は、単純な浅いコピーではうまくいかないときに、見たままのことを実行してくれます:

\begin{code}{@TEST | a1 a2 |}
a1 := { { 'harry' } } .
a2 := a1 copyTwoLevel.
(a1 at: 1) at: 1 put: 'sally'.
a1 --> #(#('sally'))
a2 --> #(#('harry'))    "状態は完全に独立しています"
\end{code}

\cmind{Object}{deepCopy}は任意の深さまでオブジェクトをコピーします。

\begin{code}{@TEST | a1 a2 |}
a1 := { { { 'harry' } } } .
a2 := a1 deepCopy.
(a1 at: 1) at: 1 put: 'sally'.
a1 --> #(#('sally'))
a2 --> #(#(#('harry')))
\end{code}

\ct{deepCopy}の問題は、相互に再帰しているような構造に対して呼び出した場合に、終了しなくなることです:

\begin{code}{NB: CANNOT TEST}
a1 := { 'harry' }.
a2 := { a1 }.
a1 at: 1 put: a2.
a1 deepCopy --> !\emph{... 終了しません!}!
\end{code}
% NB: Not a test!

正しく動くように\ct{deepCopy}をオーバーライドすることも可能ですが、\cmind{Object}{copy}がもっといい解決策を提供してくれます:

\begin{method}{オブジェクトをコピーするテンプレートメソッド}
Object>>>copy
    "レシーバーと同じようなもう一つのインスタンスを返します。
    一般的にサブクラスではpostCopyをオーバーライドし、
    shallowCopyはオーバーライドしません。"
    ^self shallowCopy postCopy
\end{method}

共有されるべきではないインスタンス変数をコピーするには\mthind{Object}{postCopy}をオーバーライドしましょう。そして、\ct{postCopy}の中では常に\ct{super postCopy}を実行すべきです。

\on{見てみたけど、システムの中にはいい例が見つかりませんでした。}

%-----------------------------------------------------------------
\subsection{デバッグ}

最も重要なメソッドがこの\mthind{Object}{halt}です。メソッドにブレークポイントを設定するには、単にメソッド本文の好きな場所に\ct{self halt}というメッセージ送信を挿入するだけで構いません。このメッセージが送られると、プログラムのその位置で実行が中断され\ind{デバッガ}が起動します。
(デバッガのさらに詳細な内容については\charef{env}を参照してください)

\sd{他の章ではhaltIf:、haltOnce、inspectOnce、flagging: isThisEverCalledなども紹介されています。}

次の最も重要なメッセージは\mthind{Object}{assert:}です。これは引数として\ind{ブロック}を取ります。ブロックが\ct{true}を返すと、実行は継続されます。そうでなければ\ct{AssertionFailure}例外が発生します。この例外がキャッチされなければ、例外の発生した位置でデバッガが開きます。\ct{assert:}は\emphind{契約による設計}をサポートするのに特に便利です。オブジェクトのパブリックなメソッドの重要な事前条件をチェックするというのが最もよくある利用方法でしょう。\cmind{Stack}{pop}は次のように簡単に実装できます:

\begin{method}{事前条件のチェック}
Stack>>>pop
    "最初の要素を返し、スタックからそれを削除します。"
    self assert: [ self isEmpty not ].
    ^self linkedList removeFirst element
\end{method}

\ct{Object>>>assert:}と\cmind{TestCase}{assert:}を混乱しないでください。後者はSUnitテスティングフレームワーク(\charef{SUnit}を参照)の中で出てきます。前者は引数としてブロックを期待する\footnote{実際には\ct{value}メッセージを解釈できる、\ct{Boolean}を含む、あらゆる引数を受け取れます}のに対し、後者は\clsind{Boolean}を期待します。共にデバッグのために有用ですが、それぞれ目的は大きく異なります。

%-----------------------------------------------------------------
\subsection{エラー処理}

このプロトコルにはランタイムエラーを通知するのに便利なメソッドがいくつか含まれます。

\ind{プリファレンスブラウザ}の\protind{debug}プロトコルでdeprecationをオンにすると、\lct{self deprecated: \emph{anExplanationString}}が送信され、現在のメソッドがもはや利用されるべきではないということを通知できます。

\ct{String}引数で代わりの選択肢を提示すべきです。
\cmindex{Object}{deprecated:}
\index{deprecation}

\begin{code}{NB: CANNOT TEST}
1 doIfNotNil: [ :arg | arg printString, ' is not nil' ]
  --> !\emph{SmallInteger(Object)>>doIfNotNil: has been deprecated. use ifNotNilDo:}!
\end{code}

メッセージの探索に失敗すると\ct{doesNotUnderstand:}が送信されます。デフォルトの実装、\ie\cmind{Object}{doesNotUnderstand:}では、その位置でデバッガが起動されます。\lct{does\-Not\-Un\-der\-stand:}をオーバーライドして他の動作をさせると便利なことがあるかもしれません。

\on{例外についての章を書いたらその章への参照を追加すること。}

\cmind{Object}{error}と\cmind{Object}{error:}は例外を発生させるために利用できる一般的なメソッドです。
(ただし、独自に定義した例外を発生させる方が、自分のコードとカーネルクラスのどちらで発生したエラーなのかを区別できるので、一般的には望ましいでしょう。)
\lr{おそらく独自の例外を自分で作成した方がいいということに触れています。(p. 208)}

Smalltalkでは規約により抽象メソッドをメソッドの本文に\lct{self sub\-class\-Res\-pon\-si\-bi\-li\-ty}と書くことによって実装します。抽象クラスが間違ってインスタンス化され、抽象メソッドを呼び出されたとしても、\cmind{Object}{subclassResponsibility}が評価されるだけです。

\begin{method}{メソッドが抽象であることを示します\footnote{訳注:この例は\pharo 1.3では異なります}}
Object>>>subclassResponsibility
    "このメソッドは、このクラスのサブクラスの振る舞いの枠組みの為に用意されています。
    サブクラスはこのメソッドを実装すべきであることに注意してください。"
    self error: 'My subclass should have overridden ', thisContext sender selector printString
\end{method}

\clsind{Magnitude}と\clsind{Number}と\clsind{Boolean}が\subind{class}{抽象}クラスの古典的な例です。この章の少し先で説明します。

\begin{code}{NB: CANNOT TEST}
Number new + 1 --> !\emph{エラー: \#+はサブクラスでオーバーライドしなければいけません}!
\end{code}

規約によりサブクラスが継承すべきではないメソッドを示すには\ct{self shouldNotImplement}を送信します。これは一般的にはクラス階層のデザインがなにかうまくいっていないというサインです。しかし、単一継承という制限があるため、こういった回避策を避けることが難しい場合もあります。
\cmindex{Object}{shouldNotImplement}
\index{inheritance!canceling}

典型的な例は、\clsind{Dictionary}で継承したけれど未実装であるというフラグを立てている\cmind{Collection}{remove:}です。(\ct{Dictionary}には代わりに\mthind{Dictionary}{removeKey:}が用意されています。)

%-----------------------------------------------------------------
\sd{ subsection{廃止予定} }
\sd{あとで書く}

\on{既に上にいくつか説明があります!エラー処理の二つ目の段落を参照して下さい。}

%-----------------------------------------------------------------
\subsection{テスト}

\protind{testing}メソッド群はSUnitのテストとはなんの関係もありません!testingメソッドはレシーバーの状態を尋ねて\clsind{Boolean}が返されるただのメソッドです。

たくさんのテスティングメソッドが\ct{Object}によって提供されています。\mthind{Object}{isComplex}は既に見ました。Objectクラスにはその他にも\mthind{Object}{isArray}、\mthind{Object}{isBoolean}、\mthind{Object}{isBlock}、\mthind{Object}{isCollection}などがあります。オブジェクトに自身のクラスを尋ねるというのはカプセル化を破ることにつながるため一般的にはこれらのメソッドの使用は避けるべきです。オブジェクトに自身のクラスを尋ねるのではなく、ただ要求を送り、オブジェクト自身がそれをどのように扱うかを決められるようにすべきでしょう。

とはいえ、これらのテスティングメソッドのいくつかは間違いなく有用です。もっとも有用なのはおそらく\cmind{ProtoObject}{isNil}と\cmind{Object}{notNil}でしょう(\patind{Null Object}\cite{Wool98a}デザインパターンを使ってこれらのメソッドを必要ではなくすることもできますが)。

% \footnote{However the \emph{Null Object} design pattern can obviate the need for even these methods. See, Bobby Woolf, ``Null Object,'' Pattern Languages of Program Design 3, Robert Martin, Dirk Riehle and Frank Buschmann (Eds.), pp. 5-18, Addison Wesley, 1998.}.

%-----------------------------------------------------------------
\subsection{初期化リリース}

\ct{Object}ではなく\ct{ProtoObject}にある最後のキーメソッドは\mthind{ProtoObject}{initialize}です。

\begin{method}{\lct{initialize} as an empty hook method}
ProtoObject>>>initialize
   "サブクラスはこのメソッドを再定義してインスタンス作成時に初期化を実行しなければいけません"
\end{method}

これが重要な理由は、\pharo では、デフォルトの\mthind{Behavior}{new}メソッドがシステム内のすべてクラスでインスタンスが新しく作成されたときに\ct{initialize}を送るからです。

\begin{method}{クラス側のテンプレートメソッドとしての\lct{new}}
Behavior>>>new
    "indexableではないレシーバー(クラス)の初期化された新しいインスタンス
    を返します。クラスがindexableだった場合は失敗します"
    ^ self basicNew initialize
\end{method}
\cmindex{Behavior}{new}

これは、単に\ct{initialize} \ind{フックメソッド}をオーバーライドすれば、あなたが作成したクラスの新しいインスタンスが自動的に初期化されるということを意味しています。通常、\ct{initialize}メソッドでは継承された全てのインスタンス変数についてクラスの\subind{class}{不変条件}を確立するために\ct{super initialize}を実行する必要があります。
(これは他のSmalltalkでは標準の振舞い\emph{ではない}ことに注意して下さい。)

%=================================================================
\section{数値}
\seclabel{Number}
驚くべきことに、Smalltalkで数値はプリミティブなデータの値ではなく本当のオブジェクトです。もちろん、数値は仮想マシンで効率的に実装されていますが、\clsindmain{Number}の階層構造はSmalltalkクラス階層の他の部分と同じく、完全にアクセス可能で拡張可能です。

\begin{figure}[ht]
\centerline {\includegraphics[width=8cm]{NumberHierarchy}}
\caption{数値のクラス階層 \figlabel{numbers}}
\end{figure}

数値は\scatind{Kernel-Numbers}カテゴリの中にあります。この階層の抽象最上位クラスは\clsind{Magnitude}です。このクラスは比較演算子をサポートするあらゆるの種類のクラスを表します。\ct{Number}は、ほとんどが抽象メソッドですが、さらに様々な算術演算やその他の演算を追加します。\clsind{Float}と\clsind{Fraction}はそれぞれ浮動小数点数と分数を表します。\clsind{Integer}も抽象クラスで、サブクラス\clsind{SmallInteger}、\clsind{LargePositiveInteger}、\clsind{LargeNegativeInteger}に区分されます。ただし、必要に応じて自動的に値が変換されるので、ほとんどのユーザーは3つの\ct{Integer}クラスの違いについて気にする必要はありません。

%-----------------------------------------------------------------
\subsection{強度}

\clsindmain{Magnitude}は\clsind{Number}クラス群の親であるだけでなく、\clsind{Character}、\clsind{Duration}、\clsind{Timespan}などのような比較演算をサポートするその他のクラスの親クラスでもあります。(複素数は比較できません。従って\clsind{Complex}数は\clsind{Number}を継承していません\footnote{訳注:\ct{Complex}クラスは\pharo 1.3に存在しません})

\mthind{Magnitude}{<}と\mthind{Magnitude}{=}は抽象メソッドです。それ以外の演算子は標準的な定義が与えられています。例えば:

\begin{method}{比較用の抽象メソッド}
Magnitude>>> < aMagnitude 
    "レシーバーが引数よりも小さいかどうかを返します"
    ^self subclassResponsibility

Magnitude>>> > aMagnitude 
    "レシーバーが引数よりも大きいかどうかを返します"
    ^aMagnitude < self
\end{method}
\cmindex{Magnitude}{>}

%-----------------------------------------------------------------
\subsection{数値}

同様に、\clsindmain{Number}では\mthind{Number}{+}、\mthind{Number}{-}、\mthind{Number}{*}、\mthind{Number}{/}などが抽象メソッドとして定義されており、それ以外の算術演算には標準的な定義が与えられています。

\ct{Number}オブジェクトは全て、\mthind{Number}{asFloat}や\mthind{Number}{asInteger}のような様々な\emph{converting}オペレーターを備えています。またそれ以外にも、\ct{Number}を実部がゼロの\clsind{Complex}のインスタンスに変換する\mthind{Number}{i}や、\clsindplural{Duration}を生成する\mthind{Number}{hour}、\mthind{Number}{day}、\mthind{Number}{week}のようなたくさんの\emphind{shortcut constructor methods}があります。

\ct{Numbers}は\mthind{Number}{sin}、\mthind{Number}{log}、\mthind{Number}{raiseTo:}、\mthind{Number}{squared}、\mthind{Number}{sqrt}などのような標準的な\emph{mathematical functions}を直接サポートしています。

\cmind{Number}{printOn:}は抽象メソッド\ct{Number>>>printOn:base:}(デフォルトの基数は10)を使って実装されています。

\emph{testing}メソッドとしては\mthind{Number}{even}、\mthind{Number}{odd}、\mthind{Number}{positive}、\mthind{Number}{negative}などがあります。当然\ct{Number}は\lct{is\-Num\-ber}をオーバーライドしています。さらに興味深い点として、\mthind{Number}{isInfinite}が\ct{false}を返すように定義されています。

\emph{truncation and round off}メソッドとしては\mthind{Number}{floor}、\mthind{Number}{ceiling}、\mthind{Number}{integerPart}、\mthind{Number}{fractionPart}などがあります。

\begin{code}{@TEST}
1 + 2.5     --> 3.5             "二つの数の和"
3.4 * 5      --> 17.0           "二つの数の積"
8 / 2         --> 4                 "二つの数の商"
10 - 8.3   --> 1.7              "二つの数の差\footnote{訳注:\pharo 1.3では表示が異なる場合があります}"
12 = 11    --> false           "二つの数の等価性"
12 ~= 11 --> true            "二つの数が異なるかどうかをテスト"
12 > 9      --> true            "より大きい"
12 >= 10  --> true            "以上"
12 < 10    --> false           "より小さい"
100@10   --> 100@10    "座標の生成"
\end{code}
\on{listingパッケージ内でタブがどう動作するかをチェックすること...}

次の例が\st では驚くほど正しく動作します:
\begin{code}{@TEST}
1000 factorial / 999 factorial --> 1000
\end{code}
\ct{1000 factorial}は他の多くの言語では計算するのが非常に難しいのですが、実際に計算できているという点に注意して下さい。これは自動変換と、数値の厳密な扱いの非常に良い例になっています。
\cmindex{Integer}{factorial}

\dothis{\ct{1000 factorial}の結果を表示してみましょう。計算よりも表示に長い時間がかかります!}

%-----------------------------------------------------------------
\subsection{浮動小数点数}

\clsindmain{Float}は\ct{Number}の抽象メソッドに浮動小数点数用の実装を与えます。

加えて興味深いことに、\ct{Float class}(\ie\ct{Float}のクラス側)は次の\emph{定数}を返すメソッドを提供しています: \mthind{Float class}{e}、\mthind{Float class}{infinity}、\mthind{Float class}{nan}、\mthind{Float class}{pi}。

\begin{code}{@TEST}
Float pi                      --> 3.141592653589793
Float infinity               --> 無限大
Float infinity isInfinite --> true
\end{code}

%-----------------------------------------------------------------
\subsection{分数}

\clsind{Fractions}は、\ct{Integer}である分子と分母を表すインスタンス変数によって表されます。\ct{Fractions}は(\cmind{Fraction}{numerator:denominator:}というコンストラクタメソッドを使用するのではなく)通常は\ct{Integer}の割り算によって生成されます。

\begin{code}{@TEST}
6/8             --> (3/4)
(6/8) class --> Fraction
\end{code}

\ct{Fraction}と\ct{Integer}または他の\ct{Fraction}を掛け合わせると\ct{Integer}になることがあります:

\begin{code}{@TEST}
6/8 * 4 --> 3
\end{code}

\lr{分数の使用を避けるには演算数のいずれかを浮動小数点数にしなければいけないということについて触れるかもしれません。例、6.0 / 8 or 6 asFloat / 8. (p. 213)}

%-----------------------------------------------------------------
\subsection{整数}

\clsindmain{Integer}は三つの具象整数クラスの親に当たる抽象クラスです。多くの抽象的な\ct{Number}のメソッドの具体的な実装を提供するだけでなく、\mthind{Integer}{factorial}、\mthind{Integer}{atRandom}、\mthind{Integer}{isPrime}、\mthind{Integer}{gcd:}など、整数独自のメソッドも数多く追加されています。

\clsindmain{SmallInteger}は、インスタンスがコンパクトに表現できるという点で特殊です。つまり、\ct{SmallInteger}は値を参照として保持する代わりに、他の整数が参照を保持するために使用している領域を直接数値を表すために使用します。オブジェクト参照の最初のビットが、そのオブジェクトが\ct{SmallInteger}かどうかを示しています。

クラスメソッドである\mthind{SmallInteger}{minVal}と\mthind{SmallInteger}{maxVal}で、\ct{SmallInteger}の範囲が分かります:

\begin{code}{@TEST}
SmallInteger maxVal = ((2 raisedTo: 30) - 1)      --> true
SmallInteger minVal = (2 raisedTo: 30) negated --> true
\end{code}

\ct{SmallInteger}がその範囲に収まらなくなると、必要に応じて自動的に\clsind{LargePositiveInteger}または\clsind{LargeNegativeInteger}に変換されます:

\begin{code}{@TEST}
(SmallInteger maxVal + 1) class --> LargePositiveInteger
(SmallInteger minVal - 1) class  --> LargeNegativeInteger
\end{code}

大きな整数も同様に、適切なタイミングで小さな整数に変換されます。

多くのプログラミング言語と同じく、整数はイテレーティブな振舞いを表すために利用できます。ブロックを繰り返し評価することに特化した\mthind{Integer}{timesRepeat:}というメソッドもあります。
\charef{syntax}で既に同様の例が出てきました:
\begin{code}{@TEST | n |}
n := 2.
3 timesRepeat: [ n := n*n ].
n --> 256
\end{code}

%=================================================================
\section{文字}

\clsindmain{Character}は\scatind{Collections-Strings}カテゴリで、\clsind{Magnitude}のサブクラスとして定義されています。\pharo 内で表示可能な文字は\lct{\$$\langle$\emph{char}$\rangle$}で表されます。例えば次のように成ります:

\begin{code}{@TEST}
$a < $b --> true
\end{code}

表示できない文字も様々なクラスメソッドを利用して生成できます。\mbox{\cmind{Character class}{value:}}はUnicode(またはASCII)を表す整数値を受け取って、対応する文字を返します。\protind{accessing untypeable characters}プロトコルには、\mthind{Character class}{backspace}、\mthind{Character class}{cr}、\mthind{Character class}{escape}、\mthind{Character class}{euro}、\mthind{Character class}{space}、\mthind{Character class}{tab}などのたくさんの便利なコンストラクタが含まれています。

\begin{code}{@TEST}
Character space = (Character value: Character space asciiValue) --> true
\end{code}

\mthind{Character}{printOn:}メソッドは優秀なので、文字を生成する三つの方法の内で最も適切な表現がどれかを理解しています:

\begin{code}{@TEST}
Character value: 1   --> Character home
Character value: 2   --> Character value: 2
Character value: 32 --> Character space
Character value: 97 --> $a
\end{code}\ignoredollar$

次のような便利な\emph{testing}メソッドが数多く組み込まれています: \mthind{Character}{isAlphaNumeric}、\mthind{Character}{isCharacter}、\mthind{Character}{isDigit}、\mthind{Character}{isLowercase}、\mthind{Character}{isVowel}など。

\ct{Character}をその文字だけを含む文字列に変換するには、\mthind{Character}{asString}メッセージを送信します。この場合、\ct{asString}と\mthind{Character}{printString}は異なる結果を生じます。

\begin{code}{@TEST}
$a asString    --> 'a'
$a                  --> $a
$a printString --> '$a'
\end{code}

ascii \ct{Character}はそれぞれ一意のインスタンスで、クラス変数である\cvind{CharacterTable}の中に保持されています:

\begin{code}{@TEST}
(Character value: 97) == $a --> true
\end{code}\ignoredollar$

しかし、0から255の範囲の外にある\ct{Characters}は一意ではありません:

\begin{code}{@TEST}
Character characterTable size                               --> 256
(Character value: 500) == (Character value: 500) --> false
\end{code}

%=================================================================
\section{文字列}

\clsindmain{String}クラスも\scatind{Collections-Strings}カテゴリで定義されています。\ct{String}は\ct{Characters}だけを保持できるインデックス付きの\ct{Collection}です。

\begin{figure}[ht]
  {\centerline {\includegraphics[width=0.4\textwidth]{StringHierarchy}}}
\caption{文字列のクラス階層 \figlabel{strings}}
\end{figure}

実のところ、\ct{String}は抽象クラスで、\pharo の\ct{Strings}は実際には具象クラスである\clsindmain{ByteString}のインスタンスです。

\begin{code}{@TEST}
'hello world' class --> ByteString
\end{code}

\ct{String}のもう一つの重要なサブクラスは\clsindmain{Symbol}です。主な違いは、\ct{Symbol}には一つの値につきたった一つのインスタンスしか存在しないということです。(これはしばしば``一意インスタンス特性''と呼ばれます。)一方、別々に生成されたまたま同じ文字の並びを持つ\ct{String}は異なるオブジェクトであることもよくあります。

\begin{code}{@TEST}
'hel','lo' == 'hello' --> false
\end{code}

\begin{code}{@TEST}
('hel','lo') asSymbol == #hello --> true
\end{code}

\noindent
もう一つの重要な違いは、\ct{String}は変更可能ですが、\ct{Symbol}は変更不能ということです。

\begin{code}{@TEST}
'hello' at: 2 put: $u; yourself --> 'hullo'
\end{code}\ignoredollar$

\begin{code}{NB: CANNOT TEST}
#hello at: 2 put: $u --> エラー!
\end{code}\ignoredollar$

文字列はコレクションなので、他のコレクションと同様のメッセージを理解できるということは見落とされがちです:

\begin{code}{@TEST}
#hello indexOf: $o --> 5
\end{code}\ignoredollar$

\ct{String}は\clsind{Magnitude}を継承していませんが、\ct{<}, \ct{=}などの通常の\protind{comparing}メソッド群をサポートしています。さらに、\cmind{String}{match:}はいくつかの基本的なglobスタイルのパターンマッチが使用できて便利です。

\begin{code}{@TEST}
'*or*' match: 'zorro' --> true
\end{code}

さらに進んだ正規表現のサポートが必要なときには、Vassili Bykovによる\pkgind{Regex}パッケージを参照してください。
\index{Bykov, Vassili}
\index{regular expression package}

文字列は比較的多くの変換メソッドをサポートしています。それらの多くは\mthind{String}{asDate}、\mthind{String}{asFileName}などのような、他のクラスの\ind{ショートカットコンストラクタメソッド}です。また、\mthind{String}{capitalized}や\mthind{String}{translateToLowercase}などのような、文字列から他の文字列に変換する便利なメソッドもたくさんあります。

文字列とコレクションについて更に進んだ話題は、\charef{collections}を参照して下さい。

\on{さらに利用できる資料がこちらにあります:
\url{http://www.dmu.com/crb/crb7.html}.}

%=================================================================
\section{真偽値}

\clsindmain{Boolean}クラスはSmalltalkという言語がどれほど多くのものをクラスライブラリに詰め込んでいるかということについて非常に興味深い洞察を与えてくれます。\ct{Boolean}は\patind{Singleton}クラスである\clsindmain{True}と\clsindmain{False}の\subind{class}{抽象}スーパークラスです。

\begin{figure}[ht]
  {\centerline {\includegraphics[width=0.5\textwidth]{BooleanHierarchy}}}
\caption{真偽値のクラス階層 \figlabel{booleans}}
\end{figure}

\ct{Boolean}の振舞いのほとんどは引数として二つの\ct{Blocks}を取る\mthind{Boolean}{ifTrue:ifFalse:}を詳しく調べることで理解できるでしょう。

\begin{code}{@TEST}
(4 factorial > 20) ifTrue: [ 'bigger' ] ifFalse: [ 'smaller' ] --> 'bigger'
\end{code}

このメソッドは\ct{Boolean}の中では抽象メソッドです。
具象サブクラスでの実装はいずれも簡単なものです:

\begin{method}{\lct{ifTrue:ifFalse:}の実装}
True>>>ifTrue: trueAlternativeBlock ifFalse: falseAlternativeBlock 
    ^trueAlternativeBlock value

False>>>ifTrue: trueAlternativeBlock ifFalse: falseAlternativeBlock 
    ^falseAlternativeBlock value
\end{method}
\cmindex{True}{ifTrue:}
\cmindex{False}{ifTrue:}

実際、これこそがOOPの本質です。メッセージがオブジェクトに送られたら、オブジェクト自身がどのメソッドで反応するかを決めるのです。今回の場合では、\ct{True}インスタンスは\emph{true}の選択肢を実行し、反対に\ct{False}インスタンスは\emph{false}の選択肢を実行します。\ct{Boolean}の全ての抽象メソッドは、\ct{True}と\ct{False}の中でこのような方法で実装されています。例えば:

\begin{method}{否定の実装}
True>>>not
    "否定--レシーバーがtrueなのでfalseを返します。"
    ^false
\end{method}
\cmindex{True}{not}

\ct{Booleans}には\mthind{Boolean}{ifTrue:}、\mthind{Boolean}{ifFalse:}、\mthind{Boolean}{ifFalse:ifTrue}などのような便利なメソッドがいくつかあります。また強欲・怠惰な論理積・論理和から好きなものを選ぶことができます。

\begin{code}{@TEST}
(1>2) & (3<4)              --> false    "両者を必ず評価します"
(1>2) and: [ 3<4 ]        --> false    "レシーバーだけが評価されます"
(1>2) and: [ (1/0) > 0 ] --> false    "引数ブロックは評価されないので、例外も発生しません"
\end{code}

最初の例では、\mthind{Boolean}{&}が\ct{Boolean}引数を受け取るので、\ct{Boolean}である部分式は両方とも評価されます。
二つ目と三つ目の例では、\mthind{Boolean}{and:}が引数として\ct{Block}を取るので、最初の式しか評価されません。\ct{Block}は最初の引数が\pvind{true}のときにだけ評価されるのです。

\dothis{\ct{and:}と\ct{or:}がどのように実装されているか想像してみてください。
そして\ct{Boolean}と\ct{True}と\ct{False}での実装を確認してみましょう。}

%=================================================================
\section{本章のまとめ}

\begin{itemize}
%  \item Send \ct{yourself} to get back the receiver at the end of a cascade.

  \item もし\ct{=}をオーバーライドしたら、同時に\ct{hash}もオーバーライドする必要があります。

  \item 自作オブジェクトのコピーを正しく実装するには\ct{postCopy}をオーバーライドして下さい。

  \item ブレークポイントを設定するには\ct{self halt}を送ります。

  \item メソッドを抽象メソッドにするには\ct{self subclassResponsibility}を返します。

  \item オブジェクトの\ct{String}表現を提供するには、\ct{printOn:}をオーバーライドして下さい。

  \item インスタンスを適切に初期化するにはフックメソッドである\ct{initialize}をオーバーライドします。

  \item \ct{Number}のメソッド群は自動的に\ct{Floats}、\ct{Fractions}、\ct{Integers}の間で変換されます。

  \item \ct{Fractions}は浮動小数点数ではなく本当の有理数を表しています。

  \item \ct{Characters}のインスタンスは一意です。

  \item \ct{Strings}は変更可能ですが、\ct{Symbols}はそうではありません。
  しかし文字列リテラルは変更できないということに注意しましょう!

  \item \ct{Symbols}は一意ですが、\ct{Strings}はそうではありません。

  \item \ct{Strings}と\ct{Symbols}は\ct{Collections}なので、一般的な\ct{Collection}のメソッドをサポートしています。

\end{itemize}

%=============================================================
\ifx\wholebook\relax\else
   \bibliographystyle{jurabib}
   \nobibliography{scg}
   \end{document}
\fi
%=============================================================

%-----------------------------------------------------------------

%%% Local Variables:
%%% coding: utf-8
%%% mode: latex
%%% TeX-master: t
%%% TeX-PDF-mode: t
%%% ispell-local-dictionary: "english"
%%% End: 
