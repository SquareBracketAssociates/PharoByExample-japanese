% $Author$
% $Date$
% $Revision$
% $Id$

% HISTORY:
% 2006-10-24 - Stef started
% 2006-12-01 - Oscar edit
% 2006-12-02 - Andrew edit
% 2007-05-04 - Oscar first draft
% 2007-07-04 - Stef review

%=================================================================
\ifx\wholebook\relax\else
% --------------------------------------------
% Lulu:
	\documentclass[a4paper,10pt,twoside]{book}
	\usepackage[
		papersize={6.13in,9.21in},
		hmargin={.75in,.75in},
		vmargin={.75in,1in},
		ignoreheadfoot
	]{geometry}
	\input{../common.tex}
	\pagestyle{headings}
	\setboolean{lulu}{true}
% --------------------------------------------
% A4:
%	\documentclass[a4paper,11pt,twoside]{book}
%	\input{../common.tex}
%	\usepackage{a4wide}
% --------------------------------------------
    \graphicspath{{figures/} {../figures/}}
	\begin{document}
	\renewcommand{\nnbb}[2]{} % Disable editorial comments
	\sloppy
\fi
%=================================================================
\chapter{Syntax in a nutshell}
\chalabel{syntax}

\sd{We should add pragmas.}
\on{Please do so.}

% \sd{It would be good to add link to the chapter where the reader can learn about conditional, exceptions and loops.}
% \on{There are links already.}

\pharo のシンタックスは他の近代的な \st 実装と同様、 \st-80 にとてもよく似ています。
そして、プログラムをピジン英語のように声に出して読めるよう \ind{シンタックス} が設計されています。

\begin{code}{}
(Smalltalk includes: Class) ifTrue: [ Transcript show: Class superclass ]
\end{code}

\noindent
\pharo のシンタックスはとてもシンプルです。
本質的には、メッセージ(\ie 式)を送信するためにのみシンタックスが存在します。
式は少数の基本要素を使って構築します.
キーワードは6つだけしかありません.さらに、制御構造のためのシンタックスも、新しいクラスを宣言するためのシンタックスもありません.
代わりに、オブジェクトにメッセージを送信することにより、ほとんどのことが達成できるのです.
例えば、if-then-else制御構造の代わりに、 \st は \clsind{Boolean} オブジェクトに \ct{ifTrue:} のようなメッセージを送ります.
また、新しい(サブ)クラスを作成するときには作成したいクラスのスーパークラスにメッセージを送ります.

%=================================================================
\section{Syntactic elements}

式は以下の要素からできています。
(i) 6つの予約語もしくは擬似変数:
\pvind{self}, \pvind{super}, \pvind{nil}, \pvind{true}, \pvind{false}, and \pvind{thisContext},
(ii) 数、文字、文字列、シンボルおよび配列を含む \emphind{リテラルオブジェクト}用定数式、
(iii) 変数宣言、
(iv) 代入、
(v) \ind{ブロック} クロージャー
(vi) メッセージ
\seeindex{pseudo-variable}{variable, pseudo}

\begin{table}\centering
	\begin{tabular}{ll}
		\toprule
		シンタックス & 意味 \\
		\midrule
		\lct{startPoint}			&	変数名 \\
		\lct{Transcript}			&	グローバル変数名 \\
		\lct{self}				&	擬似変数 \\
		\midrule
		\lct{1}				 	&	10進整数 \\
		\lct{2r101}				&	2進整数 \\
		\lct{1.5}					&	浮動小数点数 \\
		\lct{2.4e7}				&	指数表現 \\
		\lct{\$a}					&	文字 `a' \\
		\lct{'Hello'}				&	文字列 ``Hello'' \\
		\lct{\#Hello}				&	シンボル \lct{\#Hello} \\
		\lct{\#(1 2 3)}			&	リテラル配列 \\
		\lct{\{1. 2. 1+2\}}		&	動的配列 \\
		\midrule
		\lct{"a comment"} 		&	コメント \\
		\midrule
		\lct{| x y |}				&	変数 \lct{x} と \lct{y} の宣言	\\
		\lct{x := 1}				&	\lct{x} に 1 を代入 \\
		\lct{[ x + y ]}			&	\lct{x+y} を評価するブロック \\
		\lct{<primitive: 1>}		&	バーチャルマシン・プリミティブもしくはアノテーション \\
		\midrule
		\lct{3 factorial}			&	単項メッセージ \\
		\lct{3+4}					&	二項メッセージ \\
		\lct{2 raisedTo: 6 modulo: 10}		&	キーワードメッセージ \\
		\midrule
		\lct{$\uparrow$ true} 			&	値 true を返す \\
		\lct{Transcript show: 'hello'. Transcript cr }		&	式セパレーター (\lct{.})	\\
		\lct{Transcript show: 'hello'; cr}					&	メッセージカスケード (\lct{;}) \\
		\bottomrule
	\end{tabular}
	\caption{\pharo Syntax in a Nutshell\tablabel{syntax}}
\end{table}

\tabref{syntax} は様々なシンタックスの例です。
\begin{description}
\item[ローカル変数] \ct{startPoint} は変数名もしくは識別子です。
		慣習により、識別子は ``\ind{キャメルケース}'' (\ie 2番目以降の単語の先頭を大文字にし、単語を連結したもの)で記述します。
		インスタンス変数、メソッドとブロックの引数、一時変数の先頭の文字は小文字にします.
		つまり、プライベートなスコープを持つ変数の先頭を小文字にするのです.

\item[共有変数] の先頭は大文字にします。\ind{グローバル変数} 、 \ind{クラス変数} 、 \ind{プール・ディクショナリ} 、 もしくはクラス名を表します。
		\ct{Transcript} はグローバル変数であり、クラス \ct{TranscriptStream} のインスタンスです。
		\seeindex{global variable}{variable, global}
		\seeindex{pool dictionary}{variable, pool}
		\seeindex{variable!class}{class, variable}

\item[レシーバー。] \pvind{self} は現在実行しているメソッドからオブジェクトを参照するためのキーワードです。selfは「レシーバー」とも呼びます。なぜなら、selfは、メソッドを実行するメッセージを受け取るからです。
		selfには代入することができないので「擬似変数」といいます。

\item[整数。] \ct{42}のような通常の十進整数表記法に加えて、\pharo では基数表記法も使えます。
	\ct{2r101}は基数2(\ie 2進数)で101であり、10進数の5と同じです。
	\index{リテラル}
	\index{リテラル!数}

\item[浮動小数点数] は10を基数とした\ind{指数}で指定することができます: \mbox{\ct{2.4e7}}は$2.4 \times 10^7$です。
 	\index{浮動小数点数}

\item[文字。] ドル記号は\subind{リテラル}{文字}を表します: \ct{$a}\ignoredollar$ は「a」のリテラルです。
		非印刷文字のインスタンスは\ct{Character space}や\ct{Character tab}のように\clsind{Character}クラスに適切なメッセージを送ることにより得ることができます。
		
\item[文字列。] シングルクォートは\subind{リテラル}{文字列}を定義するために使います。
		クォートを含んだ文字列を表現する場合は、\ct{'G''day'} のように、クォートを重ねてください。

\item[シンボル] はそれらが文字の列で出来ている点で、文字列に似ています。
	しかし、文字列とシンボルはグローバルに一意であることが保証される点で文字列とは異なります。
		シンボルオブジェクト\ct{#Hello}はひとつだけしか存在しません。しかし、値\ct{'Hello'}を持った文字列オブジェクトは複数あるかもしれません。
		\seeindex{\#@{\textsf{\#}}}{literal symbol}

\item[コンパイル時配列] はスペースで分離したリテラルを\ct{#( )}で囲んで定義します。
		括弧内のすべてはコンパイル時に定数でなければなりません。
		例えば、\ct{#(27 (true false) abc)}は3つの要素の\subind{配列}{リテラル}です: 整数ct{27}、2つの論理値が入ったコンパイル時配列、そしてシンボル\ct{#abc}。(これが\ct{#(27 #( true false) #abc)}と同じであることに注意してください。)

\item[実行時配列。] 中括弧 \ct|{ }|は、実行時に(ダイナミックに)配列を定義します。
		要素はピリオドによって分離された式です。
		したがって、\ct|{ 1. 2. 1+2 }|は、要素 1、 2、と 1+2 を評価した結果の配列を定義します。
		(中括弧表記法は \pharo と \squeak の特有な機能です!
		他のSmalltalkでは、動的配列を明示的に構築しなければなりません。)

\item[コメント]はダブルクォートで囲います。
		\ct{"hello"} は文字列ではな\ind{くコメント}なので、\pharo コンパイラは無視します。
		コメントは複数行にまたがってもよいです。
		
\item[ローカル変数宣言。] メソッド(およびブロック中)でローカル変数を宣言するには、1つ以上の宣言を縦線\ct{| |}で囲みます。
		% \seeindex{\|@{\textsf{\|\|}}}{assignment}
		% Can't seem to index or-bars! (special char for index macro)
		\seeindex{宣言}{変数宣言}

\item[代入。] \ct{:=} は変数にオブジェクトを代入します。
%		Sometimes you will see $\leftarrow$ used instead.
%		Unfortunately, since this is not an \textsc{ascii} character, it will appear as an underscore unless
%		you are using a special font.
%		So, \ct{x := 1} is the same as \ct{x _ 1} or \ct{x UNDERSCORE 1}. You should use \ct{:=} since the other representations have been deprecated.
		\index{代入}
		\seeindex{:=@{\textsf{:=}}}{代入}
		\seeindex{\_@{\textsf{\_}}}{代入}
		\seeindex{<-@{$\leftarrow$}}{代入}

\item[ブロック。] 角括弧 \ct{[ ]} はブロック・クロージャーあるいは静的クロージャーと呼ばれるブロックを定義します。ブロックは関数を表わすファーストクラスのオブジェクトです。
		ブロックは引き数をとることも、ローカル変数を持つこともできます。
	\seeindex{[ ]@{\textsf{[ ]}}}{block}
	\seeindex{closure}{block}
	\seeindex{lexical closure}{block}

\item[プリミティブ。] \ct{<primitive: ...>}は、バーチャルマシン・プリミティブの呼出しを表わします。
	(\ct{<primitive: 1>} は \ct{SmallInteger>>>+}のVMプリミティブです。)
	プリミティブが異常終了した場合、プリミティブの後にあるコードが実行されます。
	同じシンタックスはメソッド・アノテーションにも使われます。

\item[単項メッセージ] は、レシーバー(例えば\ct{3})へ送られる単一語(例えば\ct{factorial})から構成されます。
	\index{message!unary}
	\seeindex{unary message}{message, unary}

\item[二項メッセージ] は、レシーバーに送られる、単一の引き数をとる演算子(例えば \ct{+})です。\ct{3+4}の場合、レシーバーはct{3}で、引数は\ct{4}です。
	\index{message!binary}
	\seeindex{binary message}{message, binary}

\item[キーワード・メッセージ]は、それぞれコロンで終わり、ひとつの引き数を取る複数のキーワード(例えば \ct{raisedTo:modulo:})から構成されます。
	式\ct{2 raisedTo: 6 modulo: 10} では、\emphind{メッセージセレクター} \ct{raisedTo:modulo:}はそれぞれコロンで終わり、引数 \ct{6} および \ct{10} を取り、レシーバー \ct{2} へメッセージを送ります。
	\index{message!keyword}
	\seeindex{keyword message}{message, keyword}

\item[メソッド・リターン。] \ct{^}はメソッドから値を返すために使われます。(TODO)
\md{\ct{^} always returns from the method, even if used in a block, it returns from the enclosing method.}

\item[ステートメント・シーケンス。] ピリオド(\ct{.})はステートメント・セパレーターです。2つの式の間にピリオドを置くと、それらは独立したステートメントとして扱われます。
	\seeindex{full stop}{statement separator}
	\seeindex{period}{statement separator}
	\seeindex{\ct{.}}{statement separator}

\item[カスケード。] セミコロンは一つのレシーバーへメッセージを連続して送るために使います。 \ct{Transcript show: 'hello'; cr} は、最初にレシーバーである \ct{Transcript} にキーワード・メッセージ\ct{show: 'hello'}を送り、次に、同じレシーバーへ単項のメッセージ\ct{cr}を送ります。
	\seeindex{;}{cascade}

\end{description}

クラス\ct{Number}、\ct{Character}、\ct{String}、\ct{Boolean}は、\charef{basic}でより詳細に取り上げます。

\on{Blocks are described in \charef{blocks}. (Control flow and Iterators).}

%=================================================================
\section{Pseudo-variables}

\st では、次の6つの予約キーワード、擬似変数があります:
\pvind{nil}、\pvind{true}、\pvind{false}、\pvind{self}、\pvind{super}、\pvind{thisContext}。
これらは事前に定義され、代入することができないので、擬似変数と呼びます。
\ct{true}、\ct{false}および\ct{nil}は定数です。その一方で、\ct{self}、\ct{super}および\ct{thisContext}の値はコードの実行中に動的に変化します。

\ct{true}と\ct{false}は\clsind{Boolean}クラス\clsind{True}および\clsind{False}の唯一のインスタンスです。
より詳細については、\charef{basic}を参照してください。

\pvind{self}は、常に現在実行しているメソッドのレシーバーを指します。

\ct{super}は、カレントメソッドのレシーバーを指します。しかし、\ct{super}にメッセージを送る場合、メソッド・ルックアップはスーパークラスから始まります。
更なる詳細は\charef{model}を参照してください。

\ct{nil}は未定義のオブジェクトを
表すクラスUndefinedObjectの唯一のインスタンスです。
インスタンス変数、クラス変数およびローカル変数は\ct{nil}に初期化されます。

\ct{thisContext}は、ランタイム・スタックのトップフレームを表わす擬似変数です。
現在実行している\clsind{MethodContext}あるいは\clsind{BlockClosure}を表わすと言い換えることができます。
\ct{thisContext}は、通常のプログラミングに必要ありませが、デバッガのような開発ツールの実装には不可欠です。また、例外処理と継続を実装するためにも使われます。

%=================================================================
\section{Message sends}

\pharo には3種類のメッセージがあります。
\begin{enumerate}
  \item 引数をとらない単項メッセージ。
  \ct{1 factorial}は、オブジェクト\ct{1}に、メッセージ\ct{factorial}を送ります。
  \item 1つの引き数をとる二項メッセージ。
  	\ct{1 + 2}は、オブジェクト \ct{1} に、メッセージ \ct{+} を引数 \ct{2} とともに送ります。
  \item 任意の数の引数をとるキーワードメッセージ。
  	\ct{2 raisedTo: 6 modulo: 10} は、オブジェクト \ct{2} に、メッセージセレクタ
	\ct{raisedTo:modulo:} と引数 \ct{6}, \ct{10} から構成されるメッセージを送ります。
\end{enumerate}

単項のメッセージセレクタは英小文字でから始まる英数字で記述します。
\index{message!unary}

二項メッセージ・セレクターは次の文字を使い1文字以上で記述します:
\index{message!binary}
\begin{code}{}
+ - / \ * ~ < > = @ % | & ! ? ,
\end{code}
\noindent
% [\~\!\@\%\&\*\-\+\=\\\|\?\/\>\<\,]
\on{It seems that 3 or more chars work fine, but it is not possible to have more than one ``-'' in a binary selector. Perhaps due to a conflict with parsing negative numbers?}
\ab{That's right; $-$ is weird.}
キーワード・メッセージセレクタは一連のキーワードから構成します。キーワードはそれぞれ英小文字で始まり、コロンで終わる英数字で記述します。
\index{message!keyword}

単項のメッセージは最も高い優先順位を持ちます。次に二項メッセージ、最後にキーワード・メッセージの順になります。
\begin{code}{@TEST}
2 raisedTo: 1 + 3 factorial --> 128
\end{code}
(最初に、\ct{3}へ\ct{factorial}を送ります。次に、\ct{1}へ\ct{+ 6}を送ります。最後に、\ct{2}へ\ct{raisedTo: 7}を送ります。)
式を評価した結果を示すために、式\ct{-->}結果という表記を使うことを思い出してください。

評価順は、厳密に左から右です、
\begin{code}{@TEST}
1 + 2 * 3 --> 9
\end{code}
\ct{7}ではありません。
評価順序を変更するためには括弧を使わなければなりません:
\begin{code}{@TEST}
1 + (2 * 3) --> 7
\end{code}

メッセージ送信はピリオドとセミコロンで組み立てられることもあります。ピリオドで分離された式の並びはステートメントとして順々に評価されます。
\index{statement!separator}

\begin{code}{}
Transcript cr.
Transcript show: 'hello world'.
Transcript cr
\end{code}

\noindent
これは\glbind{Transcript}オブジェクトに\ct{cr}を送り、次に、\ct{show: 'hello world'}を送り、最後に、もう一つの\ct{cr}を送ります。

一連のメッセージを同じレシーバーへ送る場合、カスケードとしてより簡潔に表現することができます。
レシーバーを、一度だけ指定し、メッセージのシーケンスをセミコロンで分割します:

\begin{code}{}
Transcript cr;
    show: 'hello world';
    cr
\end{code}
このプログラムは前のプログラムと同じ意味になります。

%=================================================================
\section{Method syntax}

\pharo のあらゆる場所(例えば、デバッガ、ワークスペース、ブラウザ)で式を評価することができます。けれども、メソッドは通常、ブラウザウィンドウやデバッガで定義します。
(外部媒体で記述したメソッドを読み込むこともできます。しかし、\pharo でプログラムする普通の方法ではありません。)

プログラムは、クラスのコンテキストの中で、1メソッドずつ開発していきます。
(クラスは既存のクラスにサブクラスを作成するようにメッセージを送ることにより定義します。したがって、クラスの定義に特別のシンタックスは必要ないのです。)

次の例には、クラス\clsind{String}にメソッド\mthind{String}{lineCount}があります。
(慣例でメソッドを\ct{ClassName>>>methodName}と表します。したがって、このメソッドは\ct{String>>>lineCount}と表します。)

\needlines{9}
\begin{method}[lineCount]{Line count}
String>>>lineCount
   "Answer the number of lines represented by the receiver,
   where every cr adds one line."
   | cr count |
   cr := Character cr.
   count := 1 min: self size.
   self do:
      [:c | c == cr ifTrue: [count := count + 1]].
   ^ count
\end{method}

構文上、メソッドは次のものから構成されます:
\begin{enumerate}
  \item 名前(\ie \ct{lineCount})と引数(この例ではひとつもない)からなるメソッドの型
  \item コメント(どんな場所にも書くことができますが、慣例では、メソッドが何を行うかの説明を一番上に書きます。)
  \item ローカル変数(\ie \ct{cr} と \ct{count})
  \item ドットによって分割した任意個の式(この例では4つあります)
\end{enumerate}

\ct{^}(\verb|^|とタイプした)が先頭にある式を評価すると、その式の値を返し、その時点でメソッドを終了します。
明示的に式を評価せずに終了するメソッドは暗黙的に\pvind{self}を返します。
\index{return!implicit}

引数とローカル変数は、常に英小文字で始めます。
英大文字で始まる名前はグローバル変数に使用します。
例えば、\ct{Character}のようなクラス名は、そのクラスを表わすオブジェクトを指すグローバル変数です。

%=================================================================
\section{Block syntax}

\ind{ブロック}は、式の評価を遅延させるためのメカニズムを提供します。
ブロックは本質的には無名の関数で、メッセージ\mthind{BlockClosure}{value}を送ることにより評価することができます。
ブロックは明示的なリターン(\ct{^}が付いた式)がなければ、本体中の最後の式の値を返します。
\seeindex{value}{BlockClosure}

\begin{code}{@TEST}
[ 1 + 2 ] value --> 3
\end{code}

ブロックは引数をとることができます。引数は先頭にコロンをつけて宣言します。
縦線でブロックの本体と引数宣言を区切ります。
1つの引数を持ったブロックを評価するためには、1つの引数を持ったメッセージ\mthind{BlockClosure}{value:}を送らなければなりません。
2つの引数をとるブロックでは\mthind{BlockClosure}{value:value:}を送ります、同様に、4つまでの引数を使えます。

\begin{code}{@TEST}
[ :x | 1 + x ] value: 2 --> 3
[ :x :y | x + y ] value: 1 value: 2 --> 3
\end{code}

4つ以上の引数を持ったブロックには、\mthind{BlockClosure}{valueWithArguments:}を使い、引数の配列を渡します。
(多くの引数が必要なブロックは設計に問題があるのかもしれません。)

メソッドのローカル変数宣言のように、ブロックにローカル変数(縦線に囲まれる)を宣言することができます。
ローカル変数は引数の後に宣言します。
\index{variable!declaration}

\begin{code}{@TEST}
[ :x :y | | z | z := x+ y. z ] value: 1 value: 2 --> 3
\end{code}

ブロックは外側環境の変数を参照することができるので、レキシカル(静的)\emph{クロージャー}です。
次のブロックは、環境を包む変数\ct{x}を指します:

\begin{code}{@TEST}
| x |
x := 1.
[ :y | x + y ] value: 2 --> 3
\end{code}

ブロックはクラス\clsind{BlockClosure}のインスタンスです。
つまり、ブロックはオブジェクトです。したがって、変数に代入することができ、他のオブジェクトと同じように引数として渡すことができます。
% For both understandability and performance, it is better for blocks to refer only to their parameters and local variables; blocks that do not refer external variables are optimized by the compiler.
% MARCUS sez: I would just delete the sentence. There is nothing optimized, accessign outer temps is as fast as inner, so the only reason to avoid accessing outer temps would be that the code is easier to understand. But that's a relatively weak argument, I think.
% However, the ability to refer (``capture'') non-local variables can be very powerful when it is needed. 

%\paragraph{Really important.} \^\ acts as an escaping mechanism. 
%Return expressions inside a nested block expression will terminate the enclosing method.
%In the example 

%\begin{script}[detect]{...} when the expression \ct{^\ x@y} is executed, the method \ct{detect:}
% escapes the current iteration and returns it. 

%TwoLevelSet>>detect: aBlock

%   firstLevel keysAndValuesDo: [ :x :v |
%      v do: [ :y | (aBlock value: x@y) ifTrue: [^x@y]]
%   ].
%   ^nil
%\end{script}


%=================================================================
\section{Conditionals and loops in a nutshell}

\st は、制御のために特別のシンタックスを使うことはしません。
代わりに、論理値、数およびコレクションに引数としてのブロックと共にメッセージを送ることにより表現します。

条件は、ブール式の結果にメッセージ\mthind{Boolean}{ifTrue:}、\mthind{Boolean}{ifFalse:}あるいは\mthind{Boolean}{ifTrue:ifFalse:}のうちの1つを送ることにより表わします。論理値に関する詳細については、\charef{basic}を参照してください。

\begin{code}{}
(17 * 13 > 220)
   ifTrue: [ 'bigger' ]
   ifFalse: [ 'smaller' ] --> 'bigger'
\end{code}
% ON: Not a test.
% My regex approach cannot handle multi-line expressions :-(

ループは、ブロック、整数あるいはコレクションにメッセージを送ることにより表現します。
ループの終了条件は繰り返し評価されるかもしれないので、ブール値ではなくブロックです。
ここに、非常に手続き的なループの一例があげます:
\index{iteration}
\index{iteration|seealso{Collection, iteration}}
\seeindex{loops}{iteration}
\seeindex{enumeration}{iteration}
\seeindex{control constructs}{iteration}

\begin{code}{@TEST | n |}
n := 1.
[ n < 1000 ] whileTrue: [ n := n*2 ].
n --> 1024
\end{code}
\cmindex{BlockClosure}{whileTrue:}

\noindent
\mthind{BlockClosure}{whileFalse:}は終了条件を逆にします。
\begin{code}{@TEST | n |}
n := 1.
[ n > 1000 ] whileFalse: [ n := n*2 ].
n --> 1024
\end{code}

\noindent
\mthind{Integere}{timesRepeat:}は、固定回数の繰り返しを実装する単純な方法です。

\begin{code}{@TEST | n |}
n := 1.
10 timesRepeat: [ n := n*2 ].
n --> 1024
\end{code}

ループ・カウンターの初期値として作用する数にメッセージ\mthind{Number}{to:do:}を送ることができます。
2つの引数は、ループ・カウンターの上限と、引数としてループ・カウンターの現在値をとるブロックです。

\needlines{4}
\begin{code}{@TEST | result |}
result := String new.
1 to: 10 do: [:n | result := result, n printString, ' '].
result --> '1 2 3 4 5 6 7 8 9 10 '
\end{code}

\damien{I think the previous example that I've just added is clearer than the one which is commented here.}
% \begin{code}{@TEST | n |}
% n := 0.
% 1 to: 10 do: [ :counter | n := n + counter ].
% n --> 55
% \end{code}

\paragraph{High-Order Iterators.}
コレクションには多くの様々なクラスがあります。コレクションの多くは同じプロトコルをサポートします。
コレクションをイテレートするための重要メッセージは
\mthind{Collection}{do:}, \mthind{Collection}{collect:}, \mthind{Collection}{select:}, \mthind{Collection}{reject:}, \mthind{collection}{detect:}および\mthind{Collection}{inject:into:}です。
これらのメッセージは、コンパクトなコードを書くことを可能にする、ハイ・レベルのイテレーターを定義します。

\clsind{Interval}は数列の始点から終点をイテレートするコレクションです。
\ct{1 to: 10}は1から10までの区間を表わします。
これはコレクションなので、メッセージ\ct{do:}を送ることができます。
引数はコレクションの各要素ごとに評価されるブロックです。

\begin{code}{@TEST | result |}
result := String new.
(1 to: 10) do: [:n | result := result, n printString, ' '].
result --> '1 2 3 4 5 6 7 8 9 10 '
\end{code}

\damien{Again, I think the previous example is clearer than the one which is commented here.}
% \begin{code}{@TEST | n |}
% n := 0.
% (1 to: 10) do: [ :element | n := n + element ].
% n --> 55
% \end{code}

\ct{collect:}は各要素を変換して、同じサイズの新しいコレクションを作ります。
\begin{code}{@TEST}
(1 to: 10) collect: [ :each | each * each ] --> #(1 4 9 16 25 36 49 64 81 100)
\end{code}

\ct{select:}と\ct{reject:}はブロックで表した条件を満たす(あるいは満たさない)要素の集合からなる新しいコレクションを作ります。
\ct{detect:}は、条件を満たす最初の要素を返します。
さらに、文字列がコレクションであることを忘れないでください。したがって、文字をイテレートすことができます。

\begin{code}{@TEST}
'hello there' select: [ :char | char isVowel ] --> 'eoee'
'hello there' reject: [ :char | char isVowel ] --> 'hll thr'
'hello there' detect: [ :char | char isVowel ] --> $e
\end{code}

コレクションは関数型の畳み込み演算\emph{fold}を\ct{inject:into:}メソッドによりサポートします。
これは、シード値で始まり、コレクションの各要素を反映する式を使って、累積的な結果を生成します。
代表例は和と積です。
\seeindex{fold}{\ct{Collection>>>inject:into}}

\begin{code}{@TEST}
(1 to: 10) inject: 0 into: [ :sum :each | sum + each ] --> 55
\end{code}

\noindent
これは\ct{0+1+2+3+4+5+6+7+8+9+10}と同等です。

コレクションに関して\charef{collections}でさらに詳しく扱います。

%=================================================================
\section{Primitives and pragmas}

\st においては、すべてはオブジェクトです。そして、あらゆることがメッセージを送ることにより発生します。
しかし、ある地点で底に当たります。
いくつかのオブジェクトは\ind{バーチャルマシン}・\ind{プリミティブ}を呼び出すことが必要になります。

例えば、下記はすべてプリミティブとして実装されています。
メモリアロケーション(new, new:)、
ビット操作(bitAnd:, bitOr:, bitShift:)、
ポインターおよび整数演算(+, -, <, >, *, / , =, ==...)、
および配列へのアクセスは、(at: , at:put:)。
\seeindex{new@{\ct{new}}}{\ct{Behavior>>>new}}

プリミティブはシンタックス\ct{<primitive: aNumber>} で呼び出します。
プリミティブを呼び出すメソッドはさらに\st コードを含んでいることがあります、そのようなコードはプリミティブが失敗した場合のみ評価されます。

\cmind{SmallInteger}{+}のコードを例にしましょう。
プリミティブが失敗した場合は、式\ct{super + aNumber}が評価され、値が返されるのです。

\needlines{6}
\begin{method}[primitive]{A primitive method}
+ aNumber 
  "Primitive. Add the receiver to the argument and answer with the result
  if it is a SmallInteger. Fail if the argument or the result is not a
  SmallInteger  Essential  No Lookup. See Object documentation whatIsAPrimitive."

  <primitive: 1>
  ^ super + aNumber
\end{method}

%The other use of primitives is to optimize some crucial methods. The idea is that the system could work 
%without the primitive but it would be slow. The following method shows that the method \ct{@} is calling the primitive 18. Here the point creation is clearly expressible in \st therefore the code after the primitive is just the creation of a point illustrating what the primitive is actually doing. Note that such a code will be never called except if the primitive would failed which is extremely rare.  

%\begin{method}[xxx]{xxx}
%Integer>>@ y 
%   "Primitive. Answer a Point whose x value is the receiver and whose y 
%   value is the argument. Optional. No Lookup. See Object documentation 
%   whatIsAPrimitive."

%   <primitive: 18>
%   ^Point x: self y: y
%\end{method}


\pharo では、角括弧シンタックスもプラグマと呼ばれるメソッド・アノテーションに使われます。
\sd{we should give an example}\ab{Please do!  Is don't know about these.}\damien{it's the third time we talk about pragmas without saying what they are and how to use them.}

%=================================================================
\section{Chapter summary}

\begin{itemize}

\item	\pharo には擬似変数と呼ばれる6つ(だけ)の予約識別子があります: \ct{true},\ct{false},\ct{nil},\ct{self},\ct{super},および\ct{thisContext}。

\item	5種類のリテラル・オブジェクトがあります。数(\ct{5}, \ct{2.5}, \ct{1.9e15}, \ct{2r111})、文字(\ct{\$a})、文字列(\ct{'hello'})、シンボル(\ct{#hello})および配列(\ct{#('hello' #hi)})

\item	文字列はシングルクォート(コメントはダブルクォート)で囲みます。
		文字列の内部のクォートを得るためには、それを重ねてください。

\item	文字列と異なり、シンボルは大域的に一意であることが保証されます。

\item	リテラル配列を定義するためには\ct{#(...)}を使います。
		動的配列を定義するためには\ct|{ ... }|を使ってください。
		\ct{#( 1 + 2 ) size -→ 3}、
		\ct|{ 1 + 2 } size -→ 1|
		の違いに注意してください

\item	3種類のメッセージがあります:
		\emph{単項} (\eg \ct{1 asString}, \ct{Array new})、
		\emph{二項} (\eg \ct{3 + 4}, \ct{'hi' , ' there'})、
		\emph{キーワード} (\eg \ct{'hi' at: 2 put: \$o})

\item	カスケードメッセージは、セミコロンによって分けられた同じターゲットへ送られるメッセージのシーケンスです.
\ct{OrderedCollection new add: #calvin; add: #hobbes; size --> 2}

\item	ローカル変数は縦線で宣言されます。
		代入には\ct{:=}を使います。
		\ct{|x| x:=1}

\item	式は場合により括弧でグループ化した、メッセージ送信、カスケード、代入により構成します。
		\emph{ステートメント}はピリオドによって分割した式です。

\item	ブロック・クロージャーは角括弧で囲まれた式です。
		ブロックは引数をとってもよく、一時変数を使うことができます。
		ブロックに適切な数の引数を伴ったvalueメッセージを送るまで、
		ブロック中の式は評価されません。\\
		\ct{[:x | x + 2] value: 4 --> 6}.

\item	制御用の専用のシンタックスはありません、条件付きでブロックを評価する単なるメッセージだけがあります。
 		\ct{(\st includes: Class) ifTrue: [ Transcript show: Class superclass ]}

\end{itemize}

%=================================================================
\ifx\wholebook\relax\else
\end{document}\fi
%=================================================================
%%% Local Variables:
%%% coding: utf-8
%%% mode: latex
%%% TeX-master: t
%%% TeX-PDF-mode: t
%%% ispell-local-dictionary: "english"
%%% End:
