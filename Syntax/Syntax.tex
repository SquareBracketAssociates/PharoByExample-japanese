% $Author$
% $Date$
% $Revision$
% $Id$

% HISTORY:
% 2006-10-24 - Stef started
% 2006-12-01 - Oscar edit
% 2006-12-02 - Andrew edit
% 2007-05-04 - Oscar first draft
% 2007-07-04 - Stef review

%=================================================================
\ifx\wholebook\relax\else
% --------------------------------------------
% Lulu:
	\documentclass[a4paper,10pt,twoside]{book}
	\usepackage[
		papersize={6.13in,9.21in},
		hmargin={.75in,.75in},
		vmargin={.75in,1in},
		ignoreheadfoot
	]{geometry}
	\input{../common.tex}
	\pagestyle{headings}
	\setboolean{lulu}{true}
% --------------------------------------------
% A4:
%	\documentclass[a4paper,11pt,twoside]{book}
%	\input{../common.tex}
%	\usepackage{a4wide}
% --------------------------------------------
    \graphicspath{{figures/} {../figures/}}
	\begin{document}
	\renewcommand{\nnbb}[2]{} % Disable editorial comments
	\sloppy
\fi
%=================================================================
\chapter{文法の要点}
\chalabel{syntax}

\sd{We should add pragmas.}
\on{Please do so.}

% \sd{It would be good to add link to the chapter where the reader can learn about conditional, exceptions and loops.}
% \on{There are links already.}

\pharo の文法は、他のモダンな \st の方言と同様 \st-80 にとてもよく似ています。
つまり、プログラムをピジン英語のように声に出して読めるように設計されています:

\begin{code}{}
(Smalltalk includes: Class) ifTrue: [ Transcript show: Class superclass ]
\end{code}

\noindent
\pharo の文法規則は必要最小限に抑えられています。
本質的には、\emph{メッセージを送信する} (\ie 式)ための文法しかありません。\damien{I don't think there is a syntax for declaring methods; in my opinion, this is just a feature of our browsers. That's why I'm commenting: and \emph{declaring methods}}
式は非常に少数の基本要素をから組み立てられます。
キーワードは 6 個しかありません。さらに、制御構造のための文法も新しいクラスを宣言するための文法もありません。
代わりにオブジェクトにメッセージを送信することによりほとんどのことが達成されます。
例えば if-then-else 制御構造の代わりに、\st は \clsind{Boolean} オブジェクトに \ct{ifTrue:} のようなメッセージを送信します。
新しい\mbox{(サブ)クラス}を作成するには、作成したいクラスのスーパークラスにメッセージを送信します。

%=================================================================
\section{文法要素}

式は以下の構成要素からなります:
(i) 6 個の予約済みのキーワードもしくは\emph{擬似変数}:
\pvind{self}、\pvind{super}、\pvind{nil}、\pvind{true}、\pvind{false}、および \pvind{thisContext}、
(ii) 数、文字、文字列、シンボルおよび配列を含む\emphind{リテラルオブジェクト}用定数式、
(iii) 変数宣言、
(iv) 代入、
(v) \ind{ブロック}クロージャー、および
(vi) メッセージ送信。
\seeindex{擬似変数}{変数, 擬似}

\begin{table}\centering
	\begin{tabular}{ll}
		\toprule
		文法 & 意味 \\
		\midrule
		\lct{startPoint}			&	変数名 \\
		\lct{Transcript}			&	グローバル変数名 \\
		\lct{self}				&	擬似変数 \\
		\midrule
		\lct{1}				 	&	10 進整数 \\
		\lct{2r101}				&	2 進整数 \\
		\lct{1.5}					&	浮動小数点数 \\
		\lct{2.4e7}				&	指数表記 \\
		\lct{\$a}					&	文字『a』 \\
		\lct{'Hello'}				&	文字列「Hello」 \\
		\lct{\#Hello}				&	シンボル \lct{\#Hello} \\
		\lct{\#(1 2 3)}			&	リテラル配列 \\
		\lct{\{1. 2. 1+2\}}		&	動的配列 \\
		\midrule
		\lct{"a comment"} 		&	コメント \\
		\midrule
		\lct{| x y |}				&	変数 \lct{x} と \lct{y} の宣言	\\
		\lct{x := 1}				&	\lct{x} に 1 を代入 \\
		\lct{[ x + y ]}			&	\lct{x+y} を評価するブロック \\
		\lct{<primitive: 1>}		&	バーチャルマシンプリミティブもしくはアノテーション \\
		\midrule
		\lct{3 factorial}			&	単項メッセージ \\
		\lct{3+4}					&	二項メッセージ \\
		\lct{2 raisedTo: 6 modulo: 10}		&	キーワードメッセージ \\
		\midrule
		\lct{$\uparrow$ true} 			&	値 true を返す \\
		\lct{Transcript show: 'hello'. Transcript cr }		&	式セパレータ (\lct{.})	\\
		\lct{Transcript show: 'hello'; cr}					&	メッセージのカスケード (\lct{;}) \\
		\bottomrule
	\end{tabular}
	\caption{\pharo の文法の要点\tablabel{syntax}}
\end{table}

\tabref{syntax} は様々な文法要素の例です。
\begin{description}
\item[ローカル変数] \ct{startPoint} は変数名もしくは識別子です。
		規約により、識別子は「\ind{キャメルケース}」 (\ie 最初の単語以外の先頭が大文字)の単語を連結したものです。
		インスタンス変数、メソッドとブロックのパラメータ、あるいは一時変数の先頭の文字は小文字でなければいけません。
		これにより変数の有効範囲がプライベートであることが読み手に伝わります。

\item[共有変数] 大文字で始まる識別子は\subind{変数}{グローバル}変数、クラス\subind{クラス}{変数}{}、\subind{変数}{プール}辞書もしくはクラス名です。
		\ct{Transcript} はグローバル変数であり、\ct{TranscriptStream} クラスのインスタンスです。
		\seeindex{グローバル変数}{変数, グローバル}
		\seeindex{プール辞書}{変数, プール}
		\seeindex{変数!クラス}{クラス, 変数}

\item[レシーバ。] \pvind{self} は、現在のメソッドを実行しているオブジェクトを参照するキーワードです。\self は「レシーバ」とも呼ばれます。なぜならこのオブジェクトは、普通はメソッドを実行するきっかけになったメッセージを受け取っているからです。
		\self は「\subind{変数}{疑似}変数」と呼ばれます。\self には代入できないからです。

\item[整数。] \ct{42} のような通常の 10 進整数に加えて、\pharo では基数表記も使えます。
	\ct{2r101} は基数 2 (\ie バイナリ)の \ct{101} であり、10 進数の 5 と同じです。
	\index{リテラル}
	\index{リテラル!数}

\item[浮動小数点数] は 10 を基数とした\ind{指数}で指定することができます: \mbox{\ct{2.4e7}} は $2.4 \times 10^7$ です。
 	\index{浮動小数点数}

\item[文字。] \subind{リテラル}{文字}リテラルにはドル記号を付けます: \ct{$a}\ignoredollar$ は『a』のリテラルです。
		非印字文字のインスタンスは、\ct{Character space} や \ct{Character tab} のように \clsind{Character} クラスに適切なメッセージ(文字の名前)を送信することにより得ることができます。
		
\item[文字列。] シングルクォートは\subind{リテラル}{文字列}を定義するために使います。
		クォートを含んだ文字列を表現したい場合は、\ct{'G''day'} のようにクォートを重ねます。

\item[シンボル] は文字の並びでできている点で文字列に似ています。
	しかし、文字列と異り、\subind{リテラル}{シンボル}はグローバルに一意であることが保証されます。
		\ct{#Hello} というシンボルオブジェクトは一つだけしか存在しません。しかし \ct{'Hello'} という値の文字列オブジェクトは複数あるかもしれません。
		\seeindex{\#@{\textsf{\#}}}{literal symbol}

\item[コンパイル時配列] はスペースで区切ったリテラルを \ct{#( )} で囲んで定義します。
		括弧内のすべての要素はコンパイル時に定数でなければなりません。
		例えば、\ct{#(27 (true false) abc)} は要素が 3 個の\subind{配列}{リテラル}\subind{リテラル}{配列}です: 整数 \ct{27}、真偽値が 2 個入ったコンパイル時配列、そしてシンボル \ct{#abc} が要素になります(これが \ct{#(27 #(true false) #abc)} と同じであることに注意してください)。

\item[実行時配列。] 中括弧 \ct|{ }|は実行時に(\subind{配列}{動的})配列を定義します。
		実行時配列の要素は、ピリオドで区切った式です。
		したがって、\ct|{ 1. 2. 1+2 }| は 1、 2、そして 1+2 を評価した結果、を要素とする配列を定義します
		(動的配列の中括弧表記は、\st の方言の中でも \pharo と \squeak に特有です!
		他の \st では、動的配列は明示的に組み立てなければなりません)。

\item[コメント] はダブルクォートで囲みます。
		\ct{"hello"} は文字列ではなく\ind{コメント}なので、\pharo コンパイラによって無視されます。
		コメントは複数行にまたがっても構いません。
		
\item[ローカル変数宣言。] ローカル変数を\subind{変数}{宣言}するには、メソッド(およびブロック)内で一つ以上のローカル変数を縦棒 \ct{| |} で囲みます。
		% \seeindex{\|@{\textsf{\|\|}}}{assignment}
		% Can't seem to index or-bars! (special char for index macro)
		\seeindex{宣言}{変数宣言}

\item[代入。] \ct{:=} は変数にオブジェクトを代入します。
%		Sometimes you will see $\leftarrow$ used instead.
%		Unfortunately, since this is not an \textsc{ascii} character, it will appear as an underscore unless
%		you are using a special font.
%		So, \ct{x := 1} is the same as \ct{x _ 1} or \ct{x UNDERSCORE 1}. You should use \ct{:=} since the other representations have been deprecated.
		\index{代入}
		\seeindex{:=@{\textsf{:=}}}{代入}
		\seeindex{\_@{\textsf{\_}}}{代入}
		\seeindex{<-@{$\leftarrow$}}{代入}

\item[ブロック。] 角括弧 \ct{[ ]} はブロッククロージャーあるいは静的クロージャーとも呼ばれる\ind{ブロック}を定義します。ブロックは関数を表す第一級のオブジェクトです。
		後で見るように、ブロックは引き数を取ることもローカル変数を持つこともできます。
	\seeindex{[ ]@{\textsf{[ ]}}}{ブロック}
	\seeindex{クロージャー}{ブロック}
	\seeindex{静的クロージャー}{ブロック}

\item[プリミティブ。] \ct{<primitive: ...>}は\ind{バーチャルマシン}\ind{プリミティブ}の呼び出しを表します
	(\ct{<primitive: 1>} は \ct{SmallInteger>>>+} のバーチャルマシンプリミティブです)。
	プリミティブの後ろにあるコードは、プリミティブが失敗した場合にだけ実行されます。
	同じ文法はメソッドアノテーションにも使われます。

\item[単項メッセージ] は単一の単語(例えば \ct{factorial})からなり、レシーバ(例えば \ct{3})に送信されます。
	\index{メッセージ!単項}
	\seeindex{単項メッセージ}{メッセージ, 単項}

\item[二項メッセージ] は演算子(例えば \ct{+})で、レシーバに単一の引数とともに送信されます。\ct{3+4}の場合、レシーバは \ct{3} で引数は \ct{4} です。
	\index{メッセージ!二項}
	\seeindex{二項メッセージ}{メッセージ, 二項}

\item[キーワードメッセージ] ではキーワードと引数を 1 個ずつ交互に並べます。キーワードはコロンで終わります。キーワードを連結した\emphind{メッセージセレクタ}と引数(キーワードの数だけ)が、メッセージとして送信されます。
式 \ct{2 raisedTo: 6 modulo: 10} では、メッセージセレクタは \ct{raisedTo:modulo:} で引数は \ct{6} と \ct{10} です。
	\index{メッセージ!キーワード}
	\seeindex{キーワードメッセージ}{メッセージ, キーワード}

\item[メソッドリターン。] \ct{^}はメソッドから値を\emphind{リターン}するために使われます(\ct{^} を入力するには \verb|^| とタイプしなければなりません)。
\md{\ct{^} always returns from the method, even if used in a block, it returns from the enclosing method.}

\item[文の並び。] ピリオド(\ct{.})は文の\emphsubind{文}{区切り}です。二つの式の間にピリオドを置くと、それらは独立した文として扱われます。
%	\seeindex{full stop}{statement separator}
	\seeindex{ピリオド}{文セパレータ}
	\seeindex{\ct{.}}{文セパレータ}

\item[カスケード。] セミコロンは、一つのレシーバにメッセージの\emphind{カスケード}(列)を送信するのに使うことができます。\ct{Transcript show: 'hello'; cr} では、最初に \ct{Transcript} にキーワードメッセージ \ct{show: 'hello'} を送信し、次に同じレシーバ(\ct{Transcript})に単項メッセージ \ct{cr} を送信します。
	\seeindex{;}{カスケード}

\end{description}

\ct{Number}、\ct{Character}、\ct{String} および \ct{Boolean} クラスについては \charef{basic} でより詳細に述べます。

\on{Blocks are described in \charef{blocks}. (Control flow and Iterators).}

%=================================================================
\section{擬似変数}

\st には次の 6 個の予約済みのキーワード、あるいは\emph{疑似変数}があります:
\pvind{nil}、\pvind{true}、\pvind{false}、\pvind{self}、\pvind{super}、および \pvind{thisContext}。
これらは事前に定義され代入することができないので、\subind{変数}{擬似}変数と呼ばれます。
\ct{true}、\ct{false}、および \ct{nil} は定数です。一方 \ct{self}、\ct{super}、および \ct{thisContext} の値は、コードの実行中に動的に変化します。

\ct{true} と \ct{false} は、\clsind{Boolean} クラスである \clsind{True} および \clsind{False} の唯一のインスタンスです。
より詳細については \charef{basic} を参照してください。

\pvind{self} は、常に現在実行しているメソッドのレシーバを指します。

\ct{super} も現在実行しているメソッドのレシーバを指します。しかし \super にメッセージを送信する場合、メソッド探索はそのメソッドを定義しているクラスのスーパークラスから始まります。
より詳細については \charef{model} を参照してください。

\ct{nil} は未定義のオブジェクトです。
\ct{nil} はクラス \clsind{UndefinedObject} の唯一のインスタンスです。
インスタンス変数、クラス変数およびローカル変数は、\ct{nil} に初期化されます。

\ct{thisContext} は、実行時スタックのトップフレームを表す擬似変数です。
言い換えれば、 \ct{thisContext} は現在実行している \clsind{MethodContext} あるいは \clsind{BlockClosure} を表します。
\ct{thisContext} は通常のプログラミングに必要ありませが、デバッガのような開発ツールを実装するには不可欠です。また、\ct{thisContext} は例外処理と継続を実装するためにも使われます。

%=================================================================
\section{メッセージ送信}

\pharo には 3 種類のメッセージ(メッセージ送信)があります。
\begin{enumerate}
  \item \emph{単項}メッセージは引数を取りません。
  \ct{1 factorial}は、オブジェクト \ct{1} にメッセージ \ct{factorial} を送信します。
  \item \emph{二項}メッセージは引き数を 1 個だけ取ります。
  	\ct{1 + 2} は、オブジェクト \ct{1} にメッセージ \ct{+} を送信します。引数は \ct{2} です。
  \item \emph{キーワード}メッセージは 1 個以上任意個の引数を取ります。
  	\ct{2 raisedTo: 6 modulo: 10} は、オブジェクト \ct{2} にメッセージセレクタ
	\ct{raisedTo:modulo:}、引数 \ct{6}、引数 \ct{10} からなるメッセージを送信します。
\end{enumerate}

単項メッセージのメッセージセレクタは、英小文字から始まる英数字で記述します。
\index{メッセージ!単項}

二項メッセージのメッセージセレクタは、次の文字セットを使って 1 文字以上で記述します:
\index{メッセージ!二項}
\begin{code}{}
+ - / \ * ~ < > = @ % | & ! ? ,
\end{code}
\noindent
% [\~\!\@\%\&\*\-\+\=\\\|\?\/\>\<\,]
\on{It seems that 3 or more chars work fine, but it is not possible to have more than one ``-'' in a binary selector. Perhaps due to a conflict with parsing negative numbers?}
\ab{That's right; $-$ is weird.}
キーワードメッセージのメッセージセレクタは、一連のキーワードを連結したものです。キーワードはそれぞれ英小文字で始まりコロンで終わる英数字で記述します。
\index{メッセージ!キーワード}

単項メッセージの優先順位が最も高くなります。次に二項メッセージ、最後にキーワードメッセージの順になります。つまり:
\begin{code}{@TEST}
2 raisedTo: 1 + 3 factorial --> 128
\end{code}
となります(最初に \ct{3} に \ct{factorial} を送信します。次に \ct{1} に \ct{+ 6} を送信します。最後に \ct{2} に \ct{raisedTo: 7} を送信します)。
式を評価した結果を示すのに\lct{\emph{式}}\ct{-->}\lct{\emph{結果}}という表記を使うことを思い出してください。

優先度を別にすると、評価順は厳密に左から右です。つまり、
\begin{code}{@TEST}
1 + 2 * 3 --> 9
\end{code}
となります。\ct{7} ではありません。
評価順を変更するためには括弧を使わなければなりません:
\begin{code}{@TEST}
1 + (2 * 3) --> 7
\end{code}

メッセージ送信はピリオドとセミコロンを使って組み立てられることもあります。ピリオドで区切られた式の並びは、文として順々に評価されます。
\index{文!セパレータ}

\begin{code}{}
Transcript cr.
Transcript show: 'hello world'.
Transcript cr
\end{code}

\noindent
これは、\glbind{Transcript} オブジェクトに \ct{cr} を送信し、次に \ct{Transcript} に \ct{show: 'hello world'} を送信し、最後に \ct{Transcript} にもう一つの \ct{cr} を送信します。

一連のメッセージを\emph{同じ}レシーバに送信する場合、\emphind{カスケード}としてより簡潔に表現することができます。
レシーバを一度だけ指定し、メッセージの並びをセミコロンで区切ります:

\begin{code}{}
Transcript cr;
    show: 'hello world';
    cr
\end{code}
これは前の例とまったく同じ意味になります。

%=================================================================
\section{メソッド文法}

式は \pharo のあらゆる場所(例えばデバッガ、ワークスペース、あるいはブラウザ)で評価することができますが、通常、メソッドはブラウザウィンドウかデバッガで定義します
(メソッドは外部メディアからファイルインすることもできますが、これは \pharo でプログラムする普通の方法ではありません)。

プログラムは、クラスのコンテキストの中で一度に 1 メソッドずつ開発していきます
(クラスは、既存のクラスにサブクラスを作成するようにメッセージを送信して定義します。したがってクラスを定義する特別の文法は必要ありません)。

以下は \clsind{String} クラスの \mthind{String}{lineCount} メソッドです
(習慣により、メソッドを \ct{ClassName>>>methodName} と表します。したがってこのメソッドは \ct{String>>>lineCount} と表します)。

\needlines{9}
\begin{method}[lineCount]{Line count}
String>>>lineCount
   "レシーバの行数を答える。
   crに出会うたび 1 行増やす。"
   | cr count |
   cr := Character cr.
   count := 1 min: self size.
   self do:
      [:c | c == cr ifTrue: [count := count + 1]].
   ^ count
\end{method}

文法上、メソッドは以下のものからなります:
\begin{enumerate}
  \item 名前 (\ie \ct{lineCount}) と引数(あれば。この例では一つも無い)からなるメソッドのパターン;
  \item コメント(どこに書いても構いませんが、メソッドが何を行うかの説明を一番上に書く習慣があります);
  \item ローカル変数 (\ie \ct{cr} と \ct{count}) の宣言; そして
  \item ドットで区切った任意個の式; ここでは式は 4 個。
\end{enumerate}

\ct{^}(\verb|^|とタイプした)が先行した任意の式は、評価するとその時点でメソッドを終了してその式の値を返します。
明示的に値を返さずに終了するメソッドは、暗黙的に \pvind{self} を \ind{リターン}します。
\index{リターン!暗黙}

パラメータとローカル変数は常に英小文字で始めなければなりません。
英大文字で始まる名前はグローバル変数とみなされます。
例えば \ct{Character} のようなクラス名は、単にそのクラスを表すオブジェクトを参照するグローバル変数です。

%=================================================================
\section{ブロック文法}

\ind{ブロック}は、式の評価を遅らせるためのメカニズムを提供します。
ブロックは本質的には無名の関数です。ブロックを評価するには、これにメッセージ \mthind{BlockClosure}{value} を送信します。
ブロックは、明示的なリターン(\ct{^} で示す)が無ければその最後の式の値を返します。
\seeindex{value}{BlockClosure}

\begin{code}{@TEST}
[ 1 + 2 ] value --> 3
\end{code}

ブロックはパラメータを持つことができます。パラメータは先頭にコロンをつけて宣言します。
縦棒でブロックの本体とパラメータ宣言を区切ります。
パラメータが 1 個のブロックを評価するには、ブロックにメッセージ \mthind{BlockClosure}{value:} と引数 1 個を送信しなければなりません。
パラメータが 2 個のブロックを評価するには、ブロックに \mthind{BlockClosure}{value:value:}を送信しなければなりません。同様にして 4 個までの引数を渡すことができます。

\begin{code}{@TEST}
[ :x | 1 + x ] value: 2 --> 3
[ :x :y | x + y ] value: 1 value: 2 --> 3
\end{code}

パラメータの数が 4 個より多いブロックには \mthind{BlockClosure}{valueWithArguments:} を用い、引数を配列にして渡さなければなりません
(パラメータが多いブロックは、しばしば設計上の問題の兆候です)。

ブロック内でもローカル変数を宣言することができます。メソッドのローカル変数宣言とまったく同様、縦棒で囲んで宣言します。
ローカル変数はパラメータの後ろに宣言します。
\index{変数!宣言}

\begin{code}{@TEST}
[ :x :y | | z | z := x+ y. z ] value: 1 value: 2 --> 3
\end{code}

ブロックはそれを取り囲む環境の変数を参照することができるので、事実上静的\emph{クロージャー}です。
次のブロックは、それを取り囲む環境の変数 \ct{x} を参照します:

\begin{code}{@TEST}
| x |
x := 1.
[ :y | x + y ] value: 2 --> 3
\end{code}

ブロックはクラス \clsind{BlockClosure} のインスタンスです。
これは、ブロックはオブジェクトなので、他の普通のオブジェクトとまったく同様変数に代入したり引数として渡すことができる、という意味です。
% For both understandability and performance, it is better for blocks to refer only to their parameters and local variables; blocks that do not refer external variables are optimized by the compiler.
% MARCUS sez: I would just delete the sentence. There is nothing optimized, accessign outer temps is as fast as inner, so the only reason to avoid accessing outer temps would be that the code is easier to understand. But that's a relatively weak argument, I think.
% However, the ability to refer (``capture'') non-local variables can be very powerful when it is needed. 

%\paragraph{Really important.} \^\ acts as an escaping mechanism. 
%Return expressions inside a nested block expression will terminate the enclosing method.
%In the example 

%\begin{script}[detect]{...} when the expression \ct{^\ x@y} is executed, the method \ct{detect:}
% escapes the current iteration and returns it. 

%TwoLevelSet>>detect: aBlock

%   firstLevel keysAndValuesDo: [ :x :v |
%      v do: [ :y | (aBlock value: x@y) ifTrue: [^x@y]]
%   ].
%   ^nil
%\end{script}


%=================================================================
\section{条件式とループの要点}

\st には制御構造のための特別の文法はありません。
代わりに真偽値、数およびコレクションに、ブロックを引数とするメッセージを送信することで制御構造を表します。

条件式は、真偽値を返す式(の値)にメッセージ \mthind{Boolean}{ifTrue:}、\mthind{Boolean}{ifFalse:} あるいは \mthind{Boolean}{ifTrue:ifFalse:} を送信することにより表します。真偽値に関する詳細については \charef{basic}を参照してください。

\begin{code}{}
(17 * 13 > 220)
   ifTrue: [ 'bigger' ]
   ifFalse: [ 'smaller' ] --> 'bigger'
\end{code}
% ON: Not a test.
% My regex approach cannot handle multi-line expressions :-(

ループは、ブロック、整数あるいはコレクションにメッセージを送信することで表します。
ループの終了条件は、繰り返し評価されるかもしれないので、真偽値そのものではなく真偽値を返すブロックでなければなりません。
以下は非常に手続き的なループの例です:
\index{繰り返し}
\index{繰り返し|seealso{Collection, iteration}}
\seeindex{ループ}{繰り返し}
\seeindex{列挙}{繰り返し}
\seeindex{制御構造}{繰り返し}

\begin{code}{@TEST | n |}
n := 1.
[ n < 1000 ] whileTrue: [ n := n*2 ].
n --> 1024
\end{code}
\cmindex{BlockClosure}{whileTrue:}

\noindent
\mthind{BlockClosure}{whileFalse:} は終了条件を逆にします。
\begin{code}{@TEST | n |}
n := 1.
[ n > 1000 ] whileFalse: [ n := n*2 ].
n --> 1024
\end{code}

\noindent
\mthind{Integere}{timesRepeat:} により単純な固定回数の繰り返しが実装できます。

\begin{code}{@TEST | n |}
n := 1.
10 timesRepeat: [ n := n*2 ].
n --> 1024
\end{code}

数にメッセージ \mthind{Number}{to:do:} を送信することができます。するとその数(レシーバ)がループカウンタの初期値になります。
\ct{to:} の引数はループカウンタの上限です。また \ct{do:} の引数はブロックで、これ自体ループカウンタの現在の値を引数として取ります。

\needlines{4}
\begin{code}{@TEST | result |}
result := String new.
1 to: 10 do: [:n | result := result, n printString, ' '].
result --> '1 2 3 4 5 6 7 8 9 10 '
\end{code}

\damien{I think the previous example that I've just added is clearer than the one which is commented here.}
% \begin{code}{@TEST | n |}
% n := 0.
% 1 to: 10 do: [ :counter | n := n + counter ].
% n --> 55
% \end{code}

\paragraph{高階イテレータ。}
コレクションには多くの異ったクラスがあります。それらの多くは同じプロトコルをサポートします。
コレクションの要素を繰り返し処理するための最も重要メッセージには、
\mthind{Collection}{do:}、\mthind{Collection}{collect:}、\mthind{Collection}{select:}、\mthind{Collection}{reject:}、\mthind{collection}{detect:} および \mthind{Collection}{inject:into:} があります。
これらのメッセージは高レベルのイテレータを定義しており、それによって非常にコンパクトなコードを書くことができます。

\clsind{Interval} は、数の並びを始点から終点まで繰り返し処理するためのコレクションです。
\ct{1 to: 10}は 1 から 10 までの \ct{Interval} を表します。
\ct{Interval} はコレクションなので、これにメッセージ \ct{do:} を送信することができます。
\ct{do:} の引数はブロックで、ブロックのパラメータがコレクションの要素にバインドされ繰り返し評価されます。

\begin{code}{@TEST | result |}
result := String new.
(1 to: 10) do: [:n | result := result, n printString, ' '].
result --> '1 2 3 4 5 6 7 8 9 10 '
\end{code}

\damien{Again, I think the previous example is clearer than the one which is commented here.}
% \begin{code}{@TEST | n |}
% n := 0.
% (1 to: 10) do: [ :element | n := n + element ].
% n --> 55
% \end{code}

\ct{collect:} は各要素を変換して同じサイズの新しいコレクションを作ります。
\begin{code}{@TEST}
(1 to: 10) collect: [ :each | each * each ] --> #(1 4 9 16 25 36 49 64 81 100)
\end{code}

\ct{select:} と \ct{reject:} は、真偽値を返すブロックを満たす(あるいは満たさない)要素からなる新しいコレクション(部分集合)を作ります。
\ct{detect:} は条件を満たす最初の要素を返します。
文字列が文字のコレクションであることを覚えておいてください。つまり文字列内の文字は繰り返し処理することができます。

\begin{code}{@TEST}
'hello there' select: [ :char | char isVowel ] --> 'eoee'
'hello there' reject: [ :char | char isVowel ] --> 'hll thr'
'hello there' detect: [ :char | char isVowel ] --> $e
\end{code}

最後に、コレクションが \ct{inject:into:} メソッドにより関数型の(高階の)\emph{畳み込み}演算もサポートすることに留意してください。
これを使うと、シード値で始まる式にコレクションの各要素を注入 (inject) して累積的な値を得ることができます。
和や積の累積は \ct{inject:into:} の典型的な応用例です。
\seeindex{fold}{\ct{Collection>>>inject:into}}

\begin{code}{@TEST}
(1 to: 10) inject: 0 into: [ :sum :each | sum + each ] --> 55
\end{code}

\noindent
これは \ct{0+1+2+3+4+5+6+7+8+9+10}と等価です。

コレクションに関しては \charef{collections} でより詳しく述べます。

%=================================================================
\section{プリミティブとプラグマ}

\st ではすべてがオブジェクトです。そしてあらゆることがメッセージ送信により起こります。
しかしある地点で底に当たります。
ある種のオブジェクトは\ind{バーチャルマシン}\ind{プリミティブ}を呼ばないと動作できません。

例えば、次のメソッドはすべてプリミティブとして実装されています。
メモリ割り当て (\mthind{Behavior}{new}、\mthind{Behavior}{new:})、
ビット操作 (\mthind{Integer}{bitAnd:}、\mthind{Integer}{bitOr:}、\mthind{Integer}{bitShift:})、
ポインターおよび整数演算 (\ct{+}、\ct{-}、\ct{<}、\ct{>}、\ct{*}、\ct{/ }、\ct{=}、\ct{==}...)、
および配列のアクセス (\mthind{Object}{at:}、\mthind{Object}{at:put:})。
\seeindex{new@{\ct{new}}}{\ct{Behavior>>>new}}

プリミティブは、文法 \ct{<primitive: aNumber>} で呼び出します。
プリミティブを呼び出すメソッドはさらに \st コードを含んでいることがあります。そのようなコードは、プリミティブが失敗した場合に\emph{だけ}評価されます。

\cmind{SmallInteger}{+}のコードを見てみましょう。
プリミティブが失敗した場合、式 \ct{super + aNumber} が評価され値が返されます。

\needlines{6}
\begin{method}[primitive]{プリミティブメソッド}
+ aNumber 
  "プリミティブ。レシーバを引数に加え、結果が SmallInteger ならそれを返す。
  引数か結果が SmallInteger でなければ失敗する。
  必須。探索無し。Object のドキュメント whatIsAPrimitive を参照のこと。"

  <primitive: 1>
  ^ super + aNumber
\end{method}

%The other use of primitives is to optimize some crucial methods. The idea is that the system could work 
%without the primitive but it would be slow. The following method shows that the method \ct{@} is calling the primitive 18. Here the point creation is clearly expressible in \st therefore the code after the primitive is just the creation of a point illustrating what the primitive is actually doing. Note that such a code will be never called except if the primitive would failed which is extremely rare.  

%\begin{method}[xxx]{xxx}
%Integer>>@ y 
%   "Primitive. Answer a Point whose x value is the receiver and whose y 
%   value is the argument. Optional. No Lookup. See Object documentation 
%   whatIsAPrimitive."

%   <primitive: 18>
%   ^Point x: self y: y
%\end{method}


\pharo では、山括弧文法はメソッドアノテーションにも使われます。これをプラグマと呼びます。
\sd{we should give an example}\ab{Please do!  Is don't know about these.}\damien{it's the third time we talk about pragmas without saying what they are and how to use them.}

%=================================================================
\section{まとめ}

\begin{itemize}

\item	\pharo には予約語が 6 個(だけ)あります。これらは\textit{擬似変数}とも呼ばれます: \ct{true}、\ct{false}、\ct{nil}、\ct{self}、\ct{super}、および \ct{thisContext}。

\item	5 種類のリテラルオブジェクトがあります: 数(\ct{5}、\ct{2.5}、\ct{1.9e15}、\ct{2r111})、文字 (\ct{$a})、文字列 (\ct{'hello'})、シンボル (\ct{#hello})、および配列 (\ct{#('hello' #hi)})。

\item	文字列はシングルクォートで、コメントはダブルクォートで囲みます。
		文字列内にクォートを書くには、クォートを重ねます。

\item	文字列と異なり、シンボルはグローバルに一意であることが保証されます。

\item	リテラル配列を定義するには \ct{#( ... )} を使います。
		動的配列を定義するには \ct|{ ... }| を使います。
		\ct{#( 1 + 2 ) size --> 3} ですが、
		\ct|{ 1 + 2 } size --> 1|
		なので注意してください。

\item	3 種類のメッセージがあります:
		\emph{単項} (\eg \ct{1 asString}、\ct{Array new})、
		\emph{二項} (\eg \ct{3 + 4}、\ct{'hi' , ' there'})、および
		\emph{キーワード} (\eg \ct{'hi' at: 2 put: $o})。

\item	\emph{カスケード}メッセージ送信では、同じターゲットにメッセージの並びを送信します。各メッセージはセミコロンで区切ります:
\ct{OrderedCollection new add: #calvin; add: #hobbes; size --> 2}。

\item	ローカル変数は縦棒で囲んで宣言します。
		代入には \ct{:=} を使います。
		\ct{|x| x:=1}。

\item	式は、メッセージ送信、カスケードおよび代入からなります。これらは括弧でグループ化されることもあります。
		\emph{文}はピリオドによって区切られた式です。

\item	ブロッククロージャーは角括弧で囲まれた式です。
		ブロックは引数を取ってもよく、一時変数を使うこともできます。
		ブロック内の式は、ブロックに \ct{value...} メッセージと正しい個数の引数を送信するまで
		評価されません。\\
		\ct{[:x | x + 2] value: 4 --> 6}。

\item	制御構造専用の文法はありません。条件付きでブロックを評価するメッセージがあるだけです。\\
 		\ct{(\st includes: Class) ifTrue: [ Transcript show: Class superclass ]}。

\end{itemize}

%=================================================================
\ifx\wholebook\relax\else
\end{document}\fi
%=================================================================
%%% Local Variables:
%%% coding: utf-8
%%% mode: latex
%%% TeX-master: t
%%% TeX-PDF-mode: t
%%% ispell-local-dictionary: "english"
%%% End:
