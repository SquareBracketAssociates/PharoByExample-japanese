% $Author$
% $Date$
% $Revision$

% HISTORY: [see also Metaprogramming2.tex]
% 2007-05-22 - Damien Pollet started (translation from French article by ...?)
% 2008-01-15 - Alex added text
% 2008-12-15 - Oscar revised
% 2009-03-24 - Stef started new chapter (acttalk ... see separate file)
% 2009-06-01 - Oscar started to revise again and add new material
% 2009-06-08 - Lukas -- unsent messages
% 2009-06-15 - Oscar completed revision
% 2009-06-16 _ Stef comments
% 2009-06-17 - Alexandre completed revision
% 2009-06-19 - Lukas comments
% 2009-07-07 - Oscar migrated to Pharo; fixed broken tests
% 2009-08-16 - Oscar indexing and cleaning up loose ends

%=================================================================
\ifx\wholebook\relax\else
% --------------------------------------------
% Lulu:
	\documentclass[a4paper,10pt,twoside]{book}
	\usepackage[
		papersize={6.13in,9.21in},
		hmargin={.75in,.75in},
		vmargin={.75in,1in},
		ignoreheadfoot
	]{geometry}
	\input{../common.tex}
	\pagestyle{headings}
	\setboolean{lulu}{true}
% --------------------------------------------
% A4:
%	\documentclass[a4paper,11pt,twoside]{book}
%	\input{../common.tex}
%	\usepackage{a4wide}
% --------------------------------------------
    \graphicspath{{figures/} {../figures/}}
	\begin{document}
	\renewcommand{\nnbb}[2]{} % Disable editorial comments
	\sloppy
\fi

%=================================================================
\chapter{リフレクション}\chalabel{reflection}

%\alex{I addressed lukas comments using editorial macros. Please, if you read this, remove this comment and remove the editorial macros, else their purpose is diminished}

\indexmain{リフレクション}
\st はリフレクティブなプログラミング言語です。
つまり、プログラムは自分自身の実行や構造を「(ランタイムシステムに)反映する(リフレクトする)」ことができます。
% \lr{not only on execution, also on the static model}
ランタイムシステムの\emphind{メタオブジェクト}を通常のオブジェクトとして\emph{具現化(reify))}することができるので、メタオブジェクトに問い合わせを行ったり、インスペクトして内容を見たりすることができます。
\st のメタオブジェクトには、クラス、メタクラス、メソッド辞書、コンパイル済みメソッド、実行時スタックなどがあります。
このようなリフレクションは\emphind{イントロスペクション}とも呼ばれ、多くのモダンなプログラミング言語でサポートされています。

\begin{figure}[ht]\centering
	\includegraphics[width=\linewidth]{reflect}
	\caption{具現化とリフレクション。\figlabel{reflect}} % \lr{not referenced, not sure if I understand it}
\end{figure}

\st では逆の操作も可能で、具現化されたメタオブジェクトを変更して、ランタイムシステムに\emph{反映}させることもできます(\figref{reflect}参照)。
このようなリフレクションは\emph{インターセッション}とも呼ばれ、主に動的なプログラミング言語でサポートされていますが、静的な言語では極めて限られています。

他のプログラム(または自分自身)を操作するようなプログラムを\emphind{メタプログラム}と呼びます。
レフレクティブなプログラミング言語では、\ind{イントロスペクション}と\ind{インターセッション}がどちらもサポートされているべきです。
イントロスペクションとは、言語自身を定義するデータ構造(オブジェクト、クラス、メソッド、実行時スタックなど)を\emph{検査}する機能です。
インターセッションとは、そのようなデータ構造を\emph{変更}できること、言い換えれば、言語の意味論やプログラムの振る舞いを、プログラム自身から変更できる機能です。
ランタイムシステムの構造を検査・変更することを\emph{構造的リフレクション}と呼び、それらの構造の解釈を変更することを\emphind{振る舞い的リフレクション}と呼びます。

この章では、主にind{構造的リフレクション}を扱います。
実用的な例を多数挙げながら、 \st がどのようにイントロスペクションとメタプログラミングをサポートしているのかを探っていきます。

%======================================
\section{イントロスペクション}

インスペクタを使うと、オブジェクトの内容を見たり、インスタンス変数の値を変更したり、オブジェクトにメッセージを送ったりすることができます。

\dothis{次のコードをワークスペースで評価してみましょう:}
\begin{code}{| w |}
w := Workspace new.
w openLabel: 'My Workspace'.
w inspect
\end{code}

このコードを評価すると、インスペクタとワークスペースがもう一つ開きます。
このインスペクタは新しく開いたワークスペースの内部状態を示しており、左ペインにはインスタンス変数(\ct!dependents!, \ct!contents!, \ct!bindings! など)、右ペインには選択したインスタンス変数の値が表示されます。
\ct!contents!インスタンス変数は、ワークスペースのテキストエリアに表示されている内容を表しますので、この時点で\ct!contents!を選択すると、インスペクタの右ペインには空文字列が表示されます。

\begin{figure}[ht]\centering
	\includegraphics[width=\linewidth]{workspaceInspector}
	\caption{\lct!ワークスペース!をインスペクトしている様子。\figlabel{workspaceInspector}}
\end{figure}

\dothis{この空文字列のところに\ct!'hello'!と入力して、\emph{accept}してみましょう。}
これで\ct!contents!変数の値が変りますが、まだワークスペースウィンドウは変化に気付かないので再描画されません。
ウィンドウを更新させるには、インスペクタの下ペインで\ct!self contentsChanged!を評価します。

%-----------------------------------------------------------------
\subsection{インスタンス変数へのアクセス}

インスペクタはどのように動いているのでしょうか?
\st では、インスタンス変数はすべて保護されています。
理論上は、アクセサメソッドがそのクラスに定義されていない限り、他のオブジェクトからインスタンス変数にアクセスすることはできません。
しかし、実際にはインスペクタはアクセサがなくてもインスタンス変数にアクセスできます。これは\st がリフレクションの機能を持っているからです。
\st では、クラスはインスタンス変数を名前または数値の添字で定義します。
インスペクタは、\ct!Object!クラスで定義されているメソッド \lct{instVarAt: \emph{index}} または \lct{instVarNamed: \emph{aString}} を使ってインスタンス変数にアクセスします。数値の添字 \lct{\emph{index}} で位置を指定するか、インスタンス変数名 \lct{\emph{aString}} を指定します。インスタンス変数に新しい値を代入するには、\ct!instVarAt:put:! か \ct!instVarNamed:put:! を使います。
\mthindex{Object}{instVarAt:}
\mthindex{Object}{instVarNamed:}
\mthindex{Object}{instVarAt:put:}
\mthindex{Object}{instVarNamed:put:}

例えば次のコードを評価すると、最初のワークスペースで変数 \ct!w! に束縛されていたオブジェクトの内部状態を変更できます。

\begin{code}{}
w instVarNamed: 'contents' put: 'howdy!'; contentsChanged
\end{code}

\important{\emph{注意:} これらのメソッドは開発ツールを作る分には便利ですが、通常のアプリケーション記述には使わない方が良いでしょう。リフレクティブなメソッドはオブジェクトのカプセル化境界をまたいでしまうので、理解も保守も難しいコードの原因となってしまいます。}
% \lr{Why? The access does not show up when looking for all readers/writers in the code browser.}

\ct!instVarAt:!と\ct!instVarAt:put:!はどちらも\ind{プリミティブメソッド}です。プリミティブメソッドは、バーチャルマシンの原始的操作として実装されています。
これらのメソッドのコードを調べれば、特殊な\ind{プラグマ}構文 \ct!<primitive: N>! (\ct!N!は整数)が使われていることがわかるでしょう。
% \lr{actually this is the syntax of pragmas (method annotations), \ct!primitive:! is just a special kind of pragma}

\needlines{5}
\begin{code}{}
Object>>>instVarAt: index 
	"プリミティブ。オブジェクト内の固定長配列で管理されているインスタンス変数を返す。"
	!\textbf{<primitive: 73>}!
	"固定長配列の範囲を超えてアクセスする。"
	^self basicAt: index - self class instSize		
\end{code}

通常はプリミティブ呼び出しの行以降のコードは実行されません。
その部分はプリミティブが失敗した場合にのみ実行されます。この例の場合では、存在しないインスタンス変数にアクセスしようとすると、プリミティブに続くコードがバックアップとして試みられます。
この仕組みにより、プリミティブメソッドが失敗したときにデバッガを起動することもできます。
プリミティブメソッドのコードは変更できますが、\pharo システムの安定性を脅かしかねないため注意してください。

\begin{figure}[ht]\centering
	\includegraphics[width=\linewidth]{allInstanceVariables}
	\caption{ある\ct!Workspace!オブジェクトのすべてのインスタンス変数を表示している様子。\figlabel{allInstanceVariables}}
\end{figure}

\figref{allInstanceVariables}は、\ct!Workspace!クラスのあるインスタンス(\ct!w!)のすべてのインスタンス変数を表示する方法を示しています。
\ct!allInstVarNames!メソッドは、与えられたクラスのすべてのインスタンス変数の名前を返します。

同様の方法で、同じ特徴を持つインスタンスを集めることもできます。
例えば\ct!SketchMorph!クラスのインスタンスのうち、\ct!owner!インスタンス変数の内容がワールドモーフであるもの(\ie 現在表示されているすべてのイメージ)を集めてみましょう。次の式を試してみてください。

\begin{code}{}
SketchMorph allInstances select: [:c | (c instVarNamed: 'owner') isWorldMorph]
\end{code}

%-----------------------------------------------------------------
\subsection{インスタンス変数に対する繰り返し処理}

\mthindex{Object}{instanceVariableValues}
\ct!instanceVariableValues!メッセージについて調べてみましょう。このメソッドは、継承したものを除くすべてのインスタンス変数の値をコレクションとして返します。
次に例を示します:
\begin{code}{@TEST}
(1@2) instanceVariableValues --> an OrderedCollection(1 2)
\end{code}

このメソッドは\ct{Object}クラスで次のように実装されています:
\needlines{9}
\begin{code}{}
Object>>>instanceVariableValues
	"レシーバのクラスで追加されたインスタンス変数の値を集めたコレクションを返す。"	
	| c |
	c := OrderedCollection new.
	self class superclass instSize + 1
		to: self class instSize
		do: [ :i | c add: (self instVarAt: i)].
	^ c
\end{code}

このメソッドは、スーパークラスで使われている最後のインスタンス変数の添字の直後から開始して、レシーバのクラスが定義しているインスタンス変数の添字に対して繰り返し処理をしています。
(\ct!instSize! メソッドは、クラスが定義している名前付きインスタンス変数の数を返します)。

%-----------------------------------------------------------------
\subsection{クラスやインターフェースへの問い合わせ}

\pharo の開発ツール(コードブラウザ、デバッガ, インスペクタなど)はすべて、これまでに見て来たリフレクティブな機能を使っています。

もういくつか、開発ツールを作る際に役立つメッセージを挙げましょう:

\lct{isKindOf: \emph{aClass}}は、レシーバが\lct{\emph{aClass}}そのいずれかのスーパークラスのインスタンスであれば真を返します。例えば:
\begin{code}{@TEST}
1.5 class                     --> Float
1.5 isKindOf: Number --> true
1.5 isKindOf: Integer   --> false
\end{code}
\mthindex{Object}{class}
\mthindex{Object}{isKindOf:}

\lct{respondsTo: \emph{aSymbol}}は、セレクタが\lct{\emph{aSymbol}}であるメソッドをレシーバが持っていれば真を返します。例えば:
\needlines{3}
\begin{code}{@TEST}
1.5 respondsTo: #floor      --> true    "Numberクラスはfloorを実装しています"
1.5 floor                            --> 1
Exception respondsTo: #, --> true    "例外クラスはグループ化できます"
\end{code}
\mthindex{Object}{respondsTo:}

\important{\emph{警告:}ここで紹介した機能は、開発ツールの記述には大変便利ですが、通常のアプリケーション作成には適していません。
オブジェクトが属するクラスや理解できるメッセージを調べようとするのは、カプセル化の原則に反しており、設計に問題がある兆候です。
しかしながら、開発ツールは普通のアプリケーションではありません。開発ツールの対象はソフトウェアそのものなので、コードの内部を詳細に調べる権利があると言えます。}

%There also exist mechanisms for introspecting on various parts of the run-time system, such as  the process scheduler, the memory manager and so on. For now we will focus on navigating through objects, classes and methods, and we will look more closely at rest of the runtime system in an other chapter.
%\on{let's not mention this if we don't actually write such a chapter!}

%-----------------------------------------------------------------
\subsection{コード・メトリクス}

\st のイントロスペクションを使って、いろいろなコード・メトリクスが簡単に調べられるという例を見てみましょう。
コード・\ind{メトリクス}とは、継承の階層の深さ、直接的・間接的なサブクラスの数、各クラスのメソッドやインスタンス変数の数、クラスで新しく定義されたメソッドやインスタンス変数の数などを計測することです。
以下に\ct!Morph!クラスのメトリクスをいくらか示します。\ct!Morph!クラスは\pharo のすべてのグラフィカル・オブジェクトのスーパークラスなので、それ自身が巨大で、かつ巨大な継承階層の根となっていることがわかります。もしかするとリファクタリングが必要かも!

\mthindex{Behavior}{allSuperclasses}
\mthindex{Behavior}{allSelectors}
\mthindex{Behavior}{allInstVarNames}
\mthindex{Behavior}{selectors}
\mthindex{Behavior}{instVarNames}
\mthindex{Behavior}{subclasses}
\mthindex{Behavior}{allSubclasses}
\mthindex{ClassDescription}{linesOfCode}
\begin{code}{}
Morph allSuperclasses size.  -->       2 "継承階層の深さ"
Morph allSelectors size.        --> 1378 "メソッドの数"
Morph allInstVarNames size. -->      6 "インスタンス変数の数"
Morph selectors size.             -->  998 "Morphで新しく定義されたメソッドの数"
Morph instVarNames size.     -->      6 "Morphで新しく定義されたインスタンス変数の数"
Morph subclasses size.          -->    45 "直接のサブクラスの数"
Morph allSubclasses size.      -->  326 "サブクラスの総数"
Morph linesOfCode.               --> 5968 "総コード行数"
\end{code}

オブジェクト指向言語において最も興味深いメトリクスの一つは、スーパークラスから継承したメソッドに拡張を加えているメソッドの数です。
このメトリクスから、クラスとそのスーパークラスの関係に関する情報が得られます。
次の節以降では、このような疑問に対する答えを見つけるために、ランタイムの情報に関する知識をどのように使えば良いのかを見ていきます。

%======================================
\section{コードをブラウズする}

\st では、すべてがオブジェクトです。特にクラスもまたオブジェクトであり、そのインスタンスを探索するための便利な機能を提供しています。
これから挙げるメッセージはほとんど\ct{Behavior}で実装されているので、どのクラスも理解できます。

前に挙げたように、クラスに\ct!#someInstance!メッセージを送ると、そのクラスのあるインスタンスが得られます。
\mthindex{Behavior}{someInstance}
\begin{code}{@TEST} % Possibly fragile!
Point someInstance --> 0@0
\end{code}

すべてのインスタンスを集めるには\ct!#allInstances!を、メモリ中にあるインスタンスの数を得るには\ct!#instanceCount!を使うことができます。

\alex{In a Pharo0.1-10342dev09.96.3, I have "ByteString instanceCount --> 63607"}
\mthindex{Behavior}{allInstances}
\mthindex{Behavior}{instanceCount}
\mthindex{Behavior}{allSubInstances}
\begin{code}{} % Cannot test this
ByteString allInstances        --> #('collection' 'position'  ...)
ByteString instanceCount    --> 104565
String allSubInstances size -->  101675
\end{code}

以下の機能は、クラスから特定の条件を満たすメソッドを抽出できるので、アプリケーションをデバッグするときに役立ちます。
\begin{itemize}
\item \mthind{Behavior}{whichSelectorsAccess:} は、指定したインスタンス変数を読み書きするすべてのメソッドのセレクタを返します。
\item \mthind{Behavior}{whichSelectorsStoreInto:} は、指定したインスタンス変数の値を変更するすべてのメソッドのセレクタを返します。
\item \mthind{Behavior}{whichSelectorsReferTo:} は、指定したメッセージを送るすべてのメソッドのセレクタを返します。
\item \mthind{Behavior}{crossReference} 各メッセージを、そのメッセージを送るメソッドの集合に関連づける % Pharo にこのメソッドは見当たらない?
\end{itemize}

\begin{code}{} % TOO FRAGILE TO TEST
Point whichSelectorsAccess: 'x'    --> #(#'\\' #= #scaleBy: ...)
Point whichSelectorsStoreInto: 'x' --> #(#setX:setY: ...)
Point whichSelectorsReferTo: #+  --> an IdentitySet(#rotateBy:about: ...)
Point crossReference --> an Array(
		an Array('*' an IdentitySet(#rotateBy:about: ...))
		an Array('+' an IdentitySet(#rotateBy:about: ...))
		...)
\end{code}

次のメッセージは継承階層も考慮した結果を返します:
\begin{itemize}
\item \mthind{Behavior}{whichClassIncludesSelector:} 与えられたメッセージを実装しているクラスを返します。
\item \mthind{Behavior}{unreferencedInstanceVariables} レシーバのクラスでもサブクラスでも使われていないインスタンス変数をリストにして返します。
\end{itemize}

\begin{code}{@TEST}
Rectangle whichClassIncludesSelector: #inspect --> Object
Rectangle unreferencedInstanceVariables            --> #()
\end{code}

\clsind{SystemNavigation}は、システム内のソースコードのブラウズや問い合わせを行うのに便利なメソッドを持つファサードです。
\ct{SystemNavigation} \mthind{SystemNavigation class}{default}を評価して得られるインスタンスを使ってシステム情報を調べられます。
例えば:

\mthindex{SystemNavigation}{allClassesImplementing:}
\begin{code}{@TEST}
SystemNavigation default allClassesImplementing: #yourself --> {Object}
\end{code}

次のメッセージの説明は不要でしょう:

\mthindex{SystemNavigation}{allSentMessages}
\mthindex{SystemNavigation}{allUnsentMessages}
\mthindex{SystemNavigation}{allUnimplementedCalls}
\begin{code}{}
SystemNavigation default allSentMessages size          --> 24930
SystemNavigation default allUnsentMessages size      --> 6431
SystemNavigation default allUnimplementedCalls size --> 270
\end{code}

中には実装されているのに使われていないメッセージもありますが、\ct{perform:}を使うなどして、ソースコードに現れない形で送られることもあるので、必ずしも不要とは限りません。
ただし、送られているのに実装されていないメッセージは、送信したときに失敗することになるので、問題がある可能性が高いと言えます。
このようなメッセージは、不完全な実装、廃止されたAPI、ライブラリ不足を示している可能性があります。

\mthindex{SystemNavigation}{allCallsOn:}
\ct!SystemNavigation default allCallsOn: #Point! は、ソースコード中で\ct!Point!を明示的にレシーバとしているすべてのメッセージを返します。

以上の機能はすべて、 \pharo のプログラミング環境(特にコードブラウザ)に統合されています。
すでにご存知だとは思いますが、メッセージをブラウズする際に使える便利なキーボードショートカットがあります。それらは、メッセージの実装を取得するための \short{m} 、メッセージの送信者を取得するための \short{n} などです。
一方で、このような問い合わせ機能が多数 \ct{SystemNavigation}クラスの\prot{browsing}プロトコルに属するメソッドとして実装されていることは、それほど知られていないかもしれません。
例えば、次の式を評価すると\ct{ifTrue:}メッセージのすべての実装をプログラム的にブラウズできます:
\mthindex{SystemNavigation}{browseAllImplementorsOf:}
\begin{code}{}
SystemNavigation default browseAllImplementorsOf: #ifTrue:
\end{code}

\begin{figure}[ht]\centering
	\includegraphics[width=\linewidth]{implementors}
	\caption{すべての\ct!\#ifTrue:!の実装をブラウズしている様子。\figlabel{implementors}}
\end{figure}

特に便利なメソッドは\ct{browseAllSelect:}と\lct{browseMethodsWithSourceString:}の二つです。以下にシステム中のすべてのメソッドのうち \super 送信を行うものをブラウズする方法を示します。最初の方法は総当たり的ですが、二つ目の方法ではより正確で、検索結果から不要なものを多少減らすことができます。
\mthindex{SystemNavigation}{browseMethodsWithSourceString:}
\mthindex{SystemNavigation}{browseAllSelect:}
\begin{code}{}
SystemNavigation default browseMethodsWithSourceString: 'super'.
SystemNavigation default browseAllSelect: [:method | method sendsToSuper ].
\end{code}

%======================================
\section{クラス、メソッド辞書、メソッド}

クラスもオブジェクトなので、他のオブジェクトと同様にインスペクト・エクスプロアできます。

\mthindex{Object}{explore}
\dothis{\ct{Point explore}を評価してみましょう。}

\figref{CompiledMethod}では、\ind{エクスプローラ}が\clsind{Point}クラスの構造を表示しています。
このクラスがメソッドを辞書として保持しており、この辞書のキーがメソッドのセレクタとなっていることがわかります。
セレクタ \ct{#*} をキーとしている要素として、\ct!Point>>>*!が逆コンパイルされた\ind{バイトコード}として表示されています。

\begin{figure}[ht]\centering
	\includegraphics[width=.5\linewidth]{CompiledMethod}
	\caption{\ct!Point!クラスと\ct!\#*!メソッドのバイトコードをエクスプロアする。\figlabel{CompiledMethod}}
\end{figure}

クラスとメソッドの関係について考えてみましょう。
\figref{MethodsAsObjects}を見ると、クラスとメタクラスは共通するスーパークラスとして \ct{Behavior} を持っていることがわかります。\ct{Behavior}が、\mthind{Behavior}{new}などのようにクラスに関するメソッドとして重要なものが定義されている場所となっています。
どのクラスもメソッドセレクタと\ind{コンパイル済みメソッド}を関連づけるメソッド辞書を持っています。
コンパイル済みメソッドは自分が所属するクラスを知っています。\figref{CompiledMethod}から、\ct{literal5}\ct{literal5}にキーをシンボル\ct{Point}、値を\ct{Point}クラスとするアソシエーションとして所属クラスの情報が含まれていることがわかります。

\begin{figure}[ht]\centering
	\includegraphics[width=\linewidth]{MethodsAsObjects}
	\caption{クラス、メソッド辞書、コンパイル済みメソッド\figlabel{MethodsAsObjects}}
\end{figure}

システムに関する情報を問い合わせる際にクラスとメソッドの関係を利用することができます。
例えば、あるクラスのメソッドのうち新しく定義されたもの、つまりスーパークラスのメソッドをオーバーライドしていないものを見つけるには、次のようにクラスから\ind{メソッド辞書}をたどります:
\mthindex{Behavior}{methodDict}
\begin{code}{}
[:aClass| aClass methodDict keys select: [:aMethod |
  (aClass superclass canUnderstand: aMethod) not ]] value: SmallInteger
  --> an IdentitySet(#threeDigitName #printStringBase:nDigits: ...)
\end{code}

コンパイル済みメソッドは、メソッドのバイトコードを保持しているだけではなく、
システムに問い合わせを行う便利なメソッドも数多く提供しています。
\ct{isAbstract}メソッド(\ct{subclassResponsibility}を送るかどうか)はそのようなメソッドのひとつです。
\ct{isAbstract}メソッドを使って、抽象クラスのすべての抽象メソッドを同定することができます。
\needlines{4}
\begin{code}{}
[:aClass| aClass methodDict keys select: [:aMethod |
  (aClass>>aMethod) isAbstract ]] value: Number
  --> #(#storeOn:base: #printOn:base: #+ #- #* #/ ...)
\end{code}
このコードでは、セレクタに対応するコンパイル済みメソッドを取得するための\ct{>>}メッセージが使われています。

% As a slightly more complex example, we can browse 

ある継承階層内で使われている \super 送信、例えばCollectionクラスのサブクラスで \super 送信を行っているメソッドの一覧をブラウズするには、以下のようにいくらか高度な問い合わせを行います。
\begin{code}{}
class := Collection.
SystemNavigation default
  browseMessageList: (class withAllSubclasses gather: [:each |
    each methodDict associations
      select: [:assoc | assoc value sendsToSuper]
      thenCollect: [:assoc | MethodReference class: each selector: assoc key]])
  name: 'Supersends of ' , class name , ' and its subclasses'
\end{code}
目的のメソッドを同定するため、クラスからメソッド辞書を経由してコンパイル済みメソッドへと探索している過程に注目してください。
\ct{MethodReference}はコンパイル済みメソッドのための軽量プロキシで、多くのツールで使われています。
\clsmthind{CompiledMethod}{methodReference}はコンパイル済みメソッドへのメソッド参照(\ct{MethodReference}のインスタンス)を返す便利なメソッドです。
\begin{code}{@TEST}
(Object>>#=) methodReference methodSymbol --> #=
\end{code}

%======================================
\section{ブラウザ環境}

\clsind{SystemNavigation}はシステムコードに対してプログラム的に問い合わせを行ったりブラウズしたりする便利な機能を提供しますが、もっとよい方法もあります。\pharo に統合されている\ind{リファクタリングブラウザ}を使うと、複雑な問い合わせを対話的にもプログラム的にも行うことができます。

例えば、\lct{Collection}クラス階層のメソッドで、メソッド自身と異なるメッセージを \super に送っているメソッドに興味があるとしましょう。
このようなコードは、普通であれば \super 送信を \self 送信に置き換えるべきなので、悪い \ind{コードの臭い} がすると考えられます。 (考えてみてください。 \super を使う必要があるのは、メソッドをオーバーライドして拡張するときだけです。継承されたメソッドにアクセスするのであれば、 \self にメッセージを送れば済みます。)

リファクタリングブラウザは、問い合わせを関心のあるクラスやメソッドに限定するエレガントな方法を提供しています。

\dothis{ブラウザを開いて\ct{Collection}クラスを見てみましょう。
クラス名を\actclick して\menu{refactoring scope>subclasses with}を選択すると、\ct{Collection}クラス階層に限定されたブラウザ環境が開きます。
この限定されたスコープの中で\menu{refactoring scope>super-sends}を選択すると、\ct{Collection}クラス階層にある \super 送信を行うすべてのメソッドを列挙した新しいウィンドウが開きます。
この中からあるメソッドをクリックして、\menu{refactor>code critics}を選んでください。
\menu{Lint checks>Possible bugs>Sends different super message}と選んでゆき、\menu{browse}を\actclick してみましょう。}

\figref{sendDifferentSuper}にあるように、\ct{Collection}クラス階層にはそのようなメソッドが19個あって、その中には \ct{super printOn:}を送っている\ct{Collection>>>printNameOn:}も含まれていることがわかります。
\begin{figure}[ht]\centering
	\includegraphics[width=\linewidth]{sendDifferentSuper}
	\caption{自身のセレクタとは異なる\super 送信を行うメソッドを探す様子。\figlabel{sendDifferentSuper}}
\end{figure}

このブラウザ環境はプログラムからでも用意できます。
例えば、次のコードでは\clsind{Collection}とそのサブクラスを扱う\clsind{BrowserEnvironment}を生成し、そこから\super 送信を行うメソッドのみを抽出して、その結果を表示するウィンドウを開きます。
\needlines{4}
\begin{code}{}
((BrowserEnvironment new forClasses: (Collection withAllSubclasses))
	selectMethods: [:method | method sendsToSuper])
	label: 'Collection methods sending super';
	open.
\end{code}{}

このコードは先に挙げた\ct{SystemNavigation}を使った例と同等ですが、よりコンパクトになっていることに注目してください。

最終的には次のようにして、自身のメソッドセレクタと異なる \super 送信を行うメソッドをプログラム的に探せます:
\needlines{6}
\begin{code}{}
((BrowserEnvironment new forClasses: (Collection withAllSubclasses))
	selectMethods: [:method | 
		method sendsToSuper
		and: [(method parseTree superMessages includes: method selector) not]])
	label: 'Collection methods sending different super';
	open
\end{code}
ここではメソッドセレクタと異なる \super メッセージを探すために、コンパイル済みメソッドのそれぞれから、(リファクタリングブラウザが使用する)構文木を取得しています。
\ct{RBProgramNode}クラスの\prot{querying}プロトコルを見て、構文木に何を尋ねることができるのかを確認してください。

%======================================
\section{実行時コンテキストにアクセスする}

これまで \st のリフレクティブな機能を使ったオブジェクト、クラス、メソッドへの問い合わせの仕方やエクスプロアの方法を見てきました。では、実行時環境についてはどうでしょう?

%-----------------------------------------------------------------
\subsection{メソッドコンテキスト}

実際のところ、実行中のメソッドに対応する実行時コンテキストはバーチャルマシンが管理しています --- 仮想イメージの中にはないのです!
しかし、\ind{デバッガ}は明らかにが実行時コンテキストにアクセスして動作しています。さらには、実行時コンテキストを探ることも、他のオブジェクトと同様に何の問題もありません。
どうなっているいるのでしょうか?

実は、デバッガには何の不思議なこともありません。
この秘密は、これまで軽く触れただけだった\pvind{thisContext}という疑似変数にあります。
メソッド中で\ct{thisContext}が参照されたときにはいつでも、そのメソッドのコンテキストが具現化され、リンクした一連の\clsind{MethodContext}オブジェクトとしてイメージから操作できるようになります。

この仕組みは簡単に実験できます。

\dothis{\ct{Integer>>>factorial}の実装に下線部の式を挿入してしてください:}

\mthindex{Object}{halt}
\begin{code}{}
Integer>>>factorial
	"レシーバーの階乗を返す。"
	self = 0 ifTrue: [!\underline{thisContext explore. self halt.}! ^ 1].
	self > 0 ifTrue: [^ self * (self - 1) factorial].
	self error: 'Not valid for negative integers'
\end{code}

\dothis{\ct{3 factorial}をワークスペースで評価します。\figref{exploringThisContext}のようにデバッガウィンドウとエクスプローラの両方が表示されはずです。}

\begin{figure}[ht]\centering
	\includegraphics[width=\linewidth]{exploringThisContext}
	\caption{\lct{thisContext}をエクスプロアしている様子。\figlabel{exploringThisContext}}
\end{figure}

貧相なデバッガへようこそ!
ここでエクスプロアされたオブジェクトのクラスをブラウズすると(\ie エクスプローラの下のペインで\ct{self browse}を評価すると)、\ct{thisContext}が、\lct{MethodContext}のインスタンスであることがわかるでしょう。\ct{sender}の連鎖からたどれるすべてのオブジェクトも同様です。
% All of these objects have been created dynamically in the image by the \st virtual machine at the point where \ct{thisContext} was referred to in the \ct{factorial} method. \lr{Not actually. In all the currently available VMs the context objects are created with every method activation, no matter if they are accessed using \ct{thisContext} or not.}

\ct{thisContext}は日常的なプログラミングで使うものではありませんが、デバッガのようなツールを実装したり、コールスタックの情報にアクセスしたりする際には必要不可欠です。
次の式を評価すると、\ct{thisContext}を使っているメソッドを見つけられます。

\begin{code}{}
SystemNavigation default browseMethodsWithSourceString: 'thisContext'
\end{code}

実は、メッセージのセンダを取得することが\ct{thisContext}の最も一般的な用途となっています。
次に示すコードは典型的な例です:
\begin{code}{}
Object>>>subclassResponsibility
	"このメッセージにより、サブクラスの挙動に関するをフレームワークを設定します。
	サブクラスがこのメッセージを実装すべきであると宣言します。"

	self error: 'My subclass should have overridden ', thisContext sender selector printString
\end{code}

\st の慣習では、\ct{self subclassResponsibility} を送るメソッドは抽象メソッドとみなされます。
しかし、\clsmthind{Object}{subclassResponsibility} はどの抽象メソッドが呼ばれたのかという情報を含むエラーメッセージをどうやって提供するのでしょうか?
答えは簡単。\ct{thisContext}にセンダを尋ねるだけです。

\lr{I think co-routines and continuations should at least mentioned here. Another very practical application that is simple and could be shown here is the ``escaper''. Store the current context into a temp or inst-var \ct{target := thisContext} and jump back to that stack frame at a later point in time using \ct{target return: 123}.}
\sd{lukas maybe we should have another chapter showing such kind of beasts. I would love to read it.
Showing how to use block to build exception and such a kind of point. I think that this chapter should be an introduction may be
we should have a Reflection applied chapter}
\lr{I would love to help writing such a chapter}
%-----------------------------------------------------------------
\subsection{賢いブレークポイント}

\mthindex{Object}{halt}
\st でブレークポイントを設定するには、メソッド内の関心のある箇所で \ct{self halt} を評価するコードを書きます。この式が実行されると\ct{thisContext}が具現化され、そのブレークポイントにおける\ind{デバッガ}のウィンドウが開きます。
しかしながら、システムで集中的に使われるメソッドにブレークポイントを設定すると問題が起きてしまいます。

例えば、\ct{OrderedCollection>>>add:}の実行を解析したいとしましょう。
このメソッド内にブレークポイントを設定すると困ったことになります。

\dothis{\emph{新しい}仮想イメージを用意して、次のブレークポイントを設定します:}
\needlines{3}
\begin{code}{}
OrderedCollection>>>add: newObject
	!\underline{self halt.}!
	^self addLast: newObject
\end{code}

仮想イメージがデバッガウィンドウすら開かずに直ちにフリーズしてしまいます!
次の点を理解すれば、問題がはっきりとわかります。 (i) \ct{OrderedCollection>>>add:}はシステムのあちこちで使われているので、変更をアクセプトしてごく間もない時点でブレークポイントが起動します。 (ii) しかし、起動した\emph{デバッガ自身}が\ct{add:}メッセージを\ct{OrderedCollection}のインスタンスに送るので、そのデバッガが邪魔されてしまいます!
このような問題を避けるには、デバッグしたい状況でのみ\emph{条件的に停止}する方法が必要です。
その用途にぴったりなのが\clsmthind{Object}{haltIf:}です。

では\ct{add:}を、例えば\ct{OrderedCollectionTest>>>testAdd}のコンテキストから呼ばれたときにだけ停止させたいとしましょう。

\dothis{また新しい仮想イメージを用意して、次のブレークポイントを設定してみましょう:}
\begin{code}{}
OrderedCollection>>>add: newObject
	!\underline{self haltIf: \#testAdd.}!
	^self addLast: newObject
\end{code}

\ct{OrderedCollectionTest}を実行してみましょう。今度はフリーズしません。
(\ct{OrderedCollectionTest}は\scat{CollectionsTests-Sequenceable}カテゴリにあります)

どう動いているのでしょうか? \clsmthind{Object}{haltIf:}を見てみましょう:
\begin{code}{}
Object>>>haltIf: condition
	| cntxt |
	condition isSymbol ifTrue: [
		"指定されたセレクタ・シンボルのメソッドがセンダの呼び出しチェーンに含まれている場合のみ停止する"
		cntxt := thisContext.
		[cntxt sender isNil] whileFalse: [
			cntxt := cntxt sender. 
			(cntxt selector = condition) ifTrue: [Halt signal]. ].
		^self.
	].
	...
\end{code}

\ct!haltIf:!は実行時スタックを\ct!thisContext!から遡りながら、呼び出し側のメソッドの名前が引数と一致するか確認します。
一致したら例外を発生させます。この例外はデフォルトの挙動として、デバッガを起動します。

\ct{haltIf:}の引数に真偽値や真偽値を返すブロックを渡してもよいですが、それらの場合は単純な処理になるので\lct{thisContext}は使わずにすみます。

%======================================
\section{理解されないメッセージをインターセプトする}
\seclabel{msgnotunderstood}

ここまで、 \st のリフレクティブな機能をオブジェクト、クラス、メソッド、実行時スタックに問い合わせを行ったりエクスプロアするのに主に使ってきました。ここからは \st システムの構造に関する知識を、メッセージをインターセプトしたり振る舞いを実行時に変えるためにどう使っていけるのかを見ていきます。

オブジェクトはメッセージを受け取ると、まず自身のクラスのメソッド辞書からメッセージに対応するメソッドを探します。
対応するメソッドが見つからなければ、\ct{Object}にたどり着くまでクラス階層を遡りつつ探します。それでも見つからなければ、オブジェクトは\ct{doesNotUnderstand:}メッセージを、元々のメッセージセレクタとメッセージへの引数を引数として\emph{自分自身に送ります}。
メソッド探索の処理が\ct{doesNotUnderstand:}の送信に対しても\clsmthind{Object}{doesNotUnderstand:}が見つかるまで繰り返され、その結果としてデバッガが起動します。

しかし、もし\ct{doesNotUnderstand:}がメソッド検索パスにある\ct{Object}のサブクラスでオーバーライドされていたらどうなるでしょう?
これはある種の非常に動的な振る舞いを実現するのに便利な方法でです。\ct{doesNotUnderstand:}をオーバーライドすることで、オブジェクトは理解できないメッセージに応答するための代替的な戦略を持つことができます。

この手法の最も一般的な応用例は、(1) \ind{軽量プロキシ}の実装と (2) 足りないコードの動的なコンパイルまたはロード、です。

%-----------------------------------------------------------------
\subsection{軽量プロキシ}

最初の例として、既存のオブジェクトのプロキシとして振る舞う「\ind{最小限のオブジェクト}」を紹介します。
このプロキシは固有のメソッドを実質的にひとつも持たないので、どんなメッセージも\ct{doesNotUnderstand:}で捕捉されることになります。このメッセージを実装すれば、プロキシは実体のオブジェクトにメッセージを移譲する前に特別な動作を行うことができます。

これがどのように実装できるのかを見てみましょう。\footnote{\pkg{PBE-Reflection}を\url{http://www.squeaksource.com/PharoByExample/}からロードできます}。

まず、\ct{LoggingProxy}を次のように定義します:
\begin{code}{}
ProtoObject subclass: #LoggingProxy
	instanceVariableNames: 'subject invocationCount'
	classVariableNames: ''
	poolDictionaries: ''
	category: 'PBE-Reflection'
\end{code}
\ct{Object}ではなく\ct{ProtoObject}のサブクラスとして定義していることに注意してください。\ct{Object}が持つ400個以上のメソッド(!)を継承したくないからです。

\begin{code}{}
Object methodDict size --> 408
\end{code}

このプロキシには二つのインスタンス変数があります。 \ct{subject} はプロキシの実体であり、\ct{invocationCount}はプロキシがインターセプトしたメッセージの数です。
二つのインスタンス変数を初期化し、インターセプトしたメッセージの数を返すアクセサを提供します。
初期化時点では、\ct{subject}変数はプロキシオブジェクト自身を指しています。
\begin{code}{}
LoggingProxy>>>initialize
	invocationCount := 0.
	subject := self.
\end{code}

\begin{code}{}
LoggingProxy>>>invocationCount
	^ invocationCount
\end{code}

この例では、理解できないすべてのメッセージをインターセプトし、トランスクリプトに表示し、メッセージ受信回数を更新し、メッセージを実体へ転送します。
\begin{code}{}
LoggingProxy>>>doesNotUnderstand: aMessage 
	Transcript show: 'performing ', aMessage printString; cr.
	invocationCount := invocationCount + 1.
	^ aMessage sendTo: subject
\end{code}

ここでちょっとした魔法の登場です。
\ct{Point}オブジェクトと\ct{LoggingProxy}オブジェクトを生成し、プロキシに「その\ct{Point}オブジェクトになれ(\mthind{ProtoObject}{become:})」と伝えます。
\seeindex{\ct{become:}}{\ct{ProtoObject>>>become:}}
\begin{code}{}
point := 1@2.
LoggingProxy new !\underline{become:}! point.
\end{code}

これにより、イメージ内にある、この\ct{Point}オブジェクトへのすべての参照をプロキシへの参照へと入れ替えます(逆も同様です)。重要なのは、これによってプロキシの\ct{subject}インスタンス変数もこの\ct{Point}オブジェクトを参照するようになることです!

\begin{code}{}
point invocationCount --> 0
point + (3@4)             --> 4@6
point invocationCount --> 1
\end{code}

これはたいていの場合うまく動作するのですが、足りない点もあります:
\begin{code}{}
point class --> LoggingProxy
\end{code}
興味深いことに、\ct{class}メソッドは\ct{ProtoObject}で実装されておらず(\ct{Object}で実装されています)、\ct{LoggingProxy}は\ct{Object}を継承していません!
\ct{class}はメッセージとして送られるのではなく、仮想マシンによって直接処理されます。
\footnote{\ct{yourself}も同様に、決して送られません。
レシーバによって仮想マシンが直接解釈する可能性のあるメッセージを次に示します:
\ct{+- < > <= >= = ~= * / \ ==}
\ct{@ bitShift: // bitAnd: bitOr:}
\ct{at: at:put: size}
\ct{next nextPut: atEnd}
\ct{blockCopy: value value: do: new new: x y}
次に示すセレクタは決して送られません。コンパイラが比較とジャンプのバイトコードにインライン展開します:
\ct{ifTrue: ifFalse: ifTrue:ifFalse: ifFalse:ifTrue:}
\ct{and: or:}
\ct{whileFalse: whileTrue: whileFalse whileTrue}
\ct{to:do: to:by:do:}
\ct{caseOf: caseOf:otherwise:}
\ct{ifNil: ifNotNil:  ifNil:ifNotNil: ifNotNil:ifNil:}
これらのメッセージを論理値ではないオブジェクトに送ろうとした場合も、
レシーバが\ct{mustBeBoolean}をオーバーライドするか、\ct{NonBooleanReceiver}例外を捕捉して適切な論理値を返せば実行を継続できます。
}% NB: Notes by Lukas Renggli

このような特別なメッセージ送信を無視するとしても、この手法では克服不可能な、\self 送信をインターセプトできないという根本的な問題があります。
\begin{code}{}
point := 1@2.
LoggingProxy new become: point.
point invocationCount --> 0
point rect: (3@4)        --> 1@2 corner: 3@4
point invocationCount --> 1
\end{code}

このプロキシは\ct{rect:}メソッド中の二度の\self 送信を数え損ねています:
\begin{code}{}
Point>>>rect: aPoint 
	^ Rectangle  origin: (self min: aPoint) corner: (self max: aPoint)
\end{code}

この方法のプロキシを使えばメッセージをインターセプトできますが、プロキシに起因する制限に気をつけてください。
\secref{wrapper}では、メッセージをインターセプトするためのより一般的な方法を見ていきます。

%-----------------------------------------------------------------
\subsection{欠けているメソッドを生成する}

理解されないメッセージをインターセプトする手法の一般的な応用例として、未実装のメソッドの動的なロードまたは生成が挙げられます。
多数のメソッドを持つ巨大なクラスライブラリを考えてみましょう。ライブラリ全体をロードする代わりに各クラスのスタブをロードしてもよいでしょう。このスタブは、そのクラスで定義されたすべてのメソッドのソースコードがどこにあるかを知っています。スタブは理解できないメッセージをすべて捕捉し、欠けているメソッドを必要に応じて動的にロードします。この処理をある時点で無効化すれば、それまでにロードされたコードが特定のクライアントアプリケーションに最低限必要なものとして保存することができます。

%\on{Stef sez: check ObjectOut -- I looked, but this seems to be very old. Depends on SqueakPage.}

この方法のシンプルな応用例を見てみましょう。次に示すクラスは必要に応じてインスタンス変数へのアクセサを自動的に追加します:
% \lr{the last statement should return the result of the message, otherwise you cannot proceed with the debugger} \alex{all redefinition of doesNotUnderstand: includes a return statement. However, I do not see your comment lukas, I tried to insert a 'self halt' in the method, I was able to proceed. I added the return in the function} \lr{Of course it depends on the exact circumstances. If you perform a message on self that returns self it does not matter, but in any other case a forgotten return can introduce strange side effects. There was no return in the listing below, but now there is and the problem is solved.}

\begin{code}{}
DynamicAcccessors>>>doesNotUnderstand: aMessage
	| messageName |
	messageName := aMessage selector asString.
	(self class instVarNames includes: messageName)
		ifTrue: [
			self class compile: messageName, String cr, ' ^ ', messageName.
			^ aMessage sendTo: self ].
	^ super doesNotUnderstand: aMessage
\end{code}
理解できないメッセージはすべてここで捕捉されます。送られたメッセージと同名のインスタンス変数があれば、自身のクラスにアクセサをコンパイルするように依頼します。その後、メッセージをもう一度送信します。

\ct{DynamicAccessors}クラスに(初期化されていない)インスタンス変数\ct{x}があり、\ct{x}のアクセサが定義されていないとしましょう。次の式を評価すると、アクセサが動的に生成されて\ct{x}の値を取得できます:
\needlines{2}
\begin{code}{}
myDA := DynamicAccessors new.
myDA x --> nil
\end{code}

\ct{DynamicAccessors}オブジェクトに最初に\ct{x}メッセージが送られたときに何が起きるか、順に見て行きましょう(\figref{DynamicAccessors}を参照)。

\begin{figure}[ht]\centering
	\includegraphics[width=\linewidth]{DynamicAccessors}
	\caption{アクセサを動的に生成する。\figlabel{DynamicAccessors}}
	% \alex{not sure whether the figure is highly necessary. The code is rather simple and the figure complete to follow in my opinion.}}
	% \on{trust me, it is useful to see all the steps.}
\end{figure}

(1) \ct{x}を\ct{myDA}に送ると (2) メソッドがクラスのメソッド辞書から探索されますが、(3) クラス階層をたどっても見つかりません。(4) 従って\ct{self doesNotUnderstand: #x}がオブジェクトに送り返され、(5) 改めてメソッド探索が行われます。\ct{DynamicAccessors}ではすぐに\ct{doesNotUnderstand:}が見つかり、(6) 文字列\ct{'x ^ x'}をコンパイルするよう依頼します。 (7) \ct{compile}メソッドが探索され、(8) 最終的に\ct{Behavior}クラスで見つかります。(9-10) コンパイル済みメソッドが\ct{DynamicAccessors}のメソッド辞書に追加されます。(11-13) 最後に再び\ct{x}メッセージが再送されますが、今度は対応するメソッドが見つかります。

この方法はインスタンス変数のセッターや、その他のお約束コード、例えばVisitorパターンで使うメソッドの生成などに使えます。

ステップ(13)で使われている\clsmthind{Object}{perform:}に注目してください。このメソッドを使うと、実行時に動的にメッセージを生成して送れます。
\begin{code}{@TEST}
5 perform: #factorial                                             --> 120
6 perform: ('fac', 'torial') asSymbol                       --> 720
4 perform: #max: withArguments: (Array with: 6) --> 6
\end{code}

%======================================
\section{メソッドラッパーとしてのオブジェクト}
\seclabel{wrapper}

これまで見てきたように、\st ではコンパイル済みメソッドも通常のオブジェクトであり、プログラマが使うことのできる多くの問い合わせ用のメソッドを用意しています。
ただし、\emph{どんな}オブジェクトでもコンパイル済みメソッドの役を演じることができるということはちょっと意外かもしれません。そのオブジェクトは、\ct{run:with:in:}およびその他いくつかの重要なメッセージに応答しさえすればいいのです。

\dothis{空のクラス\ct{Demo}を定義します。\ct{Demo new answer42}を評価して、いつもの``Message Not Understood''エラーの発生を確認してください。}

ここで、普通のオブジェクトを\ct{Demo}クラスのメソッド辞書にインストールします。

\dothis{\lct{Demo methodDict at: \#answer42 put: ObjectsAsMethodsExample new.}を評価し、再び\ct{Demo new answer42}の結果を表示してください。今度は\ct{42}という回答を得られます。}

\clsind{ObjectsAsMethodsExample}クラス定義を調べると次の2つのメソッドが見つかります:
%\alex{I would prefer having return 42 in the run:with:in: method, and not having answer42 defined in ObjectsAsMethodsExample, this could be confusing I imagine}
%\on{ObjectsAsMethodsExample is part of the standard pharo image -- it is not in PBE-Reflection}
\needlines{5}
\begin{code}{}
answer42
	^42

run: oldSelector with: arguments in: aReceiver
	^self perform: oldSelector withArguments: arguments
\end{code}

\ct{Demo}のインスタンスが\ct{answer42}メッセージを受け取るといつも通りにメソッド探索が行われますが、仮想マシンは普通のオブジェクトがコンパイル済みメソッド役を演じようとしていることを検出します。
そこで、仮想マシンは\ct{ObjectsAsMethodsExample}オブジェクトに\ct{run:with:in:}メッセージを送ります。このメッセージの引数は最初に送られたメソッドセレクタ、引数、レシーバです。
\ct{ObjectsAsMethodsExample}は\ct{run:with:in:}メソッドを実装しているので、メッセージをインターセプトして自身に移譲します。

次のようにすれば、この偽のメソッドを削除できます:
\begin{code}{}
Demo methodDict removeKey: #answer42 ifAbsent: []
\end{code}

\ct{ObjectsAsMethodsExample}をさらに詳しく調べると、スーパークラスが\ct{flushCache}、\ct{methodClass:}、\lct{selector:}のメソッドを定義していることがわかりますが、定義だけで何も実装されていません。これらのメッセージはコンパイル済みメソッドに送られる可能性があるので、コンパイル済みメソッドのふりをするために実装しておく必要があります(\ct{flushCache}はこの中で最も重要なメソッドです。他のメソッドは\clsmthind{Behavior}{addSelector:withMethod:}または\clsmthind{MethodDictionary}{at:put:}でメソッドをインストールする場合に必要になります)。

%-------------------------------------------------------------------------
\subsection{メソッドラッパーをテストカバレッジの評価に使う}

メソッドラッパーはメッセージをインターセプトする有名な方法です\cite{Bran98a}。
オリジナルの実装\footnote{http://www.squeaksource.com/MethodWrappers.html}では、メソッドラッパーは\ct{CompiledMethod}のサブクラスのインスタンスです。インストールされると、メソッドラッパーは元のメソッドの起動前後に特別な動作を実行できます。
アンインストールすると、元のメソッドがメソッド辞書の然るべき位置に戻されます。

Pharoでは\ind{メソッドラッパー}を簡単に実装できます。\ct{CompiledMethod}のサブクラスを作る代わりに\ct{run:with:in:}を実装します。実際、メソッドラッパーとしてのオブジェクトの、軽量の実装\footnote{http://www.squeaksource.com/ObjectsAsMethodsWrap.html}が存在します。しかし、これを書いている時点では、この実装は標準のPharoには組み込まれていません。

いずれにせよ、Pharoのテストランナーはまさにこの手法を使ってテストカバレッジを評価しています。
テストランナーがどのように動作するのか見てみましょう。

テストカバレッジの開始点は\clsmthind{TestRunner}{runCoverage}です:
\begin{code}{}
TestRunner>>>runCoverage
	| packages methods |
	... "カバレッジを測定するメソッドを決定する"
	self collectCoverageFor: methods
\end{code}

\clsmthind{TestRunner}{collectCoverageFor:}メソッドに、カバレッジを調べるアルゴリズムがわかりやすく書かれています:
\begin{code}{}
TestRunner>>>collectCoverageFor: methods
	| wrappers suite |
	wrappers := methods collect: [ :each | TestCoverage on: each ].
	suite := self
		reset;
		suiteAll.
	[ wrappers do: [ :each | each install ].
	  [ self runSuite: suite ] ensure: [ wrappers do: [ :each | each uninstall ] ] ] valueUnpreemptively.
	wrappers := wrappers reject: [ :each | each hasRun ].
	wrappers isEmpty 
		ifTrue: 
			[ UIManager default inform: 'Congratulations. Your tests cover all code under analysis.' ]
		ifFalse: ...
\end{code}
カバレッジを調べたい各メソッドごとにラッパーが生成され、それらがインストールされます。
ここでテストが実行され、テスト実行後にラッパーがアンインストールされます。
最終的にカバーされていないメソッドに関するフィードバックを得られます。

ラッパー自身はどのように動くのでしょうか?
\ct{TestCoverage}ラッパーには3個のインスタンス変数(\ct{hasRun}、\ct{reference}、\ct{method})があり、次のように初期化されます。

\begin{code}{}
TestCoverage class>>>on: aMethodReference
	^ self new initializeOn: aMethodReference

TestCoverage>>>initializeOn: aMethodReference
	hasRun := false.
	reference := aMethodReference.
	method := reference compiledMethod
\end{code}

installメソッドとuninstallメソッドは単純な方法でメソッド辞書を更新します:
\begin{code}{}
TestCoverage>>>install
	reference actualClass methodDictionary
		at: reference methodSymbol
		put: self

TestCoverage>>>uninstall
	reference actualClass methodDictionary
		at: reference methodSymbol
		put: method
\end{code}
\noindent
\ct{run:with:in:}メソッドは\ct{hasRun}インスタンス変数を更新し、ラッパーをアンインストールし(カバレッジが検証されたので不要になります)、元のメソッドにメッセージを再送します。
\begin{code}{}
run: aSelector with: anArray in: aReceiver
	self mark; uninstall.
	^ aReceiver withArgs: anArray executeMethod: method

mark
	hasRun := true
\end{code}
(\clsmthind{ProtoObject}{withArgs:executeMethod:}を調べて、メソッド辞書から取り除かれたメソッドがどう起動されるのか確認してください。)

たったこれだけです!

メソッドラッパーを使えば、あらゆる種類の動作をメソッドの通常処理の前後で実行することができます。典型的な応用例は計測(メソッドの呼び出しパターンの統計情報の収集)、選択的な事前・事後条件の確認、メモ化(メソッドの処理結果を選択的にキャッシュすること)です。

%======================================
\section{プラグマ}

\emphind{プラグマ}とはプログラムに付与する注釈の一種で、プログラムに関連するデータを追加することができます。
プラグマはメソッド実行に対して直接的な影響は及ぼしません。
プラグマには以下のような物を含むいくつもの使い道があります:
\begin{itemize}
\item コンパイラへの情報: \indmain{プラグマ}はコンパイラにメソッド呼び出しをプリミティブ関数として扱うよう指示します。プリミティブ関数は仮想マシンか外部プラグインで定義する必要があります。
\item 実行時処理: プラグマによって付加された情報を実行時に検証することができます。
\end{itemize}

プラグマはメソッド宣言にのみ適用できます。
メソッドには複数のプラグマを宣言でき、\st 文より前に宣言する必要があります。
プラグマはリテラルを引数に取る静的なメッセージ送信と言えます。

この章の最初でプリミティブを紹介したときにプラグマが出てきました。
プリミティブはプラグマ宣言以上の物ではありません。
\ct{instVarAt:}メソッド内の\ct{<primitive: 73>}を見てください。
このプラグマのセレクタは\ct{primitive:}で、引数はリテラル\ct{73}です。

おそらくコンパイラはプラグマの大口ユーザです。
SUnitも注釈を活用するツールです。
SUnitはテスト単体からアプリケーションのカバレッジを推定することができます。
しかし、カバレッジから一部のメソッドを除外したくなることもあるでしょう。
\ct!SplitJointTest class!の\ct!documentation!メソッドはそのような例となっています:

\begin{code}{}
SplitJointTest class>>>documentation
	<ignoreForCoverage>
	"self showDocumentation"
	
	^ 'This package provides function.... "
\end{code}

メソッドに\ct!<ignoreForCoverage>!プラグマを加えるだけで、カバレッジの適用範囲を制御できます。

%Beside the compiler, Lint is a heavy user of pragmas. Lint is a static code analyzer that flags suspicious, non-portable constructs and code that is likely to contain bugs. It may happen that a method needs to be excluded from Lint analysis. This is the case here:

%\begin{code}{}
%MorphObjectOut>>>doesNotUnderstand: aMessage 
%	"Bring in the object, install, then resend aMessage"
%	"Transcript show: thisContext sender selector; cr."
%	"useful for debugging"
%	
%	! \textbf{<lint: 'Unnecessary "= true"' rationale: 'recursionFlag may be nil' author: 'stephane.ducasse'>}!
%	...
%\end{code}	

%One of the pragmas used by Lint to filter out methods is \ct{lint:rationale:author:}.

プラグマは\clsind{Pragma}のインスタンスであり、1級のオブジェクトです。
コンパイル済みメソッドに\mthind{CompiledMethod}{pragmas}メッセージを送ると、プラグマの配列を返します。

\begin{code}{}
(SplitJoinTest class >> #showDocumentation) pragmas.
  --> an Array(<ignoreForCoverage>)
(Float>>#+) pragmas --> an Array(<primitive: 41>)
\end{code}

(以下の\ct+allNamed:in:+メソッドにより)特定のプラグマを定義しているメソッドをクラスから取得することもできます。
\ct!SplitJoinTest!には\ct!<ignoreForCoverage>!と注釈されたクラスメソッドがあります:

\begin{code}{@TEST}
Pragma allNamed: #ignoreForCoverage in: SplitJoinTest class  --> an Array(<ignoreForCoverage> <ignoreForCoverage> <ignoreForCoverage>)
\end{code}

\ct{allNamed:in:}の変種が、\ct{Pragma}のクラスメソッドとして定義されています。

プラグマは自身が定義されたメソッド(\ct{method}を使います)、そのメソッド名(\ct{selector})、そのメソッドを持つクラス(\ct{methodClass})、引数の数(\ct{numArgs})、リテラルを引数として持つかどうか(\ct{hasLiteral:}と\ct{hasLiteralSuchThat:})を知っています。

\lr{Typically pragmas are performed on an interpreter object that understands the pragma message.}

%======================================
\section{章のまとめ}

リフレクションとは、ランタイムシステムのメタオブジェクトに対して普通のオブジェクトと同様に問い合わせたり、検証したり、変更すらできる機能です。

\begin{itemize}
\item インスペクタは\ct{instVarAt:}や関連するメソッドを使って、オブジェクトの「プライベートな」インスタンス変数に問い合わせたり、変数の内容を変更します。
\item クラスの全インスタンスを取得したいときは、\ct{Behavior>>>allInstances}を送ります。
\item \ct{class}、\ct{isKindOf:}、\ct{respondsTo:}などのメソッドはメトリクスの収集や開発ツールの実装には便利ですが、通常のアプリケーションでは避けるべきです: これらのメソッドはオブジェクトのカプセル化の原則に反しており、コードの理解や保守の邪魔になります。
\item \ct{SystemNavigation}はクラス階層の探索や閲覧に便利な多くの問い合わせ方法を持つユーティリティクラスです。例えば、ある文字列をソースコードに含むメソッドをすべて見つけ出すには\ct{SystemNavigation default browseMethodsWithSourceString: 'pharo'.}を使います(遅いですが、完璧です!)。
\item どのクラスも\ct{MethodDictionary}のインスタンスを参照しています。メソッド辞書はセレクタとコンパイル済みメソッドのインスタンスを結び付けています。コンパイル済みメソッドは自身が所属するクラスを知っており、お互いの参照が循環しています。
\item \ct{MethodReference}はコンパイル済みメソッドのための軽量プロキシです。便利なメソッドを提供しており、多くの\st のツールに使われています。
\item リファクタリングブラウザの基盤の一部である\ct{BrowserEnvironment}を使えば、\ct{SystemNavigation}よりも洗練されたインターフェースでシステムに問い合わせられます。問い合わせた結果を、次の新しい問い合わせの対象範囲として使えます。GUIにとプログラム的なインターフェースのどちらも利用できます。
\item \ct{thisContext}は仮想マシンの実行時スタックを具現化した疑似変数です。主にデバッガで使われ、スタックの対話的な操作を動的に実現しています。メッセージのセンダを動的に調べる場合にも便利です。
\item 賢いブレークポイントを設定するには\ct{haltIf:}を使います。このメソッドは引数にメソッドセレクタを取り、実行時スタックのセンダが同名のメソッドである場合のみ停止します。
\item 対象へのメッセージをインターセプトする一般的な方法は「最小限のオブジェクト」をプロキシとして使うことです。プロキシが実装するメソッドをできるだけ減らし、\ct{doesNotunderstand:}を実装してすべてのメッセージ送信を捕捉します。これで追加のアクションを実行してから対象にメッセージを転送できます。
\item プロキシと主体のような二つのオブジェクトへの参照を交換するには\ct{become:}を使います。
\item \ct{class}や\ct{yourself}のように、実際には送られず仮想マシンによって解釈されるメッセージに気をつけて下さい。他にも\ct{+}\ct{-}、\ct{ifTrue:}などのメッセージは、レシーバによっては仮想マシンに直接解釈されたり、インライン展開される可能性があります。
\item 欠けているメソッドの遅延ロードまたは遅延コンパイルのために\ct{doesNotUnderstand:}をオーバーライドすることもよくあります。
\item \ct{doesNotUnderstand:}は\self 送信を捕捉できません。
\item より正確にメッセージをインターセプトするには、オブジェクトをメソッドラッパーとして使います。メソッドラッパーはコンパイル済みメソッドの代わりにメソッド辞書にインストールされます。メソッドラッパーは\ct{run:with:in:}を実装すべきです。このメッセージは仮想マシンがメソッド辞書からコンパイル済みメソッドの代わりに普通のオブジェクトを検出したときに送られます。この方法はSUnitテストランナーでカバレッジデータを集めるのに使われています。
\end{itemize}

%=========================================================
\ifx\wholebook\relax\else
   \bibliographystyle{jurabib}
   \nobibliography{scg}
   \end{document}
\fi
%=========================================================

%Other stuff:
%- anonymous classes (uses compile: and primitiveChangeClassTo:) ???
%- collect direct senders; class collaborations
%- Object primitiveChangeClassTo: become: and becomeForward: (see tests and slides with minimal object example)
%- PointerFinder?
%- anonymous classes (see slides) ?

%Test  Coverage using ObjectsAsMethodsWrap package:
%\begin{code}{}
%category := 'SCGPier'.
%w := (ObjectAsOneTimeMethodWrapper installOnClassCategory: category).
%tr := TestRunner new.
%ToolBuilder open: tr.
%[tr
%	categoryAt: (tr categoryList indexOf: 'SCGPier') put: true;
%	selectAllClasses;
%	runAll.]
%ensure: [[w do: [:each| each uninstall ]] valueUnpreemptively].
%((w select: [:each | each executed not ])
%	collect: [:each | each wrappedClass name, '>>', each selector name ]) explore.
%\end{code}
